\documentclass[12pt]{article}

\usepackage{pgfplots}
\pgfplotsset{compat=1.18} % Or adjust depending on your TeX distribution
\usepackage[utf8]{inputenc}	% Para caracteres en español
\usepackage{amsmath,amsthm,amsfonts,amssymb,amscd}
\usepackage{multirow,booktabs}
\usepackage[table]{xcolor}
\usepackage{fullpage}
\usepackage{lastpage}
\usepackage{newtxtext}
\usepackage{newtxmath}
\usepackage{enumitem}
\usepackage{fancyhdr}
\usepackage{mathrsfs}
\usepackage{wrapfig}
\usepackage{setspace}
\usepackage{calc}
\usepackage{multicol}
\usepackage{cancel}
\usepackage[retainorgcmds]{IEEEtrantools}
\usepackage[margin=3cm]{geometry}
\usepackage{amsmath}
\newlength{\tabcont}
\setlength{\parindent}{0.0in}
\setlength{\parskip}{0.05in}
\usepackage{empheq}
\usepackage{framed}
\usepackage[most]{tcolorbox}
\usepackage{xcolor}
\colorlet{shadecolor}{orange!15}
\parindent 0in
\parskip 12pt
\geometry{margin=1in, headsep=0.25in}
\theoremstyle{definition}
\newtheorem{defn}{Definition}
\newtheorem{reg}{Rule}
\newtheorem{exer}{Exercise}
\newtheorem{note}{Note}
\newtheorem{theorem}{Theorem}
\setcounter{section}{0}
\usepackage{chngcntr}
\usepackage{hyperref}
\counterwithin*{equation}{section}
\counterwithin*{equation}{subsection}


\begin{document}

\section{Info}

\begin{itemize}
    \item karinachattah@unc.edu.ar
\end{itemize}

\textbf{Temas:}

\begin{itemize}
    \item Cinemática y dinámica (mecánica)
    \item Campos eléctricos y magnéticos
    \item Circuitos 
    \item Termodinámica
\end{itemize}

\section{Measurements and magnitudes}

Measurements seek to compare a prediction with an observation, so as to test a
hypothesis. A magnitude is a number accompanied by a unit. Some magnitudes are: 

\begin{itemize}
    \item Length, measured in meters $(m)$
    \item Time, measured in seconds ($s$)
    \item  Mass, measusred in kilograms (kg)
    \item Current, measured in ampers ($A$)
    \item Temperature, measured in kelvins ($k$)
    \item Matter, measured in moles (mol)
\end{itemize}

We consider $10^3$ (e.g. kilometer) and $10^{-3}$ (e.g. milimiters) to be within
human scale.  We call mass, seconds and kilograms the \textit{mechanical units}.
We define the \textit{force unit}, or Newton, as

\begin{equation*}
    [F] = N = \text{kg} \frac{m}{s^2}
\end{equation*}

and the Pascal unit as 

\begin{equation*}
    [P] = \text{Pa} = \frac{N}{m^2}
\end{equation*}

We use scientific notation and terms which express quantities as powers of ten.
For instance, $10^{12}$ is the tera, $10^{3}$ the giga, etc.

The magnitudes hereby described are suited for algebraic manipulation. For
instance, $m \times m = m^2$, and $s \times \frac{m}{s} = m$.

\section{Vectors}

Vectors are used to express position, displacement, velocity, force,
acceleration, fields, etc. A vector $\overrightarrow{A}$ (or sometimes
$\overrightarrow{a}$) in the general sense has a direction (line), an
orientation, and a length (or magnitude). A vector also has an application
point, which denotes the point of origin of the vector. When saying $\overrightarrow{a}
= \overrightarrow{b}$, we mean that $\overrightarrow{a}$ and
$\overrightarrow{b}$ coincide in direction, magnitude and orientation,
irrespective of their application point. 

The scalar product is defined as the usual mapping in the space $\mathbb{R}^n
\times \mathbb{R} \mapsto \mathbb{R}^n$. Intuitively, the scalar product $\lambda
\overrightarrow{a}$
"streches" or "shrinks" a vector, depending on wheter $|\lambda| < 1$ or not,
and the positivty or negativity of $\lambda$ determines whether the vector
inverts its direction or not. In general, $\left| \lambda \overrightarrow{a}
\right| = \left| \lambda \right| \left| \overrightarrow{a} \right| $.

The sum of vectors, $\overrightarrow{a} + \overrightarrow{b}$, is a mapping
$\mathbb{R}^n \times \mathbb{R}^n \mapsto \mathbb{R}^n$. As usual, and in a
graphical sense, the sum corresponds to the application of the parallelogram
rule. 

\begin{shaded}
    \textbf{Parallelogram rule}. Make $\overrightarrow{a}$ and
    $\overrightarrow{b}$ coincide in their point of application. From the tip of 
    $\overrightarrow{a}$, draw a copy of $\overrightarrow{b}$, and from the tip
    of $\overrightarrow{b}$ a copy of $\overrightarrow{a}$. The corner of the
    thus generated parallelogram is the tip of $\overrightarrow{a} +
    \overrightarrow{b}$.

    Alternatively, from the tip of $\overrightarrow{a}$ write
    $\overrightarrow{b}$. Then $\overrightarrow{a} + \overrightarrow{b}$ is the
    vector which goes from the point of application of $\overrightarrow{a}$ to
    the tip of $\overrightarrow{b}$.
\end{shaded}

The sum of vectors is commutative, associative, and distributive with respect to
scalar product.

If $\overrightarrow{A}$ is a vector, we use $A_x$ and $A_y$ to denote the
projection of the vector over the axis $x$ or $y$, respectively. Using $A_x$ and
$A_y$ one forms a rectangular triangle with sides $A_x$, $A_y$ and a hypotenuse 
of length $\left| \overrightarrow{A} \right| $. 

Let $\theta$ be the angle formed
by $\overrightarrow{A}$ with the $x$-axis. Then, using trigonometry,

\begin{equation*}
    \cos \theta = \frac{A_x}{\left| \overrightarrow{A} \right| }, \qquad \sin
    \theta = \frac{A_y}{\left| \overrightarrow{A} \right| }
\end{equation*}

from which one can find $A_x, A_y$ assuming one knows $\theta$. From this
follows that $\left| \overrightarrow{A} \right| $ and $\theta$ fully determine
all the information about the vector, insofar as the allow us to determine $A_x,
A_y$. Conversely, knowing $A_x$ and $A_y$ is also sufficient to determine
$\overrightarrow{A}$, insofar as 

\begin{equation*}
    \left| \overrightarrow{A} \right|  = \sqrt{A_x^2 + A_y^2} , \qquad \frac{A_y}{A_x} =
    \frac{\left| \overrightarrow{A} \right| \sin \theta}{\left|
    \overrightarrow{A} \right|  \cos \theta} = \tan \theta \Rightarrow \theta =
    \arctan \left( \frac{A_y}{A_x} \right) 
\end{equation*}

As convention, we use $\hat{i}$ to denote the versor (vector of length 1) with
direction parallel to the $x$-axis, and $\hat{j}$ the versor with direction
parallel to the $y$-axis.

Notice that, for any vector $\overrightarrow{A}$, $A_x$ is $\hat{i}$ times
$A_x$, and $A_y$ is $\hat{j}$ times $A_y$, which means 

\begin{equation*}
    \overrightarrow{A} = A_x \hat{i} + A_y \hat{j}
\end{equation*}

When writing $\overrightarrow{A}$ in this way, we say we write it in term of its
components $x, y$. In terms of linear algebra, it's not hard to see that we are
simply expressing that $\hat{i}, \hat{j}$  form a basis of $\mathbb{R}^2$. Thus,
it is equivalent to write 

\begin{equation*}
    A_x = \left| \overrightarrow{A} \right|  \cos \theta, \qquad A_y = \left|
    \overrightarrow{A} \right| \sin \theta
\end{equation*}

and 

\begin{equation*}
    \overrightarrow{A} = \left| \overrightarrow{A} \right| \left( \cos \theta
    ~ \hat{i} + \sin \theta ~ \hat{j}\right) 
\end{equation*}

From this follows as well that 

\begin{align*}
    \overrightarrow{A} + \overrightarrow{B} 
    &= \left( A_x \hat{i} + A_y \hat{j} \right) + (B_x \hat{i} + B_y \hat{j}) \\ 
    &= \hat{i}\left( A_x + B_x \right)  + \hat{j}\left( A_y + B_y \right) 
\end{align*}

which means the sum of vectors has as components the sum of the components.

The scalar product of two vectors, $\overrightarrow{A} \cdot
\overrightarrow{B}$, is a scalar defined as 

\begin{equation*}
    \overrightarrow{A} \cdot \overrightarrow{B} = \left| \overrightarrow{A} \right| \left| \overrightarrow{B} \right| \cos \theta
\end{equation*}

where $\theta$ is the angle formed by the two vectors. The scalar product is
positive if $\cos \theta$ is positive, which occurs for $0 < \theta \leq 90$. It
is negative if $\cos \theta $ is negative, i.e. if $90 < \theta \leq 180$.
Clearly, $\overrightarrow{A} \cdot \overrightarrow{B} = 0 \iff \theta = 90$.

In general, from the definition follows that

\begin{equation*}
    \overrightarrow{A} \cdot \overrightarrow{B} = A_x B_x + A_y B_y
\end{equation*}

The vectorial product $\overrightarrow{A} \times \overrightarrow{B}$ is a vector
perpendicular to the plane formed by $\overrightarrow{A}$ and
$\overrightarrow{B}$. Its module is $\left| \overrightarrow{A} \right| \left|
\overrightarrow{B} \right| \sin \theta $, and its direction is given by what's
called the right-hand rule.

\pagebreak 

\subsection{Excercises}

\begin{shaded}
    \textbf{(2)} Sean los vectores $\overrightarrow{A} = 2\hat{i} + 3\hat{j}$
    $\overrightarrow{B} = 4\hat{i} -2 \hat{j}$ y $\overrightarrow{C} = -\hat{i}
    + \hat{j}$.
    Determinar la magnitud y el ángulo (representación polar) de los vectores
    resultantes $\overrightarrow{D} = \overrightarrow{A} + \overrightarrow{B} +
    \overrightarrow{C}$ y $\overrightarrow{E} = \overrightarrow{A} +
    \overrightarrow{B} - \overrightarrow{C}$. Resolver analítica y
    gráficamente.
\end{shaded}

(Analytical solution.) We'll use $A_x, A_y$ to denote the components of the
vector $\overrightarrow{A}$, and same for all other vectors. We know the
components of $\overrightarrow{D}$ are 

\begin{equation*}
    D_x = A_x + B_x + C_x = 2 + 4 - 1 = 5, \qquad D_y = 3 - 2 + 1 = 2
\end{equation*}

from which readily follows that $\left| D \right| = \sqrt{5^2 + 2^2} = \sqrt{29}
\approx 5.385$. Similarly, 

\begin{equation*}
    E_x = 2 + 4 + 1 = 7, \qquad E_y = 3 - 2 -1 = 0
\end{equation*}

from which follows that $\left| E \right| = \sqrt{7^2} = 7 $.

Now, we must recall that 

\begin{equation*}
    \theta_{\overrightarrow{Z}} = \arctan \left( \frac{Z_y}{Z_x} \right) 
\end{equation*}

for any $\overrightarrow{Z}$.


We need not memorize this: it is trigonometrically clear that 
$Z_x = \cos \theta_{\overrightarrow{Z}} \left| \overrightarrow{Z} \right| $ and 
$Z_y = \sin \theta_{\overrightarrow{Z}} \left| \overrightarrow{Z} \right| $, and
therefore 

\begin{equation*}
    \frac{Z_y}{Z_x} = \tan \theta
\end{equation*}

And $\arctan$ is the inverse of $\tan$. Anyhow, for $\overrightarrow{E}$ and
$\overrightarrow{D}$ we have 

\begin{equation*}
    \theta_{\overrightarrow{E}} = \arctan\left( \frac{E_y}{E_x} \right) =
    \arctan\left( 0 \right) = 0
\end{equation*}

\begin{equation*}
    \theta_{\overrightarrow{D}} = \arctan\left( \frac{D_y}{D_x} \right) =
    \arctan \left( \frac{2}{5} \right) \approx 0.38
\end{equation*}

\pagebreak 

\begin{shaded}
    \textbf{(3)} Can two vectors of different magnitud be combined and yield zero? What about three?
\end{shaded}

The zero vector is the only vector with magnitude zero. Let $\overrightarrow{A},
\overrightarrow{B}$ arbitrary vectors. Then 

\begin{equation*}
    \left| \overrightarrow{A} + \overrightarrow{B} \right| = \sqrt{(A_x + B_x)^2
    + (A_y + B_y)^2} 
\end{equation*}

which is zero if and only if 

\begin{equation*}
    ( A_x + B_x )^2 + ( A_y + B_y )^2 = 0
\end{equation*}

This only holds if $A_x + B_x = A_y + B_y = 0$. But 

\begin{equation*}
    A_x + B_x = 0 \Rightarrow A_x = -B_x, \qquad A_y + B_y = 0 \Rightarrow A_y =
    -B_y
\end{equation*}

But then 

\begin{equation*}
    \left| A \right| = \sqrt{A_x^2 + A_y^2} = \sqrt{(-B_x)^2 + (-B_y)^2} =
    \sqrt{B_x^2 + B_y^2}  = \left| B \right| 
\end{equation*} 

$\therefore ~ \left| \overrightarrow{A} + \overrightarrow{B} \right| = 0 \iff
\left| \overrightarrow{A} \right| = \left| \overrightarrow{B} \right| $.

It is simple to see that three vectors of different magnitude can add to zero.

\pagebreak 

\begin{shaded}
    Assume $A + B + C = 2\hat{i} + \hat{j}$ and $A = 6\hat{i}-3\hat{j}, B =
    2\hat{i} + 5\hat{j}$. Find the components of $C$. Solve analytically and
    graphically.
\end{shaded}

We know 

\begin{equation*}
    6 + 2 + C_x = 2, \qquad -3 + 5 + C_y = 1
\end{equation*}

from which follows that $C_x = -6, C_y = -1$.

\pagebreak 

\begin{shaded}
    \textbf{(5)} $A$ and $B$ have a magnitud of $3m, 4m$ respectively. The angle between them
    is $\theta = 30$ degrees. Find their scalar product.
\end{shaded}

Their scalar product is 

\begin{equation*}
    ( \left| B \right|  \cos \theta ) \left| A \right| 
\end{equation*}

Recall that 

\begin{equation*}
    \text{Angle in degrees} = \text{Angle in radians} \cdot \frac{180}{\pi}
\end{equation*}

Thus, thirty degrees equates to $30 \frac{\pi}{180} \approx 0.523$ radians. Then
the scalar product is 

\begin{equation*}
    4 \cos (0.523) \times 3 \approx 10.395
\end{equation*}




\pagebreak 

\begin{shaded}
    \textbf{(6)} Find the angle between $A = 4 \hat{i} + 3 \hat{j}$ and 
    $B = 6\hat{i} - 3 \hat{j}$.
\end{shaded}

Recall that 

\begin{equation*}
    A \cdot B = \left| A \right| \left| B \right| \cos \theta
\end{equation*}

where $\theta$ is the angle between the vectors. This readily entails that 

\begin{equation*}
    \frac{A \cdot B}{\left| A \right|\left| B \right|  } = \cos \theta
\end{equation*}

or equivalently that 

\begin{equation*}
    \theta = \arccos \left( \frac{A \cdot B}{\left| A \right|\left| B \right|  } \right)  
\end{equation*}

Now, $A \cdot B = 4 \times 6 + 3 \times -3 = 24 - 9 = 15$ and $\left| A \right|
\left| B \right| = 5 \cdot 6.708 = 33.541 $.

Therefore, 

\begin{equation*}
    \theta = \arccos \left( \frac{15}{33.541} \right) = \arccos \left( 0.447
    \right) = 1.107
\end{equation*}

\pagebreak 

\begin{shaded}
    \textbf{(7)} Let $\overrightarrow{v} = \left( \frac{1}{3}, \frac{2}{3} \right) $ be the vector of
    components. Find the components of the vector of module 5 whose direction 
    and orientation (sentido) are those of the given vector.
\end{shaded}

Assume $\overrightarrow{x} = (x_1, x_2)$ is of magnitude $5$. Any vector whose
direction and orientation are the same than those of $\vec{v}$ is "a stretching"
of $\vec{v}$. In other words, for $\vec{x}$ to satisfy the requirements, we must
have 

\begin{equation}
    \vec{x} = \lambda \vec{y}
\end{equation}

for some $\lambda \in \mathbb{R}$. (Furthermore, $\lambda > 0$ since otherwise
orientation is not preserved.) 

Now, from equation $(1)$ follows that

\begin{equation}
    \|\vec{x}\| =  \lambda \|\vec{y}\|
\end{equation}

since the magnitude of a scaled vector is the scaled magnitude of the vector.
Equation $(2)$ simplifies to 

\begin{equation}
    \|\vec{x}\|  = \lambda \sqrt{1 / 9 + 2 / 3} = \frac{\lambda
    \sqrt{7} }{3}
\end{equation}

From this readily follows that $\frac{3}{\sqrt{7} }\|\vec{x}\| = \lambda$. But
it is a hypothesis that $\|\vec{x}\| = 5$. Therefore, 

\begin{equation}
    \lambda = \frac{3}{\sqrt{7} } \cdot 5 = \frac{15}{\sqrt{7} }
\end{equation}

In other words, 

\begin{equation}
    \vec{x} = \frac{15}{\sqrt{7} } \vec{v}
\end{equation}




\end{document}



