\documentclass[a4paper, 12pt]{article}

\usepackage{bussproofs}
\usepackage[utf8]{inputenc}
\usepackage[T1]{fontenc}
\usepackage{textcomp}
\usepackage{amssymb}
\usepackage{amsmath, amssymb}
\newtheorem{problem}{Problem}
\newtheorem{example}{Example}
\newtheorem{lemma}{Lemma}
\newtheorem{theorem}{Theorem}
\newtheorem{problem}{Problem}
\newtheorem{example}{Example} \newtheorem{definition}{Definition}
\newtheorem{lemma}{Lemma}
\newtheorem{theorem}{Theorem}


\begin{document}


$a \sim b$ si y solo si $3$ divide a $a - b$; demostrar que $\sim$ es de equivalencia.

Para demostrar que es una relación de equivalencia, hay que ver que sea 
transitiva, reflexiva y simétrica. 

\textit{(1: Transtividad)} Asuma $a \sim b$ y $b \sim c$: queremos probar que
entonces $a \sim c$. Puesto que $a \sim b$, por definición, tenemos que 

\begin{equation}
a - b = 3q
\end{equation}
para algún $q \in \mathbb{Z}$. Y puesto que $b \sim c$, tenemos que 

\begin{equation}
    b - c = 3q'
\end{equation}

para algún $q' \in \mathbb{Z}$. Por definición,

\begin{equation}
    a \sim c \iff 3 \mid (a - c)
\end{equation}

Usando la ecuación (1) vemos que $a = 3q + b$, y que $c = b - 3q'$. Por lo tanto,
podemos sutituir $a$ y $c$ en (3) y obtenemos 

\begin{align*}
    a \sim c &\iff 3 | (3q + b - (b - 3q)) \\ 
             &\iff 3 \mid (3q - 3q') \\ 
             &\iff 3 \mid 3(q - q')
\end{align*}

lo cual es obviamente cierto. 

Hemos visto que asumir $a \sim b$ y $b \sim c$ conduce a $a \sim c$. Luego la relación es transitiva.

\textit{(2 : Reflexividad)} Queremos ver que $a \sim a$. Pero $a \sim a$ si y solo si 

\begin{align*}
    3 \mid (a - a) \iff 3 \mid 0
\end{align*}

Cualquier número divide a $0$. La relación es reflexiva. 

\textit{(3: Simétrica)} Asuma que $a \sim b$. Queremos ver que entonces $b \sim a$. Nuestro supuesto nos dice que 

\begin{align*}
    a - b = 3q
\end{align*}

para algún $q \in \mathbb{Z}$. Multiplicando ambos lados por $-1$, tenemos 

\begin{align*}
    -(a - b) = -3q \iff b - a = 3(-q)
\end{align*}

(pongo el paréntesis in el $(q)$ para que se vea claro que tres por $(-q)$ da
lo de la izquierda). Entonces, por definición, $3$ divide a $b - a$. Entonces $b \sim a$.


\pagebreak 

Sea $A$ el conjunto de todos los subconjuntos finitos de $\mathbb{N}$. Decimos
que dos conjuntos $X, Y$ que pertenezcan a $A$ están relacionados si tienen la
misma cardinalidad. Es decir, $X \sim Y \iff |X| = |Y|$.

Sea $\mathbb{N}_k = \left\{ 1, 2, \ldots, k \right\} $; es decir, el conjunto
de los primeros $k$ números naturales. Para ser rigurosos, hagamos
$\mathbb{N}_0 = \emptyset$; o sea que $\emptyset$ es el conjunto de los $0$
primeros números naturales.

Claramente, $[\mathbb{N}_1]$ es la clase de equivalencia de todos los elementos
de $A$ con cardinalidad $1$; $[\mathbb{N}_2]$, la de los elementos de $A$ con
cardinalidad $2$; etc. (Si no entendés por qué, revisá la definición de clase
de equivalencia. Si después de revisarla sigue sin ser claro, escribime y te lo
aclaro).

Por lo tanto, para cada $i \in \mathbb{N}$, los conjuntos de cardinalidad $i$
están en la clase de equivalencia de $\mathbb{N}_i$, y este último 
conjunto puede tomarse como su representante.

La cantidad de clases de equivalencia es infinita, porque hay una 
por cada cardinalidad (es decir, una por cada número natural,
y hay infinitos naturales).

\pagebreak 

$u_1 = 3, u_2 = 5, u_n = 3u_{n-1} - 2 u_{n-2}$. Probar que $u_n = 2^n + 1$.

El caso base es $n = 3$, donde $u_3 = 3u_2 - 2u_1 = 3 \times 5 - 2 \times 3 =
15 - 6 = 9$. Efectivamente se cumple que $2^3 + 1 = 8 + 1 = 9$.

Usaremos inducción fuerte. Nuestra hipótesis inductiva será 

\begin{align*}
    u_j = 2^j + 1, ~ ~ ~ ~ 1 \leq j \leq k
\end{align*}

con $k \in \mathbb{Z}$ arbitrario. (Acordate que inducción "normal" es asumir 
que vale para algún $k$, pero inducción fuerte es asumir que vale 
para todos los números naturales menores o iguales a algún $k$, que 
es lo que dice nuestra HI).

Nuestra tesis es 

\begin{align*}
    u_{k+1} = 2^{k+1} + 1
\end{align*}

Usemos la definición recursiva de $u_n$; nos dice que, si $j \geq 3$,

\begin{align*}
    u_{k+1} = 3u_k - 2u_{k-1}
\end{align*}

Como $k$ y $k-1$ son menores o iguales a $k$, nuestra hipótesis inductiva vale. 
(Fijate que por esto teníamos que usar inducción fuerte, porque nos queda 
$u_{k-1}$, entonces hay que asumir algo no sólo respecto de $k$ sino de 
los números menores a $k$). Aplicando la HI, 

\begin{align*}
    u_{k+1} &= 3\left( 2^{k} + 1 \right) - 2\left( 2^{k-1} + 1  \right) \\
            &=3\times 2^{k} + 3 - 2^k - 2 \\ 
            &=3\times 2^k - 2^k + 1 \\ 
            &=2^k(3 -1 ) + 1 \\ 
            &=2^k\times 2 + 1 \\ 
            &= 2^{k+1} + 1
\end{align*}

que es lo que queríamos demostrar.

\pagebreak 

$n^2 \leq 2^n$ para todo $n > 3$.

Caso base con $n = 4$ es fácil. 

HI: $k^2 \leq 2^k$. 

Tesis: $(k+1)^2 \leq 2^{k+1}$

Fijate que 

\begin{align*}
    (k+1)^2 \leq 2^{k+1} &\iff k^2 + 2k + 1 \leq 2^{k} \times 2
\end{align*}

Si multiplicamos ambos lados de nuestra HI por $2k$, tenemos 

\begin{align*}
    2k^2  \leq 2^{k+1}
\end{align*}

Si pudiéramos demostrar que $k^2 +2k + 1 \leq 2k^2$, por transitividad, tendríamos nuestra tesis.
Pero

\begin{align*}
    k^2 + 2k + 1 \leq 2k^2 &\iff 2k + 1 \leq 2k^2 - k^2 \\& \iff 2k+1 \leq k^2(2 - 1) \\&\iff 2k-1 \leq k^2
\end{align*}

Esto es lo que demostraste en el inciso $(a)$, o sea que sabemos que es verdad. Entonces, 
efectivamente $k^2  + 2k + 1 \leq 2k^2 \leq 2^{k+1}$. O sea que, por transitividad,

\begin{align*}
    k^2 + 2k + 1 \leq 2^{k+1}
\end{align*}

\pagebreak 

HI: $8 \mid 3^k - 1$

Tesis: $8 \mid 3^{2k + 2} - 1$


De la HI se sigue que $8 \mid (3^k - 1) \times 3^{2+k}$, es decir 

\begin{align*}
    8 \mid 3^{2k+2} - 9 \times 3^k
\end{align*}

De esto se sigue que $8q = 3^{2k+2} - 9\times 3^k$ para algún $q \in
\mathbb{Z}$, lo cual implica que $8q + 9\times 3^k = 3^{2k+2}$. Ahora bien, $9$
divide a $3^2$ y por lo tanto divide a $3^2 \times c$ para cualquier $c$; es
decir que $9$ divide a $3^m$ para cualquier $m \geq 2$. Entonces podemos
escribir $3^k$ como $9q'$, y nos queda

\begin{align*}
    8q + 9 \times 9q' = 3^{2k+2} \Rightarrow 8q + 9q'' = 3^{2k+2}
\end{align*}

donde $q'' = 9q'$.

Ahora, el truco inteligente es ver que $8x + 9y$ siempre tiene resto $1$.















\begin{align*}
    \Sigma = \left\{ 1, 2, 3, 4, 5, 6 \right\} 
\end{align*}

Si $Om$

\begin{align*}
    \phi_i
\end{align*}

















\end{document}



