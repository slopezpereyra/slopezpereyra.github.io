\documentclass[a4paper, 12pt]{article}

\usepackage{bussproofs}
\usepackage[utf8]{inputenc}
\usepackage[T1]{fontenc}
\usepackage{textcomp}
\usepackage{amssymb}
\usepackage{newtxtext} \usepackage{newtxmath}
\usepackage{amsmath, amssymb}
\newtheorem{problem}{Problem}
\newtheorem{example}{Example}
\newtheorem{lemma}{Lemma}
\newtheorem{theorem}{Theorem}
\newtheorem{problem}{Problem}
\newtheorem{example}{Example} \newtheorem{definition}{Definition}
\newtheorem{lemma}{Lemma}
\newtheorem{theorem}{Theorem}


\begin{document}



A phrase structure grammar is a set of rewriting or transformation rules of the
form $\varphi \to  \psi$, with $\varphi$ and $\psi$ strings of symbols.

\textit{Example.} The grammar with rules 

\begin{align*}
    S &\to AB \\ 
    A &\to C \\ 
    CB&\to Cb \\ 
    C &\to A
\end{align*}

provides the derivation 

\begin{align*}
    D = (S, AB, CB, Cb, ab)
\end{align*}

which has a diagram structure.





\end{document}



