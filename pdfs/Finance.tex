\documentclass[a4paper, 12pt]{article}

\usepackage{pgfplots}
\pgfplotsset{compat=1.18} % Or adjust depending on your TeX distribution
\usepackage[utf8]{inputenc}	% Para caracteres en español
\usepackage{amsmath,amsthm,amsfonts,amssymb,amscd}
\usepackage{multirow,booktabs}
\usepackage[table]{xcolor}
\usepackage{fullpage}
\usepackage{lastpage}
\usepackage{newtxtext}
\usepackage{newtxmath}
\usepackage{enumitem}
\usepackage{fancyhdr}
\usepackage{mathrsfs}
\usepackage{wrapfig}
\usepackage{setspace}
\usepackage{calc}
\usepackage{multicol}
\usepackage{cancel}
\usepackage[retainorgcmds]{IEEEtrantools}
\usepackage[margin=3cm]{geometry}
\usepackage{amsmath}
\newlength{\tabcont}
\setlength{\parindent}{0.0in}
\setlength{\parskip}{0.05in}
\usepackage{empheq}
\usepackage{framed}
\usepackage[most]{tcolorbox}
\usepackage{xcolor}
\colorlet{shadecolor}{orange!15}
\parindent 0in
\parskip 12pt
\geometry{margin=1in, headsep=0.25in}
\theoremstyle{definition}
\newtheorem{defn}{Definition}
\newtheorem{reg}{Rule}
\newtheorem{exer}{Exercise}
\newtheorem{note}{Note}
\newtheorem{theorem}{Theorem}
\setcounter{section}{0}
\usepackage{chngcntr}
\usepackage{hyperref}
\counterwithin*{equation}{section}
\counterwithin*{equation}{subsection}



\begin{document}

\section{Finance with certainty}

A financial operation may be understood as a loan. A lender gives a borrower a
certain amount of money, under the condition that the borrower shall return it
with interest. We assume the interest is previously agreed upon, which means we
are dealing with financial \textit{certainty}.

The initial capital es the amount of money being borrowed. The final capital 
is the amount that is returned. The \textit{plazo} (term) is the time the
operation lasts. The interest is the difference between the initial and the
final capital. In general, the interest is proportional to the initial capital.

Since the interest depends on the initial capital, generally what is done is
setting an \textit{interest rate}. The interest rate is the interest
corresponding to each unit of capital per each unit of time. 

\begin{shaded}
    \textbf{Example.} For example, $5\%$ to 30 days, or $10\%$ monthly, or daily
    $0.001$ ($0.1\%$), or annual $25\%$.

    \textbf{Example.} An initial capital of $10.000$ with an interest rate of 
    $5\%$ every $30$ days, we have an interest in $30$ days of $10.000 \times 0.05 = 500$. 

    \textbf{Example.} If an interest of $4000$ is paid for a loan of 
    $200.000$ for $45$ days, the interest rate is to 45 days is $2\%$:

    \begin{equation*}
        4000 / 200.000 = 0.02
    \end{equation*}
\end{shaded}

If the interest rate is set to an amount of time $t$ less than the term of the loan, 
we have different systems. 

\begin{enumerate}
    \item \textit{Simple capitalization system}: 
In the simple capitalization system, the interest is directly proportional
to the term: 

        $$F = I (1 + i t)$$

    where $F, I$ are the final and initial capitals, $i$ is the interest rate, 
    and $t$ is the time expressed in the time unit of the interest.
    \item \textit{Composite capitalization system}: Once we reach time $t$, we
        operate as if we had removed the
        money from the loan and re-inserted the amount obtained.

        \begin{equation*}
            F = C(1+i)^t, \qquad i = C\left( (1+i)^t - 1 \right) 
        \end{equation*}

        where $C$ is the capital we have at the moment (with accumulated
        interests) and $t$ is the number of terms that passed. 
\end{enumerate}


\begin{shaded}
    \textbf{Example of simple capitalization.} An interest rate of $5\%$ to 30
    days and an initial capital of $10.000$ gives an interest of: $500$ in 30
    days, $1000$ in 60 days, $1500$ in 90 days, etc. Notice that each time 30
    days pass, we increment our money by $500 = 10.000 \times 0.05$.

    Notice that after $30$ days we would have $10.500$, and after $60$ we would
    have $11.000$. But there's a problem: $11.000$ is less than what would be 
    obtained with an interest rate of $5\%$ over the $10.500$ we had after the
    first $30$ days.

    In other words, if the $5\%$ of $10.000$ is $500$, the $5\%$ of $10.500$ is
    more than $500$. So we see that as time goes by, with the simple
    capitalization system, the growth becomes less and less relative to 
    the capital at hand.
\end{shaded}

\begin{shaded}
    \textbf{Example of composite capitalization.} An interest rate of $5\%$
    every 30 days with initial capital of 10.000. Then the cumulative interest
    will be: 

    \begin{enumerate}
        \item $500$ in 30 days. 
        \item $1025$ in 60 days ($500 + 0.05 \times 10.500$)
        \item $1576$ in 90 days ($1025 + 0.05 \times 11.025$)
    \end{enumerate}
\end{shaded}


\begin{shaded}
    \textbf{Problem.} Fixed term to 35 days with $I = 100.000$. An automatic renovation is done in
    the following two vencimientos. The interest on each term are $2.4\%, 2.4\%,
    2.8\%$. Compute the final capital after $105$ days if: $(a)$ total
    renovation is done, $(b)$ partial renovation is done. 

    \textbf{Solution.} The first term goes from day $0$ to 35, the second from
    day 35 to 70, and the third from day 70 to 105. The interest rate of each of
    these terms was specified above.

    With total renovation, this would work as follows. On day 35, after the
    first term, we would have $100.000 \times 1.025$. On day 70 we would apply
    the interest of the term to the total capital so far accrued, so we would
    have $100.000(1.025)(1.024)$. Lastly, on day $105$, we would apply the
    interest again to the total capital accrued, resulting in
    $100.000(1.025)(1.024)(1.028) = 107898.88 = F$. with an interest of 
    $F - I = 7898.88$.

    With partial renovation, on day 35 we would again have $100.000 \times
    1.025$. On day 70, we would add to our accrued capital the interest of the
    term applied to the initial capital: $100.000(1.025 + 0.024)$. Then, on day
    $105$, we would have $100.000(1.025 + 0.024 + 0.028)$. (If this is unclear,
    apply the distributive law and it will be.)
\end{shaded}

\subsection{Nominal and effective rates}

The TNA (Tasa Nominal Anual) is an annual rate that serves as reference to compute rates in different
terms through direct proportionality. For instance, if to 30 days we have a TNA
of $25\%$, this means the that the rate to 30 days is 

\begin{equation*}
    I_{\text{30 days}} = TNA \times \frac{\text{rate in days}}{365} = 0.25
    \times \frac{30}{365} = 0.0205 = 2.05\%
\end{equation*}

Notice that $30 / 365$ is the ``how much are 30 days to a year``.

The TEA (Tasa Efectiva Anual) is a rate equivalent to the effective rate being
applied, but corresponding to an annual term. For instance, if for a fixed term
to 30 days we have a rate of $2.05\%$, then the composite capitalization is 

\begin{equation*}
    1 \times (1.0205)^{\frac{365}{30}} = 1.2800
\end{equation*}

which means the TEA is $28\%$.

\subsection{Continuous capitalization}

If the TNA is $r$, the final capital in a year over an initial capital $C$, with
continuous capitalization, is 

\begin{equation*}
    C \times \lim_{m \to \infty}\left( 1 + \frac{r}{m} \right)^m = C \times e^r
\end{equation*}

Then 

\begin{equation*}
    TEA = e^t - 1
\end{equation*}

meaning that the final capital es 

\begin{equation*}
    C(t) = C \times e^{rt}
\end{equation*}

When modeling, we need to take $r$ as $r(t)$, i.e. as a rate that may
continuously vary in time, which adds extra complexity.

\subsection{Discounted value}

When comparing two quantities in different time instances, we determine the 
\textit{discounted value} or \textit{actual value} of a capital. Let $C_1$ the
capital at time $t_1$, $C_2$ the capital at time $t_2$, with $t_1 < t_2$.
Then via composite capitalization with rate $i$ 

\begin{equation*}
    C_{1} = \frac{C_2}{(1+i)^{t_2 - t_1}}
\end{equation*}

is the discounted value from $C_2$ at time $t_1$.

\pagebreak 

\subsection{Excercises}

\begin{shaded}
    \textbf{Ejercicio 2}. An individual borrows $50.000$ with a term of 30 days. 
    Simple capitalization was agreed upon with an annual nominal interest rate
    (TNA) of
    $33\%$.

    $(a)$ Determine the amount the individual will have to pay in interest.

    $(b)$ Determine the interest rate of the operation with a term of 30 days. 

    $(c)$ Compute the interest rate of the operation if instead of 30 days 
    the term was 180, 90, or 60.
\end{shaded}

Recordemos que en el sistema de capitalización simple, el interés es
directamente proporcional al plazo y no varía con el capital acumulado, sino que
depende sólo del capital inicial. En general, 

\begin{equation}
    F = I \left( 1 + it \right) 
\end{equation}

Como la tasa de interés nominal anual es $33\%$, la tasa cada 30 días es 

\begin{equation*}
    \frac{30}{365} \times 0.33 \approx 0.0271 = 2.71\%
\end{equation*}

Por lo tanto, al cabo de los 30 días, el interés será $0.0271 \times 50.000 =
1356.1648$. La tasa de interés en 180, 90 y 60 días es:


\begin{equation*}
    \frac{60}{365} \times 0.33 = 0.0542, \qquad \frac{90}{365} \times 0.33 =
    0.0813, \qquad \frac{180}{365} \times 0.33 = 0.1627
\end{equation*}

or equivalently $5.42\%, 8.13\%$ and $16.27\%$.

\pagebreak 

\begin{shaded}
    \textbf{Ejercicio 2.} Una persona depositó el 1 de marzo de 2025 300.000 en una caja de ahorro en
    la que se aplica una tasa de interés constante con capitalización simple. Se
    sabe que el 1 de agosto de 2025 el capital disponible es de 314.229.

    $(a)$ ¿Cuál es la tasa de interés diaria y cuál es la tasa anual (365 días)
    que se aplica en esta cuenta?
    $(b)$ ¿Cuánto podría retirarse si el capital se deja depositado hasta el 1 de noviembre de 2025 y se mantiene
la misma tasa?
    $(c)$ ¿En qué fecha el monto ascenderá a 317.949?
\end{shaded}

Del 1 de marzo al 1 de agosto hay 153 días. Notemos que 

\begin{equation*}
    300.000 \times ( 1 + I_{153} ) = 314.299 \Rightarrow I_{153} = 0.0476
\end{equation*}

Es decir que el interés fue del $4.76\%$. Entonces,

\begin{equation*}
    I_{153} = TNA \times \frac{153}{365} \Rightarrow TNA = \frac{365}{153}
    \times 0.0476 = 0.1135
\end{equation*}

Es decir que la tasa nominal anual es del $11.35\%$. La tasa de interés diaria
es 

\begin{equation*}
    0.1135 \times \frac{1}{365} = 0.00031 = 0.0031\%
\end{equation*}

Si el capital se deja hasta el 1 de noviembre, el tiempo total será 245 días,
resultando en un interés de

\begin{equation*}
    0.1135 \times \frac{245}{365} = 0.0761
\end{equation*}

y por lo tanto un capital acumulado de 322.830. 

El monto ascenderá a 317.949 si el interés acumulado es 17.949, es decir si
aumentamos nuestro capital un $5.983\%$. Esto sucederá después de $k$ días,
donde

\begin{equation*}
    0.1135 \times \frac{k}{365} = 0.05983 \iff k = 0.05983 \times
    \frac{365}{0.1135} \iff k = 192.40
\end{equation*}

Es decir, sucederá después de $193$ días.

\pagebreak
\section{Finance}

A financial market is a market where financial instruments are exchanged, such
as basic and derivative assets. Most financial markets are regulated so that no
fraud occurs. However, there are off-the-counter markets (mercado extrabursátil)
where instruments are negotiated directly between two parts and are regulated
differently than standard financial markets.

In financial markets, one exchanges financial instruments such as:

\begin{itemize}
    \item Actions (shares) y CEDEAR. 
    \item Divisas 
    \item Commodities (Bienes de consumo, e.g. trigo, petróleo, etc)
    \item Bonos, obligaciones negociables
    \item ETF (Exchange Traded Funds).
\end{itemize}

One might also buy \textit{productos derivados}:

\begin{itemize}
    \item Forwards
    \item Futuros 
    \item Opciones 
    \item Derivados sobre tasas
\end{itemize}

A share (acción) is a financial instrument that represents ownership over some
of the equal fractions in which the social capital of a corporation is divided.
In other words, it represents ownership over a part of a company. Some shares
are quoted (cotizadas) publicly in the stock exchange (la bolsa). This means t
hey can be negotiated in the financial market. A share can provide income by
periodically giving the shareholders part of the company's money. The valuo of a
share varies according to the law of demand and supply.

With regards to currencies, what is quoted is the \textit{tipo de cambio}
(exchange rate), i.e. the cost of a currency with regards to another (generally
the local one).

Commodities comprise consumer goods, raw materials, and services which are
quoted (cotizados) in the market. In general, whatever depends on a future and
is not a value is understood to be a commodity. Grain, cotton, energy materials
such as gas and oil, metals, and meat, are some examples of commodities.

A bond (bono) is a debt issued by private entities or the state. By debt we mean
that it is a promise to pay a certain amount in the future. There is a certain
terminology here:

\begin{itemize}
    \item Amortizar: To return the capital borrowed, in part or in total. 
    \item Bonos con cupón: Periodic coupons are paid according to some coupon
        rate. 
    \item Bonos cupón cero o bonos a descuento: No coupon is paid: the capital
    is amortizado upon expiration to a nominal price. The price of emission is
    inferior to said nominal price.
\end{itemize}

Bonds have associated some \textit{tasa de retorno}.

\subsection{Financial derivative}

A financial contract is a financial derivative if its value depends on another
instrument $S$, called the \textit{underlying}. The payoff of a derivative at
the moment of its expieration is a function of the value of its underlying, 

\begin{equation}
    \text{Payoff} = f\left( S(t_{\text{end}}) \right) 
\end{equation}

where $S(t)$ is the price of the underlying at the expiration time
$t_{\text{end}}$.

A \textit{forward contract} is an agreement to sell or buy an underlying
in some future time $T$ to some price $K$. The contract specifies what the
underlying is, the time $T$ and the price $K$. No matter the market price of the
underlying at time $T$, it will be sold/bought at the agreed-upon-price $K$.
They are negotiated off-the-counter. 

In a \textit{forward}, depending on whether the investor is buying or selling
the underlying, we say he's \textit{long} or \textit{short}. Long is the one
buying the underlying, short is the one who sells it.

The payoff of a forward is the value of the contract at the time of its
expiration, and there's an opposition between the long and short positions. 

\begin{equation}
    \text{Payoff} = \begin{cases}
        S(t_{\text{end}}) - K & \text{long position}\\ 
        K - S(t_{\text{end}}) & \text{short position}
    \end{cases}
\end{equation}

\pagebreak 

\begin{shaded}
    \textbf{Ejercicio 2}: Un inversor entra short en un contrato forward por 100000 libras
    esterlinas (GBP) por dólares estadounidenses a un tipo de cambio de 1,3000
    dólares por libra. ¿ Cuánto gana o pierde el inversor si la tasa de cambio al
    finalizar el contrato es: a) 1 GBP = 1,2900 UD? b) 1 GBP = 1,3200 UD?
\end{shaded}

The investor commits himself to \textit{sell} 100.000 GBP with an underlying
value of 1,3 dolars per pound upon contract maturity. $(a)$ If, when the
contract matures, the price $S(t_{\text{end}})$ is 1,29, then the investor will
win since he will sell to a price higher than the market price. In particular,
he will win $0.01 \times 100000 = 1000$.

\pagebreak 

\begin{shaded}
    \textbf{Ejercicio 3}. Un inversor entra en una posición long en un contrato futuro
    sobre algodón cuando el precio de futuros es de 3,50 dólares por
    kilogramo. El contrato es para la entrega de 50000 kilogramos. ¿Cuánto gana
    o pierde el comerciante si el precio del algodón al final del contrato es a)
    4,82 dólares por kilogramo y b) 3,15 dólares por kilogramo? Interpretar el
    resultado en el diagrama de payoff correspondiente.
\end{shaded}

$(a)$ En la posición long, el inversor se compromete a comprar 50.000kg de
algodón a 3,50 la unidad. Claramente, si en la maduración del contrato el precio
es $4.82$, el inversor estará comprando algodón a un precio menor al valor del
mercado, lo cual significa que está ahorrando dinero. En particular, estará
ahorrando $4,82 - 3,50 = 1,32$ dólares por unidad, es decir un total de 
$1,32 \times 50.000 = 66.000$ dólares.

$(b)$ Mismo razonamiento, pero en este caso pierde $(3,50 - 3,15)50.000 =
17.500$ dólares.

\pagebreak 

\begin{shaded}
    \textbf{Ejercicio 4}: El 1 de julio de 2024 una compañía entró en un contrato forward
    para comprar 10 millones de yenes japoneses el 1 de enero de 2025. El 1 de
    setiembre de 2024 entró en un contrato forward para vender 10 millones de
    yenes japoneses el 1 de enero de 2025. Describir cuál es el payoff de esta
    estrategia.
\end{shaded}

El primero de enero de 2025, la compañía deberá comprar 10 millones de yenes al
valor del 1 de Julio de 2024, que denotamos como $p_1$, y vender 10 millones de
yenes al valor del 1 de septiembre de 2024, que denotamos como $p_2$. Sea $p_3$ el
valor del yen al 1 de enero de 2025.

El payoff será claramente lo ganado/perdido en una operación más lo
ganado/perdido en la otra, es decir la suma de los payoffs. Teniendo en cuenta
que, por separado,

\begin{equation*}
    \mathcal{P}_{\text{long}} = (p_3 - p_1)k, \qquad \mathcal{P}_{\text{short}}
    = (p_2 - p_3)k
\end{equation*}

son los payoffs de las dos operaciones,

\begin{equation}
    \mathcal{P} = \mathcal{P}_{\text{long}} + \mathcal{P}_{\text{short}}
\end{equation}

es el payoff final. La ecuación $(3)$ se simplifica a 

\begin{equation}
    \mathcal{P} = k \big( (p_3 - p_1) + (p_2 - p_3) \big) = k(p_2 - p_1)
\end{equation}

Es decir que el payoff es directamente proporcional a la diferencia del valor
del yen en los dos momentos en que se firmaron los contratos, el 1 de julio y el
1 de septiembre. Lo interesante es que el payoff se vuelve independiente del
valor del yen al momento del vencimiento del contrato.

\pagebreak 

\begin{shaded}
   \textbf{Ejercicio 5}. La empresa argentina ExportCo está actualmente
   exportando mercadería a EE.UU. y sabe que dentro de 8 meses recibirá un pago
   de 2 millones de dólares. ¿Cómo puede dicha empresa realizar una estrategia
   de cobertura ante posibles fluctuaciones en el tipo de cambio con contratos
   forward?
\end{shaded}

Puede entrar \textit{short} para vender 2 millones de dólares al precio de la
fecha para dentro de 8 meses. De ese modo, incluso si el precio del dólar baja
en los 8 meses que pasan, seguirá vendiéndolos al precio de ahora.

\pagebreak 

\begin{shaded}
\textbf{Ejercicio 6}. Un comerciante ingresa en posición short en contratos
futuros a julio sobre concentrado de jugo de naranja congelado. Cada contrato es
para la entrega de 15.000 libras del producto subyacente. El precio de futuros
actual es de 160 centavos por libra, el margen inicial es de 6000 por contrato
y el margen de mantenimiento es de 4500 por contrato. ¿Qué cambio de precio
daría lugar a una margin call? ¿En qué circunstancias se podrían retirar 2000
de la cuenta?
\end{shaded}

$(a)$ Sea $m_0$ la cantidad inicial de dinero en la cuenta marginal \textit{en
dólares}; i.e. $m_0 = 6000$. Sean $m_1, m_2, \ldots$ los subsiguientes valores
en la cuenta marginal, ajustados de acuerdo con la lógica de este tipo de
contratos. Sabemos que 

\begin{equation}
    m_{k+1} = m_k + (p_{k} - p_{k+1}) C
\end{equation}

donde $p_i$ es el precio del subyacente en el $i$-écimo momento y $C$ la
cantidad de subyacente involucrado. Notemos que los precios están en
\textit{centavos de dólar por libra}, mientras $m_{k+1}, m_k$ están en dólares.

Dicho esto, es fácil ver que por recurrencia la ecuación $(5)$ se traduce en: 

\begin{equation}
    m_{k} = m_0 + (p_0 - p_k)C
\end{equation}

Se nos pregunta qué cambio de precio daría lugar a una margin call. Es decir, se
nos pide la solución a la ecuación 

\begin{equation*}
    m_k \leq 4500
\end{equation*}

que por la recurrencia del $(6)$ resulta 

\begin{align*}
    m_0 + (p_0 - p_k)C \leq 4500\text{USD}
    &\iff 600.000 + (160 - p_k)15.000 \leq 450.000
\end{align*}

donde ponemos 450.000 y 600.000 para hablar de todas las unidades en centavos de
dólar. (Porque los precios $p_k$ están en centavos de dólar por libra.) Vemos
que la inecuación se cumple si y solo si $p_k \geq 170$. Es decir que un aumento
de $10$ centavos de dólar por libra es suficiente para conducir a una margin
call. 

$(b)$ Se retiran $2000$USD si y solo si al tiempo de finalizar el contrato el
valor final en la cuenta marginal es $m_{\text{final}} \geq m_0 + 2000 \times 100
\text{cents}$.  Pero $m_{\text{final}} = m_0 + (p_{0} -
p_{\text{final}})15.000$, con lo cual necesitamos resolver la inecuación 

\begin{align*}
&m_0 + 2000 \times 100 \geq m_0 + (p_0 - p_{\text{final}})15.000 \\ 
    \iff &\frac{ 200.000 }{15.000} \geq p_0 - p_{\text{final}} \\ 
    \iff& p_{\text{final}} \geq p_0 - 13.33 \\ 
    \iff&p_{\text{final}} \geq 146.67
\end{align*}

Es decir que una ganancia de al menos 2000USD se consigue si y solo si el precio
baja 13.33 centavos de dólar, o bien si al tiempo de maduración estamos vendiendo
el subyacente a un valor 13.33 centavos por encima del valor en dicho tiempo.



















\end{document}



