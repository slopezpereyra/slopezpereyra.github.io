\documentclass[a4paper, 12pt]{article}

\usepackage{pgfplots}
\pgfplotsset{compat=1.18} % Or adjust depending on your TeX distribution
\usepackage[utf8]{inputenc}	% Para caracteres en español
\usepackage{amsmath,amsthm,amsfonts,amssymb,amscd}
\usepackage{multirow,booktabs}
\usepackage[table]{xcolor}
\usepackage{fullpage}
\usepackage{lastpage}
\usepackage{newtxtext}
\usepackage{newtxmath}
\usepackage{enumitem}
\usepackage{fancyhdr}
\usepackage{mathrsfs}
\usepackage{wrapfig}
\usepackage{setspace}
\usepackage{calc}
\usepackage{multicol}
\usepackage{cancel}
\usepackage[retainorgcmds]{IEEEtrantools}
\usepackage[margin=3cm]{geometry}
\usepackage{amsmath}
\newlength{\tabcont}
\setlength{\parindent}{0.0in}
\setlength{\parskip}{0.05in}
\usepackage{empheq}
\usepackage{framed}
\usepackage[most]{tcolorbox}
\usepackage{xcolor}
\colorlet{shadecolor}{orange!15}
\parindent 0in
\parskip 12pt
\geometry{margin=1in, headsep=0.25in}
\theoremstyle{definition}
\newtheorem{defn}{Definition}
\newtheorem{reg}{Rule}
\newtheorem{exer}{Exercise}
\newtheorem{note}{Note}
\newtheorem{theorem}{Theorem}
\setcounter{section}{0}
\usepackage{chngcntr}
\usepackage{hyperref}
\counterwithin*{equation}{section}
\counterwithin*{equation}{subsection}



\begin{document}

\section{Finance with certainty}

A financial operation may be understood as a loan. A lender gives a borrower a
certain amount of money, under the condition that the borrower shall return it
with interest. We assume the interest is previously agreed upon, which means we
are dealing with financial \textit{certainty}.

The initial capital es the amount of money being borrowed. The final capital 
is the amount that is returned. The \textit{plazo} (term) is the time the
operation lasts. The interest is the difference between the initial and the
final capital. In general, the interest is proportional to the initial capital.

Since the interest depends on the initial capital, generally what is done is
setting an \textit{interest rate}. The interest rate is the interest
corresponding to each unit of capital per each unit of time. 

\begin{shaded}
    \textbf{Example.} For example, $5\%$ to 30 days, or $10\%$ monthly, or daily
    $0.001$ ($0.1\%$), or annual $25\%$.

    \textbf{Example.} An initial capital of $10.000$ with an interest rate of 
    $5\%$ every $30$ days, we have an interest in $30$ days of $10.000 \times 0.05 = 500$. 

    \textbf{Example.} If an interest of $4000$ is paid for a loan of 
    $200.000$ for $45$ days, the interest rate is to 45 days is $2\%$:

    \begin{equation*}
        4000 / 200.000 = 0.02
    \end{equation*}
\end{shaded}

If the interest rate is set to an amount of time $t$ less than the term of the loan, 
we have different systems. 

\begin{enumerate}
    \item \textit{Simple capitalization system}: 
In the simple capitalization system, the interest is directly proportional
to the term: 

        $$F = I (1 + i t)$$

    where $F, I$ are the final and initial capitals, $i$ is the interest rate, 
    and $t$ is the time expressed in the time unit of the interest.
    \item \textit{Composite capitalization system}: Once we reach time $t$, we
        operate as if we had removed the
        money from the loan and re-inserted the amount obtained.

        \begin{equation*}
            F = C(1+i)^t, \qquad i = C\left( (1+i)^t - 1 \right) 
        \end{equation*}

        where $C$ is the capital we have at the moment (with accumulated
        interests) and $t$ is the number of terms that passed. 
\end{enumerate}


\begin{shaded}
    \textbf{Example of simple capitalization.} An interest rate of $5\%$ to 30
    days and an initial capital of $10.000$ gives an interest of: $500$ in 30
    days, $1000$ in 60 days, $1500$ in 90 days, etc. Notice that each time 30
    days pass, we increment our money by $500 = 10.000 \times 0.05$.

    Notice that after $30$ days we would have $10.500$, and after $60$ we would
    have $11.000$. But there's a problem: $11.000$ is less than what would be 
    obtained with an interest rate of $5\%$ over the $10.500$ we had after the
    first $30$ days.

    In other words, if the $5\%$ of $10.000$ is $500$, the $5\%$ of $10.500$ is
    more than $500$. So we see that as time goes by, with the simple
    capitalization system, the growth becomes less and less relative to 
    the capital at hand.
\end{shaded}

\begin{shaded}
    \textbf{Example of composite capitalization.} An interest rate of $5\%$
    every 30 days with initial capital of 10.000. Then the cumulative interest
    will be: 

    \begin{enumerate}
        \item $500$ in 30 days. 
        \item $1025$ in 60 days ($500 + 0.05 \times 10.500$)
        \item $1576$ in 90 days ($1025 + 0.05 \times 11.025$)
    \end{enumerate}
\end{shaded}


\begin{shaded}
    \textbf{Problem.} Fixed term to 35 days with $I = 100.000$. An automatic renovation is done in
    the following two vencimientos. The interest on each term are $2.4\%, 2.4\%,
    2.8\%$. Compute the final capital after $105$ days if: $(a)$ total
    renovation is done, $(b)$ partial renovation is done. 

    \textbf{Solution.} The first term goes from day $0$ to 35, the second from
    day 35 to 70, and the third from day 70 to 105. The interest rate of each of
    these terms was specified above.

    With total renovation, this would work as follows. On day 35, after the
    first term, we would have $100.000 \times 1.025$. On day 70 we would apply
    the interest of the term to the total capital so far accrued, so we would
    have $100.000(1.025)(1.024)$. Lastly, on day $105$, we would apply the
    interest again to the total capital accrued, resulting in
    $100.000(1.025)(1.024)(1.028) = 107898.88 = F$. with an interest of 
    $F - I = 7898.88$.

    With partial renovation, on day 35 we would again have $100.000 \times
    1.025$. On day 70, we would add to our accrued capital the interest of the
    term applied to the initial capital: $100.000(1.025 + 0.024)$. Then, on day
    $105$, we would have $100.000(1.025 + 0.024 + 0.028)$. (If this is unclear,
    apply the distributive law and it will be.)
\end{shaded}

\subsection{Nominal and effective rates}

The TNA (Tasa Nominal Anual) is an annual rate that serves as reference to compute rates in different
terms through direct proportionality. For instance, if to 30 days we have a TNA
of $25\%$, this means the that the rate to 30 days is 

\begin{equation*}
    I_{\text{30 days}} = TNA \times \frac{\text{rate in days}}{365} = 0.25
    \times \frac{30}{365} = 0.0205 = 2.05\%
\end{equation*}

Notice that $30 / 365$ is the ``how much are 30 days to a year``.

The TEA (Tasa Efectiva Anual) is a rate equivalent to the effective rate being
applied, but corresponding to an annual term. For instance, if for a fixed term
to 30 days we have a rate of $2.05\%$, then the composite capitalization is 

\begin{equation*}
    1 \times (1.0205)^{\frac{365}{30}} = 1.2800
\end{equation*}

which means the TEA is $28\%$.

\subsection{Continuous capitalization}

If the TNA is $r$, the final capital in a year over an initial capital $C$, with
continuous capitalization, is 

\begin{equation*}
    C \times \lim_{m \to \infty}\left( 1 + \frac{r}{m} \right)^m = C \times e^r
\end{equation*}

Then 

\begin{equation*}
    TEA = e^t - 1
\end{equation*}

meaning that the final capital es 

\begin{equation*}
    C(t) = C \times e^{rt}
\end{equation*}

When modeling, we need to take $r$ as $r(t)$, i.e. as a rate that may
continuously vary in time, which adds extra complexity.

\subsection{Discounted value}

When comparing two quantities in different time instances, we determine the 
\textit{discounted value} or \textit{actual value} of a capital. Let $C_1$ the
capital at time $t_1$, $C_2$ the capital at time $t_2$, with $t_1 < t_2$.
Then via composite capitalization with rate $i$ 

\begin{equation*}
    C_{1} = \frac{C_2}{(1+i)^{t_2 - t_1}}
\end{equation*}

is the discounted value from $C_2$ at time $t_1$.
































\end{document}



