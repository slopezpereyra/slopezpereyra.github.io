\documentclass[12pt]{article}

\usepackage[cal=boondoxo]{mathalfa}
\usepackage{pgfplots}
\pgfplotsset{compat=1.18} % Or adjust depending on your TeX distribution
\usepackage[utf8]{inputenc}	% Para caracteres en español
\usepackage{amsmath,amsthm,amsfonts,amssymb,amscd}
\usepackage{multirow,booktabs}
\usepackage[table]{xcolor}
\usepackage{fullpage}
\usepackage{lastpage}
\usepackage{newtxtext}
\usepackage{newtxmath}
\usepackage{enumitem}
\usepackage{fancyhdr}
\usepackage{mathrsfs}
\usepackage{wrapfig}
\usepackage{setspace}
\usepackage{calc}
\usepackage{multicol}
\usepackage{cancel}
\usepackage[retainorgcmds]{IEEEtrantools}
\usepackage[margin=3cm]{geometry}
\usepackage{amsmath}
\newlength{\tabcont}
\setlength{\parindent}{0.0in}
\setlength{\parskip}{0.05in}
\usepackage{empheq}
\usepackage{framed}
\usepackage[most]{tcolorbox}
\usepackage{xcolor}
\colorlet{shadecolor}{orange!15}
\parindent 0in
\parskip 12pt
\geometry{margin=1in, headsep=0.25in}
\theoremstyle{definition}
\newtheorem{definition}{Definition}
\newtheorem{reg}{Rule}
\newtheorem{exer}{Exercise}
\newtheorem{note}{Note}
\newtheorem{theorem}{Theorem}
\begin{document}
\setcounter{section}{0}
\usepackage{chngcntr}

\counterwithin*{equation}{section}
\counterwithin*{equation}{subsection}


\begin{document}

\section{Dovetailling enumeration and the $\pi$ function}

Let $(x)_i$ denothe the power of the $i$th prime factor in the decomposition of
$x \in \omega$. Let $\varphi_e$ an arbitrary partial computable function
computed by program $P_e$. The Turing program $P_e$ runs on discrete steps, and
so we can conceive the following procedure:

\begin{enumerate}
    \item Fix $ n \leftarrow (x)_1, t \leftarrow (x)_2$. 
    \item If $P_e$ halts in $t$ steps from input $n$, halt and return
        $1$. 
    \item Otherwise, set $x \leftarrow x + 1$ and go back to step $(1)$.
\end{enumerate}

This program halts if and only if $P_e$ halts on some input, since eventually
all possible inputs are considered. We use $\pi(x)$ to denote the partial
computable function that performs the procedure above on program $x$. We observe
that if $W_x \neq \emptyset$ then $\pi(x)$ always halts, meaning that
$W_{\pi(x)} = \mathbb{N}$. On the contrary, if $W_x = \emptyset$ then $\pi(x)$
is undefined and $W_{\pi(x)} = \emptyset$.

\begin{theorem}
    There is a partial computable function $\pi^{c}_k(x)$ that takes $k$
    arguments and halts with output $c$ (independently of the arguments) if and
    only if $W_x \neq \emptyset$. Furthermore, if $W_x \neq \emptyset$, then
    $W_{\pi_k^c(x)} = \mathbb{N}$, and if $W_x = \emptyset$  then
    $W_{\pi_k^c(x)} = \emptyset$.
\end{theorem}


\small
\begin{quote}

\textbf{Proof.} Trivial to derive from all of the above.

\end{quote}
\normalsize

In general, if we use $\pi(x)$ to denote $\pi_1^1(x)$.

\section{Index set theorems}    

\begin{shaded}
    \textbf{Notation.} Let $\mathcal{F}$ be the set of partial computable
    functions. We use $\bot_F$ to denote the partial computable function which
    is undefined on all input.
\end{shaded}

A set $A \subseteq \omega$ is an index set if for all $x, y \in \omega$, 

\begin{equation*}
    x \in A \land \varphi_x = \varphi_y \Rightarrow y \in A
\end{equation*}

In other words, a set is an index set if all elements in the set index the same
partial computable function.

\begin{shaded}
    Would it not be better to set $\varphi_x \simeq \varphi_y$? Think about
    this.
\end{shaded}

Trivially, $\omega$ and $\emptyset$ are index sets. 

\begin{theorem}[Index set theorem]
    If $A$ is a non-trivial index set, then either $K\leq_1 A$ or $K \leq_1
    \overline{A}$. Furthermore, if $\bot_\mathcal{F} \in \overline{A}$ then 
    $K \leq_1 A$, and vice-versa.
\end{theorem}


\small
\begin{quote}

    \textbf{Proof.} Assume the index of $\bot_\mathcal{F}$ is in $A$ and take $y
    \in \overline{A}$. Define

    \begin{equation*}
        \phi(u, v) = \begin{cases}
            \varphi_y(v) & u \in K \\ 
            \bot  & c.c.
        \end{cases}
    \end{equation*}

    The function above is computable. The $S_n^m$ theorem ensures there is a
    total, one-to-one function $f$ s.t. $\varphi_{f(u)}(v) = \phi(u, v)$.

    If $u \in K$, then $\varphi_{f(u)} = \varphi_y$, meaning that $f(u) \in
    \overline{A}$. If $u \not\in K$, then $f(u)$ is the index of
    $\bot_\mathcal{F}$, which is in $A$.

    $\therefore $ $K \leq_1 \overline{A}$.

\end{quote}
\normalsize

\begin{theorem}[Rice's theorem]
    Let $\mathcal{C}$ any class of partial computable functions. Then 
    $A = \left\{ n : \varphi_n \in \mathcal{C} \right\} $ is not computable
    except in the trivial cases. 
\end{theorem}


\small
\begin{quote}

\textbf{Proof.} Assume the class $\mathcal{C}$ is non-trivial, meaning that $A =
\left\{ n : \varphi_n \in \mathcal{C} \right\} $ is neither $\omega$ nor
$\emptyset$. Then $K \leq_1 A$ or $K \leq_1 \overline{A}$ by virtue of the index
set theorem. Either case, $A$ is not decidable. $\blacksquare$

\end{quote}
\normalsize

\subsection{Some interesting sets}

The set of programs which halt with themselves as input ($K$), and the set of
pairs $(x, y)$ such that $P_x$ halts on input $y$ ($K_0$), are not index sets.
Index sets of interest are: 

\begin{enumerate}
    \item $K_1$, the set of programs that halt on some input:

        \begin{equation*}
            K_1 := \left\{ x : W_x \neq \emptyset \right\}  = \left\{ x : \varphi_x
            = \bot_\mathcal{F}\right\} 
        \end{equation*}

    \item \textit{Fin}, the set of programs that halt for a finite number of
        inputs.

        \begin{equation*}
            \textit{Fin} := \left\{ x : W_x \text{ is finite} \right\} 
        \end{equation*}

    \item The complement of \textit{Fin}, termed \textit{Inf}.

    \item $\mathcal{T}$, the set of total computable functions. 

    \item $\textit{Con} \subseteq \mathcal{T}$, the set of constant and total
        functions. 

    \item \textit{Cof}, the set of programs that halt on a cofinite number of
        input. 

    \item \textit{Rec}, the set of computable (recursive) programs. 

    \item \textit{Ext}, the set of programs that can be extended to total
        computable functions.
\end{enumerate}

An interesting factt is that $K \equiv_1 K_0 \equiv_1 K_1$. We already know that
$K \leq_1 K_1$, because $K_1$ is a non-trivial index set, so let us observe the
remaining relations. 

\begin{shaded}
    $(1: K_1 \leq K)$ This is intuitively clear, since deciding whether a
    program halts with itself as input would suffice to decide whether it halts
    at all. 
    
    Recall that 

    \begin{equation*}
        W_{\pi(x)} = \begin{cases}
            \mathbb{N} & x \in K_1 \\ 
            \emptyset & x \not\in K_1
        \end{cases}
    \end{equation*}

    This entails that if $x \in K_1$, then $\pi(x) \in W_{\pi(x)}$, which means
    that $\pi(x) \in K$. If $x \not\in K_1$, clearly $\pi(x) \not\in
    W_{\pi(x)}$, since said domain is the empty set, and $\pi(x) \not\in K$.
    $\blacksquare$

    $(2: K_0 \leq K)$ Observe that $x \in K \iff (x, x) \in K_0$. Define

    \begin{equation*}
        \psi(x, y, z) = \begin{cases}
            \varphi_{y}(x) & (x, y) \in K_0 \\ 
            \bot & c.c.
        \end{cases}
    \end{equation*}

    It is computable and therefore there is an index $e$ s.t. 
    $\psi(x, y, z) = \varphi_e(x, y, z)$. By virtue of the $S_n^m$ theorem,
    there is a computable and one-to-one function $f$ s.t. $\varphi_{f(e, x,
    y)}(z) = \varphi_e(x, y, z)$. Fix $e$ and let $g(x, y) = f(e, x, y)$.

    Assume $(x, y) \in K_0$. Then $\varphi_{g(x, y)}(z)$ halts for all $z$.
    In particular, it halts for 
    $\varphi_{g(x, y)}(g(x, y))$. This entails $g(x, y) \in K$.

    Similarly, assume $(x, y) \not\in K_0$. Then $\varphi_{g(x, y)}(z)$ halts
    for no input at all, i.e. $W_{g(x, y)} = \emptyset$. In particular, 
    $g(x, y) \not\in W_{g(x, y)}$, which means $g(x, y ) \not\in K$.

    $\therefore ~ K_0 \leq K$. 


    All of this suffices to show $K_0 \equiv_1 K_1 \equiv_1 K$.
\end{shaded}


\begin{definition}
    A computably enumerable set $A$ is $1$-complete if $W_e \leq_1 A$ for every
    computably enumerable set $W_e$.
\end{definition}


\begin{shaded}
    \textbf{Problem: Is $K_0$ $1$-complete?} The answer is yes. Take $W_e$
    arbitrary and c.e. Then $x \in W_e$ if and only if $(x, e) \in K_0$. Thus
    suffices to show $W_e \leq_1 K_0$. 
\end{shaded}

\subsection{Computable approximations to computations}

We write $\varphi_{e, s}(x) = y$ if $x, y, e < s$ and $y$ is the output of
$\varphi_e(x)$ in less than $s$ steps of program $P_e$. If such a $y$ exists we
say $\varphi_{e, s}(x)$ converges. We define $W_{e, s} := \text{dom}(\varphi_{e,
s})$.

Note that by def. fif $x \in W_{e, s}$ then $x, e < s$. Furthermore, 

\begin{equation*}
    \varphi_e(x) = y \iff \exists s . \varphi_{e, s}(x) = y
\end{equation*}

\begin{theorem}
    The set $\left\{ (e, x, s) : \varphi_{e, s}(x) \downarrow \right\} $ is
    computable, as is the set $W_{e, s}$.
\end{theorem}






























\end{document}



