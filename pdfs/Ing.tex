\documentclass[a4paper, 12pt]{article}

\usepackage[utf8]{inputenc}
\usepackage[T1]{fontenc}
\usepackage{textcomp}
\usepackage{amssymb}
\usepackage{newtxtext} \usepackage{newtxmath}
\usepackage{amsmath, amssymb}
\newtheorem{problem}{Problem}
\newtheorem{example}{Example}
\newtheorem{lemma}{Lemma}
\newtheorem{theorem}{Theorem}
\newtheorem{problem}{Problem}
\newtheorem{example}{Example} \newtheorem{definition}{Definition}
\newtheorem{lemma}{Lemma}
\newtheorem{theorem}{Theorem}


\begin{document}

Tips for doing good:

\begin{itemize}
    \item Do the project well 
    \item Do the take home well
\end{itemize}

Los parciales no son ambiguos.

\section{Clase 2 - 15 Aug}

\subsection{Repaso clase 1.}

Software is not only code: it's everything surrounding it. We will be thinking
about big and pricy software.

Two types of mantaining: adaptativo y correctivo (importante).

Tipica pregunta de parcial: Cuales son los desafios de la IS: Escala, Calidad,
Productividad, Consistencia, Cambios (explayando).

Escala: La escala es hacia arriba y hacia abajo; 

Productividad: Como se mide, por que sirve. 

Calidad: Funcionalidad, COnfiabilidad, USabilidad, Eficiencia, Mantenibilidad, Portabilidad. 

Cambio:

Software engineering's domain

\subsection{Cap. 3 de la bibliografia: Analisis y especificacion de requerimientos}

Para cada fase, existe una entrada y una salida. Las necesidades son abstractas, ideas; la salida es 
un detalle preciso de lo que sera un sistema. Respecto a los requerimientos, son generalmente 
faciles de comprender y especificar si la escala es pequena, pero dificil si es grande.


























\end{document}



