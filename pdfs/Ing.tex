\documentclass[a4paper, 12pt]{article}

\usepackage[utf8]{inputenc}
\usepackage[T1]{fontenc}
\usepackage{textcomp}
\usepackage{amssymb}
\usepackage{newtxtext} \usepackage{newtxmath}
\usepackage{amsmath, amssymb}
\newtheorem{problem}{Problem}
\newtheorem{example}{Example}
\newtheorem{lemma}{Lemma}
\newtheorem{theorem}{Theorem}
\newtheorem{problem}{Problem}
\newtheorem{example}{Example} \newtheorem{definition}{Definition}
\newtheorem{lemma}{Lemma}
\newtheorem{theorem}{Theorem}
\usepackage{parskip}

\begin{document}

Tips for doing good:

\begin{itemize}
    \item Do the project well 
    \item Do the take home well
\end{itemize}

Los parciales no son ambiguos.

\section{Clase 2 - 15 Aug}

\subsection{Repaso clase 1.}

Software is not only code: it's everything surrounding it. We will be thinking
about big and pricy software.

Two types of mantaining: adaptativo y correctivo (importante).

Tipica pregunta de parcial: Cuales son los desafios de la IS: Escala, Calidad,
Productividad, Consistencia, Cambios (explayando).

Escala: La escala es hacia arriba y hacia abajo; 

Productividad: Como se mide, por que sirve. 

Calidad: Funcionalidad, COnfiabilidad, USabilidad, Eficiencia, Mantenibilidad, Portabilidad. 

Cambio:

Software engineering's domain

\subsection{Cap. 3 de la bibliografia: Analisis y especificacion de requerimientos}

Para cada fase, existe una entrada y una salida. Las necesidades son abstractas, ideas; la salida es 
un detalle preciso de lo que sera un sistema. Respecto a los requerimientos, son generalmente 
faciles de comprender y especificar si la escala es pequena, pero dificil si es grande.

\pagebreak 

\section{Game}

Two types of cards: movement cards and figure cards. Figure cards are public,
movement cards are private. Figure cards are of two types: white and blue. Each
type of card (movement, figure blue, figure white) are dealt modulo the number
of players to each player.

When a a player's turn ends, if he has freed himself from all figure cards, three new figure cards are 
dealt to him.

Each time a figure is formed, it may be used in one and only one action:
freeing that figure from the former's hand, or blocking that figure from
another player's hand. If $k$ figures are formed, each figure may be used in
one and only one action still, amounting to $k$ distinct actions.

Each time a figure is used to enact an action, it changes the forbidden color 
to the color of that figure.

When a movement card is used, it goes to the maze and may be re-dealt. When a
figure card is used, it disappears.

A player's turn ends when he uses all his movement cards or he cannot form any
valid figure (or cannot find a way to do so).

The game ends when a player frees himself from all his figure cards.

\textbf{Blocking cards.} 

A player $p_i$ may form a figure owned by $p_j$. If this occurs, $p_j$ cannot
form that figure. If the figure is possesed by both $p_j$ and $p_i$, $p_i$ may
chose whether the formation has a blocking effect agains $p_j$ or if he
disposes himself of his figure card.

If $p_j$ has a blocked figure, no more of his figures can be blocked. If a
player has only 1 figure card, this card cannot be blocked. In other words, 
a figure card can be blocked iff its owner has $\geq 2$ figure cards.

If a figure card is blocked, its owner must free his other cards before being 
able to free it. New figure cards are dealt to him only once he frees (uses)
his blocked card.

Incidentally, if two figures are formed after a sequence of movement cards are used, 
the two figures can be used in any standard legal way; i.e. to block two players, or 
two free one of my figures and block another player, etc.























\end{document}



