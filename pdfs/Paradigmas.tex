\documentclass[a4paper, 12pt]{article}

\usepackage[utf8]{inputenc}
\usepackage[T1]{fontenc}
\usepackage{textcomp}
\usepackage{amssymb}
\usepackage{newtxtext} \usepackage{newtxmath}
\usepackage{amsmath, amssymb}
\newtheorem{problem}{Problem}
\newtheorem{example}{Example}
\newtheorem{lemma}{Lemma}
\newtheorem{theorem}{Theorem}
\newtheorem{problem}{Problem}
\newtheorem{example}{Example} \newtheorem{definition}{Definition}
\newtheorem{lemma}{Lemma}
\newtheorem{theorem}{Theorem}


\begin{document}

\section{Tipos ( Clase teórica 9-04 )}

Building blocks of programming languages. They divide values by sets; can think
of them as a set division of values in a language. Formally, it is a collection
of values sharing structural properties. The set of values in a type is usually
finite because they have a binary representation in the computar. For example,
32-bit integers range in $[-2147483648, 2147483647]$. This is important because
it affects behavior; e.g. if a sum of two integer surpasses this range, the
result will not be the expected one.

Primitive types are those included built-in in a language (generally booleans, ints, reals, 
characters, but not strings). 

Types are useful for documenting the language and avoid errors. Furthermore,
they are useful when dealing, for example, with memory allocation. Memory used
to be highly expensive. Thus, it was important to know how much memory
different types of values needed; this was an original distinction (values that
take so and so much memory, etc).

Then came more semantic distinctions---this is, based on logics. The larger the
set of characteristics that define a type, the less elements this type
contains. Types allow for a more structured organization of languages, which on
its turn makes it less likely to commit mistakes on runtime (assuming the
language is strongly typed).

Static type systems are those where variable types are fixed in compile time;
dynamic type systems are those where types are set in run-time.

A strongly typed system is one that imposes strong type restrictions so that
the program behaves in a predictible manner. In general, strongly typed
languages are more robust and easier to understand. The downside is that
strongly typed programs are rigid and unflexible, and thus break more easily.

Weak type systems are harder to define and more of a grey zone. Languages that
are rigid but can make some exceptions use \textit{casting}. To \textit{cast} a
type is to convert it from one type into another. Castings imposed by the
compiler (for example, the ones emerging from printing an array or a
dictionary) are usually arbitrary. In strongly typed languages, castings have
to be explicit.

Types can be atomic or composite. Composite types have multiple parts. There
are also user-defined types (vs. bult-in types). User-defined types that are
defined on rum-time (not in compile time) cannot be syntatically decomposed
into a tree.

Most operations in a language are defined with relation to types ($+, -, /, *,
\ldots$).

\textit{Sobrecaraga, polimorfismo}. Overhead (sobrecarga) is the variation in 
the meaning of a function or operator as a function of the type of its 
arguments or terms. For example, we can sum integers and floats; in 
this case, the result may be a float.  For example, multiple dispatch 
is an example of overhead.


\small
\begin{quote}

El significado de un eprador o function cambia dependiendo de los tipos 
de sus operandos o argumentos o resultado.

\end{quote}
\normalsize

A type or function is polymorphic if it can be applied to any associated type.

The difference between polymorphhism and overhead is that, in overhead, a
single symbol refers to multiple algorithms (depending on the type of the
parameters or operands)---each algorithm has different types, ant the algorithm
is chosen based on the types used. In polymorphhism, a single algorithm may
take multiple types---there is a unique implementation---the type variable is
replaced by the type in question.


\textit{Type inference.} Types can be \textit{checked} (comprobados) or
infered. In type checking, the body of the function is analyzed and the
declared types are used to check that everything matches. 

In type inference, the code of the function is looked at without declaring
types and the types are infered based on the operations used.

\pagebreak 

\section{ (Clase teórica 16-04)}




\end{document}



