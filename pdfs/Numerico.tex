\documentclass[a4paper, 12pt]{article}

\documentclass[11pt]{article}
\usepackage[utf8]{inputenc}	% Para caracteres en español
\usepackage{amsmath,amsthm,amsfonts,amssymb,amscd}
\usepackage{multirow,booktabs}
\usepackage[table]{xcolor}
\usepackage{fullpage}
\usepackage{lastpage}
\usepackage{newtxtext}
\usepackage{newtxmath}
\usepackage{enumitem}
\usepackage{fancyhdr}
\usepackage{mathrsfs}
\usepackage{wrapfig}
\usepackage{setspace}
\usepackage{calc}
\usepackage{multicol}
\usepackage{cancel}
\usepackage[retainorgcmds]{IEEEtrantools}
\usepackage[margin=3cm]{geometry}
\usepackage{amsmath}
\newlength{\tabcont}
\setlength{\parindent}{0.0in}
\setlength{\parskip}{0.05in}
\usepackage{empheq}
\usepackage{framed}
\usepackage[most]{tcolorbox}
\usepackage{xcolor}
\colorlet{shadecolor}{orange!15}
\parindent 0in
\parskip 12pt
\geometry{margin=1in, headsep=0.25in}
\theoremstyle{definition}
\newtheorem{defn}{Definition}
\newtheorem{reg}{Rule}
\newtheorem{exer}{Exercise}
\newtheorem{note}{Note}
\newtheorem{theorem}{Theorem}
\begin{document}
\setcounter{section}{8}
\usepackage{chngcntr}

\counterwithin*{equation}{section}
\counterwithin*{equation}{subsection}

\begin{document}


\begin{document}

\section{Alg. de Horner: Polynomial evaluation} 

Consider 

\begin{equation*}
    p(x) = \sum_{i=0}^n a_ix^i
\end{equation*}

We wish to compute $p(k)$ for a given $k \in \mathbb{R}$ minimizing the number
of operations. Directly computing $a_0 + a_1k_1 + \ldots$ leads to $n$ sums. The
$i$th term requires computing $k^i$, which means $i$ product operations, for a
totall of $\sum_{i=1}^n i = \frac{n(n+1)}{2}$ products.  The total number of
operations is then 

\begin{equation*}
    \Theta = n + n(n+1) / 2
\end{equation*}

The associated complexity is $\mathcal{O}(n^2)$. 

Horner's method consists of re-writing $p(x)$ so that the number of products is
reduced. One writes 

\begin{equation*}
    p(x) = a_0 + x b_0
\end{equation*}

where $b_{n-1} = a_n$ and for $0 \leq i < n - 1$:

\begin{equation*}
    b_{i-1} = a_{i} + xb_{i}
\end{equation*}

\begin{shaded}
    \begin{example}
        \normalfont
        Let $p(x) = 3 + 5x -4x^2 + 0x^3 + 6x^4$, giving $n = 4$. Then 
        $b_{3} = 6$ and 

        \begin{align*}
            &b_2 = a_3 + xb_3 = 6x,  &b_1 = a_2 + xb_2 = -4 + x(6x),\\
            &b_0 = a_1 + xb_1 = 5 + x(-4+x(6x)) 
        \end{align*}

        This finally gives

        \begin{equation*}
            p(x) = 3 + xb_0 = 3+x(5 +
            x(-4+x(6x)))
        \end{equation*}
    \end{example}
\end{shaded}

Here, one must perform $n$ sums again but only $n$ products. Thus, there are
$\Theta = n + n = 2n$ operations, giving a complexity of $\mathcal{O}(n)$ (in
the operation space). See the algorithm below:

\begin{align*}
&\textbf{input } n; a_i, i = 0, \ldots, n; x \\ 
& b_{n-1} \leftarrow a_n \\ 
& \textbf{for } i = n - 2 \textbf{ to } i = 0 \\ 
&\qquad b_{i} = a_{i+1} + x * b_{i+1} \\ 
&\textbf{od}\\
&y \leftarrow a_0 + x*b_0 \\ 
&\textbf{return } y
\end{align*}

It is easy to see in this code that the $\textbf{for}$ loop performs $n-1$
iterations, in each of which a single sum and a single product are computed. The
$n$th sum and $n$th product are performed in the computation of $y$, the final
result. 

A more polished version includes the last computatoin (the one in the assignment
of $y$) within the loop and makes no use of indexes:


\begin{align*}
&\textbf{input } n; a_i, i = 0, \ldots, n; x \\ 
& b \leftarrow a_n \\ 
& \textbf{for } i = n - 2 \textbf{ to } i = -1 \\ 
&\qquad b = a_{i+1} + x * b \\ 
&\textbf{od}\\
&\textbf{return } b
\end{align*}
 
In Python,


\small
\begin{quote}


\begin{verbatim}
def horner(coefs, x):
  n = len(coefs)-1
  b = coefs[n]

  for i in reversed(range(-1, n-1)):
    b = coefs[i+1] + x*b

  return b

\end{verbatim}

\end{quote}
\normalsize


It is trivial to adapt the code so that it returns the coefficients $b_0,
\ldots, b_{n-1}$ and not the final result, if needed.

\pagebreak 

\section{Error}

Let $r, \overline{r}$ be two real numbers s.t. the latter is an approximation of
the first. We define the \textbf{error} of the approximation to be $r -
\hat{r}$, and 

\begin{align*}
    \Delta r = \left| r - \overline{r} \right|, \qquad \delta r = \frac{\Delta
    r}{\left| r \right| }
\end{align*}

With $r$ unknown the strategy is to work with a known bound of $r$.

\pagebreak 

\section{Non-linear equations}

The general problem is to find members of the set $\mathcal{R}_f$ of roots of $f
\in \mathbb{R} \to \mathbb{R}$. The numerical strategy is to iteratively
approximate some $r \in \mathcal{R}_f$ until some pre-established threshold in
the error of approximation is met. 

More formally, the numerical strategy produces a sequence $\left\{ x_k
\right\}_{k\in \mathbb{N}}
$ which satisfies 

\begin{itemize}
    \item $\lim_{k\to \infty} \left\{ x_k \right\} = r$  for some $r \in
        \mathcal{R}_f$ 
    \item Either $e(x_k) < e(x_{k-1})$ or, more strongly, $\lim_{k\to \infty}
        e(x_k) = 0$, where $e(x_k)$ is some appropriate measure of the error of
        approximation.
\end{itemize}

\subsection{Bisection}

A very simple procedure: if a root exists in $[a, b]$, it iteratively shrinks
$[a,b]$ in halves (keeping the halves which contain the root) until the interval
is of sufficiently small length.

\begin{theorem}[Intermediate value]
    If $f$ is continuous in $[a, b]$ and $f(a)f(b) < 0$, then $\exists r \in
    \mathcal{R}_f$ s.t. $r \in [a, b]$.
\end{theorem}

Assume $f$ is continuous. A root exists in $[a, b]$ if $f(a)f(b) < 0$
(\textbf{Theorem 1}). If that is the case, the midpoint $(a+b) / 2$ is taken as
the approximation $x_0$. It is also trivial to observe that $x_0$ is \textit{at
most} at a distance of $(b-a) / 2$ from the real root, so $e_0 = |x_0 - r| \leq
(b-a) / 2$. 

If $f(x_0) = 0$ the procedure must end because a root was found. Otherwise,
sufficies to find which half of the interval contains a root computing
$f(a)f(c)$ and, if needed, $f(c)f(b)$.

The iterations may stop after reaching a maximum number of steps, when $|f(c)|$
is sufficiently close to zero, or when the error bound $|e_k| \leq (b_k - a_k) /
2$ (where $[a_k, b_k]$ is the interval of this iteration) is sufficiently small.

\begin{shaded}
    \textbf{(!)} The algorithm not always converges. Take $f(x) = 1 / x$. Clearly, it has no
    root. Yet setting $a = -1, b=1$ in the initial iteration falsely passes the
    test. (The problem obviously is that $f$ is not continuous in $[-1, 1]$.) If
    one sets
\end{shaded}

\pagebreak 

\begin{align*}
&\textbf{Input}: a, b, \delta, M, f\\  
&\textbf{Output}: \text{Tupla de la forma: } (r, \text{cota de error})\\
&f_a \leftarrow  f(a) \\ 
&f_b \leftarrow f(b) \\ 
&\qquad\\
&\textbf{if } f_a*f_b > 0 \\ 
&\qquad \textbf{return } ?\\ 
&\textbf{fi}\\
&\qquad\\
&\textbf{for } i = 1 \textbf{ to } i = M \textbf{ do } \\ 
&\qquad c \leftarrow a + (b-a) / 2 \\ 
&\qquad f_c \leftarrow f(c) \\ 
&\qquad \textbf{if } f_c = 0 \textbf{ then } \\ 
&\qquad\qquad \textbf{ return } (c, 0) \\ 
&\qquad\textbf{fi}\\
&\qquad \epsilon = \frac{b - a}{2}\\
&\qquad \textbf{if } \epsilon < \delta \textbf{ then } \\ 
& \qquad\qquad \textbf{break}\\
&\qquad\textbf{fi}\\
&\qquad \textbf{if } f_a*f_c < 0 \textbf{ then } \\ 
&\qquad \qquad b \leftarrow c \\ 
&\qquad\qquad f_b = f(b)\\
&\qquad\textbf{else } \\ 
&\qquad\qquad a \leftarrow c \\ 
&\qquad\qquad f_a = f(a)\\
&\qquad\textbf{fi} \\ 
&\textbf{od}\\
&\textbf{return } (c, \epsilon)
\end{align*}

\pagebreak 


\small
\begin{quote}

\begin{verbatim}
def bisection(f : callable, a : float, b : float, delta : float, M : int):

  s, e = f(a), f(b) # function values at (s)tart, (e)nd of interval

  if s*e > 0:
    raise ValueError("Interval [a, b] contains no root.")

  for i in range(M):

    c = a + (b-a)/2 
    m = f(c) # value of f at (m)idpoint

    if m == 0:
      return c, 0

    e = (b-a)/2
    if e < delta:
      return c, e

    if s*m < 0:
      b = c 
      e = f(b)
    else:
      a = c 
      s = f(a)
    
  return c, e

\end{verbatim}

\end{quote}
\normalsize

\pagebreak

\begin{theorem}
    If $\left\{ [a_i, b_i] \right\}_{i=0}^\infty $ are the intervals generated
    by the bisection method on iterations $i = 0, 1, \ldots$, then:

    \begin{enumerate}
        \item $\lim_{n\to
    \infty} a_n = \lim_{n \to \infty} b_n$ is a member of $\mathcal{R}_f$.
     \item If $c_n = \frac{1}{2}(a_n + b_n), r = \lim_{n \to \infty} c_n $, then 
         $|r - c_n| \leq \frac{1}{2^{n+1}}(b_0 - a_0)$
    \end{enumerate}

\end{theorem}


\small
\begin{quote}

\textbf{Proof.} \textbf{(1)} It is clear that $a_i \leq a_{i+1}$ and $b_i \geq b_{i+1}$,
since the interval on each iteration shrinks in one direction. 

$\therefore a_n, b_n$ are monotonous. 

But clearly $a_n$ is bounded by $b_0$ and $b_n$ is bounded by $a_0$. 

$\therefore $ $a_n, b_n$ are monotonous and bounded. 

$\therefore $ Their limits exist.

It is also clear that the interval shrinks to half its size on each iteration: 

\begin{equation}
    b_n - a_n = \frac{1}{2} (b_{n-1} - a_{n-1}), \qquad n \geq 1
\end{equation}

By recurrence on $(1)$,

\begin{equation}
    b_n - a_n = \frac{1}{2^n}(b_0 - a_0), \qquad n \geq 0
\end{equation}

Then 

\begin{equation}
   \lim_{n \to \infty} a_n - \lim_{n \to \infty} b_n = \lim_{n \to \infty} (a_n - b_n) 
    = \lim_{n \to \infty}  \frac{1}{2^n}(b_0 - a_0) = 0
\end{equation}

$\therefore ~ \lim_{n \to \infty} a_n = \lim_{n \to \infty} b_n$.


Since the limit of $a_n, b_n$ exists and $f$ is by assumption continuous, the
composition limit theorem applies and:

\begin{align}
    &\lim_{n \to \infty} \left( f(a_n) \cdot f(b_n) \right) \nonumber
\\
    =& \lim_{n \to \infty}    f(a_n) \cdot \lim_{n \to \infty} f(b_n) &\left\{ \text{Product of limits} \right\}  \nonumber \\ 
    =& f\left( \lim_{n \to \infty} a_n
\right) \cdot f\left( \lim_{n \to \infty} b_n \right)  &\left\{
\text{Composition limit theorem} \right\} \nonumber \\ 
    =& \left[ f(r) \right]^2 &\left\{ r = \lim_{n \to \infty} a_n \right\} 
\end{align}

The invariant of the algorithm is $f(a_n)f(b_n) < 0$. But due to the last
result,

\begin{equation*}
    \lim_{n \to \infty} f(a_n)f(b_n) \leq 0 \iff [f(r)]^2 \leq 0 \iff f(r) = 0
\end{equation*}

$\therefore r = \lim_{n \to \infty} a_n = \lim_{n \to \infty} b_n$ is a root.

\textbf{(2)} Follows directly from result $(2)$

\begin{align*}
    \left| r - c_n \right|  &= \left| r - \frac{1}{2}(b_n - a_n) \right| \\ 
                            &\leq
    \left| \frac{1}{2}(b_n - a_n) \right| \\ 
&=\left| \frac{1}{2^{n+1}}(b_0 - a_0) \right| &\left\{ \text{Result }(2)
\right\} 
\end{align*}



\end{quote}
\normalsize

\pagebreak 

\subsection{Newton's method}

\begin{shaded}
    \textbf{Taylor: repasito.} El desarrollo de una $f$ suficientemente
    diferenciable alrededor de un punto $r$ espa

    \begin{equation*}
        f(x) = f(r) + f'(r)(x-r) + \frac{f''(r)}{2!}(x-r)^2 + \ldots +
        \frac{f^{(n)}(r)}{n!}(x-r)^n + R_n(x)
    \end{equation*}

    donde $R_n(x)$ es el resto.

    Usualmente, queremos tomar $r = x + h$ , donde $x$ es una aproximación de
    $r$ y $h$ el error de aproximación. Entonces es provechoso expandir 
    $f(r)$ alreededor de su estimación $x$:

    \begin{equation*}
        f(r) = f(x+h) = f(x) + f'(x)h + \frac{f''(x)}{2!}h^2 + \ldots +
        \frac{f^{(n)}(x) }{n!}h^n + R_n(h)
    \end{equation*}

    Esto es \textbf{recontra} útil porque nos dice cuánto se diferencia $f(r)$
    de nuestra aproximación $f(x)$ (pues expresa $f(r)$ como $f(x)$ más algo).

    Usualmente $r, h$ son desconocidos pero $h$ puede acotarse. 

    El resto $R_n$ del teorema puede expresarse como sigue:

    \begin{equation*}
        f(r) = f(x+h) = f(x) + f'(x)h + \frac{f''(x)}{2!}h^2 + \ldots +
        \frac{f^{(n)}(x) }{n!}h^n + \frac{ f^{(n+1)}(\zeta) }{(k+1)!}h^{n+1}
    \end{equation*}

    para algún $\zeta \in (x, h)$. Esta forma de expresar el error de
    aproximación con el polinomio de Taylor se usará mucho.
\end{shaded}

Assume $r \in \mathcal{R}_f$ and $r = x + h$, with $x$ an approximation of $r$
and $h$ its error. Assume $f''$ exists and is continuous in some $I$ around $x$
s.t. $r \in I$. What we explained on Taylor expansions around a point gives:

\begin{equation*}
    0 = f(r) = f(x + h) = f(x) + f'(x)h + \mathcal{O}(h^2)
\end{equation*}

If $x$ is sufficiently close to $r$, $h$ is small and $h^2$ even smaller, so
that $\mathcal{O}(h^2)$ is unconsiderable:

\begin{equation*}
    0 \approx f(x) + hf'(x)
\end{equation*}

Therefore, 

\begin{equation}
    h \approx - \frac{f(x)}{f'(x)}
\end{equation}

From this follows that $r = x + h$ is approximated by 

\begin{equation*}
    r \approx x - \frac{f(x)}{f'(x)}
\end{equation*}

Since the approximation in $(5)$ truncated the terms of $\mathcal{O}(h^2)$
complexity, this new approximation is closer to $r$ than $x$ originally was. In
other words, $x - f(x) / f'(x)$ is a better approximation to $r$ than $x$
itself. 

Thus, if $x_0$ is an original approximation, we can define 

\begin{equation}
    x_{n+1} = x_n - \frac{f(x_n)}{f'(x_n)}
\end{equation}

to produce a sequence of approximations. This is the fundamental idea of
Newton's method.

\begin{align*}
&\textbf{Input: } x_0, M, \delta, \epsilon; \\ 
&v \leftarrow f(x_0) \\ 
&\textbf{if } |v| < \epsilon \textbf{ then return }  x_0 \textbf{ fi }\\ 
&\textbf{for } k = 1 \textbf{ to } k = M \textbf{ do } \\ 
&\qquad x_1 \leftarrow x_0 - \frac{v}{f'(x_0)} \\
&\qquad v \leftarrow f(x_1) \\ 
&\qquad \textbf{if } \left| x_1 - x_0 \right| < \delta \lor v < \epsilon
\textbf{ then } \\ 
&\qquad\qquad \textbf{return } x_1 \\ 
&\qquad\textbf{fi} \\ 
&\qquad x_0 \leftarrow x_1 \\ 
&\textbf{od} \\ 
&\textbf{return } x_0
\end{align*}

The predicate $\left| x_1 - x_0 \right| < \delta$ checks whether our algorithm
is adjusting $x$ in a negligible degree. If that is the case, we
should stop. 

\begin{theorem}
    If $f''$ continuous around $r \in \mathcal{R}_f$ and $f'(r) \neq 0$, then
    there is some $\delta > 0$ s.t. if $\left| r - x_0 \right| \leq \delta $,
    then: 

    \begin{itemize}
        \item $\left| r - x_n \right| \leq \delta $ for all $n \geq 1$. 
        \item $\left\{ x_n \right\} $ converges to $r$  
        \item The convergence is quadratic, i.e. there is a constant $c(\delta)$
            and a natural $N$ s.t. $\left| r - x_{n+1} \right| \leq c\left| r -
            x_n\right|^2  $ for all $n \geq N$.
    \end{itemize}

\end{theorem}



\small
\begin{quote}

\textbf{Proof.} Let $e_n = r - x_n$ be the error in the $n$th approximation. Assume $f''$ is
continuous and $f(r) = 0$, $f'(r) \neq 0$. Then 

\begin{align}
    e_{n+1} &= r-  x_{n+1}  \nonumber \\ 
&= r - \left( x_n - \frac{f(x_n)}{f'(x_n)} \right) \nonumber \\ 
&= r - x_n  + \frac{f(x_n)}{f'(x_n)}\nonumber \\
&=\frac{e_n f'(x_n) + f(x_n)}{f'(x_n)}
\end{align}

Thus, the error at any given iteration is a function of the error at the
previous iteration. Now consider the expansion of $f(r)$ as 

\begin{equation}
    f(r) = f(x_n - e_n) = f(x_n) + e_nf'(x_n) + \frac{e_n^2f''(\zeta_n)}{2}
\end{equation}

for $\zeta_n$ between $x_n$ and $r$. This equation gives 

\begin{equation}
    e_n f'(x_n) + f(x_n) = -\frac{1}{2} f''(\zeta_n)e_n^2
\end{equation}

The expression in $(5)$ is the numerator in $(3)$, whereby we obtain via
substitution: 

\begin{equation}
    e_{n+1} =  -\frac{1}{2}\frac{f''(\zeta_n)e^2_n}{f'(x_n)} 
\end{equation}

Equation $(6)$ ensures that the error scales quadratically. Now we wish to
bound the error expression in $(6)$. To bound $e_{n+1}$, we take
$\delta > 0$ to define a neighbourhood of length $\delta$ around $r$. For any
$x$ in this neighbourhood, $(6)$ reaches its maximum when the numerator is
maximized and the denominator is minimized:

\begin{equation*}
    c(\delta) = \frac{1}{2} \frac{ \max_{\left| x - r \right| \leq \delta }
    \left| f''(x) \right|  }{\min_{\left| x-r \right| \leq \delta \left| f'(x) \right|  }}
\end{equation*}

In other words, $c(\delta)$ is the maximum value which $e_{n+1}$ can take if
$\zeta_n, x_n$ are assumed to belong to the neighbourhood. Now we make two
assumptions: 

\begin{enumerate}
    \item $x_0$ belongs to the neighbourhood, i.e. $\left| x_0 - r \right| \leq
        \delta$ 
    \item $\delta$ is sufficiently small so that $\varrho := \delta c(\delta) < 1$.
\end{enumerate}

Note that, since $\zeta_0$ is between $x_0$ and $r$, assumption (1) ensures that
$\zeta_0$ is also in the neighbourhood, i.e. $\left| r - \zeta_0 \right| \leq
\delta$. Then we have:

\begin{equation*}
    \left| e_0 \right| = \frac{1}{2} \left| f''(\zeta_0) / f'(x_0) \right| \leq c(\delta)
\end{equation*}

Then:

\begin{align*}
    \left| x_1 - r \right|  
    &= \left| e_1 \right|  \\ 
    &= \left| e_0^2 \cdot \frac{1}{2} f''(\zeta_0) / f'(x_0)  \right|  \\ 
    &\leq |e_0^2| c(\delta) &\left\{ \frac{1}{2}f''(\zeta_0) / f'(x_0) \leq c(\delta) \right\}  \\ 
    &\leq |e_0| \delta c(\delta) &\left\{ \left| e_0 \right| \leq \delta
    \right\}  \\ 
    &=\left| e_0 \right| \varrho &\left\{ \varrho = \delta c(\delta) \right\} \\
    &< \left| e_0 \right|  &\left\{ \varrho < 1 \right\}  \\ 
    &\leq \delta
\end{align*}

$\therefore $ $\left| e_1 \right| < \left| e_0 \right| \leq \delta$, which means
the error decreases. This argument may be repeated inductively, giving:
 
\begin{align*}
    &\left| e_1 \right|  \leq \varrho \left| e_0 \right|  \\ 
    &\left| e_2 \right|  \leq \varrho \left| e_1 \right| \leq \varrho^2 \left| e_0 \right|   \\ 
    &\left| e_3 \right|  \leq \varrho \left| e_2 \right| \leq \varrho^3 \left| e_0 \right|   \\ 
    &\vdots
\end{align*}

In general, $\left| e_{n} \right| \leq \varrho^n \left| e_0 \right|
$. And since $0 \leq \varrho < 1$, we have $\varrho^n \to 0$ when $n \to \infty$,
entailing that $\left| e_n \right| \to 0$ when $n \to \infty$.
\end{quote}
\normalsize

\begin{theorem}
    If $f''$ is continuous in $\mathbb{R}$, and if $f$ is increasing, convex,
    and has a root, then said root is unique and Newton's method converges to it
    from any starting point.
\end{theorem}

\begin{shaded}
    Recall that $f$ is convex if $f''(x) > 0$ for all $x$. Graphically, it is
    convex if the line connecting two arbitrary points of $f$ lies above the
    curve of $f$ between those two points.
\end{shaded}


\subsection{Secant method}

In Netwon's method, 

\begin{equation*}
    x_{n+1} = x_n - \frac{f(x_n)}{f'(x_n)}
\end{equation*}

The function of interest is $f$. We cannot escape computing $f(x_n)$, but it
would be desirable to avoid the computation of $f'(x_n)$, which may potentially
be expensive. Since

\begin{equation*}
    f'(x) = \lim_{h \to x}  \frac{f(x) - f(h)}{x-h}
\end{equation*}

it is natural to suggest 

\begin{equation}
    f'(x_n) \approx \frac{f(x_n) - f(x_{n-1})}{x_n - x_{n-1}}
\end{equation}

Graphically, this means we are not using the line tangent to the point $(x_n,
f(x_n))$ but the line secant to the points $(x_n, f(x_n)$ and $(x_{n-1},
f(x_{n-1}))$. The point $x_{n+1}$ is then the value of $x$ where this secant
line has a root.

\subsection{Fixed point iteration}

The key observation is  this: if $r \in \mathcal{R}_f$, then $g(x) = x - kf(x)$
has $r$ as fixed point, for any $k \in \mathbb{R}$. Inversely, if $g$ has a
fixed point in $r$, then $r \in \mathcal{R}_f$. 

\begin{theorem}
    (1) Let $g \in C[a, b]$ and assume $g(x) \in [a, b]$ for all $x \in [a, b]$.
    Then there is a  fixed point of $g$ in $[a, b]$.

    (2) If, on top of previous conditions, $g$ is differentiable in $(a, b)$ and
    there is some $k < 1$ s.t. $\left| g'(x) \right| \leq k$ for all $x \in
    (a,b)$, then the fixed point referred in (1) is unique.
\end{theorem}

\begin{shaded}
    \begin{theorem}[Mean value theorem]
        Let $f : [a, b] \to \mathbb{R}$ continuous and differentiable on $(a,
        b)$ with $a < b$. Then there is some $c \in (a, b)$  s.t. 

        \begin{equation*}
            f'(c) = \frac{ f(b) - f(a) }{b - a}
        \end{equation*}
    \end{theorem}

    The interpretation is simple: consider the line secant to $f$ on $a, b$. The
    theorem ensures that there is some point $c$ s.t. the line tangent to $c$ is
    parallelt to said secant (equal slopes).
\end{shaded}

\small
\begin{quote}

\textbf{Proof.} (1) If $a$ or $b$ are fixed points the proof is done so assume
otherwise. Since $g(x) \in [a, b]$,  we have $g(a) > a$ and $g(b) < b$.


Take $\varphi(x) = g(x) - x$, which is continuous and defined in $[a,
b]$. Then 

\begin{equation*}
    \varphi(a) = g(a) - a > 0, \qquad \varphi(b) = g(b) - b < 0
\end{equation*}

Then $\varphi(a)\varphi(b) < 0$. Then, by the intermediate value theorem,
$\varphi$ has a root in $(a, b)$. In otherwords, there is at least one $p$ s.t. 

\begin{equation*}
    \varphi(p) = g(p) - p = 0
\end{equation*}

$\therefore $ $g(p) = p$ is a fixed point of $g$.

(2) Assume two distinct fixed points $p, q$ exist in $[a, b]$. The mean value
theorem ensures the existence of some $\zeta$ between $p, q$ (and thus in $[a,
b]$) s.t.t 

\begin{equation}
    g'(\zeta) = \frac{g(a) - g(b)}{a - b} \iff g'(\zeta)(a-b) = g(a) - g(b)
\end{equation}

By hypothesis, $\left| g'(x) \right| \leq k < 1$. Since $p, q$ are assumed to
be fixed points, equation $(1)$ gives: 

\begin{align*}
    \left| p - q \right| 
    &= \left| g(p) - g(q) \right| \\ 
    &= \left| g'(\zeta) \right| \left| p - q \right|  \\ 
    &\leq k \left| p-q \right|  < \left| p - q \right| 
\end{align*}

But this is absurd. The contradiction arises from assuming $p, q$ to be
distinct. Therefore, the fixed point is unique.


\end{quote}
\normalsize

The fixed point algorithm begins with an approximation $p_0$. Then, 

\begin{equation*}
    p_n = g(p_{n-1})
\end{equation*}

If $g$ continuous and the sequence converges, then it converges to a fixed
point, since: 

\begin{equation*}
    p := \lim_{n \to \infty} p_n = \lim_{n \to \infty} g(p_{n-1}) = g\left(
    \lim_{n \to \infty} p_{n-1} \right) = g(p)
\end{equation*}

\begin{align*}
&\textbf{Input: } p, M, \delta \\ 
&p_{\text{previous}} = p\\
&\textbf{for } i = 1 \textbf{ to } i = M \textbf{ do } \\ 
&\qquad p \leftarrow g(p) \\ 
&\qquad \textbf{if } \left| p - p_{\text{previous}} \right| < \delta \textbf{ then } \\ 
&\qquad\qquad \textbf{return } p \\ 
&\qquad\textbf{fi}\\ 
&\qquad p_{\text{previous}} = p\\ 
&\textbf{od}\\ 
&\textbf{return }p
\end{align*}


\begin{theorem}
    Let $g \in C[a, b]$ be a self-map of $[a, b]$ differentiable in $(a, b)$.
    Assume there is a constant $0 < k < 1$ s.t. $\left| g'(x) \right| \leq k$
    for all $x \in (a, b)$.

    For all $p_0 \in [a, b]$, the sequence $p_n = g(p_{n-1})$ converges to the
    unique f ixed point $p$ in $(a, b)$.
\end{theorem}


\small
\begin{quote}

\textbf{Proof.} The mean value theorem ensures that 

\begin{align*}
    \left| p_n - p \right| 
    &= \left| g(p_{n-1}) - g(p) \right| \\ 
    &= |g'(\zeta_n)||(p_{n-1} - p)| \\ 
    &\leq k \left| p_{n-1} - p \right| 
\end{align*}

with $\zeta_n \in (a, b)$. More succintly, with $e_n := p_n - p$,

\begin{equation*}
    |e_n| \leq k \left| e_{n-1} \right| \leq k \left| e_{n-2} \right| \leq
    \ldots \leq k \left| e_0 \right| 
\end{equation*}

By recurrence,

\begin{equation*}
    \left| e_n \right| \leq k^n\left| e_0 \right| 
\end{equation*}

Since $0 < k < 1$, $k^n \to 0$ when $n \to \infty$, which entails $\left| e_n
\right| \to 0$ when $n \to \infty$. It follows that $\left\{ p_n \right\} \to
p$ when $n \to \infty$.

\end{quote}
\normalsize

Now let us consider the error of this method. Take $p_n = p + e_n$ and consider
the Taylor expanssion of $g$ around $p$ evaluated at $p_n = p+e_n$:

\begin{equation}
    g(p_n) = g(p + e_n) = \sum_{i=1}^{m-1} \frac{ g^{(i)}(p) }{i!} e_n^{i} +
    \frac{f^{(m)}(\zeta_n)}{(n+1)!} e_{n}^{m}
\end{equation}

See that in $(2)$, $n$ corresponds to the iteration we are dealing with, and
thus $\zeta_n$ and $e_n$ depend on it. On the contrary, $m$ is the degree to
which we expand the series of $g$ around $p$ evaluated at $p_n$. We also assume
that $\zeta_n$ lies between $p_n$ and $p$.

By definition, $g(p_n) = p_{n+1}$ so $(2)$ is nothing but an expression for this
value. Assume $g^{(k)}(p) = 0$ for $k = 1, 2, \ldots, m-1$, but $g^{(m)}(p) \neq
0$. Then 

\begin{align*}
    e_{n+1} 
    &= p_{n+1} - p\\ 
    &=g(p_n) - g(p) \\ 
    &=\frac{ g^{(m)}\left( \zeta_n \right)  }{m!}e_n^m 
\end{align*}

More succintly,

\begin{equation*}
    e_{n+1} = \frac{ g^{(m)}\left( \zeta_n \right)  }{m!}e_n^m 
\end{equation*}

Then 

\begin{equation*}
    \lim_{n \to \infty} \left| \frac{e_{n+1}}{e_n^m} \right|  = 
    \frac{\left| g^m(p) \right| }{m!}
\end{equation*}

which is a constant. In conclusion, if the derivatives of $g$ are null in $p$ up
to the order $m-1$, the method as an order of convergence of at least $m$. Three
results follow from this fact. 

\section{Excercises}























\end{document}


