\documentclass[a4paper, 12pt]{article}

\usepackage[utf8]{inputenc}
\usepackage[T1]{fontenc}
\usepackage{textcomp}
\usepackage{amssymb}
\usepackage{newtxtext} \usepackage{newtxmath}
\usepackage{amsmath, amssymb}
\newtheorem{problem}{Problem}
\newtheorem{example}{Example}
\newtheorem{lemma}{Lemma}
\newtheorem{theorem}{Theorem}
\newtheorem{problem}{Problem}
\newtheorem{example}{Example} \newtheorem{definition}{Definition}
\newtheorem{lemma}{Lemma}
\newtheorem{theorem}{Theorem}
\usepackage{parskip}

\usepackage[top=1.5cm]{geometry}
% Define a new proof environment with smaller font
\newenvironment{proof}[1][Proof]{\par\small\noindent\textbf{#1:} }{\qed\par\normalsize}

\usepackage[cal=boondoxo]{mathalpha}

\begin{document}


\section{Godel numbering}

Let $\left\{ a_i \right\}_{i=0}^{n} $ a sequence of integers. Let 

\begin{equation*}
    a := \sum_{i=1}^{k} \mathcal{p}_i^{a_i + 1}
\end{equation*}

We see that $\left\{ a_i \right\} $ provides a pimre encoding of $a$,
insofar as $a$ can be retrieved from the sequence 
uniquely (fundametal theorem of arithmetic). This 
coding is injective but not surjective on $\omega$.

\section{Partial function enumeration}

To characterize all effectively calculable functions, we would 
like to produce a list $\mathcal{C} = \left\{ f_n \right\}_{n \in \omega} $
such that: 

\begin{itemize}
    \item The list includes all and only algorithmically computable functions 
    \item There is an efective listing of said list, i.e. an algorithmically
        computable function $g(n, x) = f_n(x)$. 
    \item Every $f_n$ is total.
\end{itemize}

\begin{theorem}
    Total partial functions are not enumerable.
\end{theorem}

\begin{proof}
    Assume $\mathcal{C}$ satisfies the conditions above. Let 
    $h(x) = g(x, x) + 1$. Since $h = \lambda x \left[ Suc \circ \left[ g \circ \left[ p_1^{1}, p_1^1 \right], p_1^1 \right]   \right] $, we know $h$ is a computable function. However, 
    we imposed the condition that $g(n, x) = f_n(x)$, which entails 
    $h(x) = f_x(x) + 1 \neq f_x(x)$.
\end{proof}


\section{Turing}


If $M = (Q, \Gamma, \mathcal{b}, \Sigma, \delta, q_1, q_0)$ is a Turing machine,
then the map $\delta$ is called a \textit{Turing program}. 
We simplify and assume $\Gamma = \Sigma = \left\{ \mathcal{b}, 1 \right\} $, 
so that we may refer to a Turing machine as 
$M = (Q, \delta)$, fixing $q_1, q_0$ as the start and 
halting states.

If $y$ is the number of $1$s on the tape when $M$ reachs 
$q_0$ with input $x$, and if $\varphi : \omega \to \omega$ is a 
number-theoretic function s.t. $\varphi(x) = y$,
we say $M$ computes $\varphi$.

\begin{definition}
    A Turing computation according to a Turing program $P$ with input $x$
    is a sequence of instantaneous description $c_0, \ldots, c_n$ such that 
    $c_0$ represents the machine in the starting state $q_1$ reading 
    the leftmost symbol of $x$, $c_n$ represents the machine in the 
    halting state $q_0$, and the transition $c_i \to  c_{i+1}$ 
    is given by the Turing program $P$.
\end{definition}

Thus, a \textit{computation} will always refer to a \textit{halting} or 
\textit{convergent} calculation. 

A partial function $\varphi(\overrightarrow{x})$ is represented 
by letting the input $\overrightarrow{x}$ be the initial configuration:

\begin{equation*}
    q_1 \alpha_1 \mathcal{b} \alpha_2 \ldots \mathcal{b} \alpha_n
\end{equation*}

where $\alpha_i$ consists of $x_i + 1$ repetitions of $1$.

\section{Basic results}

\subsection{Numbering Turing programs}

Let $\left\{ P_e \right\}_{e \in \omega} $ be an enumeration of all Turing 
programs, and assuming that $P_e$ is such that $e$
is a Gödel numbering. 

Let $\varphi_e^{(n)}$ be a partial function of $n$ variables 
computed by $P_e$, and let $\varphi_e$ denote $\varphi_e^{(1)}$.
We from now on identify each turing program $P_e$
with $\varphi_e$. 

\begin{theorem}
    Each partial computable function $\varphi_x$ has infinitely 
    many indices. Furthermore, for each $x$
    we can find a set $A_x$ of indices such that 
    $\varphi_y = \varphi_x$ for all $y \in A_x$.
\end{theorem}

\begin{proof}
    Given $\varphi_x$, consider the Turing machine $P_x$. If Turing program
    $P_x$ mentions only internal states $\left\{ q_0, \ldots, q_n \right\} $,
    add any amount of extraneous instructions of the form $q_{n+k}\mathcal{b}
    q_{n+k}\mathcal{b} R$ to create a new program for the same function.
\end{proof}

\section{Numbering Turing computations}

Let 

\begin{equation*}
    c = \overline{t} q_i s_1 \overline{s}
\end{equation*}

denote a Turing computation with $\overline{t} = t_{w}\ldots t_2t_1$,
$s_1$ the symbol being read, $\overline{s} = s_2\ldots s_v$.
Thus, the initial configuration is 

\begin{equation*}
    c_1 = q_1 s_1 s_2 \ldots s_{k+1} = q_1 s_1 \overline{s}
\end{equation*}

A Turing machine performs the transitions $c_1, \ldots, c_n$ in 
deterministic fashion. For every $c_i$, there is at most one 
consequent configuratoin $c_{i+1}$. This passing through an 
effective sequence of configurations allows for a coding 
with Gödel numbers as follows.

Once we have assigned numbers to each $s_i, q_j$, we can assign 
a number to every configuration using the prime power equation. 
In short, if $\mathcal{G}(\alpha)$, with $\alpha = q_j$ or $\alpha = s_j$,
denotes the numbering of a tape symbol or a state, we let 

\begin{equation*}
    \mathcal{G}(t_m\ldots t_1 q_i s_1 \ldots s_k) = \mathcal{p}_1^{\mathcal{G}(q_i) + 1} + \sum_{i=1}^{k} \mathcal{p}_{i+1}^{\mathcal{G}(s_i) + 1}
\end{equation*}

\section{Enumeration theorem and universal machines}

We know there is no enumeration of all total computable functions. However, 
the effective numbering for the syntax of Turing machines allows us to give an 
effective enumeration of all partial computable functions.

\begin{theorem}
    There is a partial computable function $\psi$ of two arguments such that 
    $\psi(e, x) = \varphi_e(x)$. Turing's thesis ensures there is 
    some $i$ such that $\varphi_i(e, x) = \psi(e, x)$.
\end{theorem}

\begin{proof}
    Let $M(e, x)$ be the Turing machine which computes $\psi(e, x)$ and 
    call it universal Turing machine. We provide an effective procedure which can, 
    by TT, be translated into a Turing program.

    
    \small
    \begin{quote}
    
        (1) Given input $(e, x)$, convert $e$ to $P_e$ by using the numbering of 
        Turing programs. 

        (2) Simulate $P_e$ on input $x$. The simulation begins with state 
        $q_1$ on the leftmost symbol of $x$ in standard input form. 
        If the simulation is in state $q_j$ reading symbol $s_k$, then 
        $M$ searches through the tuples (instructions) in $P_e$ until it finds 
        the indicated action on the simulation tape.
    
    \end{quote}
    \normalsize
    
\end{proof}

The enumeration theorem is crucial insofar as it guarantees the existence of a 
universal computational machine, capable of running any computable program.

\section{Parameter theorem}

\begin{theorem}[Parameter theorem]
    There is a bijective computable function $\mathcal{s}(x, y)$ such that,
    for all $x, y, z \in \omega$,

    \begin{equation*}
    \varphi_{\mathcal{s}(x, y)}(z) = \varphi_x(y, z)
    \end{equation*}
\end{theorem}

The theorem specifies that, given the Godel numbering $x$ of a function 
$\varphi_x$ of two parameters, we can fix the first parameter $y$ and computably 
find $\mathcal{s}(x, y)$ satisfying that 

\begin{equation*}
    \varphi_{\mathcal{s}(x, y)}(z) = \varphi_x(y, z)
\end{equation*}

\begin{example}
    Let $\varphi_x(y, z) = y + z$. Fix $y = 3$ and consider the resulting function 
    $f(z) = \varphi_x(3, z) = 3 + z$. Now $f(x)$ is computable and must have 
    some index $i$ such that $\varphi_i(z) = f(z)$. The parameter theorem 
    shows that we can pass computably from $x$ and the parameter 
    $y$ to such an index $i$ for $f(z)$.
\end{example}

\begin{proof}
    Let $x$ be the Godel numbering of a function $\varphi_x$ of two parameters,
    and $P_x$ be the Turing program which computes said function. 
    Let $\mathcal{S}_y$ be a Turing machine which takes as input a 
    Turing program $P$, a number $z$, and computes $P$ with input 
    $(y, z)$. Then clearly, if $y$ is held constant, 
    $\mathcal{S}_y$ with input $P_x$ and $z$ computes $\varphi_x(y, z)$.
    Clearly, $\mathcal{S}_y$ corresponds to a partial computable function
    with one parameter, and it depends directly on $x$ and $y$;
    i.e. $\mathcal{S}_y$ corresponds to a partial function $\varphi_{\mathcal{s}(x, y)}(z)$.


\end{proof}

\begin{proof}[Alternative proof]
    An alternative proof comes from von Neumann's paradigm. Let $\mathcal{Q}$ be a
    program which computes $f(y, z)$. Let $\mathcal{P}_y$ be the program 
    which, on input $z$, sets $N 1 \leftarrow y$, treating 
    $y$ as a constant, and $N 2 \leftarrow z$. Then, evidently, the 
    concatenation $\mathcal{P}_y \mathcal{Q}$ computes $f(y, z)$.
    Which entails 

    \begin{equation*}
        \Psi_{\mathcal{Q}}(y, z) \simeq \Psi_{\mathcal{P}_y\mathcal{Q}}(z) ~ \blacksquare
    \end{equation*}
\end{proof}

\begin{theorem}[Unbounded search]
    If $\theta(x, y)$ is a partial computable function, and  

    \begin{equation*}
        \psi(x) = (\mu y) \left[ \theta(x, y)\downarrow = 1 \land \left( \forall z < y \right) \left[ \theta(x, z) \downarrow \neq 1 \right]  \right] 
    \end{equation*}

    then $\psi(x) = y$ is a partial computable function.
\end{theorem}

\begin{proof}
    For a fixed $x_0$, compute $\theta(x_0, y)$ as follos. For fixed $s$,
    compute $s$ steps for each $y \leq s$ and then proceed to step 
    $s + 1$. Continue until (if ever) the first $y$ is found 
    such that $\theta(x, y)\downarrow = 1$. Output $\psi(x) = y$.
\end{proof}

\subsection{Unsolvable problems}

\textbf{Convention.} If $R \subseteq \omega^n$, then $R$ has 
the property $P$ if the set 
$\left\{ \left<x_1,\ldots,x_n \right> : R(x_1, \ldots, x_n)\right\} $ has 
property $P$, such as being computable.

\begin{definition}
    A set is computably enumerable if it is the domain of a partial computable function.
\end{definition}

Let 

\begin{equation*}
    W_e := \mathcal{D}_{\varphi_e} = \left\{ x : \varphi_e(x)\downarrow \right\} 
\end{equation*}

Any computable set $A$ is computably enumerable, since if $A$ is computable then 
$A = \mathcal{D}(\chi_A)$. A theorem establishes that a non-empty set is 
computably enumerable iff it is the range of a computable function, 
i.e. if there is an algorithm for listing its members. 

Though all computable sets are computably enumerable, not all computably 
enumerable sets are computable. 

\begin{definition}
    Let $K := \left\{ x : \varphi_x(x) \text{ is defined} \right\} = \left\{ x : x \in W_x \right\} $.
\end{definition}

\begin{theorem}
    $K$ is computably enumerable.
\end{theorem}

\begin{proof}
    The enumeration theorem guarantees that 
    $\varphi_x(x) = \psi(x, x)$ for some partial computable function $\psi$,
    and $K$ is the domain $\theta(x) = \psi(x, x)$.
\end{proof}

\begin{theorem}
    $K$ is not computable.
\end{theoremm}

\begin{proof}
    
$W_x$ is not computable. If $W_x$ had a computable characteristic function, the
following function would be computable:


\begin{equation*}
    f(x) = \begin{cases}
        \varphi_x(x) + 1 & x \in W_x \\ 
        0 & x \not\in W_x
    \end{cases}
\end{equation*}

However, $f$ cannot be computable because $f \neq \varphi_x$ for every $x$.
\end{proof}

\begin{definition}
    We define $K_0 := \left\{ \left<x, y \right> : x \in W_y \right\} $.
\end{definition}

\begin{theorem}
    $K_0$ is computably enumerable.
\end{theorem}

\begin{proof}
$K_0$ is a set of encodings of $(x, y)$ pairs s.t. $x$ is in the domain 
of $\varphi_y$. We know $\varphi_y(x) = \psi(y, x)$
due to the enumeration theorem, where $\psi$ is computable. 
Defining $\theta(\left<x,y \right>) = \psi(y, x)$, we ensure 
that $\theta(\left<x,y \right>)$ is defined iff $\psi(y, x) $ is defined,
which happens iff $x \in \mathcal{D}_{\varphi_y}$.

$\therefore $ $K_0 = \mathcal{D}_{\theta}$.

\end{proof}

\begin{theorem}
    $K_0$ is not computable.
\end{theorem}

\begin{proof}
    Note that $x \in K$ iff $\left<x, x \right> \in K_0$. Thus, if $K_0$
    had a computable characteristic function, so would 
    $K$.
\end{proof}

The halting problem consists precisely of deciding, for arbitrary $x$ and $y$,
whether $\varphi_y(x)$ converges, i.e. whether $\left<x,y \right> \in K_0$.
The previous theorem ensures the unsolvability of the halting problem.


\begin{definition}
    We write $A \leq_m B$ to say there is a computable $f$ such that 
    $f(A) \subseteq B$ and $f(\overline{A}) = \overline{B}$.
    In other words, if $x \in A$ iff $f(x) \in B$.
\end{definition}

If said $f$ is a bijection, we write $A \leq_1 B$.

\begin{definition}
    \begin{itemize}
        \item We say $A \equiv_m B$ if $A \leq_m B$ and $B \leq_m A$,
            and equivalently for $A \equiv_1 B$.
        \item $\text{deg}_m(A) = \left\{ B : A \equiv_m B \right\} $,
            and equivalently for $\text{deg}_1(A)$.
    \end{itemize}
\end{definition}

Observe that $\equiv_m, \equiv_1$ define an equivalence relation.
Their equivalence classes are called the $m$-degrees
and $1$-degrees, respectively.

\begin{theorem}
    If $A \leq_m B$ and $B$ is computable, then $A$
    is computable.
\end{theorem}

\begin{proof}
    $\chi_B$ denotes the belonging to $B$. However, for any $a \in A$,
    we know $a \in A$ iff $f(a) \in B$. So $\chi_A(x) = \chi_B\left( f(x) \right) $.
\end{proof}

If an unsolvable problem $A$ can be reduced to another problem 
$B$, then $B$ is also unsolvable. 

\begin{theorem}
    $K \leq_1 \left\{ x : \varphi_x \text{ is total} \right\} $
\end{theorem}

\begin{proof}
    Let $\mathcal{T} := \left\{ x : \varphi_x \text{ is total} \right\} $. Define 

    \begin{equation*}
        \psi(x, y) = \begin{cases}
            1 & x \in K \\ 
            \text{undefined} &\text{otherwise}
        \end{cases}
    \end{equation*}

    Clearly, $\psi$ is partially computable because the program to compute 
    $\psi(x, y)$ says: attempt to compute $\varphi_x(x)$: if this fails 
    to converge, output nothing; if it converges, then output 1 for every 
    argument $y$.

    The parameter theorem ensures there is a bijective computable function 
    $f$ such that 

    \begin{equation*}
        \psi(x, y) = \varphi_{f(x)}(y)
    \end{equation*}

    Choose $e$ such that $\varphi_e(x, y) = \psi(x, y)$, and define 

    \begin{equation*}
        f(x) = \lambda x \left[ s_1^1\left( e, x \right)  \right] 
    \end{equation*}

    Clearly, $f$ s one to one. Note that

    \begin{itemize}
        \item $x \in K \rightarrow\varphi_{f(x)} = \lambda y\left[ 1 \right] \Rightarrow \varphi_{f(x)} \text{ total} \Rightarrow f(x) \in \mathcal{T}$.
    \end{itemize}

\end{proof}





















\end{document}



