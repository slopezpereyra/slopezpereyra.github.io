\documentclass[a4paper, 12pt]{article}

\usepackage[utf8]{inputenc}
\usepackage[T1]{fontenc}
\usepackage{textcomp}
\usepackage{amssymb}
\usepackage{newtxtext} \usepackage{newtxmath}
\usepackage{amsmath, amssymb}
\newtheorem{problem}{Problem}
\newtheorem{example}{Example}
\newtheorem{lemma}{Lemma}
\newtheorem{theorem}{Theorem}
\newtheorem{problem}{Problem}
\newtheorem{example}{Example} \newtheorem{definition}{Definition}
\newtheorem{lemma}{Lemma}
\newtheorem{theorem}{Theorem}


\begin{document}

La palabra \textit{anarquismo} es imposible de definir. Comprende una rica
tradición intelectual, que retrocede a ilustración; movimiento obreros y
sindicales no del todo idénticos, como el extinto movimiento anarquista argentino, el
socialismo libertario en España, y el anarco-sindicalismo norteamericano; una
filosofía de vida y una perspectiva respecto de qué constituye una vida digna y
qué valores son fundamentales para la felicidad del género humano. 

~ 

Si a principios del siglo veinte los anarquistas eran (correctamente) considerados
un creciente peligro, y el movimiento anarquista era temido y perseguido, los
anarquistas contemporáneos son burlonamente tachados de soñadores sin causa
y desestimados como sectarios e impotentes. Esto es particularmente cierto en
Argentina, donde la historia del movimiento obrero anterior al siglo 1946 fue
borrada por el peronismo. 

~

Sin embargo, considero que el \textit{core} del pensamiento anarquista es
extremadamente valioso, que ser anarquista no implica necesariamente ser incapaz
de pragmatismo, y que todos debiéramos ser al menos un poco más anarquistas en
nuestras mentes y nuestros corazones. El segundo punto está probado si
consideramos algunos anarquistas contemporáneos cuya distancia de la impotencia
no podría ser mayor; e.g. Chomsky. Me propongo dar argumentos en favor del
primer y el segundo puntos.

~ 

\textbf{El anti-dogmatismo}. Rudolf
Rocker escribió:


\begin{quote}

Anarchism is no patent solution for an human problems,
no Utopia of a perfect social order, as it has so often been
called, since on principle it rejects all absolute schemes
and concepts. It does not believe in any absolute
truth, or in definite final goals for human development

\end{quote}

Al rechazar \textit{en principio} todo esquema absoluto, necesariamente debe
considerarse opuesto al personalismo que tan lamentablemente contamina el
pensamiento político argentino. Debe abstenerse de dar una receta universal
respecto a cómo resolver los problemas políticos, y considerar cada evento de la
historia de manera relativa.

\begin{quote}
Anarchism recognizes only the relative significance of
ideas, institutions, and social forms. It is, therefore,
not a fixed, self-enclosed social system, but rather a
definite trend in the historic development of mankind,
which, in contrast with the intellectual guardianship of
all clerical and governmental institutions, strives for the
free unhindered unfolding of all the individual and social
forces in life. 
\end{quote}

\textbf{El socialismo libertario}. Es un hecho casi auto-evidente que la
libertad de todo individuo es insegura en la medida en que la libertad de alguno
lo sea. Una sociedad en la que todos los individuos son libres es parte del
interés individual de cada uno, incluso de aquellos que momentáneamente puedan
beneficiarse del sometimiento de otros. Pero además de convenir a mi beneficio,
el sentir que la libertad es un bien es algo natural, y desear el bien a nuestro
prójimo también. 

~

Todos los días, consideramos normal tratar a nuestras familias y amigos de
manera desinteresada y solidaria, mientras consideramos que esta inclinación
debiera ser intercambiada por un egoísmo bruto en el momento en que
interactuamos con otros en una esfera social más amplia. 













\end{document}



