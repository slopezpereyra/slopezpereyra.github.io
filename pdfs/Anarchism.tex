\documentclass[a4paper, 12pt]{article}

\usepackage[utf8]{inputenc}
\usepackage[T1]{fontenc}
\usepackage{textcomp}
\usepackage{amssymb}
\usepackage{newtxtext} \usepackage{newtxmath}
\usepackage{amsmath, amssymb}
\newtheorem{problem}{Problem}
\newtheorem{example}{Example}
\newtheorem{lemma}{Lemma}
\newtheorem{theorem}{Theorem}
\newtheorem{problem}{Problem}
\newtheorem{example}{Example} \newtheorem{definition}{Definition}
\newtheorem{lemma}{Lemma}
\newtheorem{theorem}{Theorem}


\begin{document}

La palabra \textit{anarquismo} es imposible de definir. Comprende una rica (y a
mi entender hermosa) tradición intelectual que retrocede a la ilustración y al
liberalismo clásico. Comprende diversos movimientos obreros y sindicales nunca
idénticos entre sí, como el extinto movimiento anarquista argentino, el
socialismo libertario en España, y el anarco-sindicalismo norteamericano.
Comprende una filosofía de vida, una postura clara, pero no dogmática, respecto
de qué constituye una vida digna y qué valores son fundamentales para la
felicidad del género humano.

~ 

Si a principios del siglo veinte los anarquistas eran (correctamente) considerados
un creciente peligro, y el movimiento anarquista era temido y perseguido, los
anarquistas contemporáneos son burlonamente tachados de soñadores sin causa
y desestimados como sectarios e impotentes. Por lo menos, esto es así en
Argentina, donde la historia del movimiento obrero anterior a 1946 fue borrada
por el peronismo a fuerza de sangre, censura y propaganda.

~

Sin embargo, considero que el \textit{core} del pensamiento anarquista es
extremadamente valioso, que ser anarquista no implica necesariamente ser incapaz
de pragmatismo, y que todos debiéramos ser al menos un poco más anarquistas en
nuestras mentes y nuestros corazones. El segundo punto está probado si
consideramos algunos anarquistas contemporáneos cuya distancia de la impotencia
no podría ser mayor; e.g. Chomsky. Me propongo dar argumentos en favor del
primer y el segundo puntos.

~ 

Rudolf Rocker escribió con precisión:


\begin{quote}

Anarchism is no patent solution for an human problems,
no Utopia of a perfect social order, as it has so often been
called, since on principle it rejects all absolute schemes
and concepts. It does not believe in any absolute
truth, or in definite final goals for human development

[El anarquismo no es una solución patente a todos los problemas humanos, 
ni la utopía de un orden social perfecto, como frecuentemente se lo ha llamado, 
pues rechaza en principio todos los esquemas y conceptos absolutos. No cree en
ninguna verdad absoluta, ni en objetivos definitivos para el desarrollo humano.]

\end{quote}

Al rechazar \textit{en principio} todo esquema absoluto, necesariamente debe
considerarse opuesto al personalismo que tan lamentablemente contamina el
pensamiento político argentino. Debe abstenerse de dar una receta universal
respecto a cómo resolver los problemas políticos, y considerar cada evento de la
historia de manera relativa. 

\begin{quote}
Anarchism recognizes only the relative significance of
ideas, institutions, and social forms. It is, therefore,
not a fixed, self-enclosed social system, but rather a
definite trend in the historic development of mankind,
which, in contrast with the intellectual guardianship of
all clerical and governmental institutions, strives for the
free unhindered unfolding of all the individual and social
forces in life. 

[El anarquismo reconoce sólo la significancia relativa de las ideas,
instituciones, y formas sociales. Es, por lo tanto, no un sistema social 
cerrado, sino más bien una corriente definida en el desarrollo histórico de 
la humanidad que, en contraste con la tutela intelectual de todas las
instituciones clericales y gubernamentales, lucha por el desenvolvimiento 
libre e inobstaculizado de las fuerzas sociales e individuales de la vida.]
\end{quote}

En este sentido, el anarquismo es una expresión pura de dos inclinaciones
intelectuales que considero fundamentales: el pragmatismo y el anti-dogmatismo.
Para un anarquista, es imposible adorar a un líder, un dogma, o abrazar un
determinismo puro. La pregunta fundamental es \textit{cuáles son los hechos}:
sólo a partir de un análisis lo más objetivo posible de ellos puede desprenderse
la práctica. Un anarquista consecuente siempre está dispuesto a cambiar de
opinión y no entretiene certeza absoluta respecto de nada.

~

Por otra parte, en la medida en que el anarquismo comprenda un amor irrestricto
por la libertad, es evidente que un anarquista consecuente no puede ni debe
transar con ninguna forma de tiranía. Esto es una terquedad justificada: o bien
la política puede hacerse sin recurrir a la tiranía, o no debe hacerse en
absoluto. 

~

En los tiempos que corren, particularmente en Argentina, es importante notar que
\textit{ninguna} forma de tiranía significa precisamente eso: \textit{ninguna}.
La razón por la cual es importante señalarlo es que el nombre de
libertarianismo, que alguna vez perteneció al socialismo libertario y el
anarquismo, fue usurpado por una corriente que no sólo no se opone a la tiranía
del capital privado, sino que se somete servilmente a ella. La tiranía del
Estado es un problema fundamental en todo el mundo, pero en países democráticos
el público ejerce por lo menos un mínimo grado de influencia sobre el gobierno.
Por otro lado, en todas partes sucede que el capital privado es inescrutable y
imposible de influenciar. Si un liberal clásico como Smith o Rousseau observaran
cuán concentrado está el poder en unas pocas manos privadas, sin duda se les
revolvería el estómago. La oposición fundamental es oposición al poder
concentrado: si éste recae sobre manos privadas o estatales es, en términos de
principio, indiferente.


~

\textbf{El socialismo libertario}. Es un hecho casi auto-evidente que la
libertad de todo individuo es insegura en la medida en que la libertad de alguno
lo sea. Una sociedad en la que todos los individuos son libres es parte del
interés individual de cada uno, incluso de aquellos que momentáneamente puedan
beneficiarse del sometimiento de otros. Pero además de convenir a mi beneficio,
el sentir que la libertad es un bien es algo natural, y desear el bien a nuestro
prójimo también. 

~

Todos los días, consideramos normal tratar a nuestras familias y amigos de
manera desinteresada y solidaria, mientras consideramos que esta inclinación
debiera ser intercambiada por un egoísmo bruto en el momento en que
interactuamos con otros en una esfera social más amplia. 













\end{document}



