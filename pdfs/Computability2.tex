\documentclass[a4paper, 12pt]{article}

\usepackage[utf8]{inputenc}
\usepackage[T1]{fontenc}
\usepackage{textcomp}
\usepackage{amssymb}
\usepackage{newtxtext} \usepackage{newtxmath}
\usepackage{amsmath, amssymb}
\newtheorem{problem}{Problem}
\newtheorem{example}{Example}
\newtheorem{lemma}{Lemma}
\newtheorem{theorem}{Theorem}
\newtheorem{problem}{Problem}
\newtheorem{example}{Example} \newtheorem{definition}{Definition}
\newtheorem{lemma}{Lemma}
\newtheorem{theorem}{Theorem}

\newenvironment{rcases}
  {\left.\begin{aligned}}
  {\end{aligned}\right\rbrace}

\begin{document}

\begin{titlepage}
   \begin{center}
       \vspace*{1cm}

       \textbf{Computability theory}

       \small
       \vspace{0.5cm}
        FAMAF - UNC
            
       \vspace{1.5cm}
       \footnotesize
       \textbf{SLP}
       \normalsize

       \vfill
            
            
     
   \end{center}
\end{titlepage}



\begin{document}

\section{Computational paradigms}


$Pro^{\Sigma}$ is a $( \Sigma \cup \Sigma_p )$-p.r. Since each $\mathcal{P} \in Pro^{\Sigma}$
corresponds to a $\Sigma$-recursive function, this entails 
$\mathcal{R}$ is $\Sigma$-p.r. Which means there is an 
enumeration $\varphi_1, \varphi_2, \ldots$ such that every 
$\Sigma$-recursive function $f$ satisfies that 
$f = \varphi_i$ for some $i \in \mathbb{N}$.

    
\section{Halting function}

Recall that we say that $\mathcal{P}$ halts with input $\overrightarrow{s}, \overrightarrow{p}$ in $t$ steps when:

\begin{equation*}
    S_{\mathcal{P}}^{t}(1, \overrightarrow{s}, \overrightarrow{p}) = \left( n(\mathcal{P}) + 1, \overrightarrow{x}, \overrightarrow{w} \right) 
\end{equation*}

for any $\overrightarrow{x}, \overrightarrow{w}$. Recall too that, if 
$\mathcal{P}$ halts with input $\omega^{n} \times \Sigma^{*m} $, we use 
$\Psi_{\mathcal{P}}^{n, m, \#}$ to denote the value of $N1$ in the halting 
state. This lead to our definition of $\Sigma$-computability, where 
we said that $f ~ (n, m, \#)$ is $\Sigma$-computable if and only if 
there is a program $\mathcal{P}$ such that 

\begin{equation*}
    f = \Psi_{\mathcal{P}}^{n, m, \#}
\end{equation*}

(and analogously for the alphabetic case). The computability of sets was 
analogously defined, with $S$ being $\Sigma$-computable if and only if 
$\chi_S^{\omega^{n} \times \Sigma^{*m}}$ is $\Sigma$-computable. 
With regards to $\Sigma$-enumerability, we say a set $S \subseteq \omega^{n} \times \Sigma^{*m} $ is $\Sigma$-enumerable if 
it is empty or if there is a functon $F : \omega \to \omega^{n} \times \Sigma^{*m} $
such that $I_F = S$ and $F_{(i)}$ is $\Sigma$-computable. In other words, 
if there are programs $\mathcal{P}_1, \ldots, \mathcal{P}_{n+m}$ 
such that 

\begin{equation*}
    S = \text{Im}\left[ \Psi_{\mathcal{P}_1}^{1, 0, \#},\ldots, \Psi_{\mathcal{P}_n}^{1, 0, \#}, \Psi_{\mathcal{P}_{n+1}}^{1, 0, *},\ldots, \Psi_{\mathcal{P}_{n+m}}^{1, 0, *} \right] 
\end{equation*}

and the domain of each $\Psi_{\mathcal{P}_}^{n, m, z}$ is $\omega$.

Given $n, m \in \omega$, we define: 

\begin{equation*}
    \text{Halt}^{n, m} = \lambda t \overrightarrow{x}\overrightarrow{\alpha}\mathcal{P} \left[ i^{n, m}(t, \overrightarrow{x}, \overrightarrow{\alpha}, \mathcal{P}) = n(\mathcal{P}) + 1 \right] 
\end{equation*}





\end{document}



