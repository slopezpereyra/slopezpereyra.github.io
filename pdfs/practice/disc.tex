\documentclass[a4paper, 12pt]{article}

\usepackage[utf8]{inputenc}
\usepackage[T1]{fontenc}
\usepackage{textcomp}
\usepackage{amssymb}
\usepackage{newtxtext} \usepackage{newtxmath}
\usepackage{amsmath, amssymb}
\newtheorem{problem}{Problem}
\newtheorem{example}{Example}
\newtheorem{lemma}{Lemma}
\newtheorem{theorem}{Theorem}
\newtheorem{problem}{Problem}
\newtheorem{example}{Example} \newtheorem{definition}{Definition}
\newtheorem{lemma}{Lemma}
\newtheorem{theorem}{Theorem}
\usepackage{parskip}

\begin{document}

\textit{(1)} How many linear codes of length $7$ with $8$ words are there?

\textit{(2)} How many linear codes of length $7$ with $32$ words are there?

\textit{(3)} How many cyclic codes of length $7$ with $8$ words are there?

\textit{(4)} How many cyclic codes of length $23.500.002$ with $8$ words are there?

~

\textit{(1)} We know $\overrightarrow{0} \in C$ for any linear code. Let us analyze how we can build 
a code satisfying the constraints.

Each time I add a vector $\alpha$ to the code, I must add all of its linear combinations.
Thus, any set of words $\left\{ \alpha_1, \alpha_2, \ldots, \alpha_k \right\} $ "generates"
a code 

\begin{align*}
    C = \left\{ \overrightarrow{0} \right\} \cup \left\{ \text{All non-null linear combinations of } \alpha_1, \ldots, \alpha_k \right\} 
\end{align*}

It is simple to observe that there are $2^k$ linear combinations of $k$
vectors. But this includes the one that is zero. So any set of $k$ words
generates $2^{k} - 1$ words.

So, if we want to create a code with $8$ words, we must choose $3$ vectors because 
$2^3 - 1 = 7$ and there will be $7$ words plus $\overrightarrow{0} = 8$  words 
in the code. 

We are told that the length of the words is seven. So we must calculate 
in how many ways I can choose $3$ non-zero words of length $7$. Think about it 
and you'll see that there are 

\begin{align*}
    \frac{ \left( 2^7 - 1 \right) (2^7 - 2) \left( 2^7 - 3 \right)  }{3!}
\end{align*}

ways to do this. So this is the number of codes with $8$ words of 
length $7$.

\textit{(2)} Same reasoning tells we must chose $5$ words, becaue $2^5 = 32$.
There are 

\begin{align*}
    \frac{ \prod_{i=1}^{5} \left( 2^7 - i \right)  }{5!}
\end{align*}

ways to do this.


\textit{(3)} Cyclic codes are linear so what we said before still applies. But they also impose the 
following condition: if a word $w$ is in the code, all its rotations are in the code. It is 
very simple to observe that a word of length $n$ has $n$ rotations (including itself and 
excluding repetead words). So whenever we add a word of length $n$ to the code, we are in 
fact adding $n$ words.

Using this reasoning, if I add a single word of length $7$ to the code, I'm already adding 
$7$ words, which (counting the zero vector) already spans a code of $8$ words of length $7$.
There are $2^7 - 1$ words to chose from. So this is the number codes requested.

\textit{(4)} Using the same reasoning as in \textit{(3)}, there is no cyclic code with $8$
words of length 23.500.002.





























\end{document}



