\documentclass[a4paper, 12pt]{article}

\usepackage[utf8]{inputenc}
\usepackage[T1]{fontenc}
\usepackage{textcomp}
\usepackage{amssymb}
\usepackage{newtxtext} \usepackage{newtxmath}
\usepackage{amsmath, amssymb}
\newtheorem{problem}{Problem}
\newtheorem{example}{Example}
\newtheorem{lemma}{Lemma}
\newtheorem{theorem}{Theorem}
\newtheorem{problem}{Problem}
\newtheorem{example}{Example} \newtheorem{definition}{Definition}
\newtheorem{lemma}{Lemma}
\newtheorem{theorem}{Theorem}
\usepackage{parskip}

\begin{document}

We shall prove $3$-color is NP complete. In order to do this, we will prove 
$3$-SAT $\leq_{\rho}$ $3$-Color. In other words, given an instance of $3$-SAT 
of the form 

\begin{align*}
    B = \bigwedge{i=1}^{m} (l_{i1} \lor  l_{i2} \lor  l_{i3})
\end{align*}

with each literal $l_{ij}$ a case of the variables $x_1, \ldots, x_n$, we shall
provide an effective procedure that constructs a special graph $\mathcal{G}$
s.t. $\mathcal{G}$ is $3$-colorable iff $B$ is satisfiable.

\textit{(1 : Building $\mathcal{G}$)} We shall define $\mathcal{G}$ by parts; namely, 

\begin{enumerate}
    \item Two special vertices $s$ and $t$ that are connected.
    \item $n$ triangles, each connecting the vertices in $\left\{ t, v_{i}, w_{i} : 1 \leq i \leq n \right\} $
    \item $m$ triangles formed by the vertices $\left\{ b_{i1}, b_{i2}, b_{i3} : 1 \leq i \leq m \right\} $
    \item A tip $u_{ij}$ each connected to $b_{ij}$ and $s$.
\end{enumerate}

Now let us define 

\begin{align*}
    \psi(l_{ij}) = \begin{cases}
        v_{k} & l_{ij} = x_k  \\ 
        w_{k} & l_{ij} = \overline{x_k}
    \end{cases}
\end{align*}

Then we also include in $\mathcal{G}$ the sides $\left\{ u_{ij} ~ \psi(l_{ij} :
1 \leq i \leq m, 1 \leq j \leq 3) \right\} $. In other words, we connect each
tip $u_{ij}$ to either $v_k$ or $w_k$, depending on what the literal $l_{ij}$
is.

This completes the construction of $\mathcal{G}$. Now we shall prove 
$\mathcal{G}$ is 3-colorable iff $B$ is satisfiable.

\textit{(2 : Proving $\Rightarrow$)} Assume $\mathcal{G}$ has a proper coloring
of three colors or less. Since $\mathcal{G}$ contains triangles,
it must be a coloring of exactly three colors. We shall define 

\begin{align*}
    \overrightarrow{b_k} = \begin{cases}
        1 & c(v_k) = c(s) \\ 
        0 & c(v_k) \neq c(s)
    \end{cases}
\end{align*}

and prove that $B(\overrightarrow{b}) = 1$. Proving this equates to proving
there is at least one $j$ in $\left\{ 1, 2, 3 \right\} $ s.t.
$l_{ij}(\overrightarrow{b}) = 1$ for any arbitrary $i$. To prove this, we shall
take $u_{ij}$ and analyze what is color entails about the truth assignment.

The triangle $\left\{ b_{i 1}, b_{i 2}, b_{i 3} \right\} $ must contain $c(t)$
at some $b_{ij_0}$ fixed. Take $u_{ij_0}$. Note that $c(s) \neq c(u_{ij_0})
\neq c(t)$. And since $\psi(u_{ij_0})$ cannot have the color of $t$, it must be
the case that $c \left( \psi\left( u_{ij_0} \right)  \right) = c(s) $. Now consider 
these cases. 

\textit{Case 1.} If $\psi(u_{ij_0}) = v_k$, it follows that $l_{ij} = x_k$.j
Then $c(v_k) = c(s) \Rightarrow \overrightarrow{b_k} = 1 \Rightarrow l_{ij}(\overrightarrow{b}) = 1 $. $\therefore ~ B_i (\overrightarrow{b}) = 1$.

\textit{Case 2.} If $\psi(u_{ij_0}) = w_k$ then $l_{ij} = \overline{x_k}$. Since 
$c(w_k) = c(s)$ in this case, $c(v_k) \neq c(s)$ and $\overrightarrow{b}_k = 0$.
Then $l_{ij}(\overrightarrow{b}) = 1$. $\therefore  ~ B_i(\overrightarrow{b}) = 1$. 

In both cases, for an arbitrary $i$, the coloring of $\mathcal{G}$ allows us 
to define an assignment $ve^3$ that makes $B_i(\overrightarrow{b}) = 1$.
Of course, this assignment is s.t. $B(\overrightarrow{b}) = 1$. $\blacksquare$


\pagebreak


\textit{(3 : Proving $\Leftarrow$)} Assume $B$ is satisfiable by a boolean
vector $\overrightarrow{b}$. Then for any given $i$ in $[1, m]$ we have
$B_i(\overrightarrow{b}) = 1$. Then $l_{ij_{0}}(\overrightarrow{b}) = 1$ for
(at least) a fixed $j_0$, $1 \leq j_0 \leq 3$. 

Let $C = \left\{ 0, 1, 2 \right\} $ a set of colors and define $c(s) = 0, c(t) = 1$. Let 

\begin{align*}
    &c(v_k) = \begin{cases}
        c(s) & \overrightarrow{b}_k = 1 \\ 
        2    & \overrightarrow{b_k} = 0
    \end{cases} & c(w_k) = \begin{cases}
    2 &\overrightarrow{b_k} = 1 \\ 
    c(s) &\overrightarrow{b_k} = 0
    \end{cases}
\end{align*}

Clearly, $\left\{ s, t \right\} $ is properly colored and $\{t, v_i, w_i\}$ is 
properly colored. All that is left is to color the triangles 
with tips.

Let 

\begin{align*}
    c(u_{ij}) = \begin{cases}
        2 & j = j_0 \\ 
        c(t) & j \neq j_0
    \end{cases}
\end{align*}

Of course, each $\left\{ u_{ij}, s \right\} $ is properly colored. But what about 
$\left\{ u_{ij}, \psi(l_{ij}) \right\} $? Well, there are two cases to consider. 

If $j = j_0$, $c(u_{ij}) = 2$ and $l_{ij}(\overrightarrow{b}) = 1$. If
$\psi(l_{ij}) = v_k$, this means $x_{k}(\overrightarrow{b}) = 1 \Rightarrow
\overrightarrow{b_k} = 1$. Then $v_k$ is colored with $c(s) \neq c(u_{ij})$ and the coloring 
is correct. If $\psi(l_{ij}) = w_k$, entailing that $l_{ij} = \overline{x_k}$, then 
$\overrightarrow{b}_k = 0$ necessarily, in which case $c(w_k) = c(s) \neq c(u_{ij})$.

If $j \neq j_0$, then $c(u_{ij}) = c(t)$. But $\psi(l_{ij}) \in \left\{ v_k, w_k \right\} $ never
takes the color of $t$, and the coloring is correct.

All that is left is to color the triangle $\left\{ b_{i1}, b_{i2}, b_{i 3} \right\} $.
But this is trivial. Simply let $c(b_{i j_0}) = c(s)$, ensuring 
that $\left\{ b_{i j_0}, u_{i j_0} \right\} $ are properly colored,
and color the remaining two vertices with $c(t)$ and $2$ in 
any order.

We have used $\overrightarrow{b}$ to define a 3-coloring of $\mathcal{G}$. $\blacksquare$



































\end{document}



