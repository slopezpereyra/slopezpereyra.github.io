\documentclass[a4paper, 12pt]{article}

\usepackage[utf8]{inputenc}
\usepackage[T1]{fontenc}
\usepackage{textcomp}
\usepackage{amssymb}
\usepackage{newtxtext} \usepackage{newtxmath}
\usepackage{amsmath, amssymb}
\newtheorem{problem}{Problem}
\newtheorem{example}{Example}
\newtheorem{lemma}{Lemma}
\newtheorem{theorem}{Theorem}
\newtheorem{problem}{Problem}
\newtheorem{example}{Example} \newtheorem{definition}{Definition}
\newtheorem{lemma}{Lemma}
\newtheorem{theorem}{Theorem}
\usepackage{parskip}

\begin{document}

Sea $B$ instancia de $3$-sat. Construyamos el siguiente $3$-hiper grafo. Let $i
\in \left\{ 1, \ldots, n \right\}, j \in \left\{ 1, \ldots, m \right\}  $, $r
\in \left\{ 1, 2, 3 \right\}, k \in \left\{ 1\ldots m(n-1) \right\}  $.

\textbf{Vertices}. Define 

\begin{align*}
    X &= \left\{ a_{ij} \right\}  \cup \left\{ s_j \right\} \cup \left\{ h_k \right\} \\ 
    Y &= \left\{ b_{ij} \right\} \cup \left\{ t_j \right\} \cup \left\{ g_k \right\} \\ 
    Z &= \left\{ u_{ij} \right\} \cup \left\{ w_{ij} \right\} 
\end{align*}

Define 

\begin{align*}
    v_{jr} = \begin{cases}
        u_{ij} & \exists  i : l_{ir} = x_k \\ 
        w_{ij} & \exists i : l_{ir} = \overline{x_k}
    \end{cases}
\end{align*}

Let $E$ be the union of 

\begin{align*}
    E_0 &= \left\{ a_{ij}, b_{ij}, u_{ij} \right\} \\ 
    E_1 &= \left\{ a_{(i+1)j}, b_{ij}, w_{ij} \right\} \\ 
    E_2 &= \left\{ h_k, g_k, u_{ij} \right\} \cup \left\{ h_k, g_k w_{ij} \right\} \\ 
    E_3 &= \left\{ s_j, t_j, v_{jr} \right\} 
\end{align*}

Assume there is a perfect matching. Take an arbitrary $i$.
If there is some $j$ s.t. $a_{ij}$ belongs to the matching, 
then $a_{ij} \in E_0$ or $a_{ij} \in E_1$. If $a_{ij} \in E_0$,
this is if $\left\{ a_{ij}, b_{ij}, u_{ij} \right\} $ is in the matching, 
it cannot be the case that $a_{(i+1)j, b_{ij}, w_{ij}}$ is in the matching.
This inductively implies that now member of $E_1$ is in the matching 
and all members of $E_0$ are in the matching. Call this CASE 0. 

A similar reasoning reveals that if any side of $E_1$ belongs to 
the matching, all sides of $E_1$ belong to the matching and no 
side of $E_0$ belong to the matching. Call this CASE 1. 

Define 

\begin{align*}
    \overrightarrow{b_i} = \begin{cases}
        1 & \textbf{CASE 1 for }i\\
        0 & \textbf{CASE 0 for }i
    \end{cases}
\end{align*}

We must only prove there is some $l_{ir}$ s.t. $l_{ir}(\overrightarrow{b}) = 1$ 
for any $i$. There are two cases. 

If $l_{ir} = x_k$, then $v_{jr} = u_{ij}$. Then $\left\{ s_j, t_j, u_{ij} \right\} $ 
belongs to the matching. But then $\left\{ a_{ij}, b_{ij}, u_{ij} \right\} $
cannot belong to the matching. Then Case 1 holds 
and $l_{ir}(\overrightarrow{b}) = 1$.

If $l_{ir} = \overline{x_k}$, $v_{jr} = w_{ij}$. Then 
$\left\{ s_j, t_j, w_{ij} \right\} $ belongs to the matching. 
But then $\left\{ a_{(i+1) j}, b_{ij}, w_{ij} \right\} $
cannot belong to the matching and we are in case 0.
Then $l_{ir}(\overrightarrow{b}) = 1$.

In both cases, $l_{ir}(\overrightarrow{b}) = 1$.


$(\Leftarrow)$ Assume there is some $\overrightarrow{b}$ s.t. $B(\overrightarrow{b}) = 1$.
We will build a matching with edges $F_0 \cup \ldots\cup F_3$ such that 
$F_i \subseteq E_i$.

In particular, we let 

\begin{align*}
    F_0 = \left\{ \left\{ a_{ij}, b_{ij}, u_{ij} \right\} : b_i = 0  \right\} \\ 
    F_1 = \left\{ a_{(i+1)j, b_{ij}, w_{ij}} : b_i = 1 \right\} 
\end{align*}

Now, for any $i$ there is some $l_{ir}$ s.t. $l_{ir}(\overrightarrow{b}) = 1$.
I make 

\begin{align*}
    F_3 = \left\{ s_j, t_j, v_{jr} \right\} 
\end{align*}

If an edge belongs to $F_0$ and $F_3$, we must have $v_{jr} = u_{ij}$. But this
implies both $l_{jr} = x_i$ and $b_i = 0$, which is absurd. 

If an edge belongs to $F_1$ and $F_3$ then we must have 
$v_{jr} = w_{ij}$.But this implies $l_{jr} = \overline{x_k}$ and 
$b_i = 1$, which is absurd. 

So, we have a matching. But we must see that it is perfect.
For this purpose, we define: 

\begin{equation*}
    N = \left\{ z \in Z : z \text{ not covered by } F_0, F_1, F_2 \right\} 
\end{equation*}

Clearly, $|F_3| = m$. Let $p$ be the amount of $i$s s.t. $b_i = 0$
and $q$ the amount of $i$s s.t. $b_i = 1$. Clearly, $n = p + q$,
$|F_0| = mp, |F_1| = mq$. Then 

\begin{align*}
    |Z - N| = |F_0| + |F_1| + |F_3| = mp + mq + m = m(n+1)
\end{align*}

Then $|N| = |Z| - |Z - N| = 2mn - m(n+1) = m(n-1)$

Then there is a bijection $f : \left\{ 1, \ldots, m(n-1) \right\} : N $. 
We define 

\begin{align*}
    \left\{ g_k, h_k, f(k) \right\} 
\end{align*}

which has disjoint edges because it is injective and covers 
every uncovered $z$ because it is surjective. 

All sides of $Z$ are now covered and since $|X| = |Y| = |Z|$
the matching is perfect.

\end{document}



