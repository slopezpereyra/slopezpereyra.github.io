\documentclass[a4paper, 12pt]{article}

\usepackage[utf8]{inputenc}
\usepackage[T1]{fontenc}
\usepackage{textcomp}
\usepackage{amssymb}
\usepackage{newtxtext} \usepackage{newtxmath}
\usepackage{amsmath, amssymb}
\newtheorem{problem}{Problem}
\newtheorem{example}{Example}
\newtheorem{lemma}{Lemma}
\newtheorem{theorem}{Theorem}
\newtheorem{problem}{Problem}
\newtheorem{example}{Example} \newtheorem{definition}{Definition}
\newtheorem{lemma}{Lemma}
\newtheorem{theorem}{Theorem}


\begin{document}


Sabemos que $\chi(G) \geq 3$ si y solo si $G$ contiene un ciclo impar. Asuma
que $G$ contiene un ciclo impar. Sean $x_1, x_2, x_3, \ldots, x_{2k+1} \in V$ los
vértices del ciclo, tal que $x_i x_{i+1}$ es un lado para cada $i$ (con módulo, 
es decir $x_{2k+1} x_1$ es un lado). La regla de transformación que va de $G$ a
$\overrightarrow{G}$ nos dice que podemos dar una dirección a cada lado en $E$.

Como no existen caminos dirigidos de tres vértices o más en
$\overrightarrow{G}$, en  $\overrightarrow{G}$ sucede lo siguiente: Si
$\overrightarrow{x_i x_{i+1}}$ es un lado, entonces
$\overleftarrow{x_{i+1}x_{i+2}}$ es un lado. De otro modo, $\overrightarrow{x_i
x_{i+1}x_{i+2}}$ sería un camino de tres vértices. 

En $\overrightarrow{G}$, los vértices del ciclo pueden 
comenzar con $\overrightarrow{x_1 x_2}$ o con $\overrightarrow{x_2 x_1}$. De lo anterior se sigue que si 
se empieza con $\overrightarrow{x_1 x_2}$ es un lado, entonces todos los lados
del ciclo son de la forma 

\begin{align*}
    x_{2j - 1} \to x_{2j } \leftarrow x_{2j  + 1}, ~ j \in \mathbb{N}
\end{align*}

Pero sabemos que $x_{2k+1}x_1$ es un lado en $E$. Si $\overrightarrow{x_{2k+1}
x_1}$ es un lado en $\overrightarrow{G}$, resulta que
$\overrightarrow{x_{2k+1}x_1 x_2}$ es un lado de tres vértices. Si
$\overleftarrow{x_{2k+1}x_1}$ es un lado en $\overrightarrow{G}$, resulta que
$\overleftarrow{x_{2k}x_{2k+1}x_1}$ es un lado de $\overrightarrow{G}$. En
ambos casos terminamos con lados dirigidos de tres vértices. La contradicción
se sigue de asumir que $G$ contiene un ciclo impar. Y como no contiene un ciclo
impar su número cromático es $2$ $ \Rightarrow $ $G$ es bipartito.

La demostración para el caso en que, en $\overrightarrow{G}$, los vértices
del ciclo comienzan con $\overrightarrow{x_2 x_1}$ es análoga.

\pagebreak 

\begin{problem}
    Recordemos que $\mathbb{Z}_n$ denota $\left\{ 0, 1, \ldots, n - 1 \right\}
    $. Sea $G_{p, q}$ el grafo con vértices $v_{i, j}$ con $i \in \mathbb{Z}_p,
    j \in \mathbb{Z}_q$ y con lados $E = E_1 \cup E_2$, donde 

    \begin{align*}
        E_1 &= \left\{ v_{i, j} v_{i+1, j} : i \in \mathbb{Z}_p, j \in \mathbb{Z}_q  \right\}  \\ 
        E_2 &= \left\{ v_{i, j} v_{k, j + 1 } : i, k \in \mathbb{Z}_p, j \in \mathbb{Z}_q \right\} 
    \end{align*}

    Calcular $\chi(G_{p, q})$ para todo $p, q \geq 3$.
\end{problem}

\textit{Caso 1 : $q, p $ pares}. Considere los vértices del conjunto

\begin{align*}
    F_j := \left\{   v_{1, j}, v_{2, j}, \ldots, v_{p, j}  \right\}
\end{align*}

Llamaremos a este conjunto la $j$-écima fila. 

Por def. de $E_1$, los vértices de $F_0$ forman un ciclo. Pues $q$ par, forman
un ciclo par que requiere dos colores. Sean esos dos colores $0, 1$.

Ahora bien, cada vértice de $F_0$ está conectado con todos los vértices de
$F_{1}$, que a su vez constituye otro ciclo par con dos colores necesarios.
Luego, los dos colores necesarios para $F_{1}$ son distintos de $0, 1$;
digamos, $2, 3$.

Una vez en $F_3$, podemos volver a colorear el ciclo impar con $0, 1$, $F_4$
con $2, 3$, etc. En general, se puede dar un coloreo propio del tipo 

\begin{align*}
    c(v_{i, j}) = \begin{cases}
        i \mod 2 & j \text{ par } \\ 
        (i \mod 2) + 2 & j \text{ impar}
    \end{cases}
\end{align*}

Que este coloreo es propio es fácil de ver. Asuma que $v_{xy}, v_{wz}$ son
vecinos. Entonces o bien $v_{xy}v_{wz} \in E_1$ o bien $v_{xy}v_{wz} \in E_2$.
En el segundo caso, $z$ y $y$ no comparten la misma paridad y la función asigna
distintos colores a ellos. En el primer caso, los módulo de $x$ y $w$ sobre $2$
son diferentes. En particular, no hay conflicto entre $F_{q}$ y $F_1$ porque
$q$ y $1$ no comparten paridad.

~ 

\textit{Caso 2 : $q$ par, $p$ impar}. Puesto que $p$ impar, ahora resulta que $F_j$ es un ciclo 
impar para todo $j$ y necesita tres colores. El mismo razonamiento que en el caso 
anterior nos lleva a proponer

\begin{align*}
    c(v_{i, j}) = \begin{cases}
        i \mod 3 & j \text{ par } \\ 
        (i \mod 3) + 3 & j \text{ impar}
    \end{cases}
\end{align*}

Es decir, coloreamos las "filas" (los ciclos impares) con los colores $\left\{
0, 1, 2 \right\} $ en las filas pares, y con $\left\{ 3, 4, 5 \right\} $ en las
impares. Una vez más, como $q$ y $1$ no comparten paridad, no hay conflicto
entre $F_q$ y $F_1$.

\pagebreak 

\textit{(c)}   


\textit{(1)} $(A \cap  B) \times C = (A \times C) \cap (B \times C) $.

Considere el conjunto $S_1 = (A \cap B) \times C$. 

\begin{align*}
    S_1 = (A \cap B) \times C = \left\{ (x, y) : x \in A \cap B, y \in C \right\} 
\end{align*}

Ahora consideremos $S_2 = (A \times C) \cap (B \times C)$. El conjunto 
$A \times C$ son los pares $(a, c)$ con $a \in A, c \in C$; 
y el conjunto $B \times C$ son los pares $(b, c)$ con 
$b \in B, c \in C$. Se sigue que su intersección
resulta en los pares $(x, c)$ con $x \in A \cap B$. Es decir que 

\begin{align*}
    S_2 = (A \times C) \cap (B \times C) = \left\{ (x, y) : x \in A \cap B, y \in C \right\} = S_1
\end{align*}

Es decir que $(A \cap B) \times C = (A \times C) \cap  (B \times C)$.










































\end{document}



