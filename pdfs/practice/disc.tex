\documentclass[a4paper, 12pt]{article}

\usepackage[utf8]{inputenc}
\usepackage[T1]{fontenc}
\usepackage{textcomp}
\usepackage{amssymb}
\usepackage{newtxtext} \usepackage{newtxmath}
\usepackage{amsmath, amssymb}
\newtheorem{problem}{Problem}
\newtheorem{example}{Example}
\newtheorem{lemma}{Lemma}
\newtheorem{theorem}{Theorem}
\newtheorem{problem}{Problem}
\newtheorem{example}{Example} \newtheorem{definition}{Definition}
\newtheorem{lemma}{Lemma}
\newtheorem{theorem}{Theorem}
\usepackage{parskip}

\begin{document}

Let $G = (V, E)$ a bipartite graph with parts $X$ and $Y$, and let $Z \in
\left\{ X, Y \right\} $. We want to prove that there is a complete matching
from $X$ to $Y$ iff $\forall S \subseteq Z : |S| \leq |\Gamma(S)| $.

\textit{($\Rightarrow$)} Assume there is a complete matching from $X$ to $Y$.
Then there is an injective function $f : S \subseteq X \to Y$ s.t. any $x \in
S$ is associated to a distinct $y \in Y$ and $|f(S)| = |S|$. Since $G$ is
bipartite, any $f(S) \subseteq  \Gamma(S)$. Then $|S| \leq |\Gamma(S)|$. The
proof is equivalent if we take $S \subseteq Y$.

\textit{($\Leftarrow$)} We will use the matching algorithm to construct our
proof. Take $S \subseteq X$ and assume $|S| \leq |\Gamma(S)|$. Assume that,
after running the algorithm, an incomplete matching is found.

Let $S_0 \subseteq S$ be the set of rows tagged at the end of the algorithm.
Then all its neighbors $T_1 = \Gamma(S_0)$ were matched with a non-intersecting
set of rows $S_1$. On its turn, rows in $S_1$ may have neighbors that are not
matched, which inspires the definition of $T_2 = \Gamma(S_1) - T_1$.
In general, we define 

\begin{align*}
    S_i &= \Gamma(T_{i})\\
    T_{i+1} &= \Gamma(S_{i}) - \bigcup_{j=0}^{i} T_{i}
\end{align*}

This sequence of intermeddling sets, $S_{k}, T_k, S_{k-1}, T_{k-1}, \ldots, S_1,
T_1, S_0$, describes the run of the algorithm. Now, if the algorithm 
on $S_0$, it means there were no available neighbors for $S_0$; i.e. 
no way to construct the next $T_i$.

Observe that each row matches a unique column, and then $|S_i| = |T_i|$ for any
$i$. Furthermore, the $S_i$ are disjoint and the $T_i$ are disjoint. Then 

\begin{align*}
    |S| &= \sum |S_i| \\ 
        &= |S_0| + \sum |T_i| \\ 
        &= |S_0| + |T|
\end{align*}

It readily follows that $|S| > |T|$, because by assumption 
$S_0 \neq \emptyset$. Now, we will prove $T = \Gamma(S)$.

\textit{(1)} $T$ are the labeled columns, and each labeled column 
is labeled by a neighboring row. So any $t \in T$ has at least 
one neighbor in$ S$. This means $T \subseteq \Gamma(S)$.

\textit{(2)} Now assume there is some $y \in \Gamma(S)$ s.t. $y \not\in T$.
Then $y$ was not labeled. But since $y \in \Gamma(S)$ it 
could have been labeled by a neighboring row.
Then $y$ would have been labeled. Then $\Gamma(S) \subseteq T$.

Now that we know $\Gamma(S) = T$, we have $|S| > |\Gamma(S)|$.
This contradicts our assumption. Then there must 
be a complete matching.








\end{document}



