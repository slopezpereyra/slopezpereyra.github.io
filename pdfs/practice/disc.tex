\documentclass[a4paper, 12pt]{article}

\usepackage[utf8]{inputenc}
\usepackage[T1]{fontenc}
\usepackage{textcomp}
\usepackage{amssymb}
\usepackage{newtxtext} \usepackage{newtxmath}
\usepackage{amsmath, amssymb}
\newtheorem{problem}{Problem}
\newtheorem{example}{Example}
\newtheorem{lemma}{Lemma}
\newtheorem{theorem}{Theorem}
\newtheorem{problem}{Problem}
\newtheorem{example}{Example} \newtheorem{definition}{Definition}
\newtheorem{lemma}{Lemma}
\newtheorem{theorem}{Theorem}
\usepackage{parskip}

\begin{document}

I shall prove $3$-SAT is NP complete by proving SAT $\leq_\rho$ $3$-SAT. In order 
to do this, we are to provide an algorithm or effective procedure $\mathcal{A}$
that is capable of constructing an instance of $3$-SAT from an instance 
of SAT in polynomial time, and construct it in such a way that one if 
satisfiable if and only if the other one is satisfiable.  

Let 

\begin{align*}
    B = \bigwedge_{i=1}^{m} \left\{ l_{i 1} \lor  \ldots \lor l_{i r_i} \right\} 
\end{align*}

be an instance of SAT, where $r_i$ is the number of literals in the $i$th term 
of $B$. We shall define the following effective procedure $\mathcal{P}$: For any
arbitrary $B_i$, we shall construct $E_i$ as follows:

\begin{itemize}
    \item If $r_i = 1$, 
        $$E_i = \left( l_{i 1} \lor y_{ 1} \lor  y_{2} \right) \land \left\{
        l_{i 1} \lor  \overline{y_{ 1}} \lor  y_{ 2} \right\} \land \left(
    l_{i 1} \lor y_{ 1} \lor  \overline{y_{ 2}} \right) \land \left( l_{i 1}
\lor  \overline{y_{ 1}} \lor  \overline{y_{ 2}} \right)   $$
    \item If $r_{i} = 2$, 
        \begin{equation*}
            E_i = (l_{i 1} \lor  l_{i 2} \lor  y_{ 1}) \land  (l_{i 1} \lor  l_{i 2} \lor  \overline{y_{ 1}})
        \end{equation*}
    \item If $r_{i} = 3$, $E_i = B_i$.
    \item If $r_i \geq 4$, then 
        \begin{align*}
            E_i = &(l_{i 1} \lor  l_{i 2} \lor  y_{ 1}) \land (\overline{y_{ 1}} \lor  y_{ 2} \lor  l_{i 3}) \land (\overline{y_{ 2}} \lor  y_{ 3} \lor l_{i 4}) \\ 
                  &\land  \ldots \\ 
                  &\land (\overline{y_{ (r-4)}} \lor  y_{(r_i-3)} \lor  l_{i(r_i-2)}) \land (\overline{y_{ (r_i-3)}} \land l_{i(r_i-1)} \land l_{ir_i})
        \end{align*}
\end{itemize}

where each $y_{j}$ are new variables. We shall prove $E = \bigwedge_{i=1}^{m'}
E_i$ is satisfiable iff $B$ is satisfiable.

$(\Rightarrow)$ Assume $(\overrightarrow{b}, \overrightarrow{u})$ is an
assignment for the $x$, $y$ variables respectively s.t. $E(\overrightarrow{b},
\overrightarrow{u}) =1$. Then, for any arbitrary $E_i$, there is at least some
literal that evaluates to one under this assignment. Let us consider by cases. 

\textit{(1: $r_i = 1$)}. Assume $B_i(\overrightarrow{b})= 0$. Then $l_{i 1}(\overrightarrow{b}) = 0$. But since $E_i(\overrightarrow{b}, \overrightarrow{u}) = 1$, we must have 

\begin{equation*}
    (y_{ 1} \lor  y_{ 2}) \land  \ldots \land  (\overline{y_{ 1}} \lor  \overline{y_{ 2}})(\overrightarrow{u}) = 1
\end{equation*}

But it is trivial to see that any of the possible assignments makes some terms
true and others false simultaneously, which is a contradiction. So
$B_i(\overrightarrow{b}) = 1$.

\textit{(2 : $r_i = 2$)}. Assume $B_i(\overrightarrow{b}) = 0$. Since $E_{i}(\overrightarrow{b}, \overrightarrow{u}) = 1$ we must have 

\begin{equation*}
    [ y_{ 1} \land  y_{ 2} ](\overrightarrow{u}) = 1 \Rightarrow \bot
\end{equation*}

Then $B_i(\overrightarrow{b}) = 1$.

\textit{(3 : $r_i = 3$)}. Trivial. 

\textit{(4: $r_i \geq 4$)}. Assume $B(\overrightarrow{b}) = 0$. Then $E_i$ is
an expression of the form $y_{ 1} \land  (\overline{y_1} \lor  y_2) \land
(\overline{y_2} \lor  y_3) \land  \ldots \land  \overline{y_{r_i - 2}}$.
Necessarily, $y_1$ must be true, which entails $y_2$ must 
be true, which inductively entails $y_k$ is true for any $k$.
But then the last term is false. $( \bot )$

In all possible cases, $B_i(\overrightarrow{b}) = 1$ for any $i$. Then
$B(\overrightarrow{b}) = 1$.

$(\Leftarrow)$ Assume $\overrightarrow{b}$ is an assignment s.t.
$B(\overrightarrow{b}) = 1$. Take an arbitrary $B_i$. Since it is true under
$\overrightarrow{b}$, there is at least one fixed $j_0$ s.t.
$l_{ij_0}(\overrightarrow{b}) = 1$. Let us define the assignment $\overrightarrow{u}$
as follows:

\begin{align*}
    &u_1 = u_2 = \ldots = u_{j_0 - 2} = 1 \\
    &u_{j_0 - 1} = u_{j_0} = \ldots = u_{k} = 0 \\
\end{align*}

for all $k$ variables $y_1, \ldots, y_k$. We shall prove this assignment makes 
$E(\overrightarrow{b}, \overrightarrow{u}) = 1$. For this to occur, suffices 
that $E_i(\overrightarrow{b}, \overrightarrow{u}) = 1$. There are 
four possible cases.

\textit{(1: $r_i = 1$)}. In this case, each term in the series of conjuctions 
will either be anterior to the appearence of $l_{i j_0}$, posterior to it,
or will be the term with $l_{i j_0}$. If it is the term with $l_{ij_0}$ it 
will be true by assumption. If it is anterior it will contain at least 
one $y_{k}$, and by definition this will be true. If it is posterior 
it will contain at least one $\overline{y}_{i k}$ and it will be true.


\textit{(2 : $r_i = 2$)}. By def. of $E_i$, both terms contain all 
$l_{i j}$, so both terms will contain $l_{i j_0}$ and will 
be true.

\textit{(3 : $r_i = 3$)}. Trivial.

\textit{(4: $r_i \geq 4$)}. Observe that $E_i$ will be of the form 

\begin{align*}
    E_i &= (l_{i 1} \lor  l_{i 2} \lor  y_{ 1}) &\text{$y_{ 1}(\overrightarrow{u}) = 1$ by def} \\ 
        &=(\overline{y_{ 1}} \lor  y_{ 2} lor l_{i 3}) &\text{True for the same reason} \\ 
        &\vdots \\ 
        &=(\overline{y_{ (j_0 - 3)}} \lor y_{ j_0 - 2} \lor  l_{i (j_0 -1)})\\
        &=(\overline{y_{ (j_0 - 2)}} \lor y_{ ( j_0 - 1 )} \lor  l_{i j_0})\\
        &=(\overline{y_{ (j_0 - 1)}} \lor y_{ j_0} \lor  l_{i ( j_0 + 1)})\\
        &\vdots \\ 
        &=(\overline{y_{r_i - 2}} \lor  l_{i( r_i - 1 )} \lor  l_{ir})
\end{align*}

all true.







\end{document}



