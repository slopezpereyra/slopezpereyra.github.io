
\usepackage{newtxtext} \usepackage{newtxmath}
\usepackage{graphicx}
\graphicspath{ {../Images/} }
\usepackage[utf8]{inputenc}
\usepackage[T1]{fontenc}
\usepackage{textcomp}
\usepackage{amssymb}
\usepackage{amsmath, amssymb}
\newtheorem{problem}{Problem}
\newtheorem{example}{Example}
\newtheorem{lemma}{Lemma}
\newtheorem{theorem}{Theorem}
\newtheorem{problem}{Problem}
\newtheorem{example}{Example} \newtheorem{definition}{Definition}
\newtheorem{lemma}{Lemma}
\newtheorem{theorem}{Theorem}


\clearpage 

\section{Baby Brooks}

\begin{quote}
    \textbf{Notational note.} I use $\mathcal{G}$ to denote the Greedy 
    algorithm.
\end{quote}

We wish to prove that if $G = (V, E)$ is connected and non-regular, then
$\chi(G) \leq \Delta$.

Let $x_0 \in V$ be s.t. $d(x_0) = \delta$. Since $G$ is
connected, running BFS from $x_0$ adds all vertices to the BFS tree. Let
$\mathcal{O}^{-1}$ be the ordering of the vertices s.t. $z$ is the $i$th vertex
if it was the $i$th one to be added by BFS. Trivially, $x_0$ is the first
vertex in $\mathcal{O}^{-1}$. Let $\mathcal{O}$ be the reverse order, with
$x_0$ last. We will prove $\mathcal{G}$ colors $G$ with at most $\Delta$ colors
if it uses the ordering $\mathcal{O}$.


Observe that, in the BFS run, every $x \neq x_0$ is inserted by a neighbor that
was already in the tree. In other words, in the $\mathcal{O}^{-1}$ order, every
vertex has a neighbor that precedes him in the order. Consequently, in
$\mathcal{O}$, every $x \neq x_0$ has a neighbor that succeeds him in the
order. 

It follows that the worst case scenario for the coloring of $x \neq x_0$ is
that it has $d(x) - 1$ preceding neighbors. $\therefore $ $\mathcal{G}$
eliminates at most $d(x) - 1 \leq \Delta - 1$ colors. Then $x$ can be colored
with a color in $\left\{ 1, \ldots, \Delta \right\} $.

When $\mathcal{G}$ reaches $x_0$ it eliminates at most $d(x_0) = \delta$
colors. Since $G$ is non-regular $\delta < \Delta$.
$\therefore $ There is at least one color for $x_0$
in $\left\{ 1, 2, \ldots, \Delta \right\} $.

\end{quote}
\normalsize

\pagebreak

\section{Max flow, min cut}

Let $f$ a flow over a network $\mathcal{N}$. We want to prove two things:
\textit{(1)} $v(f) \leq Cap(S)$ for any cut $S$ and \textit{(2)} $f$ is maximal
iff there is a cut $S$ s.t. $v(f) = Cap(S)$.

\textit{(1)} We know $v(f) = f(S, \overline{S}) - f(\overline{S}, S)$. Since
$f(A, B)$ is a sum over $f$ and $0 \leq f(\overrightarrow{ab}) \leq
c(\overrightarrow{ab})$ for any $\overrightarrow{ab} \in E$, 

$$v(f) = f(S, \overline{S}) - f(\overline{S}, S) \leq f(S, \overline{S})$$

The same logic implies $f(S, \overline{S}) \leq c(S, \overline{S}) = Cap(S)$.
Then $v(f) \leq f(S, \overline{S}) \leq Cap(S)$. $\blacksquare$

\textit{(2: $\Leftarrow$)} Assume there is a cut $S$ s.t. $v(f) = Cap(S)$. Let
$g$ an arbitrary flow. Then $v(g) \leq Cap(S) = v(f)$. Then $f$ is maximal.
Furthermore, it is trivial by definition of $Cap$ that $Cap(T) \geq v(f)$ for
any cut $T$. Then $Cap(T) \geq Cap(S) \Rightarrow S$ is minimal.

\textit{(2 : $\Rightarrow$)} Assume $f$ is maximal. Let

$$
S = \left\{ s \right\} \cup \left\{ x \in V : \exists f\text{-camino
aumentante entre $s$ y $x$} \right\} 
$$

$S$ is a cut because, if $t \in S$, there is an augmenting path 
$s \ldots t$ and the flow can be augmented, which contradicts 
that $f$ is maximal.

Recall that $v(f) = f(S, \overline{S}) - f(\overline{S},S)$. The first 
term in the difference is

\begin{align*}
    f(S, \overline{S}) &= \sum_{x \in S, z \not\in S, \overrightarrow{xz} \in E}
    f(\overrightarrow{xz})
\end{align*}

Let $\overrightarrow{xz} \in E$ a side in the range of the sum above.
Then there is an augmenting path $s \ldots x$ and there is no 
augmenting path $s \ldots z$. But $\overrightarrow{xz} \in E$ and 
$s\ldots x \ldots z$ is a path. Since it cannot be an augmenting path,
we must have $f(\overrightarrow{xz}) = c(\overrightarrow{xz})$. 
Then $f(\overrightarrow{xz}) = c(\overrightarrow{xz})$ for all $x \in S$, $z
\not\in S, \overrightarrow{xz}\in E$. Therefore

\begin{align*}
    f(S, \overline{S}) = \sum_{\ldots} f(\overrightarrow{xz}) = \sum_{\ldots}
    c(\overrightarrow{xz}) = Cap(S)
\end{align*}

Now consider the second term in the difference:

\begin{align*}
    f(\overline{S}, S) = \sum_{w \not\in S, x \in S, \overrightarrow{wx} \in E}
    f(\overrightarrow{wx})
\end{align*}

Let $\overrightarrow{wx}$ an arbitrary side in the sum above. Again,
there must be an augmenting path $s \ldots x$, but not one 
$s \ldots w$. But $\overrightarrow{wx}$ is a side, and then 
$s \ldots \overleftarrow{xw}$ is not augmenting only if 
$f(\overleftarrow{xw}) = 0$. This means $f(\overrightarrow{wx}) = 0$
for all $\overrightarrow{wx}$ in the range of the sum above.

$\therefore $  $v(f) = Cap(S) - 0 = Cap(S)$. $\blacksquare$

\pagebreak

\section{Networks}

\subsection{Edmond-Karp: Complexity}

The complexity of E.K. is determined by the complexity of finding the
$f_i$-augmenting paths, the number of such paths, and the complexity 
of updating the flow with such paths. 

To find $f_i$-augmenting paths, EK uses BFS with a complexity $O(m)$. Updating
the flow has complexity $O(n)$. So the question is what is the number $\varphi$
of $f_i$ augmenting paths (or runs of BFS) that are used.

The number of such paths is bounded by the number of sides times the number of
times a side may become critical. We shall determine how many times a side may
become critical.

Let $f_1, f_2, \ldots$ be the iterations of EK. Let $\overrightarrow{xz}$ be a
side that became critical at iteration $k$. There are two options: it either
saturated being forward or emptied being backward.

\textit{Saturated being forward.} If $\overrightarrow{xz}$ saturated being
forward, then $p_k = s \ldots \overrightarrow{xz} \ldots t$ is the form of the
augmenting path used to determine $f_k$. 

Assume $\overrightarrow{xz}$ becomes critical again at some iteration $j$, so
that $p_j = s \ldots \overrightarrow{xz} \ldots t$ is an augmenting path. For
this to occur there must exist some iteration $i$, with $k < i < j$, such that
$p_i = s \ldots \overleftarrow{xz} \ldots t$ is an augmenting path. In other
words, $z$ must have returned some of the flow to $x$. Since we are using EK,
the augmenting paths are of minimal length, and the distance between two
vertices never decreases. This means 

\begin{align*}
    d_j(t) &\geq d_i(x) + b_i(x)  \\ 
           &=d_i(z) + 1 + b_i(x) \\ 
           &\geq d_k(z) + 1 + d_k(x) \\ 
           &=d_k(x) + 1 + 1 + b_k(x) \\ 
           &=d_k(t) + 2
\end{align*}

\textit{Emptied being backward.} Now, assume $\overrightarrow{xz}$ is a side
that saturated being backward on the $k$th iteration; this is, that $p_k = s
\ldots \overleftarrow{xz} \ldots t$ was the path found. Assume it empties again
at some iteration $j$, so that $p_j = s \ldots \overleftarrow{xz} \ldots t$ is
the $f_j$ path. Then there is some $i$, with $k < i < j$, such that $p_i = s
\ldots \overrightarrow{xz} \ldots t$ is a path; i.e. the side was used forward.
Now, all these paths are of minimal length, because we are using 
E.K. And in all succesive paths, the distance from $s$ to $t$ never 
decreases. Then,

\begin{align*}
    d_j(t) &\geq d_i(t) \\ 
           &= d_i(z) + b_i(z) \\ 
           &= d_i(x) + 1 + b_i(z) \\ 
           &\geq d_k(x) +1 + b_k(z) \\ 
           &=d_k(z) + 1 + 1 + b_k(z) \\ 
           &= d_k(t) + 2
\end{align*}

In both cases, the distance from $s$ to $t$ had to increase by at least 
two unites. But the distance from $s$ to $t$ is bounded by $n$; this is, 
it is $O(n)$. Then a side may become critical $O(\frac{n}{2}) = O(n)$ times.
Since there are $m$ sides, the number of times an arbitrary side will become 
critical is $O(mn)$.

$\therefore $ The complexity is $O(mn) \left[ O(n) + O(m) \right] = O(mn)O(m) = O(m^2n) $.


\pagebreak 

\subsection{Edmond-Karp: Augmenting paths are non-decreasing}

Let $A = \left\{ x \in V : d_{k+1}(x) < d_k(x) \right\} $ and
assume $A \neq \emptyset$. Let $x_0 \in A$ be the vertex whose distance
$d_{k+1}(x_0)$ from $s$ is minimal; i.e. $d_{k+1}(x_0) \leq d_{k+1}(y) ~
\forall y \in A$. Since $x_0 \in A$, $d_{k+1}(x_0) < d_k(x_0) \leq \infty$.
Then there exists a $\mathcal{P}_{k+1} = s \ldots x_0$ of minimal length.
Let $z$ be the predecessor of $x_0$ in this path.

By definition of $d_f$, the length of the path is $d_{k+1}(x_0)$. Because the
path is of minimal length to $x_0$, it is of minimal length to any predecessor
of $x_0$ in it, including $z$. Then $d_{k+1}(z) = d_{k+1}(x_0) - 1$. This
implies $z \not\in A$.

There are two possible cases: either $\overrightarrow{xz} \in E$
or $\overrightarrow{zx} \in E$.

\textit{(Case 1):} If $\overrightarrow{zx_0} \in E$, then
$d_{k+1}(z) < d_{k+1}(x_0)$. Since $z \not\in A$,

\begin{align*}
    d_k(z) \leq d_{k+1}(z) < d_{k+1}(x_0) < \infty
\end{align*}

Since $d_k(z) < \infty$, there is an $f_{k}$-augmenting path from $s$ to $z$.
Then, in principle, $s \ldots z x$ could be  an augmenting path. But
if this were the case, 

$$d_k(x_0) \leq d_k(z) + 1 \leq d_{k+1}(z) + 1 = d_{k+1}(x)$$ 

which implies $x_0 \not\in A$ ($\bot$). $s\ldots zx$ is not $f_k$-a.p. This can
only happen if $f_k(\overrightarrow{zx_0}) = c(\overrightarrow{zx_0})$ (the
side is saturated). Since the side is used in a $f_{k+1}$-a.p. it must be the
case that $f_{k+1}(\overrightarrow{zx_0}) < c(\overrightarrow{zx_0})$. This
means $\overrightarrow{zx}$ was used backwards in the $k$th iteration, or
rather that $f_k = s \ldots \overleftarrow{x_0z} \ldots t$.

Since this is Edmond-Karp, augmenting paths are of minimal length. Then 

\begin{align*}
    d_{k}(z) &= d_k(x_0) + 1   \\ 
             &> d_{k+1}(x_0) + 1 \\ 
             &=d_{k+1}(z) + 2 \\ 
             & \geq d_{k}(z) + 2
\end{align*}

which is absurd.

\textit{1.3.2} If  $\overrightarrow{xz}$ is a side then again

\begin{align*}
    d_k(z) \leq d_{k+1}(z) < d_{k+1}(x_0) < \infty
\end{align*}

Then there is an $f_k$-a.p. $s \ldots z$ and, at least in principle, 
$s \ldots \overrightarrow{zx}$ could also be augmenting. But if this were the case,
we would have 

$$
d_k(x_0) = d_k(z) + 1 \leq d_{k+1}(z) + 1 = d_{k+1}(x_0) 
$$

which would imply $x_0 \not\in A$ ($\bot$). Then $s \ldots \overrightarrow{zx_0}$ is 
not augmenting, which means the side $\overrightarrow{x_0z}$ cannot 
be used backwards. This can only be true if $f_k(\overrightarrow{x_0z} = 0$.
But since the side is used backwards in the $f_{k+1}$-augmenting path,
it must be used forward in this iteration. Which means $s \ldots \overrightarrow{x_0z} \ldots t$
is augmenting. But then 

\begin{align*}
    d_{k}(z) &= d_{k}(x_0) + 1 \\ 
             &> d_{k+1}(x_0) + 1 \\ 
             &= d_{k+1}(z) +2 \\ 
             &\geq d_{k}(z) + 2 
\end{align*}

which is absurd.

\textit{Conclusion.} In both cases a contradiction arises. The contradiction
comes from assuming $A \neq \emptyset$. Then $A = \emptyset$. $\blacksquare$


\pagebreak

\section{Dinitz: Complexity}


We will prove the complexity of Dinitz is $O(n^2m)$. 

Recall that Dinitz builds succesive auxiliary networks with BFS, uses DFS to find blocking 
paths in them, and uses these paths to update the flow. Since the level 
of $t$ augments in each succesive network, there are at most $O(n)$ auxiliary 
networks. So the complexity of Dinitz is 

\begin{align*}
    O(n) \left[ \text{Comp. of building A.N.} + \text{Comp. of finding blocking flow in A.N.} \right] 
\end{align*}

The complexity of building the A.N. is the complexity of BFS,  which is $O(m)$.
The process of finding the blocking path varies is the Western and the original
algorithms. 

\textit{Original version.} The original Dinitz algorithm enforced the following
invariant: In any given auxiliary network, all vertices have edges connecting
to the next level. This invariant implies that the DFS run always reaches $t$
without need to backtrack. In consequence, the complexity of the DFS run is
$O(n)$. The complexity of updating the flow is $O(n)$, and each time this happens 
at least one side is saturated and removed. Then there are $O(m)$ paths.
$\therefore $ The complexity of finding the paths and updating the flow
is $O(n \times m)$.

However, there is some extra cost associated to preserving this invariant. Each
time a path is found, saturated sides are removed with a prunning operation.
The prunning operation goes from the highest to the lowest levels in the
network, checks for vertices whose exiting edge is saturated, and deletes them.
This means that, for each vertex, a process of $O(1)$ complexity checks if it
is saturated. This happens each time a path is found, so $O(n \times m)$ times.
Once these sides have been detected, they and their neighbors are removed,
which has complexity $O(d(x))$. Since this happens at worst for all vertices,
the deleting operation is $O\left( \sum_{x \in V} d(x) \right) = O(m)$. The
total cost of enforcing the invariant is then $O(n \times m) + O(m)$, which
means the complexity of finding the blocking paths and updating the flow is
$O(n \times m) + O(n \times m) + O(m) = O(n \times m) $. Then the complexity of
Dinitz is $O(n^2m)$.

\textit{Western version.} The Western version finds blocking paths and updates 
the flow inside a while loop with three clauses. The first clause is 
running BFS to advance; i.e. setting $x = s$ and iteratively making 
$x$ be the first neighbor of $x$ until $t$ is reached. This clause executes 
as long as there is a neighbor and we call it $A$.

The second clause considers the case where $x$, the vertex we are 
traversing, has no neighbors in the next level. Then the algorithm
backtracks to its predecessor, removes the side leading from it to $x$,
and attempts $A$ once more. We call this part $R$.

The last clause considers the case where $t$ is reached. Here, 
the flow is augmented using the path found and all saturated 
sides are removed.

Thus, finding the blocking flow and updating the path can be modeled 
as a word $A \ldots I A \ldots R A \ldots R A\ldots I$; i.e. as a succession 
of words of the form $A \ldots X$ with $X \in \left\{ R, I \right\} $.

The process $R$ has complexity $O(1)$ because it simply involves going
backwards and deleting a side. The process $A$ has complexity $O(1)$ because it
simply involves moving forward, but at most $O(n)$ successions of $A$ may 
occur. This means $O(A \ldots A) = O(n)$. The process $I$ has complexity $O(n)$
because the flow is updated across at most all vertices. Then  
$O(A\ldots I) = O(n) + O(n) = O(n)$ and $O(A\ldots R) = O(n) + O(1) = O(n)$.
The question is how many words of the form $A \ldots X$ exist. At worst, all
paths are traversed in the process of finding $t$, so there are $O(m)$ words of
this form. Then the complexity of the run is 

\begin{align*}
    O(m) \left[ O(n) + O(n) \right]  = O(nm)
\end{align*}

Then the complexity of Dinitz is $O(n^2m)$.

\pagebreak

\subsection{Wave: Complexity}

We know the distance between $s$ and $t$ in auxiliary networks is increasing.
The distance is bounded by $n$. $\therefore $ There are $O(n)$ auxiliary
networks. Now, each auxiliary network is first constructed and then 
used to find a blocking flow. Then the complexity of Wave is 

\begin{align*}
    O(n) \left[ F + B \right] 
\end{align*}

where $F$ is the complexity of finding a blocking flow in an auxiliary network 
and $B$ the complexity of building an auxiliary network. To build the auxiliary 
network we use BFS. $\therefore  $ $B = O(m)$. Let us examine $F$.

To find blocking flows we attempt to balance in forward and backward waves.
When going forward, let $V$ be the steps where a side is saturated,
$\overline{V}$ those where a side is not saturated. When going 
backward, let $S$ be the steps where a side is emptied, $\overline{S}$ those 
there a side is not emptied. 

\textit{Complexity of $V$}. Assume $\overrightarrow{xz}$ is a side and is
saturated in a forward wave. To saturate again it must empty at least a little
bit before. If it empties, then $z$ was blocked and returned the flow to $x$;
and since $z$ is blocked, $\overrightarrow{xz}$ will never again be used. Then
each side can saturate at most once. $\therefore $ the complexity of $V$ is
$O(m)$.

\textit{Complexity of $S$}. Assume  $\overrightarrow{zx}$ is a side and 
it empties in a backward wave. Since $x$  returned flow, it 
is blocked and $z$ will not send flow to $x$ a gain. Then $\overrightarrow{zx}$
cannot be emptied ever again. $\therefore $ The complexity of $S$ is $O(m)$.

\textit{Complexity of $\overline{V}$}. When a vertex sends flow to 
its neighbors, it saturates all sides except perhaps one.
Then, for any given vertex, each forward wave will send 
partial flow through at most one side. $\therefore $ The 
complexity of $\overline{V}$ is $O(n) \times \varphi$, with 
$\varphi$ the number of forward waves.

\textit{Complexity of $\overline{S}$}. When a vertex returns flow 
to its predecessors, it empties all sides except perhaps 
one. Then, for any given vertex in a backward wave, 
at most one side is partially emptied. $\therefore $ The 
complexity of $\overline{S}$ is $O(n) = \psi$ where $\psi$ 
is the number of backward waves.

Now, it is trivialy to see $\varphi = \psi$ and there are 
at most $n$ forward waves. $\therefore ~ O(\varphi) = n$ . 

$\therefore $ The complexity of $\overline{S}, \overline{V}$ are both 
$O(n) \times O(n) = O(n^2)$.

$\therefore $ $F = O(n^2) + O(n^2) + O(m) + O(m) = O(n^2) + O(m)$.

But $O(n^2) + O(m) = O(n^2)$, because $m \leq \binom{n}{2} = O(n^2)$.

$\therefore $ $F = O(n^2)$.

$\therefore $ The complexity of Dinitz is $O(n) \left[ O(m) + O(n^2) \right] = O(n)O(n^2) = O(n^3) $





\pagebreak

\section{Codes}

\subsection{Hamming bound}

Let $C \subseteq \left\{ 0, 1 \right\}^{n} $. We want to prove 

\begin{align*}
    |C| = \frac{2^n}{\sum_{i=0}^{t} \binom{n}{i}}
\end{align*}

Let $A = \bigcup_{v \in C} D_t(v)$. Recall that $D_t(v)$ is the set of words in 
$\left\{ 0, 1 \right\}^n$ that are at a Hamming distance of $t$ or less from $v$.

Let $S_v(r) = \left\{ w \in \left\{ 0, 1 \right\}^n : d_H(v, w) = r  \right\}
$. Then it follows by definition that $D_t(v) = \bigcup_{r=0}^{t} S_v(r)$. Of course,
this union is disjoint. It follows that 

\begin{align*}
    A = \bigcup_{v \in C} \bigcup_{r = 0}^{t} S_v(r)
\end{align*}

and 

\begin{align*}
    |A| = |C| \times \sum_{r=0}^{t} |S_v(r)|
\end{align*}

So now we must only determine $|S_v(r)|$. But this is easy to do if we consider
that any $w \in S_v(r)$ differs from $v$ by exactly $r$ bits, and is fully
determined by this difference. In other words, there is a bijection between any
$w \in S_v(r)$ and the set of the $r$ bits out of all $n$ possible bits that
make up the difference between $w$ and $v$. This readily entails that $|S_v(r)|
= \binom{n}{r}$. This readily gives 

\begin{align*}
    |A| &= |C| \times \sum_{r=0}^{t} \binom{n}{r} \\ 
    \Rightarrow |C| &= \frac{|A|}{\sum_{r=0}^{t} \binom{n}{r} |}
\end{align*}

We do not know the cardinality of $A$, but since $A \subseteq \left\{ 0, 1 \right\}^n $
we know $|A| \leq 2^n$. Then

\begin{align*}
    |C| \leq \frac{2^n}{\sum_{r=0}^{t} \binom{n}{r}}
\end{align*}


\pagebreak


\subsection{$\delta(C) = \min \left\{ j : \exists S \subseteq H_{*n} : |S| = j \land  S \text{ is LD} \right\} $}

\begin{quote}
    \textbf{Notation.} I use $H_{*n}$ to denote the set with the $n$ columns
    of $H$. I use $H^{(i)}$ to denote the $i$th column of $H$.
\end{quote}

Let $s = \min \left\{ j : \exists S \subseteq H_{*n} : |S| = j \land  S \text{
is LD} \right\} $. This implies there are $s$ columns $H^{(j_1)}, \ldots,
H^{(j_s)}$ s.t. $\sum x_i H^{( j_i )} = 0$ for $x_1, \ldots, x_s$ not all null.

\textit{(1)} Let $w := \sum x_i e_{j_i}$ where $e_{k}$ is the vector with all
zeroes except at the $k$th coordinate. Since not all $x_i$ are zeroes, $w \neq
0$. Now, 

\begin{align*}
    Hw^t &= H \left( x_1 e_{j_1} + \ldots + x_s e_{j_s} \right)^t \\ 
         &= x_1 H e_{j_1}^t + \ldots + x_s H e_{j_s}^t \\ 
         &= \sum x_i H^{(j_i)} &\left\{ \text{Because } He_j^t = H^{(j)} \right\}  \\ 
         &= 0
\end{align*}

Then $w \in Nu(H) = C$. But $|w| \leq s$ and $w \neq 0$. We know $\delta = \min
\left\{ |x| : x \in C, c \neq 0 \right\} $.

$\therefore ~ \delta \leq |w| \leq s$.

\textit{(2)} Let $v \in  C$ s.t. $\delta = |v|$. Then there are 
$i_1, \ldots, i_{\delta}$ s.t. $v = e_{i_1} + \ldots + e_{i_\delta}$.
Since $v \in  C, Hv^t = 0$, which using the same 
logic as before gives $\sum H^{(i_j)} = Hv^t = 0$.

This implies $\left\{ H^{(i_1)}, \ldots, H^{(i_{\delta})} \right\} $ is LD.

$\therefore  s \leq \delta$.

\textit{(3)} Points \textit{(1)} and \textit{(3)} imply $s = \delta$.

\pagebreak 

\subsection{Three statements around a generating polynomial}

Let $C$ a code of length $n$ and dimension $k$ with generating polynomial 
$g(x)$. We will prove: 


\small
\begin{quote}

\begin{enumerate}
    \item $C = \left\{ p(x) : gr(p) < n \land  g(x) \mid p(x) \right\} := C_1$
    \item $C = \left\{ v(x) \odot g(x) : v(x) \in F[x] \right\} := C_2$ 
    \item $gr(g) = n - k$
    \item $g(x) \mid (1 + x^n)$
\end{enumerate}

\end{quote}
\normalsize

\textit{(1 and 2) :} We shall prove $C_1 \subseteq C_2 \subseteq C \subseteq C_1$.

Let $p(x) \in C_1$.Then there is some $q(x)$ s.t. $p(x) = g(x) q(x)$ and 
$gr \left( g(x)q(x) \right) < n $.

$\therefore ~ g(x) q(x) = g(x) \odot q(x) \in C_2$.

$\therefore ~ C_1 \subseteq C_2$.

Now let $p(x) = v(x) \odot g(x) \in C_2$ with $v(x)$ an arbitrary polynomial.
Then

\begin{align*}
    p(x) &= \left( v_0 + v_1x + \ldots + v_{gr(v)} x^{gr(v)} \right)  \odot g(x) \\ 
         &= v_0 \odot g(x) + v_1 \left( x \odot g(x) \right)  + v_2 \left( x_2 \odot g(x) \right)  + \ldots + v_{gr(v)} \left( x^{gr(v)} \odot g(x) \right)  \\ 
         &= v_0 g(x) + v_1 Rot\left( g(x) \right)  + v_2 Rot^2\left( g(x) \right)  + \ldots  +v_{gr(v)} Rot^{gr(v)}\left( g(x) \right)  
\end{align*}

All rotations of $g(x)$ belong to $C$. 

$\therefore $ $p(x) \in C$.

$\therefore C_2 \subseteq C$.

Now let $p(x) \in C$. By definition, $gr(p) < n$, which implies $p(x) = p(x)
\mod (1 + x^n)$. We know

\begin{align*}
    p(x) = g(x) q(x) + r(x)
\end{align*}

for some $q(x), r(x)$ s.t. $gr(r) < gr(g)$. Then

\begin{align*}
    p(x) &= \left( g(x) q(x) + r(x) \right)  \mod (1 + x^n)\\ 
         &= g(x) \odot q(x) + \left( r(x) \mod (1 + x^n) \right)  \\ 
         &= g(x) \odot q(x) + r(x) &\left\{ \text{Since $gr(r) < gr(g) < n$} \right\} 
\end{align*}

$\therefore $ $r(x) = p(x) + g(x) \odot q(x)$. 

We know $p \in C$ and $g(x) \odot q(x) \in C^2 \subseteq C$. 

$\therefore $ $r(x) \in C$.

But since $g$ is generating polynomial, it is the unique polynomial with the
least non-null degree in $C$.

$\therefore ~ gr(r) < gr(g) \Rightarrow r(x) = 0$.

$\therefore g(x) \mid p(x)$ and then $p(x) \in C_1$.

$\therefore C \subseteq C_1$

\textit{(3)} Let $p(x) \in C$. Then there is $q(x)$ s.t. $p(x) = g(x) q(x)$
with $n > gr(p) = gr(g) + gr(q)$. This readily implies $gr(q) < n - gr(g) < n$.
Then $g(x) q(x) \in C$.

This entails there is a bijection between $C$ and the set of polynomials of
degree $\< n - gr(g)$. Then 

\begin{align*}
    |C| &= |\left\{ p(x) : gr(p) < n - gr(g) \right\} | \\ 
    \iff ~ 2^k &= 2^{n-gr(g)} \\ 
    \iff k &= n - gr(g) \\ 
    \iff ~ gr(g) &= n - k ~ \blacksquare
\end{align*}

\textit{(4)} Divide $(1 + x^n)$ by $g(x)$ to obtain 

\begin{align*}
    1 + x^n = g(x) q(x) + r(x)
\end{align*}

with $gr(r) < gr(g)$. Taking the modulus, 

\begin{align*}
    0 &= (1 + x^n) \mod (1 + x^n) \\ 
      &=\left( g(x) q(x) + r(x) \right) \mod (1 + x^n) \\ 
      &= \left( g(x) \odot q(x) \right)  + \left( r(x) \mod 1 + x^n \right)  \\ 
      & g(x) \odot q(x) = r(x)
\end{align*}

because $gr(r) < gr(g) < n$.

$\therefore ~ r(x) = g(x) \odot  q(x) \in C$.

But $gr(r) < gr(g)$. $\therefore  r(x) = 0$ and $g(x) \mid (1 + x^n)$.



\pagebreak

\section{Matchings}

\subsection{Konig}

We want to prove that any bipartite and regular graph $G = (V, E) $ has a
perfect matching. Let $X, Y$ be the two parts of $G$. For any $W \subseteq V$
let $E_W := \left\{ wu \in E : w \in W \right\} $.

\textit{(1)} Let $S \subseteq X$ and $l \in E_S$. It follows that

\begin{align*}
    \exists x \in S, y \in Y : l = xy = yx 
\end{align*}

$\therefore y \in \Gamma(x)$. And since $x \in  S$ we have $y \in \Gamma(S)$ 
and $l \in E_{\Gamma(S)}$.

$\therefore $ $E_S \subseteq E_{\Gamma(S)}$ and $|E_S| \leq |E_{\Gamma(S)}|$.

\textit{(2)} Let us calculate $|E_W|$ when $W \subseteq X$.

Observe that $E_W = \bigcup_{w \in W} \left\{ wv : v \in \Gamma(w) \right\} $.
Furthermore, the union is disjoint, because $wv \in E_W \Rightarrow w \in X
\Rightarrow v \in Y$. Then

\begin{equation*}
    |E_W| = \sum_{w \in W} |\Gamma(w)| = \sum_{w \in W} d(w)
\end{equation*}

Since $G$ is regular, $d(w) = \delta = \Delta$. 

$\therefore $ $|E_W| = \Delta |W|$

\textit{(3)} Using what we established in \textit{(1)}, it follows from \textit{(2)} that

\begin{equation*}
    |S| \Delta \leq |\Gamma(S)| \Delta \Rightarrow |S| \leq |\Gamma(S)| 
\end{equation*}

This holds for any $S \subseteq X$. Then Hall's theorem implies there is a
complete matching from $X$ to $Y$. To prove it is perfect, we must prove $|X| =
|Y|$.

But since $X, Y$ are the two parts of $G$, $E = E_X = E_Y$. Then $|E_X| =
|E_Y|$, which implies $|X| \Delta = |Y| \delta \Rightarrow |X| =
|Y|$.

Alternatively, since there is a complete matching from $X$ to $Y$, $|X| \leq
|Y|$. But the choice of $X$ over $Y$ was arbitrary, and then the same holds for
$Y$. Then $|X| = |Y|$.

In both caes the matching is perfect.



\pagebreak 

\subsection{Hall}

Let $G = (V, E)$ a bipartite graph with parts $X$ and $Y$, and let $Z \in
\left\{ X, Y \right\} $. We want to prove that there is a complete matching
from $X$ to $Y$ iff $\forall S \subseteq Z : |S| \leq |\Gamma(S)| $.

$(\Rightarrow)$ The proof is trivial, because if such matching exists, it
induces an injective function $f : X \to Y$ s.t. $f(x) \in \Gamma(x)$. Since it
is an injection, $|f(S)| = |S|$ for any $S$. Then $f(S) \subseteq \Gamma(S)
\Rightarrow |S| \leq |\Gamma(S)|$.

$(\Leftarrow)$ Assume the Hall condition $|S| \leq |\Gamma(S)|$ holds. Assume
that, after running the algorithm to find a maximal matching, an incomplete
matching is found. We will build $S \subseteq X$ that violates our assumption
(we could use $S \subseteq Y$ without loss of generality).

\textit{(1)} Let $S_0$ be the set of rows unmatched and $T_1 = \Gamma(S_0)$.
Observe that, by assumption, $S_0 \neq \emptyset$, and all columns in $T_1$
have a match that is not in $S_0$. Let $S_1$ the set of rows matching columns
of $T_1$ and $T_2 = \Gamma(S_1) - T_1$. Generally, 

\begin{align*}
    S_i &= \text{Rows matching with } T_i \\ 
    T_{i+1} &= \Gamma(S_i) - \bigcup^{j=i}_{j=0} T_j
\end{align*}

The algorithm stops only when it is revising a row and this row has no
available neighbors; this is, it only stops passing from a $S_i$ to a $T_{i+1}$
when $T_{i+1} = \emptyset$. Furthermore, since each column only labels a single
row (that of its match), and $T_i$ "creates" $S_i$, we have $|S_j| = |T_j|$. 

Define $S = \bigcup S_i, T = \bigcup T_i$, and note that all the $S_i$ are
disjoint and all the $T_i$ are disjoint. Then

\begin{align*}
    |S| &= \sum |S_i| \\ 
        &= |S_0| + \sum |T_i| \\ 
        &= |S_0| + |T|
\end{align*}

$\therefore ~ |S| > |T|$ (since $S_0 \neq \emptyset$).

We must only prove now that $T = \Gamma(S)$. 

\textit{(1)} $T$ are the labeled columns, and each column is labeled by a row in $S$.
Each row only labels its neighbors. This implies $T \subset \Gamma(S)$.

\textit{(2)} Assume $y \in \Gamma(S)$ and $y \not\in T$. Then $y$ was 
not labeled. But since $y \in \Gamma(S)$ there is an 
$x \in S$ s.t. $y \in \Gamma(x)$. Then each time 
the algorithm passes through $x$ it should label 
$y$, which contradicts the fact that $y$ is not 
labeled. Then $y \in T$. Then $\Gamma(S) \subseteq T$.

$\therefore \Gamma(S) = T$ and $|S| > |\Gamma(S)|$. But this contradicts 
the hypothesis that the Hall condition holds. The 
contradiction comes from assuming there wasn't a 
complete matching. $\therefore $ There is a complete
matching. $\blacksquare$

\pagebreak
\section{P-NP}

\subsection{2-Color is polynomial}

To prove $2$-color is polynomial, we must provide an algorithm $\mathcal{A}$
that correctly decides whether any given $G = (V, E)$ is $2$-colorable in
polynomial time. We will first provide $\mathcal{A}$ and then show its
correctness and its belonging to $P$.

The algorithm takes an arbitrary non-colored vertex to be the root of its
connected component and colors it with $1$. Within each connected component, it
runs BFS from the given root to explore it. Each time BFS
pivots over a vertex $p \in V$ and enqueues its neighbors, the algorithm also
colors each neighbor with $3 - c(p)$, thus ensuring that all colors are in the
range $\left\{ 1, 2 \right\} $.

It is important to note that the color of any given vertex is fully determined
by the parity of its level in the BFS tree. Since the root at level zero is set
to $1$, all vertices in the second level are colored with $2$, those in the
third with $1$, and so on.

\begin{align*}
    &j := 1\\
    &\textbf{while } j < n \textbf{ do } \\
    &\qquad r = \text{arbitrary non-colored vertex} \\ 
    &\qquad c(r) = 1 \\ 
    &\qquad queue = \left\{ r \right\} \\ 
    & \qquad\textbf{while } queue \neq \emptyset \textbf{ do } \\ 
    & \qquad \qquad p = pop(queue) \\ 
    &\qquad\qquad \textbf{for } x \in \Gamma(p) \textbf{ do } \\ 
    & \qquad \qquad \qquad \textbf{if } x \text{ not colored} \textbf{ do} \\ 
    &\qquad \qquad \qquad \qquad c(x) = 3 - c(p)\\ 
    &\qquad \qquad \qquad \qquad push(queue, x)\\
    &\qquad\qquad\qquad\textbf{fi}\\
    &\qquad\qquad \textbf{od}\\ 
    &\qquad\textbf{od}\\
    &\textbf{od}\\
    &\textbf{for } \left\{ x y \right\}  \in E \textbf{ do } \\ 
    &\qquad \textbf{if } c(x) = c(y) ~  \textbf{do } return \textbf{ False fi} \\ 
    &\textbf{od}\\
    &return \textbf{ True}
\end{align*}

The inner while runs BFS with a slight modification to color vertices when they
are enqueued and is thus $O(m)$. It is executed per each connected component
and the number of connected components is $O(n)$. $\therefore $ The algorithm 
is polynomial. 

That the algorithm correctly decides that a graph is two-colorable is trivial.
To prove that it also correctly decides that a graph is not two-colorable, we
shall prove a negative answer entails the graph contains an odd cycle.

Assume the algorithm was executed over $G = (V, E)$ and returned
$\textbf{False}$. Then there is some $\overrightarrow{xy} \in E$, in a
particular connected component $\mathcal{C} \subseteq G$, s.t.
$c(x) = c(y)$. Let us presume, without loss of generality, that 
$x$ was enqueued before $y$. Let us denote with $r$ the root 
of $\mathcal{C}$ from which BFS was ran.

Assume $x$ enqueues $y$. Then $c(y) = 3 - c(x) \neq c(x)$, a contradiction.
Then $x$ does not enqueue $y$. But this can only happen if, when $x$ is the at
front of the queue, $y$ was already enqueued by some other vertex. 

In particular, there is a vertex $w$ that is the vertex of greater level common
to $x$ and $y$ in the BFS tree---i.e. the vertex from which $x$ and $y$
diverge---. Let $\eta(w)$ be the level of $w$ in the BFS
tree.

Consider the cycle $w \ldots x y \ldots w$, which exists because all these
vertices belong to $\mathcal{C}$. There are $\eta(x) - \eta(w)$ edges from $w$ to
$x$, and $\eta(y) - \eta(w)$ edges from $y$ to $w$. There is one 
extra edge for $xy$. The total amount is then 

\begin{align*}
    \eta(x) - \eta(w) + \eta(y) - \eta(w) + 1 &= \eta(x) + \eta(y) - 2\eta(w) + 1
\end{align*}

By assumption, $\eta(x)$ and $\eta(y)$ are both greater than $\eta(w)$.
$\therefore $ $\eta(x) + \eta(y) > 2\eta(w)$ and length of the path is greater
than zero (sanity check). 

Since $c(x) = c(y)$, $\eta(x) \equiv \eta(y) \mod 2$ and therefore $\eta(x) +
\eta(y)$ is even. Then the length of the cycle is odd. 

$\therefore $ $C_{2k+1} \subseteq \mathcal{C}$ and $\chi(G) \geq 3$.


\pagebreak

\subsection{3SAT es NP-Completo}.

Let $B = B_1 \land  \ldots B_m$ an instance of SAT with variables 
$x_1, \ldots, x_n$. We build an instance of $3$-SAT 
by transforming each $B_i$ into an $E_i$ as follows:

\textit{Complete.}

\begin{align*}
E_i = (e_1 \lor  e_2 \lor y_1) \land  (\overline{y_1} \lor  y_2 \lor  e_3) \land  (\overline{y_2} \lor  y_3 \lor  e_4) \lor \ldots (\overline{y_{k-3}} \lor e_{k-1} \lor e_{k})
\end{align*}

We want to prove

\begin{align*}
    \exists \overrightarrow{b} : B(\overrightarrow{b}) = 1 \iff \exists \overrightarrow{\alpha} : \tilde{ B }(\overrightarrow{b}, \overrightarrow{\alpha}) = 1
\end{align*}

\textit{($\Leftarrow$)} Asume $B(\overrightarrow{b}) = 0$. Then
$D_i(\overrightarrow{b}) = 0$ for some $i$. Let 
 $e_1, \ldots, e_k$ be the literals in $D_i$.

If $k = 3$ a contradiction ensues trivially. If $k = 2$, then $D_i = e_1 \lor
e_2$ and then $E_i = (e_1 \lor  e_2 \lor y_1) \land  (e_1 \lor  e_2 \lor
\overline{y_1})$. Since $D_i = 0$, $e_1 \lor  e_2 = 0$ and therefore $e_1 = e_2
= 0$ From this follows  $E_i = y_1 \land \overline{y_1} = 1$. $(\bot)$

If $k = 1$ then $e_1 = 0$ and therefore $E_i = (y_1 \lor  y_2) \land  (y_1 \lor
\overline{y_2}) \land  (\overline{y_1} \lor  y_2) \land  (\overline{y_1} \lor
\overline{y_2}) = 0$. But by assumption $E_i = 1 (\bot)$.

If $k\geq 4$ we must observe that, since $D_i(\overrightarrow{b}) =
0$, we have $e_1 = e_2 = \ldots = e_k = 0$. Then this literals 
are neutral elements in the disjunctions and can be ignored.
Since $E_i(\overrightarrow{b}, \overrightarrow{\alpha}) = 1$, its first term 
is true; in other words, $e_1 \lor  e_2 \lor  y_1 = 1
\Rightarrow y_1 = 1$. In all the following cases (except the last),
$E_i = \overline{y_{i-1}} \lor  y_i$ must be true; this is,
$y_i \Rightarrow y_{i+1}$ is true. But the last term is $\overline{y_{k-3}}$,
which cannot be true because $y_1$ and $ y_1 \Rightarrow y_2 \Rightarrow \ldots
\Rightarrow y_{k-3}$. ($\bot$)

\textit{($\Rightarrow$)} Assume $B(\overrightarrow{b}) = 1$. For $k = 1, k = 2$, define 
$y_i = 0$ for all $i$. $\therefore $ $D_i(\overrightarrow{b}) = 1 \Rightarrow
E_i(\overrightarrow{b}, \overrightarrow{\alpha}) = 1$. For $k = 3$ the result 
is trivial. Let us consider the case $k \geq 4$.

Since $D_i(\overrightarrow{b}) = 1$ is a true disjunction, at least one $e_r$
is true under $\overrightarrow{b}$ . Define the following assignment:

\begin{align*}
    &y_1 = y_2 = \ldots = y_{r-2} = 1 \\ 
    &y_i = 0 \text{ para todos los demás $i$}
\end{align*}

Then

\begin{align*}
    E(\overrightarrow{b}, \overrightarrow{\alpha}) &= \left( e_1 \lor e_2 \lor y_1 \right) &\left\{ \text{True because } y_1 = 1 \right\}  \\ 
    \land &\left( \overline{y_1} \lor  y_2 \lor  e_3 \right) &\left\{ \text{True because } y_2 = 1 \right\}  \\ 
          &\vdots& \\ 
    \land &(\overline{y_{r-3}} \lor  y_{r-2} \lor  e_{r-1}) &\left\{ \text{True because } y_{r-2} = 1 \right\}  \\ 
    \land&(\overline{y_{r-2}} \lor  y_{r-1} \lor  e_r) &\left\{ \text{True because } e_{r} = 1 \right\}  \\ 
    \land&(\overline{y_{r-1}} \lor y_{r} \lor  e_{r+1}) &\left\{ \text{True because } y_{r-1} = 0 \right\}  \\ 
         &\vdots &\\ 
    \land &(\overline{y_{k-3}} \lor e_{k-1} \lor  e_k) &\left\{ \text{True because } y_{k-3} = 0 \right\} 
\end{align*}

$\therefore $ Our assignment makes $\tilde{ B }$ true.

\pagebreak 


\subsection{3-Color es NP-Completo}

Sabemos que $3$-Color $\in NP$. La idea es ver que $3-SAT \leq_p 3-COLOR$. 
Debemos crear un grafo $G = (V, E \cup F) $ tal que $B$ es satisfactible 
si y solo si $\chi(G) \leq 3$.

Let $B = D_1 \land  \ldots \land  B_m$ with variables $x_1, \ldots, x_n$ and each $B_i
= (l_{i1} \lor l_{i2} \lor  l_{i3})$. Let $\mathcal{G} = (V, E)$ a graph 
formed as follows: 

\begin{itemize}
    \item A nucleus $t$ from which $n$ triangles with sides $\left\{ t, v_1, w_1 \right\}, \ldots, \left\{ t, v_n, w_n \right\}  $ form.
    \item $m$ claws, each with a triangle $\left\{ b_{i1}, b_{12}, b_{13} \right\} $, and such that from each $b_{ij}$ sprouts a tip $u_{ij}$.
    \item A source $s$ connected to $t$ and to every tip $u_{ij}$.
\end{itemize}

Now, let $\psi : \left\{ l_{11}, \ldots, l_{m3} \right\} \to V$ a function that maps a literal to $v_k$ if the literal is $x_k$, and to $w_k$ if the literal is $\overline{x_k}$. In other words, 

\begin{align*}
    \psi(l_{ij}) := \begin{cases}
        v_k & l_{ij} = x_k \\ 
        w_k & l_{ij} = \overline{x_k}
    \end{cases}
\end{align*}


We will use $\psi$ to create our graph $G = (V, E \cup F)$ by letting


\begin{align*}
    F = \left\{ u_{ij}\psi(l_{ij}) : 1 \leq i \leq m, 1 \leq j \leq 3 \right\} 
\end{align*}

In other words, we connect each $u_{ij}$ to either $v_k$ or $w_k$, depending 
on whether $l_{ij} = x_k$ or $\overline{x_k}$. Now that we have defined 
$G$, we must only prove that $B$ is satisfiable iff $G$ is $3$-colorable.

\textit{Proof of $(\Leftarrow)$ :} Assume $\chi(G) \leq 3$. Since $G$ has
triangles, $\chi(G) = 3$. Let $\overrightarrow{b}_k = \left[ c(v_k) = c(s)
\right] $.

We must prove $B_i(\overrightarrow{b}) = 1$ for all $1 \leq i \leq m$. Take an
arbitrary $B_i$. 


\begin{quote}

\textit{(1)} The triangle $\left\{ b_{i1}, b_{i 2}, b_{i 3} \right\} $
must have $c(b_{ij}) = c(t)$ for some $j$. Then, since $\left\{ b_{ij}, u_{ij}
\right\} $ is a side, $c(u_{ij}) \neq c(t)$. And since $\left\{ u_{ij}, s
\right\} $ is a side, $c(u_{ij}) \neq c(s)$. $\therefore $ $u_{ij}$ was
colored with the third color.

\textit{(2)} Since $\left\{ u_{ij}, \psi(l_{ij}) \right\} $ is a side, and
$\left\{ \psi(l_{ij}), t \right\} $ is a side, $c(\psi(l_{ij})) = c(s)$.

\end{quote}

We have established that $c(\psi(l_{ij})) = c(s)$. By definition, we have two
cases. 

\begin{quote}
    \textit{(1)} If $\psi(l_{ij}) = v_k$, $l_{ij} = x_k$ and $c(v_k) = c(s)$, which means $\overrightarrow{b}_k = 1$. Then $l_{ij}(\overrightarrow{b}) = 1$. Then $D_i(\overrightarrow{b}) = 1$.

    \textit{(2)} If $\psi(l_{ij}) = w_k$, then $l_{ij} = \overline{x_k}$. Since $\left\{ v_k, w_k \right\} $ is a side, these vertices have different colors. Then $c(v_k) \neq c(s)$. Then $\overrightarrow{b}_k = 0$. Then $l_{ij}(\overrightarrow{b}) = \overline{x_k}(\overrightarrow{b}) = 1$. Then $D_i(\overrightarrow{b}) = 1$.
\end{quote}

In both cases, $D_i(\overrightarrow{b}) = 1$. This holds for any $i = 1, 2,
\ldots, m$. Then $B(\overrightarrow{b}) = 1$. $\blacksquare$

\textit{Proof of $(\Rightarrow)$ :} Assume there is some $\overrightarrow{b}$
s.t. $B(\overrightarrow{b}) = 1$. Let $C = \left\{ \mathcal{C}_s,
\mathcal{C}_t, \mathcal{C} \right\} $ a set of three colors, and let $c(s) =
\mathcal{C}_s, c(t) = \mathcal{C}_t$, and 

\begin{align*}
    &c(v_k) = \begin{cases}
        \mathcal{C}_s & b_k = 1 \\ 
        \mathcal{C} & b_k = 0
    \end{cases}, &c(w_k) = \begin{cases}
    \mathcal{C} & b_k = 1 \\ 
    \mathcal{C}_s & b_k = 0
    \end{cases}
\end{align*}

Clearly, we are ensuring $c(v_k) \neq c(w_k) \neq c(t)$, so the triangles
$\left\{ t, v_i, w_i \right\} $ are all properly colored. And of course,
$\left\{ t, s \right\} $ is also properly colored. All we have to do is look at
the claws.

First, since $\exists \overrightarrow{b} : B(\overrightarrow{b}) = 1$, then
$D_i(\overrightarrow{b}) = 1$ for all $i$. This means, in an arbitrary $D_i$,
there is at least one $l_{ij}$ s.t. $l_{ij}(\overrightarrow{b}) = 1$. Let's 
color the tips of the claw as follows:

\begin{align*}
    c(u_{ir}) = \begin{cases}
        \mathcal{C} & r = j \\ 
        \mathcal{C}_t & r \neq j
    \end{cases}
\end{align*}

Clearly, $\left\{ u_{ir}, s \right\} $ is properly colored. What about $\left\{ u_{ir}, \psi(l_{ir}) \right\} $? There are two cases: 

\begin{quote}
    \textit{(Case $r \neq j$) :} In this case, $c(u_{ir}) = \mathcal{C}_t$, and since $\psi(l_{ir})$ is either a $v$ or a $w$, $c(\psi(l_{ir})) \neq \mathcal{C}_t$. 

    \textit{(Case $r = j$) :} Here, $c(u_{ir}) = \mathcal{C}$. If $l_{ij} = x_k$, $\psi(l_{ij}) = v_k$. By definition of $l_{ij}$, $l_{ij}(\overrightarrow{b}) = 1$. Then $c(\psi(l_{ij})) = c(v_k) = \mathcal{C}_s$. 
\end{quote}

So, in both cases we are properly coloring $\left\{ u_{ir}, \psi(l_{ir})
\right\} $.

Now that we have colored the tips $u_{ir}$, we only have to color the triangles
in the claw, $\left\{ b_{i 1}, b_{i 2}, b_{i 3} \right\} $. Let $c(b_{ij}) =
\mathcal{C}_t$ and the other two be colored in any of the possible ways. Clearly, the triangle is properly colored. Furthermore, 

\begin{itemize}
    \item $\left\{ b_{ij}, u_{ij} \right\} $ is properly colored, because $c(b_{ij}) = \mathcal{C}_t, c(u_{ij}) = \mathcal{C}$. 

    \item $\left\{ b_{ir}, u_{ir} \right\} $ with $r \neq j$ is properly colored, because $c(u_{ir}) = \mathcal{C}_t$ and $c(b_{ir}) \neq \mathcal{C}_t$.
\end{itemize}

We have given a proper coloring of all vertices using $\overrightarrow{b}$.

\subsection{Trisexual marriage}

\end{document}



