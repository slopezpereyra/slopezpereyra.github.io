\documentclass[a4paper, 12pt]{article}
\usepackage{verse, gmverse}
\usepackage[utf8]{inputenc}
\usepackage[T1]{fontenc}
\usepackage{textcomp}
\usepackage{newtxtext} \usepackage{newtxmath}
\usepackage{titlesec}
\titleformat{\section}[block]{\Large\bfseries\filcenter}{}{1em}{}
\titleformat{\subsection}[hang]{\normalfont\filcenter\MakeUppercase}{}{1em}{}
\setcounter{secnumdepth}{0}

\title{DIEZ AÑOS DE NOCHE}


\begin{document}

\date{}
\maketitle

\pagebreak

\hspace*{\fill}\textit{Facilis descensus Averno}\\
\hspace*{\fill}—Virgilio

\pagebreak
\tableofcontents
\pagebreak
\Huge
\centerline{Prefacio}
\normalsize

~

~ Jung señaló que todos tenemos miles de años; la neurociencia, a mi juicio,
corrobora esta tesis. Nuestra experiencia está formada por vestigios primitivos
y arquetipos; la piel de nuestra vida es proyectada desde el esqueleto de la
especie. A la ciencia le compete describir este hecho en términos de
estructuras cerebrales, homologías, y una explicación evolutiva de la
consciencia. A la poesía, intuirlo o develarlo mediante imágenes y símbolos.

Quiero decir que, en la poesía, he pretendido una rara suerte de arqueología de
la mente. He deseado reducir la alta resolución de mi vida individual a una
resolución más baja donde lo inesencial se ha diluido. He querido librar a mi
experiencia de su densidad y traducirla a algo más simple y aproximativo, a un
rostro tan abreviado y primigenio que todos puedan verse en él.

Espero que en el pausado ritmo de estos versos encuentres al menos un indicio
de algo que siempre ha sido tuyo, un fósil donde asome un rostro milenario. En
realidad, tú y yo somos el mismo.


~

\hspace*{\fill}S.L.P
~

\pagebreak

\subsection{Quisiera ser un hombre simple}
~

\begin{verse}
    
quisiera ser un hombre simple\\
~

quisiera sospechar que todavía\\
algo de mí no se ha perdido, algo…!\\
~ 

algo que ansiosa de crueldad y noche\\
ceñiste dentro de una azucena muerta\\
~ 

algo que un tenue gesto de tus manos\\
hizo desvanecer, o algo…\\
~ 

a todas horas\\
un huevo lleno de ávidas serpientes\\
presiente la fractura inevitable\\
~ 

¿lo sientes? en mi carne tenebrosa\\
la cobra de tu amor ha eclosionado\\
~ 

quisiera ser un hombre más sencillo\\
o un perro que envejece serenamente triste\\
~ 

entonces nuestro amor no habría perdido\\
ni un solo sueño y no florecerían\\
estas escamas negras en mi vida…\\
~ 

no soy un hombre simple, en todo siento\\
el eco de un milenio anocheciente\\
~ 

e incluso en tu ternura, que es perfecta,\\
presagio un universo de sospechas\\
y en bífidos eclipses anochezco…!\\

\end{verse}

\pagebreak

\subsection{A Paulina}

\begin{quote}

    \footnotesize
\hspace*{\fill}{Nuestra alma, pozo de agua viva.}\\
\hspace*{\fill}—Orígenes de Alejandría

\end{quote}
\normalsize
~
\begin{verse}
aminorabas la sombra del mundo\\
~ 

en el brocal de tus ojos un agua\\
de melancólica y rara frescura\\
~ 

un solo día\\
aminoraron la sombra… ese día\\
fue una certeza que yo estaba vivo\\
hasta la luz sideral era mía…\\
~ 

¿sabes? no soy feliz… me siento una celda \\
con una estrecha ventana por donde\\
ves, tenebrosa de muerte, mi vida\\
~ 

y en sus rincones sin luz amontonas\\
piedras y piedras sin ser y rencores\\
soy una luna sin alma que expira\\
\end{verse}

\pagebreak

\subsection{Carta de un suicidio no consumado}
~
\footnotesize
\begin{quote}
\hspace*{\fill}{ Though I walk through the valley of the shadow \\\hspace*{\fill} of death,
I will fear no evil: for thou art with me.\\
\hspace*{\fill}—Psalms, 23:4
\end{quote}

\normalsize
\begin{verse}

ves en el corazón de mi agonía\\
un interior de luz inmarcesible?\\
ves en mi palidez un imposible\\
rubor? ves en mi cáscara sombría\\
cómo mi carne nunca estará fría\\
si doy feliz el paso ineludible?\\
sí… aunque mi muerte me es incognoscible\\
sigue siendo, después de todo, mía…\\
no llorarás por mí, no verás una\\
serpiente amanecer en mi retrato\\
tuya será la paz que me consume\\
y en el último eclipse de la luna\\
oirás el rechinar de mi zapato\\
y olerás en la muerte mi perfume\\
\end{verse}

\pagebreak 
    
\subsection{Desprecio}
~
\begin{verse}
en la fangosa carne de las cosas\\
siento cómo tus manos configuran\\
la cruz de mi castigo…!\\
~

en un cenit estúpido mi vida\\
vomita las serpientes que he tragado\\
luna tras luna, manso y sometido\\
~

detesto que mis tardes hayan sido\\
un cáliz putrefacto que a miserables gotas\\
llené con la ilusión de ser tu amigo\\
~

detesto haberte amado como un perro\\
detesto la penumbra de tu boca\\
y el junio sepulcral que me ha parido…!\\
\end{verse}

\pagebreak

\subsection{Todo ya ha sido escrito}
~
\begin{verse}
    
Solos entre volúmenes y escritos,\\
muertos acaso en otra Alejandría,\\
hay un libro, una página y un día\\
que aguardan, tras lectores infinitos\\
e innúmeros, los ojos destinados\\
al ritmo de su prosa o de su verso.\\
Hay un libro entre muebles olvidados\\
que me espera en su plácido universo.\\
Su autor, anónimo o indescifrable\\
para estos ojos vanos y fortuitos,\\
será un oscuro Nombre: un innombrable\\
doble de mis volúmenes y escritos.\\
En él, frutos de una memoria y una\\
tristeza casi mías, está la noche\\
que ya escribí: la ignominiosa luna,\\
los libros y la pena y el reproche.\\
Busco la dicha de encontrar acaso\\
las hojas que figuran estas hojas,\\
el término inicial de mis congojas\\
y el rojo más antiguo de este ocaso.\\
Lo escrito es una letra evanescente\\
de un alfabeto ajeno y una historia\\
indescifrable. Hay un antecedente\\
de todo lo que esboza mi memoria,\\
de mis congojas, lágrimas y penas,\\
de esta penumbra y esta noche hueca.\\
Y aguarda en las efímeras arenas\\
del tiempo, que es su vasta biblioteca.\\
\end{verse}

\pagebreak
\subsection{Un árbol era mi sueño}
~

\begin{verse}
Un árbol era mi sueño\\
la noche que te soñaba,\\
y las floridas estrellas\\
su hojarasca milenaria.\\
~ 

—\textit{¿Y qué es un sueño?}—dijiste.\\
—\textit{¿Y qué es soñar?}—preguntabas.\\
Mientras, la copa del árbol\\
y el cielo se entrelazaban.\\
~ 

Fuiste esa noche, soberbia\\
de soledad y de escarcha,\\
como la sombra que turba\\
el centro de una esmeralda.\\
~ 

Apenas recuerdo el auge\\
que quiso emprender el alba\\
cuando, al fin, abrí los ojos\\
(el árbol se marchitaba…).\\
~ 

Y yo te dije mi sueño.\\
Y tú no dijiste nada.\\
Y solo quebró el silencio\\
el crepitar de una rama.\\
\end{verse}

\pagebreak 

\subsection{A mí mismo}

\scriptsize
\begin{quote}
    \hspace*{\fill}¿Conocéis el miedo del que se adormece?\\
\hspace*{\fill}—Friedrich Nietzsche
\end{quote}
\normalsize
~

\begin{verse}
    
En la penumbra larga te cuestiono.\\
Entiendo que fue pródiga en olvido\\
esta época de lunas que has vivido\\
con algo de piedad y algo de encono.\\
A veces sueñas una arcana fuente\\
de la que brotan otros lentos sueños.\\
(Ni tú ni yo hemos sido nunca dueños\\
de los crueles jardines de tu mente.)\\
Otras intuyes que la oscura muerte\\
del padre ha sido eterna, y que las cosas,\\
hasta las más incólumes y hermosas,\\
están echadas a esa misma suerte.\\
Algunas hablas con la amada muerta,\\
pidiéndole que vuelva a los confines\\
de este mundo de lunas y jazmines,\\
aunque ya le han cerrado aquella puerta.\\
O en la serena y oprobiosa aurora\\
que pesa sobre el odio de tu frente\\
sabes que ha sido un sueño aquella fuente,\\
y que es inútil anhelarla ahora.\\
No sé qué dios oscuro y vengativo\\
ha prodigado el sueño y la vigilia;\\
no sé el consuelo pobre que te auxilia\\
del horror de soñar estando vivo.\\
Te espanta ir a dormir, pero es en vano.\\
Mientras escribes estas líneas sientes\\
el murmurar del agua de las fuentes\\
de un mundo menos cierto y más arcano.\\
\end{verse}

\pagebreak 
\subsection{Éramos libres}
~

\begin{verse}
nada era presa del eclipse\\
~ 

en cierta región de tu carne\\
un trueno era tejido y destejido\\
y un eco de cristales nos hundía\\
~ 

como una mosca muerta\\
el lunar de mi boca\\
~ 

qué vi en la gema de tu vientre?\\
mi espuma floricida castigaba\\
la vía láctea de tus pechos?\\
~ 

no sé… sé que llorabas\\
que fuiste frágil\\
~ 

yo fui terrible, marchito\\
~ 

recuerdas esa piedra vengativa\\
con que crispé tu cadáver?\\
mi mano lapidaria la recuerda...\\
~ 

recuerdas? todavía\\
no éramos presa éramos libres\\
del eclipse\\

\end{verse}

\pagebreak 
\subsection{Desde lejos}
~

\begin{verse}
escuchas el patético suspiro
de un eco crepitante, un arrorró\ldots?
~

es una hogaza pobre de pan envejecido 
que alguna vez ardió en mi corazón 
~

o es un ladrido estúpido que Cristo,
borracho de rencores y crepúsculos,
silencia con la muerte de una flor
\end{verse}

\pagebreak 
\subsection{Una esmeralda negra}
~

\begin{verse}
    
era obsidiana antigua lo que formaba tu alma?\\
una esmeralda negra?\\
~ 

vi una vez —fue hace ya tiempo\\
todavía eran eternos los jazmines—\\
arenas minerales en tu pecho\\
tu boca una fractura de rubíes\\
un mar y caracolas con espuma\\
corales luminosos de tu vientre\\
~ 

eras un monumento del agua y de la tierra\\
y en cuarzos y cristales\\
vi tu dios y tu barro originarios\\
~ 

una esmeralda negra?\\
\end{verse}

\pagebreak 
\subsection{Pasará}
~

\begin{verse}
    
en el abyecto corazón del día\\
sé que una luna sanadora duerme\\
~ 

arde como una lúcida serpiente 
que siente que su encierro se deshace
~

puede que la esperanza esté perdida,\\
pero hay un redentor oculto siempre\\
en la más inocente de las cosas,\\
y el odio no es eterno ni es un alma\\
lo que asesina la tuya\\
~

por eso duermes en la noche sucia:\\
una moneda antigua y misteriosa\\
te aguarda bajo las aguas\\
\end{verse}

\pagebreak 

\subsection{Debe pagarse el sueño de los muertos}
~ 

\begin{verse}
y ahora que mis padres están muertos\\
y en mí la sombra bárbara se cierra\\
¿quién pagará mi bóveda de tierra\\
y el sueño de mis párpados abiertos?\\
~ 

en vano urdí una flor en los desiertos\\
en vano amé la luz... ya no me aterra\\
la cáscara nocturna que me encierra\\
en la frescura negra de los huertos\\
~ 

y ahora te pienso, patio de jazmines\\
donde una vez soñé ser enterrado\\
en ese tiempo en que te vi sembrado\\
por ciertos hacedores de jardines…\\
~ 

¿pero quién sabe de ese antiguo sueño?\\
¿del túmulo sin luz quién es el dueño?\\
\end{verse}

\pagebreak
\subsection{De tiempo todo está lleno}

\scriptsize
\begin{quote}
    \hspace*{\fill}And to the sun we all will bow\\
\hspace*{\fill}And say, good-bye – but just for now!\\
\hspace*{\fill}—Dylan Thomas
\end{quote}
\normalsize

\begin{verse}
De tiempo todo está lleno,\\
de ti está todo vacío:\\
casas, ventanas y puertas,\\
y nuestros cinco sentidos.\\
~ 

Sólo tiempo, sólo tiempo:
nada más ha sido escrito 
donde fue escrita la historia 
de todos los seres vivos.
~ 

Pero en el sueño hay estelas,\\
pero en el sueño hay vestigios 
de algo que tiene tu rostro:
un eco, una voz, un indicio\ldots
~ 

Porque nuestro sueño acecha\\
lo verdadero, lo vívido.\\
Por eso a veces se abre\\
la flor del ojo dormido.\\
Por eso a veces llenamos\\
las sábanas de rocío.\\
~ 

A nadie, despierto, veo.\\
Alguien, dormido, yo he visto.\\
\end{verse}

\pagebreak 
\subsection{Hastío}
~ 

\begin{verse}
    
una mosca está libando\\
de tu inocencia muerta\\
~ 

un sueño de lombrices iracundas\\
en esto que se ha muerto entre tú y yo\\
~

milenios: son milenios lo que tuerce\\
la sangre milenaria\\
y solo tiempo se conjura en esto\\
que hoy ya no quiso ser\\
~ 

¿sabes? cuando es de noche espero\\
la hora de volver a hundir las manos\\
en el espeso fango de la vida…\\
~ 

y sé que como yo también esperas\\
y el sueño que gestamos está lejos\\
y sabes, como yo, que se ha perdido\\
\end{verse}

\pagebreak
\subsection{Soñé con todas las almas}
~ 

\begin{verse}
soñé con todas las almas\\
~

el polvo de muchos siglos\\
las ensuciaba\\
~

amor (yo dije) quisiera\\
volver al agua\\
floreciente de tus pechos\\
volver al agua...!\\
~

todas sintieron el eco\\
de esa plegaria\\
~

todas supieron el sueño\\
que yo soñaba:\\
~

volver al agua\\
\end{verse}

\pagebreak

\subsection{Selene}
~ 
\begin{verse}
    
I wish to give the windings of the moon to you\\
the blunt interrogation of the dawn\\
mercilessly posed before the soul of men\\
~ 

the streets unwalked the libraries\\
a page randomly taken—violently taken from a secret book\\
and mysticism and dew\\
~ 

I desire to offer all that I’ve never possessed\\
but in what I live ungotten to you\\
~ 

the locust-stricken regions of a tree\\
the unsuspected darkness of the rose\\
the never solitary loneliness of me\\
under a winter rain someday perhaps maybe\\
~ 

I wish to offer silence\\
dead and weary silence to you\\
~ 

and that is all I am to offer to you\\
~ 

this is the eternal water of generous cosmologies\\
the stars reflected clear, as well eternal, in the water\\
the silence of a dreaming child—these are my offerings to you\\
~ 

this I confess before the night amidst the speeches of the water:\\
that time the sands of time the cruelty of time I give to you\\
~ 

this I confess before the moon that evil star:\\
O yes I have pursued all things which may evoke you\\
~ 

before you I extend this sort of death\\
that rises from the heart unto the lips\\
~ 

I seek in every rose the ceasing of the rose\\
and in a whispering zest\\
I speak your name I speak\\
the vicious letters of your earthly name\\
\end{verse}

\pagebreak
\subsection{Los mitos y los sueños}
~ 

\begin{verse}
    
Los mitos y los sueños son escritos\\
en los secretos muros de la mente\\
desde la misma primigenia fuente\\
de símbolos ociosos y arquetipos.\\
~ 

En sus oscuros ámbitos, los ritos\\
— el fuego y las especias, la paciente\\
y cristiana oración— son infinitos.\\
Eterna es la manzana y la serpiente.\\
~ 

Eterno es el puñal y es el hermano,\\
eternos son los dioses y los días,\\
y eternas son las arduas simetrías\\
de que se nutre el corazón humano.\\
~ 

Son pocas pero viejas, viejas cosas:\\
los mitos y las lunas y las rosas...\\
\end{verse}

\pagebreak 

\subsection{Nacimiento}
~ 
\begin{verse}
    
Decís que sí y la sombra se deshace.\\
Al fin sos algo más que tu latido.\\
La angustia de la noche te atraviesa.\\
~ 

De pronto sos un pétalo que nace,\\
esclavo de los vientos y el olvido,\\
y sos la muerte que a soñar empieza.\\
\end{verse}

\pagebreak
\subsection{La sangre no descansa}
~ 

\begin{verse}
    
la sangre no descansa, no reposa\\
~ 

eso es lo que dijiste aquella noche\\
en que fui el asesino de una rosa\\
~ 

la sangre no reposa, no descansa\\
~ 

eso es lo que sentiste aquella tarde\\
que yo entenebrecía el agua mansa\\
~ 

la sangre, nuestra sangre milenaria\\
~ 

(la luna negra de tu pena humilde\\
temblaba suplicando una plegaria)\\
~ 

la sangre no descansa: un eco es ella\\
~ 

un eco que a tu vientre se remonta\\
desde la luz de la primer estrella\\
\end{verse}

\pagebreak 
\subsection{Génesis}
~ 
\begin{quote}
\footnotesize    
    \hspace*{\fill} Porque en todas las tardes de esta vida\\ \hspace*{\fill}muy poco nace, pero mucho muere...!\\
\hspace*{\fill}— César Vallejo

\end{quote}
\normalsize

\begin{verse}

He soñado la génesis del hombre,\\
en una noche llena de sospechas.\\
Soñé empapado de sudor y luna,\\
y en un estado inquieto de consciencia.\\
Soñé que allí no había una manzana,\\
ni prístinos adanes ni arduas evas,\\
ni bien ni mal (siquiera se insinuaba\\
un ángulo formando alguna recta…!).\\
He soñado la génesis del hombre,\\
partícipes ni el agua ni la esperma.\\
(Tal vez, en el más plácido escenario,\\
un soplo era el principio de la gesta…)\\
He soñado la génesis del hombre,\\
y he vuelto de mi sueño libre apenas,\\
porque en todas las noches de esta vida\\
muy poco nace, pero mucho sueña…!\\

\end{verse}

\pagebreak
\subsection{El mal remedio}
~ 

\begin{verse}
    
ignoro el bálsamo preciso\\
aunque mi herida es aparente\\
~ 

siempre la sanación es más oscura\\
que la oscura fiebre\\
\end{verse}

\pagebreak
\subsection{Here as we timeless lay}
~ 

\begin{verse}
    
here as we timeless lay presume that all is timeless\\
that each new life is but an instant’s way\\
and this ungotten child we have prefigured in a dream\\
already bears the guilt of humankind\\
~ 

yes, in the birth of each new drop of life\\
await the seeds of all that is unfading\\
~ 

you’ll see as I see now that you have been\\
a wife to me as you were the wife of Abraham\\
and that our child awaited in the rose\\
that some forgotten eyes saw rise and fall amid the snow\\
\end{verse}

\pagebreak
\subsection{Eras como la aurora}
~ 

\begin{verse}
    
Eras como la aurora, penumbra y mediodía.\\
La aurora nos enseña que siempre es necesaria\\
la muerte, con su noche profunda y milenaria,\\
para el que sueña triste con ver la luz del día.\\
~ 

Yo siempre he sido tuyo, tú siempre has sido mía;\\
desde antes que la mano incólume y precaria\\
de un dios crepuscular le diera a mi embrionaria\\
esencia alguna forma, tu sangre en mí ya ardía.\\
~ 

Mil muertes es el alma, mil muertes ominosas\\
para alumbrar apenas la faz de nuestras vidas,\\
para olvidar a un hombre, para olvidar un día.\\
~ 

Pero la aurora alumbra las más oscuras cosas\\
en este nuevo mundo de estrellas ya vencidas,\\
y hasta a una mosca nutre la flor en su agonía…!\\
\end{verse}

\pagebreak
\subsection{Visión}
~ 

\begin{verse}
    
En el antiguo patio se han urdido\\
las hebras de un oscuro laberinto\\
donde todo es igual, pero distinto\\
(como lo es en el sueño y el olvido).\\
Igual se ve la vasta luna, apenas\\
luciendo la penumbra de la rosa,\\
que acaso ni es igual ni es otra cosa\\
que antes, como arena en las arenas.\\
Y, entonces, vos. (Da igual si fue la vida,\\
el sueño o la memoria: acaso verte\\
revela que la vida y que la muerte\\
son sueño y son memoria a su medida.)\\
Sos vos, como una sombra dolorosa,\\
al mismo tiempo vívida y dudosa…!\\
\end{verse}

\pagebreak
\subsection{A portrait of the artist}
~ 

\begin{verse}
I'm but the ruins of an ancient world\\
and those that pass me by can only speculate\\
which were the gods that ruled my pagan faith\\
which were the locusts of my obscure fall\\
\end{verse}

\pagebreak 
\subsection{Primer sueño}
~ 

\begin{verse}
En el jardín oscuro de los sueños\\
dos luces, aunque tenues, verdaderas\\
obnubilaron mis sentidos muertos\\
y de algún modo los vivificaron.\\
Sentí en mi carne el pulso de la luna;\\
la tumba donde ayer estaba ciego\\
me propagó un agudo escalofrío\\
de estrellas y de inviernos.\\
Recuerdo esos dos ojos crueles;\\
sé que una voz eterna acompañaba\\
su paso por la noche inescrutable,\\
y sé que habló de sombras y de amores\\
llenándome de lágrimas y umbrías.\\
Y en este mundo largo y sin aurora\\
mis lágrimas de niño\\
eran un agua plácida y serena.\\
El mundo estaba siendo propalado,\\
y al ritmo de las tiernas alboradas\\
vi que el amor es sabio y es eterno.\\
\end{verse}

\pagebreak

\subsection{Despedida}
~ 

\begin{verse}
    
Cuando dijimos adiós\\
bajo la luna sombría,\\
sentí lo que sintió Dios\\
la noche del primer día.\\
\end{verse}

\pagebreak

\subsection{Historia de un abandono}
~ 

\begin{verse}


no te despiertes:\\
cuando te diga adiós\\
por toda nuestra casa\\
relumbrará mi voz\\
~ 

relumbrará y al despertar\\
dirás \textit{se ha ido}...\\
~ 

(habrá un aliento a mar)

\end{verse}

\pagebreak
\subsection{Un poema español}
~ 

\begin{verse}
    
en las aguas de la vida
grabé tu nombre y el mío:
\textit{olvido}.

en las aguas de la vida
cuánto he escrito, amor,
y cuánto llevó el río…!
\end{verse}

\pagebreak 

\subsection{Cementerio}
~ 

\begin{verse}
    
déjenme con la muerta\\
déjenme con el muerto\\
~ 

déjenme con la hierba\\
tierna del cementerio\\
~ 
déjenme fuera\\
de aquél mi cuerpo\\
~ 

déjenme, déjenme,\\
déjennos.\\
\end{verse}

\pagebreak
\subsection{Rajastán}
~ 

\begin{verse}
    
Desierto:
desierto de sed.
¿Adónde,
adónde hallaré
el oasis
del mundo, tu piel?

Adónde.
Adónde.

No sé.
\end{verse}

\pagebreak
\subsection{Promesa}

\begin{verse}
Será tu vientre, mujer,
la orilla: el niño la mar
donde los enamorados
saben romper a llorar.
~ 

Y, arca de toda la vida,
siempre desolado arcón,
volveré un día a llenar
de tiempo tu corazón.
\end{verse}

\pagebreak
\subsection{A una muerta}
~ 
\scriptsize
\begin{quote}
    \hspace*{\fill}    Thy word is a lamp unto my feet, 
    and a light unto my path.\\
\hspace*{\fill}—Psalms 119:105
\end{quote}

\normalsize
\begin{verse}
   
me sobrecoge una voz
me hunde una voz a lo lejos \\
~ 

una voz como de un día\\
pasado un triste elemento
~ 

«santiago: en ráfagas suaves\\
de lunas y de luceros\\,
tumba de un ansia florida
y cálida, te recuerdo»
~ 

«santiago: el día atesora\\
siempre un último destello\\
~

un último corazón,\\
rival del odio, tenemos»\\
~

amor: a su íntimo origen\\
a su remoto comienzo\\
todo al oírte quisiera\\
regresar límpido, tierno...\\
~ 

amor: acércame el soplo\\
de nuestro roce primero\\
~

pon al pie de lo más grande\\
el beso de lo pequeño\\
y, floreciéndolo, admite\\
que todo nazca de nuevo...!
\end{verse}

\pagebreak
\subsection{¿Desde cuándo?}

\begin{verse}

para qué dios fue nuestro amor un mártir?
a que ídolos enfermos se ofreció\ldots!
~ 

hoy que todas las cosas son de polvo 
vemos la sangre seca en los altares
sin discernir el turbio sacrificio\ldots
~ 

cansado, nuestro cuerpo que hoy es uno
se quiebra, se enemista, es segregado:
alguien rueda a la noche, alguien al día
~ 

y no has de preguntarte desde cuándo:
la luz que cesa siempre ha sido sombra,
la flor que muere siempre ha estado muerta

   
\end{verse}

\pagebreak
\subsection{Natalicio}
~ 

\begin{verse}
    
fue un trece de junio...\\
~ 

un horizonte de gallos\\
lloraba en la madrugada\\
su pésame cotidiano\\
~ 

mis ojos estaban puestos
en los últimos veranos,
el último hijo, la última
lluvia del último campo
~

todo lloró un padrenuestro:\\
el aire, el trébol, el álamo\\
~ 

el trigo lejos reía…\\
~ 

se me llevó hasta los brazos\\
de mi madre: estaba muerto\\
(pero aún estaba llorando)\\
~ 

«ha muerto, y era su nombre\\
apenas, casi Santiago»\\
~ 

todo fue un trece de junio,
y hallé que todo era blanco 
~

todo fue un trece de junio,
estaba muerto y tenía\\
un año menos que un año\\
\end{verse}

\pagebreak 
\subsection{Mi casa}

\scriptsize
\begin{quote}

\end{quote}
\normalsize

\begin{verse}

casas, casas como pueblos 
de desalentados odios:
rencores atardecidos
donde atardecen mis ojos
hasta que miro mi cuerpo
como un paisaje de lobos
~

casas que no hallan espacio
casas que pueblo y asombro 
con una luz que no entiendo 
para qué llevo, ni cómo\ldots
~

tumbas de un viento de enero 
que quiso ser amoroso
~

yo necesito que triunfes,
niño que, antes del polvo 
fui, pero no sigo siendo 
sino en mi sitio más hondo 
~

yo necesito que triunfes:
hay un latido remoto 
que persevera, vencido, 
pero una vez victorioso\ldots
~

entonces vendrá el olvido 
más redentor y más hondo 
~ 

mi casa tendrá la paz 
de los pausados arroyos
~

cada quien será juzgado
será el sepulcro del odio
e iremos bajo la tierra
a descansarnos de todo\ldots




\end{verse}


\pagebreak
\subsection{Al hogar familiar}

\begin{verse}
   

Desierto: siempre te quedan 
los manantiales del odio. 
~ 

(Mi cuerpo erizado y mustio
busca en las aguas su rostro.)
~

La habitación se despuebla
de cosas que ya no somos:
allí dormimos un día,
allí despertamos otro 
con un recóndito tigre
apretado entre nosotros.
~ 

Mi cuerpo es un arenal,
el roce un árido soplo.
~ 

Desierto: siempre te quedan 
los manantiales del odio.


\end{verse}

\pagebreak
\subsection{Es tu última tormenta}
~ 

\begin{verse}
    
soy hermano del diablo y en mi puño
se desploma el destino de los hombres
~

hijo de dios: te han olvidado
y el arca es relamida por las aguas
~

es tu última tormenta
\end{verse}

\pagebreak

\subsection{What if they see me}
~ 
\begin{verse}
    
What if they see me in those deserted nights\\
when sleepless and in hope of something precious\\
the moon above me crackling like a tender bread\\
I doubt the truths I’ve built my faith upon?\\
~ 

What if they see my aged demeanour\\
in the solitude of days\\
when lost among the dreams of stars\\
I fail to grasp the names of these gods I pray to?\\
~ 

What if, somehow unaccountably, they hear\\
the voices and the groans my thoughts are made of\\
the evilness, the weariness, the ignorance they’re made of?\\
~ 

What if they see me like a quiet river\\
a shoreless sea in which depths of purple blue\\
devils and gods are fused without a sorrow?\\
~ 

Would they not turn their faces, close their teary eyes with tears of hatred?\\
Would they not speak thus to their hearts:\\
turn your pristine gaze away, love not this ugly soul\\
kiss not the lips that ate the ancient apple\\
reject the ghost, the murderer of all?\\
~ 

What if they see me once, just once, the way my furniture can see me\\
when I dread alone the corridors of this abandoned home\\
free from the masquerades and foolish words\\
through which I’m who they see, they know and, sometimes, love?\\
\end{verse}

\pagebreak

\subsection{A un ídolo de piedra}

\begin{verse}
    
eres el testimonio de unas manos
que hace doscientos años fueron viejas
~

en tus imperfecciones
como en terribles cárceles sin rejas
aún persevera un hombre
~

el hombre que te condenó a la vida
de un símbolo: a la vida
de todo aquello que no tiene nombre
\end{verse}

\pagebreak 

\subsection{Sólo el amor aguarda}

\begin{verse}
sólo el amor aguarda 
no la tierra ni la osamenta cruda
~ 

no el críptico silencio de aquel odio 
que atosigó tu sangre
~ 

(lo olvidarán los que te sobrevivan)
~ 

sólo el amor aguarda:
por el amor serás juzgado


\end{verse}

\pagebreak 

\subsection{Cuento infantil}
~ 
\begin{verse}
    
—\textit{Nunca te voy a olvidar},\\
pronunció la blanca luna.\\
Conversaban con el mar\\
de su amor y su fortuna.\\
~ 

Con el pasar de las olas\\
el mar se perdió en el mar.\\
La luna, triste, fue a hablar\\
con los campos de amapolas.\\
~ 

—\textit{¿Dónde fue el mar, amapolas?},\\
pronunciaba apenas. ¡Pobre…!\\
Así conversaban sobre\\
cómo era sentirse solas.\\
~ 

— \textit{Nunca lo voy a olvidar},\\
dijo la luna otra vez,\\
y volvió llorando al mar,\\
nadie sabe bien por qué.\\
~ 

Hay quienes dicen acaso\\
que por eso está tan lejos,\\
que fue buscando el ocaso\\
(esto lo cuentan los viejos).\\
~ 

Que así del mar las espumas\\
fueron nacidas: del llanto\\
que dio a la luna quebranto\\
y la vio envuelta de brumas.\\
~ 

Pero yo la he visto a veces\\
hablar en tonos hermosos.\\
—\textit{Está hablando con los peces},\\
 dicen algunos, dudosos.\\
 ~ 

La he visto hablar en la bruma\\
y sé que no habla a las olas,\\
ni a las suaves caracolas\\
que lame y besa la espuma.\\
~ 

—\textit{Te busqué en los campos}—dice—,\\
\textit{pues te perdí con las olas. \\
Te soñé en las amapolas\\
y en las espigas te quise.}\\
~ 

—\textit{Nunca me fui, luna triste}\\
—dice alumbrado de estrellas—,\\
\textit{y nunca vos me perdiste\\
ni con las olas ni en ellas.\\
~ 

Aunque, en el plácido abismo\\
de mi universo, las olas \\
me esfumen, yo soy el mismo.}\\
Y así conversan a solas...\\
\\

\end{verse}

\pagebreak

\subsection{Todas las rosas son la de tu huerta}
~ 

\begin{quote}
   
\scriptsize
\hspace*{\fill}{La rosa, que aquí ve tu ojo exterior,\\\hspace*{\fill} florece así en Dios desde la eternidad.}\\
\hspace*{\fill}—Angelus Silesius
\end{quote}
\normalsize
\begin{verse}
    

La inmarcesible rosa de tu huerta,\\
que el curso de las albas ha erigido,\\
con pétalos de lágrimas y olvido\\
ha despertado esta mañana, muerta.\\
~ 

Todas las rosas —la que sigue abierta\\
y la que para siempre se ha perdido—\\
son una rosa indivisible y cierta,\\
como uno es lo soñado y lo vivido.\\
~ 

Y es hora de escoger: ¿vale la pena\\
derramar esas lágrimas hermosas?\\
¿No son acaso eternas estas cosas,\\
como el tiempo medido por la arena?\\
~ 

Durante el alba plácida y desierta,\\
todas las rosas son la de tu huerta.\\

\end{verse}

\pagebreak 
\subsection{Paraná}
~ 

\begin{verse}
Desde el oriente yermo la tarde se derrama
sobre la anciana orilla de un agua que delira
con ser la viva sangre de un Cristo que nos llama
con algo de tristeza y con algo de mentira…!
~

El agua enrojecida, nocturna, te proclama.
(Quisiera, vida, verte tal como Dios te mira:
llena de luz y sombra, como una diurna trama
tejida con los hilos de una luna que expira…!)
~

Recuerdo la asombrosa navaja de tu aliento,
que destejió las hebras de mi carne florida
cuando sentí en tu boca la muerte de una estrella.
~

El río, antigua lágrima de Cristo, con el viento
se lleva esas memorias, y en ellas nuestra vida…
(Hoy sé que en esta arena no hay pies que dejen huella.)
\end{verse}

\pagebreak
\subsection{En casa de mi padre}

\begin{verse}
   Los débiles cristales de esta casa 
   me devuelven a la vereda anciana.
   El patio está vacío. En la ventana 
   mi rostro se interroga. Nadie pasa 
   tras el cristal especular. No olvido 
   ver a mi padre hurgar su biblioteca
   oscuro y minucioso, ni la mueca 
   que deformó su rostro anochecido.
   Un torpe ayer de penas y de asombros,
   la vida se conjura en la memoria.
   El pórtico y el patio son escombros,
   y tratan de decir alguna historia.
   Donde vivió mi padre, la fortuna 
   dejó esta ruina trágica y serena.
   (En el terrible espejo está la luna,
   que permanece incólume y ajena.)
   Sobre mi pensamiento pesan cosas 
   que yo juzgué del sueño o del olvido,
   que trato de sentir que son hermosas,
   que no lo son y que jamás lo han sido.
   Los tristes tragaluces ya han urdido
   su lobreguez arcana. En la veranda 
   la noche inescrutable y gris ablanda 
   todas las cosas que él había erigido.
\end{verse}

\pagebreak
\subsection{Algo}

\begin{verse}
    
Jamás en el crepúsculo he de verte,
ni en el sudario de una noche triste
fundiéndote con todo lo que existe.
Jamás serás mi pérdida o mi suerte.
~ 

Y, sin embargo, en la tiniebla inerte
hay algo de tu imagen que subsiste;
algo que nunca, pero nunca fuiste 
sino después de tu ilusoria muerte.
~ 

Algo que intuyo con sigilo eterno
en las intrascedencias de los días;
en esta luz de cáscaras sombrías
que nos presagia el odio del invierno.
~ 

En todo te sospecho, vaga sombra.
En nada tu memoria no me asombra.


\end{verse}

\pagebreak
\subsection{La aurora de tu muerte}

\begin{verse}

la ayer estéril tierra de mis sueños 
siembras desde la aurora de tu muerte
~ 

tu ausencia como un símbolo siniestro
procura enriquecerla hasta que nazca 
tu rostro, que ya es menos que el olvido
~

y a este mundo de anhelos y arquetipos
vengo a llenar de furia la copa tu vida
vengo a morder el alba como un perro 
~ 

vengo a salvarme de tu propia muerte,
vengo a decirme, justo en el instante
en que tu rostro quiere entretejerse, 
\textit{ya vuelve, amigo mío, estás soñando\ldots}



\end{verse}

\pagebreak
\subsection{No love goes to waste}
~ 

\begin{verse}
    
Here in the night below the moon\\
I’ve cried that love is vain, and every word\\
we whisper to our lover’s ears is wind forever gone.\\
But no love goes to waste, I now suspect—\\
my eyes glow with the light of every orb\\
that on this ancient sky has ever slept\\
as I pronounce with weakened words the truth.\\
O, no love goes to waste, I do not doubt;\\
it is forever kept away from us, cruel passing dust,\\
warm in the heart of those who never knew\\
nor will know that our whispers never died.\\
And as we slowly fade into the baleful mist,\\
you stranger have to know you’ll light your way\\
if you remember this I say, not fool nor wise:\\
that no love goes to waste, no memory dies.\\
\end{verse}








\end{document}



