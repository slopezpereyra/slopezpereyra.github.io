\documentclass[a4paper, 12pt]{article}

\usepackage[utf8]{inputenc}
\usepackage[T1]{fontenc}
\usepackage{textcomp}
\usepackage{amssymb}
\usepackage[cal=boondoxo]{mathalfa}
\usepackage{mlmodern}
\usepackage{amsmath, amssymb}
\usepackage{tikz}
\usetikzlibrary{arrows.meta, positioning}
\newtheorem{problem}{Problem}
\newtheorem{example}{Example}
\newtheorem{lemma}{Lemma}
\newtheorem{theorem}{Theorem}
\newtheorem{problem}{Problem}
\newtheorem{example}{Example} \newtheorem{definition}{Definition}
\newtheorem{lemma}{Lemma}
\newtheorem{theorem}{Theorem}
\DeclareMathAlphabet{\mathcal}{OMS}{cmsy}{m}{n}

\begin{document}


Sea PA la aritmética de Peano. Dada una fórmula $\psi$, usamos $\left<\psi
\right>$ para denotar su número de Gödel.

~

El lema del punto fijo de Gödel establece que si $\varphi(x)$ es una fórmula con
una variable libre, entonces existe una sentencia $\psi$ tal que 

\begin{equation*}
  \text{PA} \vdash \psi \leftrightarrow \varphi( \left< \psi \right> )
\end{equation*}

Es decir, $\psi$ es una sentencia que dice "$\varphi$ es verdadera para el
número de Gödel de $\psi$". Es decir, $\psi$ es auto-referencial.

~ 

Que existe auto-referencialidad es un quilombo bárbaro porque nos permite
construir sentencias extrañas. Por ejemplo, si $\varphi(x)$ fuera el predicado 
"$x$ es el número de Gödel de un axioma de PA", entonces

\begin{equation*}
  \psi = \text{"$\psi$ es un axioma de PA"}
\end{equation*}

Si queremos hacer la auto-referencialidad más obvia, podemos pensar que $\psi$
dice "soy un axioma de PA". Esta sentencia es falsa, pero pueden construirse
sentencias auto-referenciales verdaderas. Por ejemplo, 

\begin{equation*}
  \varphi(x) := \exists \phi \in \mathcal{S} . \left< \phi \right> \neq x
\end{equation*}

con $\mathcal{S}$ el conjunto de sentencias, tiene la sentencia asociada 

\begin{equation*}
  \psi \leftrightarrow \varphi(\left< \psi \right>)
\end{equation*}

que equivale a "existe una sentencia cuyo número de Gödel es distinto al mío".

~ 

El verdadero horror surge cuando tomamos una teoría consistente $T$ que es o
bien PA o bien una extensión de PA, y cuyos axiomas son recursivamente
enumerables. Sea $\mathcal{P}_T(x)$ la fórmula "$x$ es el número de Gödel de una
sentencia deducible en $T$". Por la aritmetización de la sintaxis,
$\mathcal{P}_T(x)$ es expresable en el lenguaje de la aritmética. Entonces, por
el teorema del punto fijo de Gödel,

\begin{equation*}
  \text{PA} \vdash \psi \leftrightarrow \neg\mathcal{P}_T\big( \left< \psi \right> \big)
\end{equation*}

Es decir, $\psi$ dice de sí misma que es no deducible de $T$. Si $\psi$ fuera
falsa, tendríamos que $\psi$ es deducible de $T$, con lo cual el teorema de
corrección nos dice que $\psi$ es verdadera. Esto contradice la hipótesis de que
$T$ es consistente. Por lo tanto, $T \vDash \psi$. 

$\therefore $ (\textbf{Teorema de incompletitud.}) Existen
sentencias verdaderas que no son demostrables en $T$.

\pagebreak 

\section{Introducción}

Informalmente, un \textit{quine} es un programa sin input cuyo output es su
propio código. Se trata de un caso particular de una clase más general de
fenómenos caracterizados por la \textit{auto-referencialidad indirecta}. Decimos
que existe auto-referencialidad indirecta cuando una sentencia se denota a sí
misma de manera no-explícita a través de un código u otra forma adecuada de
representación. 

~ 

Los \textit{quines} tienen algunas aplicaciones prácticas, por ejemplo en
producción de virus, pero en rigor su valor difícilmente exceda el de una
curiosidad teórica. A pesar de esto, a nuestro juicio tienen un valor
ilustrativo para nada despreciable: ejemplifican de manera concreta una
aplicación del teorema de la recursión de Kleene. Dicho teorema no es
insignificante: como veremos más adelante, guarda una relación directa con la
currificación funcional y con la posibilidad de transformar programas
sintácticamente sin alterar su semántica. En la medida en que los
\textit{quines}, en su inútil excentricidad, sirvan para ilustrar
el teorema de la recursión de Kleene, merecen nuestra apreciación.

\section{Auto-referencialidad directa e indirecta}

~ 

La auto-referencialidad, en todas sus formas, es un problema clásico en la
lógica. La auto-referencialidad \textit{directa}, en la que una sentencia habla
explícitamente de sí misma, cuenta con mayor antigüedad y ha recibido un
tratamiento más extensivo. Sus ejemplos más paradigmáticos 
son la paradoja del mentiroso, la paradoja de Epiménides, y la menos conocida
paradoja de Curry. La última, por ejemplo, permite demostrar cualquier sentencia
$\varphi$ usando reglas deductivas válidas. Considere por ejemplo la sentencia 


\begin{equation*}
  \varphi := \text{El profesor Pagano nos pondrá un diez.}
\end{equation*}

y la siguiente sentencia asociada:
\begin{equation*}
  \psi := \text{Si esta sentencia es verdad, el profesor Pagano nos pondrá diez}.
\end{equation*}

Veamos que la existencia de esta sentencia auto-referencial nos permite
demostrar $\varphi$. En primer lugar, notemos que si asumimos el antecedente
"esta sentencia es verdadera", se sigue inmediatamente el consecuente $\varphi$.
Por lo tanto, $\psi$ es verdadera. Pero como $\psi$ es verdadera, y $\psi$ dice
que de ser verdadera se sigue $\varphi$, tenemos que $\varphi$ es verdadera.
$\therefore $ El profesor Pagano nos pondrá diez.

~ 

Toda sentencia, verdadera o falsa, puede demostrarse de este modo, con lo cual
el hecho de que $\varphi$ sea realmente verdadera es incidental\footnote{Último
chiste del artículo.}. La solución de la paradoja de Curry no es clara: la
conclusión es lógicamente impecable y se deriva estrictamente de la
auto-referencialidad directa de $\psi$. Pero, habiendo ejemplificado los
problemas que acarrea la auto-referencialidad directa, ¿qué hay de la
auto-referencialidad indirecta? ¿Qué rol juega en la lógica en general y, más
concretamente, la computación teórica?

~

El ejemplo más paradigmático de auto-referencialidad indirecta, que a la postre
muestra la importancia crucial de este concepto en la computación teórica, es el
teorema de incompletitud de Gödel.  Si $T$ es la aritmética de Peano (PA) o una
extensión consistente de ella, entonces el teorema de incompletitud garantiza la
existencia de una sentencia $\psi$ tal que


\begin{equation*}
  \text{PA} \vdash \psi \leftrightarrow \neg\mathcal{P}_T\big( \left< \psi \right> \big)
\end{equation*}

donde $\left<\psi \right>$ es el número de Gödel de la sentencia $\psi$ y
$\mathcal{P}_T(x)$ es el predicado "$x$ es deducible de la teoría $T$". En otras
palabras, $\psi$ es una sentencia que dice "no soy demostrable en $T$". La
auto-referencialidad es indirecta en la medida en que $\psi$ es equivalente a
una sentencia que contiene $\left<\psi \right>$.

~ 

La auto-referencialidad indirecta parece evadir las paradojas asociadas a la
auto-referencialidad directa. Constituye, por lo tanto, una herramienta lógica
poderosa. Para comprender por qué un \textit{quine} es una forma de
auto-referencialidad indirecta, debemos presentarlo teóricamente como un caso
particular del teorema de la recursión de Kleene.

\section{Kleene}

Sea $\mathcal{F}$ el conjunto de funciones parciales computables. La
aritmetización de la sintaxis dada por Gödel establece una relación biyectiva
entre el conjunto de programas (en el sentido de máquinas de Turing u otro modelo equivalente)
y los números naturales. Por lo tanto, como $\mathcal{F}$ es recursivamente
enumerable, podemos dar la enumeración 

\begin{equation*}
  \varphi_1, \varphi_2, \ldots
\end{equation*}

tal que $\varphi_k$ es la función computada por el programa $k$. 

\subsection{Currificación y el teorema $S_n^m$}

Un resultado significativo dado por Kleene es que la currificación de una
función computable es computable. Informalmente, esto significa que existe un
programa tal que, dado otro programa de $n$ variables, devuelve un programa de $1
\leq m < n$ variables que es funcionalmente equivalente al programa original
con el valor de las primeras $n - m$ variables fijas. Usualmente, este resultado
es conocido como el teorema $S_n^m$.

~

Más formalmente, si $\varphi_p(u, x)$ es la función de dos
argumentos computada por $p$, existe una función computable $S(k, p)$ tal que
$\varphi_{S(k, p)}(x) = \varphi_p(k, x)$. Este resultado, que puede demostrarse
dentro del paradigma de Turing o, tal vez más intuitivamente, utilizando un
paradigma imperativo equivalente, se generaliza para funciones $\varphi(x_1, x_2,
\ldots, x_n)$ de $n$ argumentos. 

\begin{theorem}[Teorema $S_n^m$]
  Sea $\mathcal{P}$ un programa arbitrario. Existe una función primitiva
  recursiva $S_n^m : \mathbb{N}^m \times \mathbb{P}$ tal que 

  \begin{equation*}
    \varphi_\mathcal{P}(x_1, \ldots, x_n, y_1, \ldots, y_m) \simeq
    \varphi_{\mathcal{S}(y_1,\ldots, y_m, \mathcal{P})} \left(
    x_1, \ldots, x_n) \right) 
  \end{equation*}

\end{theorem}


\small
\begin{quote}

\textbf{Prueba.} Interpretemos $\mathcal{P}$ como una concatenación de
instrucciones en un paradigma imperativo equivalente al paradigma de Turing.
Convengamos que las variables de un programa son un
conjunto enumerable $v_1, v_2, \ldots$, y que
$\varphi_\mathcal{P}(x_1,\ldots, x_n)$ se corresponde con ejecutar $\mathcal{P}$
desde el estado en que las variables $v_1, \ldots, v_n$ toman los valores $x_1,
\ldots, x_n$. Sea
$\mathcal{Q}_i : \mathbb{N} \to \mathbb{P}$ la función tal que
$\mathcal{Q}_i(x)$ devuelve el programa que asigna a la variable $i$ el valor
$x$. Es fácil ver que $\mathcal{Q}_i$ es primitiva recursiva. Entonces,
definimos 

\begin{equation*}
  S_n^m(y_1, \ldots, y_n, \mathcal{P}) := \mathcal{Q}_{n+1}(y_1); \ldots;
  \mathcal{Q}_{n+m} (y_m); \mathcal{P}
\end{equation*}

donde $\alpha;\beta$ es la concatenación de las instrucciones $\alpha, \beta$.
La concatenación de palabras es primitiva recursiva y por lo tanto $S_n^m$ es
primitiva recursiva. Es evidente que $S_n^m(y_1, \ldots, y_n, \mathcal{P})$ es el programa
que ejecuta $\mathcal{P}$ desde un estado en que las variables $n+1, \ldots, n+m$ 
toman los valores $y_1, \ldots, y_m$. Por lo tanto, 

\begin{equation*}
  \varphi_{S_n^m(y_1,\ldots, y_m, \mathcal{P})}(x_1,\ldots, x_n) \simeq
  \varphi_{\mathcal{P}}(x_1, \ldots, x_n, y_1, \ldots, y_m) ~ \blacksquare
\end{equation*}



\end{quote}
\normalsize


El resultado teórico que más nos interesa por su relación con los \textit{quines}
es el teorema de la recursión de Kleene, cuya demostración depende del teorema
$S_n^m$. El teorema de la recursión garantiza que, para toda
$f \in \mathcal{F}$, existe un programa $e$ tal que $e$ y $f(e)$ computan la
misma función, o equivalentemente $\varphi_e(x) \simeq \varphi_{f(e)}(x)$. El
teorema se demuestra constructivamente. 

\begin{theorem}[Teorema de la recursión de Kleene]
  Sea $f \in \mathcal{F}$. Entonces existe $e$ tal que $\varphi_e(x) \simeq
  \varphi_{f(e)}(x)$.
\end{theorem}


\small
\begin{quote}

\textbf{Prueba.} Sea 

$$\varphi_{f(S(u, u))}(x) =: g(u, x)$$ 


Como $g$ es computable, existe un $n$ tal que $g(u, x) = \varphi_n(u, x)$. Si
definimos $e := S(n, n)$, resulta entonces

\begin{align*}
  \varphi_e(x) 
  &=\varphi_{S(n, n)}(x) \\ 
  &=\varphi_n(n, x) \\ 
  &=g(n, x) \\ 
  &=\varphi_{f(S(n, n))}(x) \\ 
  &=\varphi_{f(e)}(x)
\end{align*}
~

Por lo tanto, los programas $e$ y $f(e)$ computan la misma función. 

\end{quote}
\normalsize


\begin{figure}[ht]
  \centering
  \begin{tikzpicture}[>=Stealth, node distance=2.5cm, every node/.style={align=center}]
    % Nodes
    \node (gux) [draw, circle] {\( g(u, x) \)};
    \node (Snx) [draw, circle, right=of gux] {\( \varphi_n(u, x) \)};
    \node (e) [draw, circle, right=1.8cm of Snx] {\( e := S(n,n) \)};
    \node (fdef) [draw, circle, above=of gux] {\( \varphi_{f(S(u, u))}(x) \)};
    \node (f) [draw, rectangle, below=of e] {\( f(e) \)};
    \node (phife) [left=1cm of f] {\( \varphi_{f(e)} \)};
    \node (phie) [left=5cm of phife] {\( \varphi_e \)};

    % Arrows
    \draw[->] (fdef) -- node[left] {def} (gux);
    \draw[->] (gux) -- node[above] {for some $n$} (Snx);
    \draw[->] (Snx) -- node[above] {} (e);
    \draw[->] (e) -- node[right] {\(f\)} (f);
    \draw[->, dashed] (e) -- (phie);
    \draw[->, dashed] (f) -- (phife);
    
    % Equal functions
    \draw[<->, thick, double equal sign distance] (phie) -- (phife)
      node[midway, below] {\( \varphi_e = \varphi_{f(e)} \)};
  \end{tikzpicture}
  \caption{Diagrama ilustrando la construcción del punto fijo en el teorema de
  la recursión de Kleene}
  \label{fig:kleene-recursion}
\end{figure}

~

Intuitivamente, es importante tomar del teorema de la recursión de
Kleene la idea de que podemos transformar cualquier programa en uno nuevo que,
aunque difiera en su sintaxis, compute la misma función. Más aún, dicha
transformación y su construcción son computables, y constituyen una forma de
auto-referencialidad indirecta: el programa $f(e)$ tiene como input un programa
funcionalmente equivalente (es decir, idéntico en términos de computabilidad),
pero no necesariamente igual.

~
~

El teorema está intrínsecamente relacionado con el concepto de \textit{quine}
porque garantiza que un programa puede operar sobre una referencia de sí mismo.
Si bien $e$ y $f(e)$ computan la misma función, difieren en el hecho de que
$f(e)$ tiene "conocimiento" de $e$. El teorema es lo suficientemente fuerte como
para garantizar la existencia de \textit{quines} en todo modelo de computación
equivalente al de Turing. 

\begin{theorem}[Existencia de \textit{quines}]
  Si $\mathcal{Q}(e)$ es un programa que imprime $e$, por el teoremea de la
  recursión de Kleene existe $e$ tal que $\mathcal{Q}(e) = e$. Por lo tanto, $e$
  es un programa cuyo output es sí mismo.
\end{theorem}




























\end{document}



