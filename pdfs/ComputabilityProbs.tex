\documentclass[a4paper, 12pt]{article}

\usepackage[utf8]{inputenc}
\usepackage[T1]{fontenc}
\usepackage{textcomp}
\usepackage{amssymb}
\usepackage{newtxtext} \usepackage{newtxmath}
\usepackage{amsmath, amssymb}
\newtheorem{problem}{Problem}
\newtheorem{example}{Example}
\newtheorem{lemma}{Lemma}
\newtheorem{theorem}{Theorem}
\newtheorem{problem}{Problem}
\newtheorem{example}{Example} \newtheorem{definition}{Definition}
\newtheorem{lemma}{Lemma}
\newtheorem{theorem}{Theorem}

\begin{document}

\section{Comentario preliminar}

Usamos $\varphi$ para denotar un elemento arbitrario del alfabeto $\{\#, *\}$. Usamos la notación $f \sim (n, m, \varphi)$ para decir "$f$ es de
tipo $(n, m, \varphi)$".

Muchos de estos problemas están resueltos en inglés. Esto es una desgracia
porque es poco democrático. La razón es que los ejercicios fueron hechos con Vim
+ VimTex, y los atajos de teclado de Vim y LaTex son muy imprácticos si el
teclado está configurado en castellano. Si hubiera escrito estos problemas con
la intención de compartirlos, habría resuelto esta incompatibilidad. Lo cierto
es que los escribí sin intención de compartirlos. 

No asuma que los ejercicios son correctos. Allí donde lo sean, no asuma que la
solución dada es la más elegante. Fueron hechos por un estudiante, nada más.
Estimo que otros estudiantes podrán sacar provecho de ellos.

\section{Guia 2 : Infinituplas}


\begin{problem}
    Demostrá por inducción: 
    Para todo $x \in \mathbb{N}$ hay una infinitupla única $\overrightarrow{s}
    \in \omega^{[\mathbb{N}]}$ tal que

    \begin{align*}
        x = \prod_{i=1}^{\infty} pr(i)^{s_i}
    \end{align*}
\end{problem}

\small
\begin{quote}




El caso base es trivial. Supongamos que la afirmación se cumple para todo $n \leq k$. El
teorema fundamental de la aritmética asegura que $k + 1 = p_1 \cdot \ldots \cdot
p_m$, donde $p_i$ es primo. Supongamos que la factorización anterior está ordenada (es decir,
$p_{j + 1} > p_j$ para todo $j \in [1, m]$). Entonces $k + 1 = p_m \cdot q$ con $q =
p_1 \cdot \ldots \cdot p_{m - 1}$.

\footnotesize
\begin{quote}
\textit{Subdemostración.} Demostraremos $k + 1 = p_m \cdot q \Rightarrow q \leq k$.
Supongamos que la premisa se cumple y la consecuencia no. Dado que $q > k$, tenemos
$q \cdot x > k + 1$ para todo $x > 1$. Entonces $q \cdot x > k + 1$ para todo $x$
que sea primo. Entonces $q \cdot p_m \neq k + 1$, lo cual es una contradicción. Entonces,
si $k + 1 = q \cdot p_m$, tenemos $q \leq k$. $\blacksquare$
\end{quote}
\small

Dado que $q \leq k$, mediante la hipótesis inductiva, $q$ toma la forma productoria
del teorema anterior. Entonces $k + 1 = q \cdot pr(j)$ donde $pr(j) = p_m$. Entonces el
teorema se cumple para todo $n \in \mathbb{N}$.

\end{quote}
\normalsize

\begin{problem}
    Prove that $S = \{(x, @^x) : x \equiv 0 \mod 2\}$ is $\{@\}$-effectively
    enumerable.
\end{problem}


\small
\begin{quote}

\textit{Prueba corta.} Se hace demostrando que $S$ es $\Sigma$-efectivamente
computable. Esto es fácil: se da un procedimiento que verifica, dado un input
$(x, \alpha)$, si $x$ es par y si $\alpha = @^x$. Demostrando que es
$\Sigma$-efectivamente computable, demostramos que es $\Sigma$-enumerable.

\textit{Prueba larga}. Damos explícitamente el programa que enumera a $S$.

Lo hacemos notando que $*^{\leq}$ es $\Sigma$-efectivamente computable bajo
cualquier orden $\leq$ de $\Sigma$, ya que la función es bastante algorítmica por
naturaleza. (Si no se convence, escriba el procedimiento efectivo de esta
función.) Sea $\mathbb{P}_{\text{número a palabra}}$ el procedimiento que, dado
un valor $x \in \omega$, calcula $*^{\leq}(x)$. Entonces definimos $\mathbb{P}$
como el procedimiento que tomando un valor $x \in \omega$ hace lo siguiente:

\footnotesize
\begin{quote}
\textit{(0)} Computa $(x)_1, (x)_2$.

\textit{(1)} Usa $\mathbb{P}_{\text{número a palabra}}$ para calcular $*^{\leq}(x_2)$
y lo guarda en $\alpha$. 

\textit{(2)} Comprueba si $(x)_1$ es par; si lo es continúa, si no lo es va a
\textit{(5)}

\textit{(3)} Comprueba si $\alpha = @^{(x)_1}$. Si lo es, continúa, si no lo es
va a \textit{(5)} 

\textit{(4)} Devuelve $(x_1, \alpha)$ y termina. 

\textit{(5)} Devuelve $(0, \varepsilon)$ y termina.
\end{quote}
\small

\textit{Ejemplo.} Considera $\mathbb{P}(6)$. En \textit{(0)}, esto asigna $x_1 = 1$,
$x_2 = 1$.  La primera palabra en $\Sigma$ es $@$. El programa encuentra la
tupla $(1, @$. Como $1$ es impar va a \textit{(5)} y devuelve $(0, \epsilon)$.

Considera la tupla $(2, @@@@)$. Sabemos que existe algún $x \in \omega$ tal que
$\mathbb{P}(x) = (2, @@@@)$ (aquí uso la notación matemática de manera flexible). Dado que $@@@@$ es la cuarta palabra en $\Sigma$, $x$ es tal que
$x = \langle 2, 4, (x)_3, (x)_4, \ldots \ldots\rangle $. Por ejemplo, $2^2 + 3^4 =
85$ o $2^2 + 3^4 + 5^{17} = 762939453210$ satisfarán esto.

\end{quote}
\normalsize

\section{Guia 4}

\begin{problem}
Si $M$ es una máquina de Turing, entonces $\delta$ es una función $\Sigma$-mixta.
\end{problem}

\small
\begin{quote}

Se dice que una función es una función $\Sigma$-mixta si $\mathcal{D}_f \subseteq
\omega^n \times \Sigma^{m}$ para algunos $n, m \geq 0$ y $\mathcal{I}_f \subseteq
\omega$ o $\mathcal{I}_f \subseteq \Sigma^{}$. La función $\delta$ no satisface
ninguna de estas propiedades; por ejemplo, su dominio es un conjunto de estados
$Q \times \Gamma \not\subseteq \Sigma^{*m}$.
\end{quote}
\normalsize



\section{Guías 5 y 6}

\begin{problem}
    Encuentre funciones que definan recursivamente a $R = \lambda t \left[  2^t
    \right]$.
\end{problem}


\small
\begin{quote}

Seamos claros con los tipos. Pues $R \sim (1, 0, \#)$ y la recursión se hará
claramente sobre una variable numérica, debemos encontrar $f \sim (0, 0, \#), g
\sim (2, 0, \#)$ tales que $R(0) = f$ y $R(t + 1) = g\left( R(t), t \right) $.
Evidentemente $R(0) = 1 \Rightarrow f = C_1^{0, 0}$. Puesto que $R(t + 1) = 2^{t
+ 1} = 2^t \times 2$ tenemos que $g = \lambda x \left[  2\cdot x \right] \circ
\left[ p_1^{2, 0} \right] $. 

\begin{quote}
    \textit{Observación.} Aunque $g$ involucra, a fines prácticos, una sola variable
    numérica, la definimos de modo tal que su dominio es $\omega^2$. Esto es
    para respetar los tipos exigidos por la recursión primitiva.
\end{quote}

\end{quote}
\normalsize

\begin{problem}
    Lo mismo para $R = \lambda t \left[ t!  \right]$.
\end{problem}
   

\small
\begin{quote}

Los tipos de $f$ y $g$ serán igual que en el ejercicio anterior. Pero como esta
recursión sí involucra al factor $t$, el segundo argumento de $g$ ya no será
superfluo. Es fácil ver que $f = C_{1}^{0, 0}$. Dado que $R(t+1) = t!(t+1)$
tenemos que 

\begin{align*}
    g = \lambda xy \left[ x\cdot y  \right] \circ \left[ p_1^{2, 0}, Suc \circ
    p_2^{2, 0} \right] 
\end{align*}

\begin{quote}
    (Recuerde que para recursión de función numérica sobre variable numérica,
    requerimos $R(t+1, \vec{x}, \vec{\alpha}) = g(R(t), t, \vec{x},
    \vec{\alpha})$).
\end{quote}


\end{quote}
\normalsize

\begin{problem}
    Lo mismo para $R = \lambda t x_1 \alpha_1 \alpha_2 \left[ t \cdot x_1  \right]$.
\end{problem}

Seamos rigurosos con los dominios y observemos que la $f$ y la $g$ buscadas son
tal que $f \sim (1, 2, \#), g \sim  (3, 2, \#)$. Es evidente entonces que  $f =
C_0^{1, 2}$. Pues $R(t + 1, x, \alpha, \beta) = t \cdot x_1$ tenemos simplemente
que 

\begin{align*}
    g = \lambda xyz\alpha\beta \left[ x \cdot y  \right] \circ \left[ p_2^{3,
    2}, p_3^{3, 2} \right] 
\end{align*}

\begin{problem}
    Sea $\Sigma = \left\{ @, !, ? \right\} $. Encuentre $f, g$ tales que $R(f,
    g) = \lambda t x_1 \left[ !@!!!!?^t  \right]$.
\end{problem}


\small
\begin{quote}

Pues hacemos recursión sobre una variable numérica de $R(f, g) \sim (2, 0, *)$, requerimos
que $f \sim  (1, 0, *), g \sim (2, 1, *)$. Observe que $R(f, g)(0, x_1) =
!@!!!!\varepsilon $. Luego $f = C_{!@!!!!}^{1, 0}$. Observe que $R(t + 1, x) =
!@!!!!?^t ?^{t+1} = R(t) ?^{t+1}$. Luego 

\begin{align*}
    g = \lambda \alpha\beta \left[ \alpha\beta \right]  \circ \left[ p_3^{2, 1},
    \lambda \alphax \left[  \alpha^x  \right] \circ \left[ C_?^{2, 1}, p_1^{2, 1} \right] \right]
\end{align*}

Es fácil observar, reemplazando las variables, que

\begin{align*}
    &\lambda \alpha\beta \left[ \alpha\beta \right]  \circ \left[ p_3^{2, 1},
    \lambda  x\alpha \left[  \alpha^x  \right] \circ \left[Suc \circ p_1^{2, 1},  C_?^{2, 1},  \right] \right] \left( t, x, R(t) \right) =R(t) ?^{t + 1}
\end{align*}

\end{quote}
\normalsize


\begin{problem}
    Si $\Sigma = \left\{ @, !, ? \right\} $, encuentre $f, \mathcal{G}$ tales
    que $R(f, \mathcal{G}) = \lambda \alpha_1 \alpha  \left[ |\alpha|_1 +
    |\alpha|_{@}  \right]$
\end{problem}


\small
\begin{quote}

Otra vez seamos explícitos con los dominios. Pues $R(f, g) \sim (0, 2, \#)$
tenemos $f \sim (0, 1, \#), g \sim (1, 2, \#)$.

Es evidente que $R(f, \mathcal{G}), \alpha_1, \epsilon) = |\alpha|$. Luego $f =
\lambda \alpha \left[ |\alpha|  \right]$. Veamos que 

\begin{align*}
    R(\alpha_1, \alpha a) = \begin{cases}
        R(\alpha_1, \alpha) & a \neq @ \\ 
        R(\alpha_1, \alpha) + 1 & a = @
    \end{cases}
\end{align*}

Tomando la familia indexada de funciones 
$\mathcal{G} = \left\{ (!, p_1^{1, 2} ), (?, p_1^{1, 2}), 
    (@, Suc \circ p_1^{1, 2} )\right\}  $, obtenemos efectivamente que $R(f,
    \mathcal{G})$. 

\end{quote}
\normalsize

\begin{problem}
    Encuentre $f, \mathcal{G}$ tales que $R(f, \mathcal{G}) = \lambda
    \alpha_1\alpha \left[ \alpha_1 \alpha  \right]$
\end{problem}


\small
\begin{quote}

Evidentemente, $f = \lambda \alpha \left[ \alpha  \right]$. Pues $R(f,
\mathcal{G})(\alpha_1, \alpha a) = \alpha_1 \alpha a$, observamos que
$\mathcal{G} = \left\{ a \in \Sigma : (a, d_a \circ p_3^{0, 3}) \right\}  $.
Entonces, es evidente que

\begin{align*}
    R(f, \mathcal{G})(\alpha_1, \alpha a) &= \mathcal{G}_a \left( \alpha_1,
    \alpha, R(f, \mathcal{G})(\alpha_1, \alpha) \Right) \\ 
                                          &= (d_a \circ p_3^{3, 0)}) \left(
                                          \alpha_1, \alpha, R(f,
                                      \mathcal{G})(\alpha_1, \alpha) \right)  \\ 
                                          &=R(f, \mathcal{G})(\alpha_1, \alpha)a
\end{align*}

Por ejemplo, $R(f, \mathcal{G})(!?!, ?@) = R(f, \mathcal{G})(!?!, ?) @ = (R(f,
\mathcal{G})(!?!, \varepsilon )?)@ = ((!?!)?)@ = !?!?@$.

tal como deseábamos.

\end{quote}
\normalsize

\begin{problem}
    Demuestre que $\mathcal{F} = \lambda xy\alpha\beta \left[ \alpha^x = \beta  \right]$ es
    $\Sigma$-p.r. 
\end{problem}


\small
\begin{quote}

    Cuidado con los dominios: $\mathcal{D}_\mathcal{F} = \omega^2 \times
    \Sigma^{*}^2$, aunque la variable $y$ de la expresión lambda no sea
    utilizada.
Es fácil ver que 

\begin{align*}
    \mathcal{F} = \lambda \alpha\beta \left[  \alpha = \beta  \right] \circ
    \left[ \lambda x\alpha \left[  \alpha^x \right] \circ \left[ p_1^{2, 2},
    p_3^{2,2} \right] , p_4^{2, 2}\right] 
\end{align*}

\end{quote}
\normalsize

\begin{problem}
    Demuestre que el conjunto $ S = \left[ (x, y, \alpha, \beta, \gamma) \in
    \omega^2 \times \Sigma^{*}^3 : x \leq |\gamma| \right] $ es $\Sigma$-p.r. 
\end{problem}

Observe que 

\begin{align*}
    \chi_S^{\omega^2 \times \Sigma^{*}^3} = \lambda xy \left[ x \leq y  \right]
    \circ \left[ p_1^{2, 3}, \lambda \alpha \left[ |\alpha|  \right] \circ
    p_5^{2, 3} \right] 
\end{align*}

Puesto que $\lambda xy \left[ x \leq y  \right]$ es $\Sigma$-p.r. y $\lambda
\alpha \left[ |\alpha|  \right]$ también, $\chi_S^{\omega^2 \times
\Sigma^{*}^3}$ es $\Sigma$-p.r. 

~

$\therefore $ $S$ es $\Sigma$-p.r. 

\begin{problem}
    Sea $\Sigma = \left[ @, ? \right] $. Demuestre que 

    \begin{align*}
        f :  \left\{ (x, y, \alpha) : x \leq y \right\}    &\mapsto \omega \\ 
        (x, y, \alpha) &\mapsto  \begin{cases}
            x^2 & |\alpha| \leq y \\ 
            0 & |\alpha| > y
        \end{cases}
    \end{align*}

    es $\Sigma$-p.r. 
\end{problem}


\small
\begin{quote}

Sean 

\begin{align*}
    S_1 &= \left\{ (x, y, \alpha) \in \omega^2 \times \Sigma^{*} : x \leq y
\land |\alpha| \leq y \right\} \\ 
        S_2 &= \left\{ (x, y, \alpha) \in \omega^2 \times \Sigma^{*} : x \leq y
\land |\alpha| > y \right\} 
\end{align*}

Evidentemente, $S_1 \cap S_2 = \emptyset$. Es claro que cada conjunto
corresponde a uno de los casos de $f$, y que $S_1 \cup S_2 = \mathcal{D}_f$.

Ahora bien, la función $f_1 := \lambda xy\alpha \left[ x^2  \right]$ es
evidentemente $\Sigma$-p.r. Lo mismo aplica a la función $f_2 := C_0^{2, 1}$. Más aún,
es fácil probar que $S_1, S_2$ son $\Sigma$-p.r. (esto lo dejamos). Luego,
puesto que la restricción de una función $\Sigma$-p.r. a un dominio
$\Sigma$-p.r. es a su vez una función $\Sigma$-p.r., tenemos que $f_1_{\mid S_1},
f_2_{\mid S_2}$ son $\Sigma$-p.r.  Luego $f = f_1_{\mid S_1} \cup f_2_{\mid S_2}$ es $\Sigma$-p.r. 

\end{quote}
\normalsize

\begin{problem}
    Pruebe que la función $\lambda x x_1 \left[  \sum_{t=1}^{t=x} Pred(x_1)^{t}
    \right]$ es $\Sigma$-p.r. 
\end{problem}


\small
\begin{quote}

\textit{(1)} Evidentemente, $\lambda xy \left[ Pred(x)^y  \right] = \lambda xy
\left[ x^y  \right] \circ \left[ Pred \circ p_1^{2, 0}, p_2^{2, 0} \right] $ es
$\Sigma$-p.r. 

\textit{(2)} Considere la función $G := \lambda xy x_1 \left[ \sum_{t=x}^{t=y}
Pred(x_1)^t  \right]$. Pues $Pred(x_1)^t$ es $\Sigma$-p.r. sabemos que $G$ es
$\Sigma$-p.r. Evidentemente, la función del ejercicio es 

\begin{align*}
    G \circ \left[ C_1^{2, 0}, p_1^{2, 0}, p_2^{2, 0} \right] 
\end{align*}

Luego es $\Sigma$-p.r. 

\end{quote}
\normalsize

\begin{problem}
    Lo mismo para $\mathcal{F} := \lambda xyz \alpha \beta \left[
    \mathop{\subset}_{t=3}^{t=z+5} \alpha^{Pred(z) \cdot t}
\beta^{Pred(Pred(|\alpha|))}   \right]$
\end{problem}

Observe que $\mathcal{D}_{\mathcal{F}} = \left\{ (x, y, z, \alpha, \beta) \in
\omega^3 \times \Sigma^{*}^2 : z \geq 1 \land |\alpha| \geq 2 \right\} $, pues
la función $Pred$ no está definida para el valor cero.  

\textit{(1)} Observe que 

\begin{align*}
    f_1 &:= \lambda xy\alpha\beta \left[ \alpha^{Pred(x) y} \right] = \lambda x\alpha \left[ \alpha^x
    \right] \circ \left[ \lambda xy \left[ Pred(x).y \right] \circ [ p_1^{2, 2}, p_2
    ], p_3^{2, 2} \right]  \\ 
        f_2 &:= \lambda xy\alpha\beta \left[  \beta^{Pred(Pred(|\alpha|))}
        \right] = \lambda x\alpha \left[  \alpha^x \right] \circ \left[ Pred
        \circ \left[ Pred \circ \left[ \lambda \alpha \left[ |\alpha|  \right]
    \circ p_3^{2, 2} \right]  \right], p_4^{2, 2}  \right] 
\end{align*}

Luego 

\begin{align*}
    f := \lambda xy\alpha\beta \left[ f_1(x, y, \alpha, \beta) f_2(x, y, \alpha,
    \beta)  \right] = \lambda \alpha \beta \left[ \alpha\beta  \right] \circ
    \left[ f_1, f_2 \right] 
\end{align*}

es $\Sigma$-p.r. Esta es la función que está dentro de la concatenación. 

\textit{(2)} Sea $G := \lambda xyz\alpha\beta \left[
\mathop{\subset}_{t=x}^{t=y} f(z, t, \alpha, \beta)  \right]$. Sabemos que, dado
que $f$ es $\Sigma$-p.r., $G$ es $\Sigma$-p.r. Ahora bien, 

\begin{align*}
\mathcal{F} = G \circ \left[ C_3^{3, 2}, \lambda x \left[ x + 5  \right] \circ
p_3^{3, 2}, p3^{3, 2}, p_4^{3, 2}, p_5^{3, 2} \right] 
\end{align*}

Luego $\mathcal{F}$ es $\Sigma$-p.r. 

\begin{problem}
    Use que $x \in \mathbb{N}$ es primo si y solo si $x > 1 \land \left(
    (\forall t \in \omega)_{t \leq x} ~ t = x \lor \neg(t \mid x) \right) $ para
    demostrar que $\lambda x \left[ x \text{ es primo }  \right]$ es $\Sigma$-p.r. 
\end{problem}


\small
\begin{quote}

Definamos $P_1 = \lambda  \left[  x > 1 \right], P_2 = \lambda x \left[ \left( \forall t
\in \omega \right)_{t\leq x} t = x \lor \neg \left( t \mid x \right)
\right]$. Observe que el predicado $P' = \lambda tx \left[  t = x \lor \neg (t \mid
x) \right]$ es $\Sigma$-p.r. (se deja al lector). Pues $P'$ es $\Sigma$-p.r.
tenemos que $P_2 = \lambda x \left[ (\forall t \in \omega)_{t \leq x} P'(t, x)
\right]$ es $\Sigma$-p.r. Dado que $\mathcal{D}_{P_1} = \mathcal{D}_{P_2}$
podemos tomar $P = P_1 \land P_2$ y $P$ es $\Sigma$-p.r. Es evidente que $P =
\lambda x \left[  x \text{ es primo} \right] $.

\end{quote}
\normalsize
\pagebreak

\begin{problem}
    Pruebe que $ L = \left\{ (x, \alpha, \beta) \in \omega \times \Sigma^{*}^2 : \left(
    \exists t \in \omega \right) ~ \alpha^x = \beta^t  \right\} $ es $\Sigma$-p.r. 
\end{problem}


\small
\begin{quote}

El predicado siendo cuantificado es trivialmente $\Sigma$-p.r. y así lo es a su
vez $\omega$ (pues $\chi_{\omega}^{\omega} = C_1^{1, 0$). Fijemos un elemento
    arbitrario $(x, \alpha, \beta) \in L$. Pues
    $\alpha^x = \beta^t$, se sigue que $|\alpha|x = |\beta| t$. Si $t >
    |\alpha|x$ es imposible que la igualdad se cumpla, asumiendo que $|\beta|
    \neq 0$. Si $\beta = \varepsilon $ entonces o bien $x = 0$ o $\alpha =
    \varepsilon $, en cuyos casos la igualdad vale para cualquier $t \in
    \omega$, incluyendo $t = 0$. Se sigue que en todo caso $t \leq |\alpha|x$.


    Ahora que dimos con una cota para $t$, observe que $f = \lambda x\alpha
    \left[  |\alpha|x  \right]$ es $\Sigma$-p.r. (se deja al
    lector). Luego

    \begin{align*}
        \chi_{L}^{\omega \times \Sigma^{*}^{2}} = \lambda x x_0 \alpha\beta
        \left[ \left( \exists t \in \omega \right)_{t \leq x} 
            \alpha^{x_0}=\beta^t   \right] \circ \left[ f \circ [ p_1^{1, 2},
            p_2^{1, 2} ] \
    , p_1^{1, 2}, p_2^{1, 2}, p_3^{1, 2}  \right] 
    \end{align*}

    que es $\Sigma$-p.r. 


\end{quote}
\normalsize


\pagebreak


\begin{problem}
    Sea $\Sigma = \left\{ @, ? \right\} $. Demuestre que 

    \begin{align*}
        L = \left\{ (x, \alpha, \beta) \in \mathbb{N} \times \Sigma^{*}\times
        \Sigma^{+} : \left( \exists \gamma \in \Sigma^{*} \right)  @\beta@ = \gamma?\alpha ?\gamma^{R}\right\} 
    \end{align*}

    es $\Sigma$-p.r. 
\end{problem}


\small
\begin{quote}

\textit{(1)} Sea $P_0 = \lambda \alpha\beta\gamma\left[ @\beta@ =
\gamma?\alpha?\gamma^R  \right]$. Para demostrar que es $\Sigma$-p.r. observe
que $\lambda \alpha \left[  \alpha^R  \right] = R\left(C_{\varepsilon }^{0, 2},
\left\{ a \in \Sigma : \left( a, \lambda \alpha\beta \left[  \alpha\beta
\right] \circ \left[ p_3^{0, 4}, p_4^{0, 4} \right]  \right)  \right\}\right) $. Pues
tomar la recíproca de una palabra es una función $\Sigma$-p.r. se sigue
fácilmente que $P_0$ es $\Sigma$-p.r. 

\textit{(2)} Sea $\left( x, \alpha, \beta \right) \in  L $ un elemento
arbitrario. Considere $\gamma \in \Sigma^{*}$ t.q. $@\beta@ =
\gamma?\alpha?\gamma^{R}$. Evidentemente, 

\begin{align*}
    |@\beta @| =| \gamma?\alpha?\gamma^{R}| &\Rightarrow |\beta| + 2 = 2 + 2|\gamma|
    + |\alpha| \\ 
                                          &\Rightarrow |\beta| - |\alpha| =
                                          2|\gamma|
\end{align*}

Si $|\beta| - |\alpha|$ es par obtenemos que $(|\beta| - |\alpha|)/2$ es una
cota. Si es impar entonces $(|\beta| - |\alpha| + 1) / 2$ es una cota. Como este
último valor es superior a $|\gamma|$ en ambos casos, lo tomamos como la cota de
$t$. Es trivial observar que $f := \lambda \alpha\beta \left[ (|\alpha| - |\beta| +
1) / 2  \right]$ es $\Sigma$-p.r., pues la división entera es $\Sigma$-p.r. 

\textit{(3)} Tenemos entonces que 

\begin{align*}
    \chi_L^{\omega\times \Sigma^{*}^2} = \lambda xy\alpha\beta &\left[ \left(
    \exists \gamma\in \Sigma^{*} \right)_{|\gamma| \leq x} ~ @\beta@ =
\gamma?\alpha?\gamma^{R}   \right] \\ 
            \circ & \left[ f \circ \left[ p_2^{1, 2}, p_3^{1, 2} \right],
            p_1^{1, 2}, p_2^{1, 2}, p_3^{1, 2}  \right] \\ 
                \land &~ \lambda x\alpha \left[ x \neq 0 \land \alpha \neq
                \epsilon  \right] \circ \left[ p_1^{1, 2}, p_3^{1, 2} \right] 
\end{align*}

Observe que ambos predicados tienen el mismo dominio y en consecuencia están
bien definidos. El segundo predicado asegura que respetemos que los elementos de
$L$ son de $\mathbb{N} \times {\Sigma^{*}} \times \Sigma^{+}$. Pues la función
dada arriba es claramente $\Sigma$-p.r., hemos probado que $L$ es $\Sigma$-p.r. 

\end{quote}
\normalsize





\begin{problem}
    Pruebe que 

    \begin{align*}
        L = \left\{ (x, \alpha, \beta) \in \omega \times \Sigma^{*}^2 : \left(
        \exists t \in Im(pr) \right) ~ \alpha^{Pred(Pred(x)) \cdot
    Pred(|\alpha|)} = \beta^t  \right\} 
    \end{align*}

    es $\Sigma$-p.r. 
\end{problem}


\small
\begin{quote}

    \textit{(1)} Sea $P_0 = \lambda x_0 x_1\alpha\beta \left[
    \alpha^{Pred(Pred(x_0))\cdot Pred(|\alpha|)} = \beta^{x_1}  \right]$. Salteamos la
    prueba de que $P_0$ es $\Sigma$-p.r. porque es mecánica.

    \textit{(2)} Sabemos que $\chi_{L}^{\omega \times  \Sigma^{*}^2}$ es el
    predicado $P = \lambda x_0 \alpha \beta \left[ (\exists t \in Im(pr)) ~
    P_0(x_0, t, \alpha, \beta)  \right]$. 

    Sea $(x_0, \alpha, \beta) \in L$ un elemento arbitrario y $t \in Im(pr)$.

    \begin{align*}
        \alpha^{(x_0 - 2)(|\alpha| - 1)} = \beta^{t} \Rightarrow |\alpha|(x_0 -
        2)(|\alpha| - 1) = |\beta| t
    \end{align*}

    Un razonamiento similar al de un ejercicio anterior nos lleva a concluir que
    $t \leq |\alpha|(x_0 - 2)(|\alpha| - 1)$. Es muy fácil ver que $f := \lambda
    x\alpha\left[  (x-2)(|\alpha| - 1)  \right]$ es $\Sigma$-p.r. (de hecho, si
    demostró que $P_0$ es $\Sigma$-p.r. ya demostró que $f$ es $\Sigma$-p.r.).

    \textit{(3)} Se sigue de todo lo anterior que 

    \begin{align*}
    \chi_L^{\omega\times \Sigma^{*}^2} = \lambda xy\alpha\beta &\left[ \left(
    \exists t \in Im(pr) \right)_{t\leq x} \alpha^{(2 - y)(|\alpha| - 1)} =
\beta^t   \right] \\ 
            \circ & \left[ f \circ \left[ p_1^{1, 2}, p_2^{1, 2} \right],
            p_1^{1, 2}, p_2^{1, 2}, p_3^{1, 2}   \right] 
    \end{align*}

\textit{(4)} Como última observación, vea que $Im(pr)$ es $\Sigma$-computable.
En efecto, 

\begin{align*}
    \chi_{Im(pr)}^{\omega} = \lambda x \left[ \neg \left(\left( \exists t \in \omega
    \right)_{t \leq x} ~ t \mid x \right)  \right]
\end{align*}

Pues $\omega$ es $\Sigma$-p.r. y $t \mid x$ es $\Sigma$-p.r. tenemos que
$\chi_{Im(pr)^{\omega}}$ es $\Sigma$-p.r. Luego, $\chi_{L}^{\omega \times
\Sigma^{*}^2}$ es $\Sigma$-p.r. 







\end{quote}
\normalsize


\begin{problem}
    Use the \textbf{U rule} to find a predicate $P$ s.t. $M(P) = \lambda
    x[\text{integer part of } \sqrt{x}]$ .
\end{problem}


\small
\begin{quote}

Let $f(x)$ denote the integer part of $\sqrt{x}$. If $f(x) = y$ then $y^2 \leq x
\land (y+1)^2 > x$. Then letting $P = \lambda xy\left[ y^2 \leq x \land (y +
1)^2 > x \right] $ ensures that $M(P(x, y)) = f(x)$.


\end{quote}
\normalsize


\begin{problem}
    Is is true that $Ins^{\Sigma} \cap Pro^{\Sigma} = \emptyset$? And is it true
    that $\lambda i \mathcal{P} [I_i^{\mathcal{P}}]$ has domain $\left\{ (i,
    \mathcal{P}) \in \mathbb{N} \times Pro^{\Sigma}: i \leq n(\mathcal{P})
\right\} $?
\end{problem}


\small
\begin{quote}

Both statements are false. A single instruction in $Ins^{\Sigma}$ can be a
program (as long as it is not a GOTO statement to a non-existent label).
Furthermore, $\lambda i \mathcal{P} [I_i^{\mathcal{P}}]$ is defined for $i = 0$
(it maps to $\varepsilon$) and for $i \geq n(\mathcal{P} )$ (it also maps to
$\varepsilon$).

\begin{problem}
    Prove: If $\mathcal{P}_1, \mathcal{P}_2 \in Pro^{\Sigma}$ then $\mathcal{P}_1
    \mathcal{P}_1 = \mathcal{P}_2 \mathcal{P}_2 \Rightarrow \mathcal{P}_1 =
    \mathcal{P}_2$.
\end{problem}

This follows from the theorem that guarantees that any program $\mathcal{P} \in
Pro^{\Sigma}$ is a \textit{unique} concatenation of instructions. Let
$\mathcal{P}_1 = I_1^{\mathcal{P}_1} \ldots I_{n(\mathcal{P}_1)}^{\mathcal{P}_1}$ and $\mathcal{P}_2 = I_1^{\mathcal{P}_2}
\ldots I_{n(\mathcal{P}_2)}^{\mathcal{P}_2}$. Assume $\mathcal{P}_1\mathcal{P}_1 =
\mathcal{P}_2 \mathcal{P}_2$. Then 

\begin{align*}
    I_1^{\mathcal{P}_1} \ldots I_{n(\mathcal{P}_1)}^{\mathcal{P}_1}
    I_1^{\mathcal{P}_1} \ldots I_{n(\mathcal{P}_1)}^{\mathcal{P}_1} = 
    I_2^{\mathcal{P}_2} \ldots I_{n(\mathcal{P}_2)}^{\mathcal{P}_2}
    I_2^{\mathcal{P}_2} \ldots I_{n(\mathcal{P}_2)}^{\mathcal{P}_2}
\end{align*}

Then, it is a theorem that $I_k^{\mathcal{P}_1} =
i_k^{\mathcal{P}_2}$. From this follows directly that $\mathcal{P}_1 =
\mathcal{P}_2$. $\blacksquare$

\end{quote}
\normalsize

\begin{problem}
    Give true or false for the following statements.
\end{problem}


\small
\begin{quote}


\textit{Statement 1: If $S_{\mathcal{P}}(i, \overrightarrow{s},
\overrightarrow{\alpha}) = (i, \overrightarrow{s}, \overrightarrow{\alpha})$
then $i \not\in [1, n(\mathcal{P})]$}. The statement is false. It could be the
case that $i \not\in [1, n( \mathcal{P} )]$, in which case we would say the program
halted. However, consider the program 

\begin{align*}
    L1 ~ GOTO ~ L1 
\end{align*}

Evidently, $S_{\mathcal{P}}(1, \overrightarrow{s}, \overrightarrow{\alpha}) =
(1, \overrightarrow{s}, \overrightarrow{\alpha})$, and $1 \leq 1 \leq
n(\mathcal{P})$ .

\textit{Statement 2. Let $\mathcal{P} \in Pro^{\Sigma}$ and $d$ an instantaneous
description whose first coordinate is $i$. If $I_i^{\mathcal{P}} = N_2
\leftarrow N_2 + 1$, then $$S_{\mathcal{P}}(d) = \left( i+1, \left( N_1,
Suc(N_2), N_3, \ldots \right), (P_1, P_2, P_3, \ldots)  \right) $$}

The statement is true via direct application of the $S_{\mathcal{P}}$ function.

\textit{Statement 3. Let $\mathcal{P} \in Pro^{\Sigma}$ and $(i,
\overrightarrow{s}, \overrightarrow{\sigma})$ an instantaneous description. If
$Bas(I_i^{\mathcal{P}}) = IF ~ P_3 ~ BEGINS ~ a ~ GOTO ~ L_6$ and $[P_3]_1 = a$,
then $S_{\mathcal{P}}(i, \overrightarrow{s}, \overrightarrow{\sigma}) = (j,
\overrightarrow{s}, \overrightarrow{\sigma})$, where $j$ is the least number $l$
s.t. $I_{l}^{\mathcal{P}}$ has label $L_6$}.

Because $[P_3]_1 = a$, the value of $S_{\mathcal{P}}(i, \overrightarrow{s},
\overrightarrow{\sigma})$ must indeed contain the instruction that has label
$L_6$. This instruction is the $j$th instruction for some $j$, etc. The
statement is true.


\end{quote}
\normalsize

\begin{problem}
    Let $\Sigma = \{@, !\}$. Give a program that computes $f : \{0, 1, 2\}
    \mapsto \omega $ given by $f(0) = f(1) = 0, f(2) = 5$.
\end{problem}


\small
\begin{quote}


Evidently $f \sim (1, 0, \#)$ and so we must find some $\mathcal{P} \in
Pro^{\Sigma}$ s.t. $\Psi_{\mathcal{P}}^{1, 0, \#}(x) = f(x)$. The program must
let $N_1$ hold the value $0$ if the starting state is either $[\![ 0 ]\!]$ or
$[\![ 1 ]\!]$, and the value $5$ if the starting state is $[\![ 2 ]\!]$. In all
other cases, it must not halt, to ensure that the domain of
$\Psi_{\mathcal{P}}^{1, 0, \#}$ is the same as that of $f$. The desired program
is 

\begin{align*}
    &N_2 \leftarrow N_1\\
    &N_2 \leftarrow N_2 - 1\\
    &IF ~ N_2 \neq 0 ~ GOTO ~L_1 \\ 
    &GOTO ~ L_4 \\ 
    L_1 ~ &N_2 \leftarrow N_2 - 1 \\ 
    &IF ~  N_2 \neq 0 ~ GOTO ~ L_2\\
    &GOTO L_3\\
    L_2~& GOTO ~ L_2 \\ 
    L_3 ~ & N_1 \leftarrow N_1 + 1\\
    & N_1 \leftarrow N_1 + 1\\
    & N_1 \leftarrow N_1 + 1 \\ 
    & GOTO ~ L_5 \\
    L_4 ~ & N_1 \leftarrow 0 \\ 
    L_5~& SKIP
\end{align*}

If $\mathcal{P}$ denotes this program, it is evident that $\mathcal{P}$ only
halts for starting states $[\![ x_1 ]\!]$ with $x_1 \in \{0, 1, 2\}$.
Thus, the domain of $\Psi_{\mathcal{P}}^{1, 0, \#}$ is precisely
$\mathcal{D}_f$. It is easy to verify that, more generally,
$\Psi_{\mathcal{P}}^{1, 0, \#} = f$.

\end{quote}
\normalsize

\begin{problem}
    Using the same alphabet as in the previous problem, find $\mathcal{P} \in
    Pro^{\Sigma}$ that computes $\lambda xy[x + y]$.
\end{problem}


\small
\begin{quote}

The desired program is 

\begin{align*}
    L_1 ~ &IF ~ N_2 = 0 ~ GOTO ~ L_3 \\ 
          &N_1 \leftarrow N_1 + 1 \\ 
          &N_2 \leftarrow N_2 - 1 \\ 
          &GOTO ~ L_1\\
    L_3 ~ & SKIP
\end{align*}

\begin{problem}
    Same for $C_0^{1, 1}_{\mid \{0, 1\} \times  \Sigma^{*}}$
\end{problem}


\small
\begin{quote}


Since the domain of the constant function is restricted t o $\{0, 1\} \times
\Sigma^{*}$, we must ensure the program only halts for states $[\![ x_1, x_2,
\alpha ]\!]$ s.t. $x_1,x_2 \in \{0, 1\}$. Thus, the program is 

\begin{align*}
    &N_1 \leftarrow N_1 - 1 \\
    &N_2 \leftarrow N_2 - 1 \\
    &IF N_2 \neq 0 ~ GOTO ~ L_1 \\ 
    & IF N_1 \neq 0 ~ GOTO ~ L_1 \\ 
    &GOTO ~ L_2 \\ 
    L_1~& GOTO ~ L_1 \\ 
    L_2~ &SKIP
\end{align*}

\end{quote}
\normalsize

\end{quote}
\normalsize



\begin{problem}
    Same for $\lambda i\alpha[[\alpha]_i]$ (same alphabet).
\end{problem}

\small
\begin{quote}


\begin{align*}
    &IF ~ N_0 \neq 0 ~ GOTO ~ L_1 \\ 
    &P_1 \leftarrow \varepsilon \\ 
    & GOTO ~ L_{100} \\
    L_1 ~ & N_1 \leftarrow N_1 - 1 \\ 
    L_2 ~ & N_1 \leftarrow N_1 - 1 \\ 
          &P_1 \leftarrow {}^{\curvearrowright} P_1\\
    &IF ~ N_1 \neq 0 ~ GOTO ~ L_2\\
    &IF ~ P_1 ~ STARTSWITH ~ @ ~ GOTO ~ L_2 \\ 
    &IF ~ P_1 ~ STARTSWITH ~ ! ~ GOTO L_3\\ 
    & GOTO L_{100}\\
    L_3~&P_1 \leftarrow ~ !\\
    L_2 ~ &P_1 \leftarrow  @ \\ 
    L_{100} ~ & SKIP
\end{align*}

\textit{Example.} Let $\alpha = @!!@@$. Assume we give $[\![ 4, \alpha ]\!]$.
Since $4 \neq 0$ we go to $L_1$ immediately. Here $N_1$ is set to three. Then
$N_1$ is set to two and $P_1$ is set to $!!@@$. Since $N_1 \neq 0$, $N_1$ is now
set to $1$ and $P_1$ to $!@@$. Once more, $N_1$ is now set to $0$ and $P_1$ to
$@@$. Since now $N_1 = 0$ , we know the starting character of $P_1$ is the one
we looked for. We set $P_1$ to be its first character (if $P_1 = \varepsilon$ it
has no first character and nothings needs to be done, because this means the
input $[\![ x_1, \alpha ]\!]$ had $x_1 > |\alpha|$). The other cases also work.

\end{quote}
\normalsize







\end{document}





