\documentclass[a4paper, 12pt]{article}

\usepackage[utf8]{inputenc}
\usepackage[T1]{fontenc}
\usepackage{textcomp}
\usepackage{amssymb}
\usepackage{newtxtext} \usepackage{newtxmath}
\usepackage{amsmath, amssymb}
\newtheorem{problem}{Problem}
\newtheorem{example}{Example}
\newtheorem{lemma}{Lemma}
\newtheorem{theorem}{Theorem}
\newtheorem{problem}{Problem}
\newtheorem{example}{Example} \newtheorem{definition}{Definition}
\newtheorem{lemma}{Lemma}
\newtheorem{theorem}{Theorem}
\usepackage{parskip}

\begin{document}

   
\section{P1}

Let us recall that $P : \mathcal{P}(\Omega) \mapsto [0, 1]$ satisfies the following three 
conditions:

\begin{itemize}
    \item $\forall \zeta \in \subseteq \Omega : 0 \leq P(\zeta) \leq 1$
    \item $P(\Omega) = 1$
    \item $\forall \left\{ \zeta \right\}_{I \subseteq \mathbb{N}} : \zeta_i \cap  \zeta_j = \emptyset : P(\bigcup_{i \in I}) \zeta_i = \sum_{i \in I} P(\zeta_i) $
\end{itemize}

In general, I use $\omega$ to denote $|\Omega|$, and we should recall that whenever $P$ is a constant 
map (i.e. when events are equiprobable) we have $\forall A \subseteq \Omega : P(A) = \frac{|A|}{\omega}$.

To keep notation brief, we shall speak of probabilities $P(A)$ without specifying that 
$A \subseteq \Omega$.

We recall the following properties, which follow strictly from the aforementioned 
facts:

\begin{itemize}
    \item $P(\emptyset) = 0$
    \item $P(A^c) = 1 - P(A)$
    \item $A \subseteq B \Rightarrow P(A) \leq P(B)$
    \item $P(A \cup B) = P(A) + P(B) - P(A \cap B)$
\end{itemize}

\pagebreak

\begin{problem}[Problem 2 of the sheet]
    Prove that $A \subseteq B \subseteq \Omega \Rightarrow P(B- A) = P(B) - P(A)$ and also
    $P(A) \leq P(B)$.
\end{problem}

Let $A \subseteq B \subseteq \Omega$.

\begin{align*}
    B - A &= \left\{ x \in B : x \not\in A \right\}  \\ 
    &=\left\{ x \in \Omega : x \in B \land  x\not\in A \right\} \\
    &= \left\{ x \in \Omega : x \in B \land x \in A^c \right\}  \\ 
    &= B \cap A^c
\end{align*}

Since $P(A \cup B) = P(A) + P(B) - P(A \cap B)$, we have 
$$P(A \cap B) = P(A) + P(B) - P(A \cup B)$$ 

Then 

\begin{align*}
    P(B - A) &= P(B \cap A^c) \\ 
             &=P(B) + P(A^c) - P(B \cup A^c)\\ 
             &=P(B) + \left[ 1 - P(A) \right] - P(B \cup A^c)
\end{align*}

It is easy to see that, since $A \subseteq B$, $B \cup A^c = \Omega$.
We readily obtain


\begin{align*}
    P(B - A) &= P(B) + \left[ 1 - P(A) \right] - P(\Omega) \\ 
    &=P(B) + 1 - P(A) - 1 \\ 
    &= P(B) - P(A)
\end{align*}

\textit{quod erat demonstrandum}. And since $P(B - A) = P(B) - P(A) \geq 0$, we must have 
$P(B) \geq P(A)$.

\pagebreak 

\begin{problem}[3]
    
\end{problem}

 
For ease of mind, let us write here that 

\begin{align*}
    P(A_i) = \begin{cases}
        .22 & i = 1 \\ 
        .25 & i = 2 \\ 
        .28 & i = 3
    \end{cases}
\end{align*}

We are also given 

\begin{equation*}
    P(A_1 \cap A_2) = .11 ~ ~ ~ ~ ~ P(A_1 \cap A_3) = .005 ~ ~ ~ ~ ~ ~ ~ P(A_1 \cap A_3) = .07 ~ ~ ~ ~ ~ ~ P(A_1 \cap A_2 \cap A_3) = .01
\end{equation*}

    (1 : $P(A_1 \cup A_2)$ Observe that $P(A_1 \cap A_2) \neq P(A_1)P(A_2)$,
    which entails the events are not independent. We then have 

    \begin{align*}
        P(A_1 \cup A_2) &= P(A_1) + P(A_2) - P(A_1 \cap A_2) \\ 
                        &= .22 + .25 - .11 \\ 
                        &= .36
    \end{align*}

    (2: $P(A_1^c \cap A_2^c \cap A_3)$) Observe that 

    \begin{align*}
        P(A_1^c \cap A_2^c \cap A_3) &= P( \left[ A_1^c \cap A_2^c \right] \cap A_3^c ) \\ 
                                     &= P(\left[ A_1 \cup A_2 \right]^c \cap A_3^c )\\ 
                                     &=P(( \left[ A_1 \cup A_2 \right] \cup A_3 )^c ) \\ 
                                     &= P(\Omega^c) \\ 
                                     &= P(\emptyset) \\ 
                                     &= 0
    \end{align*}


    (3: $P( (A_1^c \cap A_2^c) \cup A^c )$)

    \pagebreak 

    \begin{problem}[4]
        
    \end{problem}

    Se nos dice que cinco empresas deben firmar contratos de un grupo de $3$ 
    contratos posibles. Se nos dice además que cada empresa firma a lo sumo 
    un contrato. 

    
    \small
    \begin{quote}
    
    Mucho ojo: no se nos dice que cada contrato se da a lo sumo a una empresa, sino que 
    cada empresa firma a lo sumo un contrato. Es decir que varias empresas podrían firmar 
    el mismo contrato.
    \end{quote}
    \normalsize

    Si contamos la opción "no firma ningún contrato", hay $4$ opciones para cada 
    una de las $5$ empresas; es decir, hay $\omega = 4^5$ puntos en el espacio 
    muestral.

    Si cada evento es equiprobable, la probabilidad de que la tercera empresa
    reciba un contrato es 

    \begin{align*} \frac{| \left\{ x \in \Omega : 3\text{era empresa firma
    contrato} \right\}  |}{\omega} \end{align*}

    El numerador puede calcularse restando a $\omega$ la cantidad de casos en que 
    la tercera empresa no firma un contrato. Claramente, la cantidad de tales 
    casos es $4^4$; es decir, hay $4^5 - 4^4 = 768$ casos en que 
    la tercera empresa firma algún contrato. Lo cual nos da una probabilidad 
    de $768 / 4^5 = .75$.

    \textit{Solución alternativa}. Sea $\alpha$ una palabra sobre el alfabeto $\left\{ 0,\ldots,3 \right\} $
    de longitud $5$. ¿Cuántas tales palabras hay? Naturalmente, $4^5 = \omega$.
    Asumamos que $\alpha_3 \neq 0$; es decir que el tercer símbolo de $\alpha$
    es no-nulo. ¿Cuántas palabras así hay? Naturalmente, $4^4 \times 3 = 768$.
    Es decir, $\frac{768}{4^5} = .75$ de las palabras tienen $\alpha_3 \neq 0$,
    lo cual coincide con nuestro resultado anterior.


    \pagebreak 

    \begin{problem}[5]
        
    \end{problem}


    From a set of $25$ buses, $8$ present flaws. $5$ are randomly (in this
    context, uniformly) chosen. We are therefore dealing with equiprobable
    events.  

    There are $\omega = \binom{25}{5}$ possible ways of selecting the $5$ buses. 
    There are $\binom{8}{4}$ possible ways of selecting $4$ out of the $8$
    flawed buses and $\binom{25-8}{1} = \binom{17}{1}$ ways of selecting the 
    remaining bus from the set of non-flawed buses. From this follows that the desired probability is

    \begin{align*}
        \frac{\binom{8}{4} \cdot\binom{17}{1}}{\binom{25}{5}}= \frac{70 \cdot 17}{53130} = .022
    \end{align*}


    With regards to the probability that at least $4$ have flaws, we must take 
    into account the cases where $4$ have flaws and the cases where $5$ have 
    flaws, which are clearly disjoint. The probability then is

    \begin{equation*}
        .022 + \frac{\binom{8}{5}}{53130} = .022 + .001 = .023
    \end{equation*}

    
    \small
    \begin{quote}
    
    \textbf{Note.} This is obviously a problem involving the hypergeometric distribution, 
    closely related to the binomial distribution. But since we have not studied 
    distributions yet, we cannot use this fact.
    
    \end{quote}
    \normalsize
    

    \pagebreak 

    \begin{problem}[6]
        
    \end{problem}

    Let $A, B, C, D, E$ denote the five faculty members. Two papers from a set
    of five are drawn to decide who will be chosen. 

    Observe that the order does not matter; i.e. drawing $A$ and then $B$ is the 
    same than drawing $B$ and then $A$. It follows that 

    \begin{equation*}
        \Omega = \mathcal{P}_2(\left\{ A, \ldots, E \right\} )
    \end{equation*}

    where $\mathcal{P}_i(\zeta) = \left\{ S \in \mathcal{P}(\zeta) : |S| = i \right\} $.
    Naturally, $\omega = 5 \times 4 \cdot \frac{1}{2} = 10$, where we 
    divide by $2$ to exclude equivalent pairs (e.g. $A, B$ and $B, A$).

    
    \small
    \begin{quote}
    
    Alternatively, we could have reasoned that $\omega = \binom{5}{2} = 10$, the number 
    of $2$-element subsets of a $5$-element set.
    
    \end{quote}
    \normalsize

    ($a$) We are asked for the probability of the event $\left\{ A, B \right\} $. It 
    should be obvious that all events $S \in \Omega$ are equiprobable, 
    which entails $P(\left\{ A, B \right\}) = \frac{1}{10} = .1 $.

    
    \small
    \begin{quote}
    
    Alternatively, we could have reasoned the following. There are two ways in
    which $A$ and $B$ may be chosen: $A$ is chosen first and then $B$, or $B$
    is chosen first and then $A$. This gives $\frac{1}{5} \cdot \frac{1}{4} +
    \frac{1}{5} \cdot \frac{1}{4} = \frac{2 \cdot}{5\cdot4} = \frac{2}{20} =
    .1$
    
    \end{quote}
    \normalsize

    ($b$) We are asked for the probability that the selection contains $C$ or
    $D$. It is straightforward to reason that there are $\binom{3}{2} = 3$ sets
    that do not contain neither $C$ nor $D$. From which readily follows that 
    there are $10 - 3 = 7$ sets containing $C$, $D$ or both. $\therefore $
    The desired probability is $\frac{7}{10}$.

    ($c$) Let us change the notation a bit. Let $\left\{ 1,\ldots, 5 \right\} $
    be the professors we used to call $A, \ldots, E$. Let $a_i := \left\{ 3, 6,
    7, 10, 14 \right\} $ be the set of years of teaching of each professor,
    assuming $a_1$ corresponds to $A$, $a_2$ to $B$, etc. We are asked for the
    probability that the selected pair $\left\{ j, k \right\} $ satisfies 
    $a_j + a_k \geq 15$.

    There are two ways to solve this problem: one slow but direct, one 
    pretty but a bit more clever.

    \textit{Direct solution.} It is easy to see that, of all pairs $j, k$, only the following 
    satisfy the requirement:

   
   \small
   \begin{quote}
   
   
    \begin{itemize}
        \item $1, 5 \mapsto a_1 + a_5 = 17$ 
        \item $2, 4 \mapsto a_2 + a_4 = 16$
        \item $2, 5 \mapsto  a_2 + a_5 = 20$.
        \item $3, 4 \mapsto a_3 + a_4 = 17$
        \item $3, 5 \mapsto  a_3 + a_5 = 21$
        \item $4, 5 \mapsto a_4+a_5=24$
    \end{itemize}
   
   \end{quote}
   \normalsize
  
   So only $6$ out of the $10$ possible pairs satisfy the relationship, giving us 
   the desired probability: $\frac{6}{10} = \frac{3}{5} = .6$.

   \textit{Pretty solution.} Draw the $4\times 5$ boolean matrix $\mathcal{A}$
   whose coefficients $\mathcal{A}_{ij}$ are $1$ if $a_i + a_j \geq 15$, $0$
   otherwise. Since upper and lower diagonal entries are equivalent (the matrix
   is symmetric), and because $i \neq j$ in our experiment, the diagonal of the
   matrix should not be considered. This gives the representation

   \begin{align*}
   \begin{bmatrix} 
       ~ & 0 & 0 &0 &1 \\ 
       ~&~&0&1&1 \\ 
       ~&~&~&1&1 \\ 
       ~ &~&~&~&1
   \end{bmatrix} 
   \end{align*}

   where $6$ out of $10$ relevant entries are $1$.


   \pagebreak 

   \begin{problem}[7]
       
   \end{problem}

   Let $M := \left\{ m_1, \ldots, m_4 \right\} $ and $W := \left\{ w_1, \ldots,
   w_4 \right\} $ be alphabets denoting the men and women, respectively. The
   sample space $\Omega$ consists of all permutations in $M \cup W$, which
   readily entails $\omega = 8!$.

   $(a)$ Consider the event $\zeta$ when at least one women $w \in W$ is among the first three elements in the 
   sampled permutation. Then $w$ could be the first, the second or the third element in the permutation, 
   and we impose no condition on the rest of the elements. These events are evidently disjoint. So, 
   if we denote with $e_i$ the event where $w$ is the $i$th element of the permutation, we have 

   \begin{equation*}
       P(\zeta) = \frac{ P(e_1) + P(e_2) + P(e_3) }{8!}
   \end{equation*}

   Each $e_i$ may occur in $7!$ ways, since we fix $w$ at the $i$th element and we must only 
   choose from the remaining $7$ assistants.


   $\therefore $ $P(\zeta) = \frac{ 3 \cdot 7! }{8!} = \frac{3}{8}$

   
   \small
   \begin{quote}
   
   \textbf{Note.} If you are interested in being very formal, this are the rigorous 
   steps taken above.
   
   (1) $\zeta = e_1 \cup e_2 \cup  e_3 \Rightarrow P(\zeta) = P(e_1 \cup e_2 \cup e_3)$.

   (2) $( \forall i, j : e_i \cap e_j = \emptyset ) \Rightarrow P(\zeta) = P(e_1) + P(e_2) + P(e_3)$. 
   
   (3) Since events are equiprobable, $P(e_i) = |e_i| / \omega$. 

   (4) $|e_i| = 7!$ because $e_i$ is a permutation of $7$ elements.

   (5) $P(\zeta) = |e_1|/\omega + |e_2| / \omega + |e_3|/\omega = \frac{3\times 7!}{8}$.
   
   \end{quote}
   \normalsize
   


   $(b)$ Let $\varrho$ denote the event where, after the first five meetings,
   all female assistants have been met. Let $(p_1,\ldots, p_8) \in \varrho$ be an arbitrary
   permutation, and denote it with $\overrightarrow{p}$. (Remember that an event is a subset of the sample space, and
   hence a set, so the expresssion $\overrightarrow{p} \in \varrho$ is well defined.)


   The definition of $\varrho$ entails that one and only one $m_j$ exists in $p_1, \ldots, p_5$,
   since all $w_1, \ldots, w_4$ must lie in this sequence. So the number of 
   ways in which we may construct $\overrightarrow{p}$ (i.e. the cardinality 
   of $\varrho$) is readily determined by the number of ways in which we can 
   place exactly one $m_j$ among the first elements of $\overrightarrow{p}$.

   There are $5$ positions to place $m_j$, and $4$ elements in $M$ to choose
   from. Assuming $m_j$ was placed at the $k$th position, we know there are
   $4!$ ways of placing the $4$ women among the remaining positions in $p_1,
   \ldots, p_5$. So there are $5 \times 4 \times 4!$ ways to construct $p_1,
   \ldots, p_5$ for $\overrightarrow{p} \in \varrho$. 

   The positions $p_6, p_7, p_8$ must be chosen from the remaining 
   $3$ men, so there are $3 \times 2$ possibilities. 

   $\therefore $ $|\varrho| = 5\times 4 \times 4! \times 3 = 60 \times 4!$.

   $\therefore $ $P(\varrho) =  \frac{ |\varrho| }{\omega} = \frac{60 \times 4!}{8!} = .036$
  
   \pagebreak

   \section{Conditional probability} 

   \begin{problem}[8]
       A box contains $6$ red balls and $4$ green balls. A second box contains 
       $7$ red balls and $3$ green balls. A ball is randomly chosen from 
       the first box and placed into the second. Then a ball is drawn 
       from the second box and placed into the first.
   \end{problem}


   (1) Let $R_i$ denote the event of choosing a red ball in the $i$th draw, and
   $G_i$ the event of choosing a green ball in the $i$th draw. Evidently,
   $P(R_1) = \frac{6}{10}$. 

   If the event $R_1$ occurs, when the second draw is made, the second box 
   contains $8$ red balls and $3$ green balls, entailing that 
   $P(R_2 \mid  R_1) = \frac{8}{11} $.

   So, using the conditional probability formula, the event when both balls are
   red, $R_1 \cap R_2$, has probability $P(R_1 \cap R_2) = P(R_1) \cdot P(R_2
   \mid R_1) = \frac{6}{10} \cdot \frac{8}{11} = .436$.

   (2) We now inquire the probability that the number of red and green balls in the 
   first box are the same at the beginning and the end of the experiment. Naturally, 
   this entails either drawing both times a red or both times a green ball. We have 
   already computed the probability of drawing both times a red ball. The probability 
   of drawing both times a green ball is similarly computed: 

   \begin{align*}
       P(G_1 \cap G_2) &= P(G_1) \cdot P(G_2 \mid G_1) \\ 
       &= \frac{4}{10} \cdot \frac{4}{11} \\ 
       &= .145
   \end{align*}


   Since $R_1 \cap R_2$ and $G_1 \cap G_2$ are obviously disjoint, the probability that 
   either of them occurs is simply $.436 + .145 = .581$, the sum of the probabilities 
   of both events.

   \pagebreak 


   \begin{problem}[10]
       Given events $A, B$ with $P(B) > 0$, prove $P(A \mid B) + P(\overline{A} \mid B) = 1$.
   \end{problem}

   ($a$) We know 

   \begin{equation*}
       P(A \mid B) = \frac{P(A \cap B)}{P(B)}, ~ ~ ~ P(\overline{A} \mid B) = \frac{P(\overline{A} \cap B)}{P(B)}
   \end{equation*}

   It follows 

   \begin{align*}
       P(A \mid B) + P(\overline{A} \mid B) &= \frac{P(A \cap B) + P(\overline{A} \cap B)}{P(B)}  \\ 
                                            &= \frac{P\left((A \cap B) \cup (\overline{A} \cap B)\right)}{P(B)} &\left\{ A \cap B \text{  and } \overline{A} \cap B \text{ are disjoint}\right\} \\ 
                                            &= \frac{ P\left( ( A \cup \overline{A} ) \cap B \right)    }{P(B)} \\ 
                                            &= \frac{ P(\Omega \cap B) }{P(B)} \\ 
                                            &= \frac{P(B)}{P(B)} \\ 
                                            &= 1
   \end{align*}

   \textit{quod erat demonstrandum}.

   $(b)$ Assume $P(B \mid A) > P(B)$. We want to prove $P(\overline{B} | A) <
   P(\overline{B})$. 

   By assumption,

   \begin{align*}
       \frac{P(A \cap B)}{P(A)} > P(B) &\Rightarrow P(A \cap B) > P(B)P(A)
   \end{align*}


   We want to prove

   \begin{align*}
       \frac{P(\overline{B} \cap A)}{P(A)} < P(\overline{B}) \equiv P(\overline{B} \cap A) < P(\overline{B})P(A)
   \end{align*}

   \pagebreak 

   \begin{problem}[]
       One every $25$ adults have a disease. Let $s$ be a subject. If $s$ has the disease, the diagnostic 
       test is positive $.99$ of the times. If $s$ does not have the disease, the diagnostic test 
       is positive $.02$ of the times.
   \end{problem}

   ($a$) We are asked to find the probability of a result being positive. The law of total 
   probability readily states that

   \begin{align*}
       P(\text{positive}) &= P(\text{positive} | \text{enfermo})P(\text{enfermo}) + P(\text{positive} | \text{sano})P(\text{sano})  \\ 
       &= .99 \cdot \frac{1}{25} + .02 \cdot \frac{24}{25} \\ 
       &= .395 + .0192 \\ 
       &= .058
   \end{align*}

   (b) We are requested to find $P(\text{enfermo} | \text{positivo})$. Observe that we know the 
   "reverse" of this; i.e. $P(\text{positivo} | \text{enfermo})$. This type of 
   scenario calls for Bayes theorem, which states 

   \begin{align*}
       P(\text{enfermo}|\text{positivo}) &= \frac{P(\text{positivo}| \text{enfermo})P(\text{enfermo})}{P(\text{positivo})} \\ 
                                         &= \frac{.99 \cdot \frac{1}{25}}{.058} \\ 
                                         &=.682
   \end{align*}

   (c) Similarly, 

   \begin{align*}
       P(\text{sano} | \text{negativo}) &= \frac{P(\text{negativo} | \text{sano}) P(\text{sano})}{P(\text{negativo})} \\ 
                                        &= \frac{.98 \cdot \frac{24}{25}}{1-.058} \\ 
                                        &=.998
   \end{align*}


   \pagebreak 

   \begin{problem}[12]
       
   \end{problem}

   Recall that there are two equivalent definitions of independence. Two events 
   $\varphi, \psi$ are independent if $P(\varphi \cap \psi) = P(\varphi)P(\psi)$,
   or equivalently if $P(\varphi \mid \psi) = P(\varphi)$. We must use both 
   definitions in order to prove the properties.

   $(a)$ We are required to prove $P(\overline{A} \cap B) = P(\overline{A})P(B)$. Observe that 

   \begin{align*}
       P(\overline{A} \cap B) = P(B) - P(A \cap B)
   \end{align*}

   (This follows from the fact that $B = (A \cap B) \cup (\overline{A} \cap B)$ and the fact that 
   the events in the union are obviously disjoint.) Then, due to the independence of 
   $A$ and $B$,

   \begin{align*}
       P(\overline{A} \cap B) &= P(B) - P(A)P(B)\\ 
                              &= P(B)\left[ 1 - P(A) \right]  \\ 
                              &=P(B)P( \overline{A} )
   \end{align*}

  \textit{quod erat demonstrandum}.

  $(c)$ By DeMorgan's law, $(\overline{A} \cap \overline{B}) = \overline{(A \cup B)}$. Then 

  \begin{align*}
      P(\overline{A} \cap \overline{B}) &= 1 - P(A \cup B) \\
                                        &=1 - \left[ P(A) + P(B) - P(A \cap B) \right]  \\ 
                                        &=1 - \left[ P(A) + P(B) - P(A)P(B) \right] \\ 
                                        &=1 - P(A) - P(B) + P(A)P(B) \\ 
                                        &=P(\overline{A}) - P(B) + P(A)P(B) \\ 
                                        &=P(\overline{A}) - P(B)\left[1 - P(A) \right]  \\ 
                                        &=P(\overline{A}) - P(B)P(\overline{A}) \\ 
                                        &=P(\overline{A})(1 -P(B)) \\ 
                                        &=P(\overline{A})P(\overline{B})
  \end{align*}

  \pagebreak 

  \begin{problem}[13]
      A collection $\chi$ of 10 items has 2 satisfying the predicate $\varphi : \chi \mapsto \left\{ 0, 1 \right\} $.  
      Two random samples $x_1, x_2$ are taken from $\chi$. Let $A = \left\{ \varphi(x_1) \right\} $, $B = \left\{ \varphi(x_2) \right\} $.
      Compute $P(A), P(B), P(A \cap B)$. Are $A$ and $B$ independent?
  \end{problem}

  Obviously, $P(A) = \frac{2}{10} = \frac{1}{5}$ and 

  \begin{align*}
      P(B) &= P(B \mid A)P(A) + P(B \mid \overline{A})P(\overline{A}) \\ 
           &= \frac{1}{9} \frac{2}{10} + \frac{2}{9} \frac{8}{10} \\ 
           &=.\frac{2}{90} + \frac{16}{90} \\ 
           &=\frac{18}{90} \\ 
           &= \frac{2}{10} \\ 
           &= \frac{1}{5}
  \end{align*}

  We know 

  \begin{equation*}
      P(A \cap B) = P(B \mid A)P(A) = \frac{1}{9} \frac{2}{10} = \frac{1}{45}
  \end{equation*}

  Then $P(A)P(B) = \frac{1}{5} \cdot \frac{1}{5} \neq P(A \cap B)$. The events are not independent.

\pagebreak 

\begin{problem}[14]
    From a deck of 52 spanish cards, 4 players $p_1, \ldots, p_4$ receive $13$ cards each.
\end{problem}

$(a)$ Let $w, x, y, z$ be the four types of cards. The probability of $p_1$ receiving 
the 13 cards of type $w$ is $1 / \binom{52}{13}$. The same logic gives that the probability of 
all of them receiving the corresponding hands is 

\begin{equation*}
    P(\zeta) = \prod_{i=0}^{3} \frac{1}{\binom{52 - 13i}{13}}
\end{equation*}

where $\zeta$ denotes the corresponding event.

$(b)$ If a player has all cards of same type, it has all cards of that type
(13). There are $4!$ ways of distributing the types among the players. Then the
probability of this event is $4! P(\zeta)$.

$(c)$ There are $4$ aces: $ \mathcal{A} = x_{13}, y_{13}, w_{13}, z_{13}$. There are 
$\binom{48}{13}$ hands without these elements. $\therefore $ The probability 
is $1 / \binom{48}{13}$.

$(d)$ For each player to receive an element in $\mathcal{A}$, each must get exactly 
one such element.

The probability of $A$ receiving exactly one $A$ is $\frac{4}{\binom{51}{12}}$, where 
$4$ accounts for which element it receives and $\binom{51}{12}$ is the number of 
hands containing that fixed ace. Similar logic gives that the probability of all 
players reciving exactly one ace is 

\begin{align*}
    4! \prod_{i=0}^{3} \frac{1}{\binom{51 - 12i}{12}}
\end{align*}


\pagebreak 

\section{P2}

(0.a) Assuming any number in $\mathbb{N}_{100}$ has the same probability of being chosen on a list 
$(a_1, \ldots, a_5)$, we have

\begin{equation*}
    P(\overrightarrow{x} \in \Omega) = \frac{1}{|\Omega|} = \frac{1}{100^5}
\end{equation*}

(0.b) A random variable associated to this experiment might be the sum of the elements 
in the list: $X(\overrightarrow{a}) = \sum a_i$.

~ ~ ~ 

(1.a) A probability mass function must satisfy $\sum p(x_i) = 1$. Only the second given 
function satisfies this.

(1.b) 

\begin{align*}
    P(2 \leq X \leq 4) = P(X = 2) + P(X = 3) + P(X=4) = .1 + .1 + .3 = .5
\end{align*}

(1.c) The cumulative distribution function of $X$ describes $P(X \leq x)$  for every $x$
in the range of the random variable. Thus, we have

\begin{align*}
    P(X \leq x) = \begin{cases}
        0 & X < 0 \\ 
        .4 & 0 \leq X < 1 \\ 
        .5 & 1 \leq X < 2 \\ 
        .6 & 2 \leq X < 3 \\ 
        .7 & 3 \leq X < 4\\ 
        1 & 4 \leq X
    \end{cases}
\end{align*}

(1.d) Asssume $P(x) = k(5 - x)$ for $x = 0,\ldots, 4$. We need 

\begin{align*}
    k \sum_{x=0}^{5}(5 - x) = 1 \iff k \left[ 5^2 + 5 \cdot 4 + \ldots + 5 \right] = 1
\end{align*}

In other words, we need $k \left[ 5 + 4 + 3 + 2 +1 \right] = k\left[ 15 \right] = 1$, 
i.e. we need $k = \frac{1}{15}$.


(2.a) The probability that at most three lines are in use is 

\begin{align*}
    \sum_{x = 0}^{3} p(x) = .1 + .15 + .2 + .25 = .7
\end{align*}

(2.e) Let us study the event where 2, 3 or 4 lines are not being used.
Two lines not being used equates to using $\leq 4$ lines. Three lines 
not being used equates to $\leq 3$ lines not being used. 4 lines not 
being used equates to $\leq 2$ lines being used. In other words, 
the event is

\begin{equation*}
    E = (X \leq 4) \cup (X \leq 3) \cup (X \leq 2) = (X \leq 4)
\end{equation*}

The probability is then $\sum_{x \leq 4} p(x) = .9$

(2.f) For at least 4 lines not being used we need to four lines not being used,
or 5 not being used, or 6 not being used. This means using 2, 1 or 0 lines. 
The probablity is then $.45$.

(3.a) Let us recall that for any $x_j \in Im(X)$

\begin{align*}
    F(x_i) = \sum_{i=1}^{j} p_X(x_i)
\end{align*}

From this readily follows that 

\begin{align*}
    p_X(x_j) &= F(x_j) - \sum_{i=1}^{j-1} p_X(x_i) \\ 
             &=F(x_j) - F(x_{j - 1})
\end{align*}

So we have a nice formula to derive $p_X(x_j)$ given the CDF. In this case, we have

\begin{align*}
    p_X(x) = \begin{cases}
        .3 & x = 1 \\ 
        .1 & x = 3 \\ 
        .05 & x = 4 \\ 
        .15 & x = 6 \\ 
        .4 & x = 12
    \end{cases}
\end{align*}

It is easy to verify that $\sum_{x \in Im(X)} p_X(x) = .3+.1+.05+.15+.4 = 1$.

(3.b) 

\begin{align*}
    P(3 \leq X \leq 6) &= F(6) - F(3) \\ 
                       &= .6 - .4 \\ 
                       &= .2 \\ 
    P(X \geq 4) &= 1 - P(X < 4) \\ 
                &=1 - F(4) \\ 
                &=1 - .45 \\ 
                &= .55
\end{align*}

(4) Five persons $S = \left\{ s_1, \ldots, s_5 \right\} $. Only $s_1, s_2$ have property $R$.
Samples are drawn from $S$ randomly; on each draw property 
$R$ is verified. Let $X$ be the number of verifications made 
until a sample satisfying $R$ is drawn.

(4.a) Evidently, $\Omega = \left\{ T, NT_1, NT_2 NNT_1, NNT_2, NNNT_1,
NNNT_2\right\} $, where $N$ denotes a negative draw and $T_i$ a positive test
comming from drawing $s_1$ or $s_2$. This model gives $X : \Omega \mapsto
\mathbb{R} \right\} $ defined as $X(\omega) = |\omega|_N$ the number of $N$s in
$\omega$.

The events are clearly not independent, since once a non-positive draw is made,
the probability of obtaining a positive draw increases. Let us observe

\begin{align*}
    p_X(0) &= P(T) = \frac{2}{5} \\ 
    p_X(1) &= P(NT_1 \cup NT_2) = \frac{3}{5} \cdot \frac{2}{4} = \frac{3}{10} \\ 
    p_X(2) &= P(NNT_1 \cup NNT_2) = \frac{3}{5} \cdot \frac{2}{4} \cdot \frac{2}{3} = \frac{1}{5} \\ 
    p_X(3) &= \frac{3}{5} \cdot \frac{2}{4} \cdot \frac{1}{3} = \frac{1}{10}
\end{align*}

It is easy to verify that that $\sum p_X(X) =  1$. The probability that $R$ is
not true in the first two draws is simply $p_X(2) + p_X(3) = \frac{3}{10}$.

(5.a) Let $X$ denote the number of points traversed. To find $p_X$ we need
only examine the following. First of all, at least one point is traversed,
which entails $Im_X = \mathbb{N}$. Now, the probability that only one
point is traversed is simply $\frac{1}{3}$. The probability that two
points are traversed is $\frac{2}{3} \cdot \frac{1}{3}$. That of three
points being traversed is $\frac{2}{3} \cdot \frac{2}{3} \cdot
\frac{1}{3}$. In general, 

\begin{align*}
    p_X(x) = \left( \frac{2}{3} \right)^{x - 1} \frac{1}{3}
    = \frac{2^{x-1}}{3^x}
\end{align*}

Observe that the sum of this p. mass function is a geometric series and

\begin{align*}
    \frac{1}{3} \sum_{x = 1}^{\infty} \left( \frac{2}{3} \right) ^{x-1}
    = \frac{1}{1-\frac{2}{3}} = \left( \frac{1}{3} \right)  3 = 1
\end{align*}

which is what we expect.

The CDF is 

\begin{align*}
    F(n) &= \sum_{x=1}^{n} p_X(x) \\  
    &= \sum_{x=1}^{n} \frac{2^{x-1}}{3^x} \\ 
\end{align*}



















\end{document}



