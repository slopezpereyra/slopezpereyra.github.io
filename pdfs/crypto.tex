\documentclass[a4paper, 12pt]{article}

\usepackage[utf8]{inputenc}
\usepackage[T1]{fontenc}
\usepackage{textcomp}
\usepackage{amssymb}
\usepackage{newtxtext} \usepackage{newtxmath}
\usepackage{amsmath, amssymb}
\newtheorem{problem}{Problem}
\newtheorem{example}{Example}
\newtheorem{lemma}{Lemma}
\newtheorem{theorem}{Theorem}
\newtheorem{problem}{Problem}
\newtheorem{example}{Example} \newtheorem{definition}{Definition}
\newtheorem{lemma}{Lemma}
\newtheorem{theorem}{Theorem}
\usepackage{parskip}

\usepackage[top=1.5cm]{geometry}
% Define a new proof environment with smaller font
\newenvironment{proof}[1][Proof]{\par\small\noindent\textbf{#1:} }{\qed\par\normalsize}

\usepackage[cal=boondoxupr]{mathalpha}


\begin{document}

\section{Preliminaries}
   
Let $\mathcal{G}, \mathcal{E}, \mathcal{D}$ enote the key \textit{generation},
\textit{encription}, and \textit{decription} algorithms, respectively. 
Let $\mathcal{M}$ denote a finite and non-empty message space 
and $\mathcal{K}$ a finite \textit{key space} which is the 
range of $\mathcal{G}$. Clearly, the domain 
of $\mathcal{E}$ is $\mathcal{K} \times \mathcal{M}$,
and the range of $\mathcal{D}$ is $\mathcal{M}$.
Let $\mathcal{C}$ denote the range of $\mathcal{E}$. 

The algorithm $\mathcal{E}$ is allowed to be probabilistic, so that 
$\mathcal{E}(m)$ might produce different outputs even if $m$
is fixed. We write $c \leftarrow \mathcal{E}_k(m)$ to denote the 
(possibly probabilistic) process by which $m$ is encripted 
with key $k$ producing a ciphertext $c$.

We assume (without loss of generality) that $\mathcal{G}$
chooses $k \in \mathcal{K}$ uniformely, and thus we say 
the uniform distribution. Thus, $K$ is defined as the 
random variable which is the output of $\mathcal{G}$.
Similarly, $M$ is the random variable that denotes the 
message being encripted at a particular time, nad 
$Pr\left[ M = m \right] $ denotes the probability 
that the message takes on the value $m \in M$.
The distribution of $M$ is not set by the encryption scheme 
but simply reflects the probability of particular messages being sent by the
different parties involved. $K$ and $M$ are assumed to be independent.

Fixing an encryption scheme and a distribution over $\mathcal{M}$
determines a distribution over $\mathcal{C}$ given by 
choosing a key $k \in \mathcal{K}$ (via $\mathcal{G}$) and 
a message $m \in \mathcal{M}$ (according to the distribution 
of $M$). We let $C$ denote the random variable de\not\in g the 
resulting ciphertext, so that for $c \in \mathcal{C}$, 
$Pr\left[ C = c \right] $ is the probability of producing 
the cyphertext $c$.

\begin{definition}
    An encryption scheme $(\mathcal{G}, \mathcal{E}, \mathcal{D})$ with message space 
    $\mathcal{M}$ is \textbf{perfectly secret} if, for 
    every probability distribution over $\mathcal{M}$, 
    every $m \in \mathcal{M}$, and every $c \in \mathcal{C}$ such that 
    $P(C = c) > 0$:

    \begin{equation*}
        P\left( M = m \mid C = c \right)  = P(M = m)
    \end{equation*}
\end{definition}

\begin{theorem}
    An encryption scheme is perfectly secret if the distribution of the 
    ciphertex does not depend on the plaintext; i.e. for any $m_1, m_2 \in \mathcal{M}$,
    and every $c \in \mathcal{C}$:

    \begin{equation*}
        P\left( \mathcal{E}_K(m_1) = c \right) = P\left( \mathcal{E}_K(m_2) = c \right) 
    \end{equation*}
\end{theorem}

The theorem simply states that if the ciphertext contains no information 
whatsoever about the plaintext, i.e. that if it's impossible to distinguish 
an encryption of $m_1$ from an encription of $m_2$, then there is perfect 
secrecy.

\begin{definition}
    Let $PrivK_{\mathcal{A}, \pi}^{eav}$ be the computable function given by the following 
    effective procedure:

    \begin{itemize}
        \item Simulate an adversary $\mathcal{A}$ which outputs a pair $m_0, m_1 \in \mathcal{A}$.
        \item Generate $k = \mathcal{G}(m_i)$, with $i$ chosen uniformily 
            from $\left\{ 0, 1 \right\} $.
        \item Let $c \leftarrow \mathcal{E}_k(m_i)$.
        \item Let $\mathcal{A}$ randomly output a bit $b$.
        \item If $b = i$ output $1$, if $b \neq i$ 
            output 0.
    \end{itemize}
\end{definition}

\begin{definition}
    An encryption scheme $\pi = \left( \mathcal{G}, \mathcal{E}, \mathcal{D} \right) $ 
    with message space $\mathcal{M}$ is perfectly indistinguishable if, 
    for every agent $\mathcal{A}$ which outputs a pair $m_0, m_1 \in \mathcal{M}$,

    \begin{equation*}
        P \left( \text{PrivK}_{\mathcal{A}, \pi}^{eav} = 1 \right) = \frac{1}{2}
    \end{equation*}
\end{definition}

\begin{theorem}
    An encryption scheme is perfectly secret if and only if it is perfectly 
    indistinguishable.
\end{theorem}

\subsection{One-time pad}

The one-time pad encryption scheme, given $\mathcal{M} = \mathcal{C} =
\mathcal{K} = \left\{ 0, 1 \right\}^{\ell} $, is defined by: 

\begin{itemize}
    \item $\mathcal{G}$ chooses $k \in \mathcal{K}$ uniformily.
    \item $\mathcal{E}$, given $k \in \mathcal{K}$ and $m \in \mathcal{M}$,
        outputs $c := k \circ m$
    \item $\mathcal{D}$, given $k \in \mathcal{K}$ and $c \in \mathcal{C}$,
        outputs $m := k \circ c$
\end{itemize}

Here, $a \circ b$ denotes the bitwise exclusive-or (XOR) 
operation. The scheme is correct because, fixing $k$,
for every $m$ it holds 

\begin{equation*}
    \mathcal{D}_k \left( \mathcal{E}_k (m) \right) = k \circ k \circ m = m
\end{equation*}

\begin{theorem}
    One-time pad encryption is perfectly secret.
\end{theorem}

\begin{proof}
    Observe that 

    \begin{align*}
        P\left( C = c \mid M = m \right) &= P \left( \mathcal{E}_K(m) = c \right)  \\ 
                                         &= P\left( m \circ K = c \right)  \\ 
                                         &= P\left( K = m \circ c \right)  \\ 
                                         &= 2^{-\ell}
    \end{align*}
\end{proof}

where the final equality holds because $K$ is chosen uniformily as one 
in $2^{\ell}$ binary strings. Now observe that, fixing a distribution 
over $\mathcal{M}$, for any $c \in \mathcal{C}$,

\begin{align*}
    P\left( C = c \right)  &= \sum_{m \in \mathcal{M}} P\left( C = c \mid M = m \right) P\left( M = m \right)  \\ 
                           &= 2^{-\ell} \sum_{m \in \mathcal{M}} P(M = m')\\ 
                           &= 2^{-\ell}
\end{align*}

Furthermore, 

\begin{align*}
    P\left( M = m \mid C = c \right)  &= \frac{ P\left( C = c \mid M = m \right) P(M = m)  }{P(C = c)} \\ 
                                      &= \frac{  2^{-\ell}  P(M = m) }{2^{-\ell}} \\ 
                                      &= P(M = m)
\end{align*}

$\therefore $ The scheme is perfectly secret.


\subsection{Limitations of perfect secrecy}

An inherent drawback of perfect secrecy is that the key space must always 
be at least as large as the message space. If all keys are the same length, and the 
message space consists of strings of some fixed length, this implies that the key 
is at least as long as the message. 

A second limitation is that a key $k$ may only be used once, insofar as using it 
twice or more leaks information.

\begin{theorem}
    If $(\mathcal{G}, \mathcal{E}, \mathcal{D})$ is a perfectly secret encryption 
    scheme, then $|\mathcal{K}| \geq |\mathcal{M}|$.
\end{theorem}

\begin{proof}
    Assume the scheme is perfectly secret and $\left| \mathcal{K} \right| < \left| \mathcal{M} \right| $. Consider the uniform distribution over $\mathcal{M}$ and let 
    $c \in \mathcal{C}$ be a ciphertext occurring with non-zero probability. Let 
    $\mathcal{M}(c)$ be the set of all possible messages that are 
    possible decryptions of $c$:

    \begin{equation*}
        \mathcal{M}(c) := \left\{ m : m = \mathcal{D}_k(c) \text{ for some } k \in \mathcal{K} \right\} 
    \end{equation*}

    Clearly, $\left| \mathcal{M}(c) \right| \leq \left| \mathcal{K} \right| < \left| \mathcal{M} \right|   $. 
    Then there is some $m \in \mathcal{M}$
    such that $m \not\in \mathcal{M}(c)$. But then 

    \begin{equation*}
        P(M = m \mid C = c) = 0 \neq P(M = m)
    \end{equation*}

    which means the scheme is not perfeclty secret.
\end{proof}

Informally, if the consequent were not true,
then given a ciphertext $c$, at least one possible message $m$
could be ruled out, which contradicts the perfect secrecy 
assumption.



\begin{theorem}[Shannon's theorem]
    Assume $\left| \mathcal{M} \right| = \left| \mathcal{C} \right| = \left| \mathcal{K} \right| $. Then a scheme is perfectly secret if and only if: 

    \begin{itemize}
        \item Every $k \in \mathcal{K}$ is chosen uniformily by $\mathcal{G}$.
        \item For every $m \in \mathcal{M}, c \in \mathcal{C}$,
            there is a unique $k \in \mathcal{K}$ such that 
            $\mathcal{E}_k(m)$ outputs $c$.
    \end{itemize}
\end{theorem}

( Proof in page 36. ) This theorem provides a way to decide if a scheme 
is perfectly secret. The first condition is easy to check, and the second condition 
can be proven or contradicted without having to deal with any 
probabilities.

\pagebreak

\section{Private-key (Symmetric) Cryptograph}

\begin{definition}[Informal concrete definition]
    A scheme is $(t, \epsilon)$-secure if any adversary running 
    for some amount of time $w \leq t$ succeedes in breaking the 
    scheme with probability $p \leq \epsilon$.
\end{definition}

\begin{definition}[Informal asymptotic definition]
    A scheme is secure if any adversary running for time 
    $O(p(n))$, with $n$ a security parameter,
    succeedes in breaking the code with probability 
    smaller than any inverse polynomial in $n$.
\end{definition}

\begin{example}
    Assume an adversary running for $n^3$ minutes can succeede 
    in breaking the scheme with probability $2^{40} 2^{-n}$.
    When $n \leq 40$, $40^3$ minutes (approx. $6$ weeks)
    breaks the code with probability $1$, so this is not 
    useful. When $n = 500$, an adversary running for 
    $200$ years breaks the scheme with probability roughly 
    $2^{-500}$.
\end{example}

The security parameter increases the time required to run the scheme,
as well as the length of the key, so a balance between the desired 
security and the practicality of the scheme is to be reached.

\subsection{The asymptotic approach}

\begin{definition}
    An algorithm is efficient if it runs in polynomial time.
\end{definition}

We allow all algorithms to be randomized or probabilistic. Any 
such algorithm may make a non-deterministic decision at any step 
of its execution. Randomness is essential to cryptography, to the 
point that honest parties must be probabilistic. Furthermore, 
randomization gives attackers additional power, and it makes sense 
to consider the strongest possible opponent. 

\begin{definition}
    A function $\varphi : \mathbb{N} \to \mathbb{R}^+_0$ is negligible if for 
    every polynomial $p \in \mathcal{P}$ there is an 
    $N$ such that $f(n) < \frac{1}{p(n)}$ for all
    $n > N$.
\end{definition}

\begin{example}
    The functions $2^{-n}, n^{- \log n}$ are negligible,
    though they approach zero at different rates. For instance, 
    $2^{-n} < n^{-5}$ if and only if $n > 5 \log n$, with $n = 23$
    being the least number which satisfies this. Alternatively, 
    $n^{-\log n} < n^{-5}$ if and only if $\log n > 5$, with 
    $n = 33$ being the least number which satisfies this.
\end{example}

\begin{theorem}
    Let $\varphi_1, \varphi_2$ be negligible functions. Then 
    $\varphi_1 + \varphi_2$ is negligible, and 
    for any poltnomial $p$, 
    $p(n) \cdot \varphi_1(n)$ is negligible.
\end{theorem}

\begin{proof}
    Complete.
\end{proof}

The theorem guarantees that if certain event occurs with only 
negligible probability, such event occurs with negligible 
probability even if the experiment is repeated polynomially 
many times.

A corollary of the theorem is that if a function $g$ is not 
negligible, then neither is the function $f = g / p$  for any 
positive polynomial $p$.

\begin{definition}[Informal security definition]
    A scheme is secure if, for every probabilistic polynomial-time 
    adversary $\mathcal{A}$ carrying out an attack of some 
    formally specified type, the probability that $\mathcal{A}$
    succeeds (where "succeed" is formally specified) is 
    negligible.
\end{definition}

The definition is asymptotic because it is posible that, for small 
values of $n$ (the security parameter), an adversary may succeed with 
high probability. But the point is that there is some $N$ such that,
if $n > N$, the probability of the attacker succeeding becomes less 
than $1 / p(n)$ for any polynomial $p$.

\subsection{Computationally secure encryption}

\begin{definition}
    A private-key encryption scheme is a tuple of probabilistic 
    polynomial-time algorithms $(\mathcal{G}, \mathcal{E}, \mathcal{D})$
    such that: 

    \begin{itemize}
        \item $\mathcal{G}$ takes as input $1^n$ (the security 
            parameter written in unary) and outputs a key $k \in \mathcal{K}$,
            where it is assumed that $|k| \geq n$.
        \item $\mathcal{E}$ takes as input $k \in \mathcal{K}$ and 
            $m \in \left\{ 0,1 \right\}^{*} $ and outputs a 
            $c \in \mathcal{C}$.
        \item $\mathcal{D}$ takes as input $k \in \mathcal{K}, c \in \mathcal{C}$, and outputs $m \in \mathcal{M}$ deterministically.
        \item For every $k \leftarrow \mathcal{G}(1^n), n \in \mathbb{N}, m \in \left\{ 0,1 \right\}^{*} $, $\mathcal{D}_g\left( \mathcal{E}_k(m) \right) = m $
    \end{itemize}
\end{definition}

If $(\mathcal{G}, \mathcal{E}, \mathcal{D})$ is such that $\mathcal{E}_k$ is defined only for messages $m \in \left\{ 0, 1 \right\}^{\ell(n)} $, we say $(\mathcal{G}, \mathcal{E}, \mathcal{D})$ is a fixed-length 
private-key encryption scheme for messages of length $\ell(n)$.

Considering an adersary $\mathcal{A}$  that runs in polynomial time,
we define $\text{Priv}_{\mathcal{A}, \pi}^{eav}$ just as before,
except that we now admit that $\mathcal{A}$ may guess which of 
two messages was encrypted with probability \textit{negligibly}
better than $50\%$. Since now the scheme is parametrized by 
$n$, the security parameter, we write 

\begin{equation*}
    P(\text{PrivK}^{eav}_{\mathcal{A}, \pi}(n) = 1)
\end{equation*}

to denote the probability that $\mathcal{A}$ correctly guesses 
which of two messages was encrypted.

\begin{definition}
    A private-key encryption scheme $\pi = (\mathcal{G}, \mathcal{E}, \mathcal{D})$ has indistinguishable encryptions in the presence of an 
    eavesdropper, or is EAV-secure, if for all probabilistic 
    polynomial-time adversaries $\mathcal{A}$, there is a 
    negligible function $\varphi$ such that:

    \begin{equation*}
        \forall n \in \mathbb{N} : P\left[ \text{PrivK}_{\mathcal{A}, \pi}^{eav}(n) = 1 \right] \leq \frac{1}{2} + \varphi(n)
    \end{equation*}

    where the probability is taken over the randomness used by 
    $\mathcal{A}$ and the randomness used in the experiment.
\end{definition}

\subsection{Psuedorandom generators and stream ciphers}

A pseudorandom generator $G$ is a deterministic algorithm which 
transforms a short, uniform string called the seed into a longer,
pseudorandom output string. 

\begin{definition}
    Let $\ell$ a polynomial and $G$ a deterministic polynomial-time 
    algorithm such that, for any natural $n$ and binary string 
    $s$ of length $n$, $G(s)$ is a string of length $\ell(n)$.
    We say $G$ is a pseudorandom generator if: 

    \begin{itemize}
        \item For every $n$, $\ell(n) > n$.
        \item For any PPT algorithm $D$, there is some $\varphi \in \mathcal{N}$ such that 

            \begin{equation*}
                \left| P\left[ D\left( G(s) \right) = 1 \right] - P \left[ D(r) = 1 \right]   \right| \leq \varphi(n)
            \end{equation*}
        where the first probability is taken over the uniform choice 
        of $s$ and the randomness of $D$, and the second probability 
        is taken over the uniform choice of $r \in \left\{ 0, 1 \right\}^\ell(n) $ and the randomness of $D$.
    \end{itemize}

    We call $\ell$ the expansion factor of $G$.
\end{definition}

A stream cipher is a pair of deterministic algorithms $(Init, GetBits)$
where: 

\begin{itemize}
    \item Init takes an input seed $s$, an optional initialization vector
        $\overrightarrow{v}$, and outputs an initial state $q_0$.
    \init GetBits takes as input $q_i$ and outputs a bit $y$ and an updated 
    state $q_{i+1}$. (In practice, $y$ is usually a block of several bits.)
\end{itemize}

Given $\ell$ an expansion factor, an algorithm $G_\ell$ may be defined as
mapping inputs of length $n$ to outputs of length $\ell(n)$. The 
algorithm runs Init and then repeatedly runs GetBits $\ell$ times.

\begin{align*}
    &\text{Input}: s, \overrightarrow{v}\\ 
    &\text{Output}: \overrightarrow{y} \\ 
    &\\ 
    &q_0 := Init(s, \overrightarrow{v}) \\ 
    &\textbf{for } i = 1 \textbf{ to } \ell \textbf{ do } \\ 
    &\qquad (y_i, q_i) := GetBits(q_{i-1}) \\ 
    &\textbf{od} \\ 
    &\textbf{return } \overrightarrow{y}
\end{align*}



\subsection{Reduction proofs}

If a problem is hard, then we may prove a scheme is secure by 
showing that any efficient adversary $\mathcal{A}$ is reducible to 
an algorithm $\mathcal{A}'$ which is a solution to the hard 
problem. Let $X$ be a hard problem. A security proof may look as follows: 

\begin{itemize}
    \item Fix some efficient (PPT) adversary $\mathcal{A}$ attacking 
        the scheme $\pi$, and let $\epsilon(n)$ be its 
        success probability.
    \item Construct $\mathcal{A}'$ that attempts to solve 
        $X$ using $\mathcal{A}$ as a subroutine. So, 
        given an input instance $x$ of the problem $X$, $\mathcal{A}'$
        will simulate for $\mathcal{A}$ an instance of $\pi$
        such that: 

        \begin{itemize}
            \item Running as a subroutine of $\mathcal{A}'$, $\mathcal{A}$ attacks $\pi$. 
            \item If $\mathcal{A}$ succeeds, then $\mathcal{A}'$ solves 
                $x$ at least with inverse polynomial probability 
                $1 / p(n)$.
        \end{itemize}
    \item The first two items entail that $\mathcal{A}'$ solves 
        $X$ with probability $\epsilon(n) / p(n)$. If $\epsilon(n)$ 
        is not negligible, then this last probability isn't either. 
        Furthermore, if $\mathcal{A}$ is efficient, we have found an 
        efficient algorithm $\mathcal{A}'$ to solve $X$, which 
        contradicts the initial assumption. 
    \item We conclude no efficient adversare $\mathcal{A}$ can succeed 
        in breaking $\pi$ with non-negligible probability. In 
        other words, $\pi$ is computationally secure.
\end{itemize}


\subsection{A secure fixed-length encryption scheme}

One-time padding consisted in using a uniformly sampled 
pad $\mathcal{p} \in \left\{ 0, 1 \right\}^{\ell} $ to encrypt/decrypt 
a message as 

\begin{equation*}
    m \overbrace{\rightarrow }^{\text{encryption}} c := m + \mathcal{p} \overbrace{\rightarrow }^{\text{decryption}} m = c + \mathcal{p}
\end{equation*}

An insight from previous sections is that we can generate 
a pseudorandom pad $\mathcal{p}$ via a seed $\mathcal{s}$,
and share only $\mathcal{s}$ as key. Thus, we defeat one of the 
limitations faced; namely, our key becomes shorter than our message.
In terms of security, a pseudorandom string "looks random" to any 
polynomial-time adversary.

The scheme is as follows. Fix $\ell$ some message length and let 
$G$ be a pseudorandom generator with expansion factor $\ell$.
Let $\pi = (\mathcal{G}, \mathcal{E}, \mathcal{D})$
be as follows:

\begin{itemize}
    \item $\mathcal{G}(1^n)$ outputs a uniform $k \in \mathcal{K}$ 
        of length $n$, the security parameter. 
    \item $\mathcal{E}$ works by computing $\mathcal{p} \leftarrow G(k)$,
        and then $c := m + \mathcal{p}$.
    \item $\mathcal{D}$ computes $\mathcal{p} \leftarrow G(k)$
        and then $m = \mathcal{p} + c$ .
\end{itemize}

\begin{theorem}
    If $G$ is a pseudorandom generator, the scheme provided above 
    is a fixed-length private-key encryption scheme that has 
    indistinguishable encryptions in the presence of an eavesdropper.
\end{theorem}

\begin{proof}
    Long. Page 89.
\end{proof}

\subsection{CPA Security}

\begin{definition}
    A private-key encryption scheme $\pi = (\mathcal{G}, \mathcal{E}, \mathcal{D})$ has indistinguishable encryptions under a chosen-plaintext attack, or is CPA-secure, if for all probabilistic polynomial-time adversaries $\mathcal{A}$, there is a function $\varphi \in \mathcal{N}$ such that: 

    \begin{equation*}
        P \left[ \text{PrivK}_{\mathcal{A}, \pi}^{\text{cpa}}(n) = 1 \right]  \leq \frac{1}{2} + \varphi(n)
    \end{equation*}
\end{definition}

The experiment PrivK$_{\mathcal{A}, \pi}^{\text{cpa}}$ is simply the 
chosen-plaintext attack experiment: 

\begin{itemize}
    \item $k \in \mathcal{K}$ is generated (secretely) and $\mathcal{A}$ is given 
        access to $\mathcal{E}_k$.
    \item $\mathcal{A}$ outputs two messages of equal length, $m_0, m_1 \in \mathcal{M}$.
    \item $i \in \left\{ 0, 1 \right\} $ is chosen uniformily and 
        $m_i$ is encoded with $\mathcal{E}$.
    \item $\mathcal{A}$ continues to have oracle access to $\mathcal{E}$,
    and ventures (outputs) a guess $b \in \left\{ 0, 1 \right\} $.

    \item If $ b = i$ the result is 1, zero otherwise.

\end{itemize}


\begin{theorem}
    If $(\mathcal{G}, \mathcal{E}, \mathcal{D})$ is a private-key encryption 
    scheme that is CPA-secure, it is CPA-secure for 
    multiple encryptions.
\end{theorem}

\begin{theorem}
    If $\pi = (\mathcal{G}, \mathcal{E}, \mathcal{D})$ is a 
    fixed-length private-key encryption scheme that is CPA-secure, there is an arbitrary length
    private-key encryption scheme $\pi' = (\mathcal{G}, \mathcal{E}', \mathcal{D})$.
\end{theorem}

\begin{proof}
    Assume (w.l.g.) that $\pi$ encodes messages of $1$ bit. Let 
    $m = m_0 m_1 \ldots m_\ell$  be a message of $\ell$ bits.
    Then $c = \mathcal{E}(m_0)\mathcal{E}(m_1)\ldots\mathcal{E}(m_\ell)$ 
    is an encryption of $m$. That the encryption is secure follows 
    from the theorem preceding the one being proved here.
\end{proof}

\section{Constructing CPA-secure schemes}

\subsection{Pseudorandom functions}

For convenience, let us use $\mathcal{B}$ to denote 
$\left\{ 0,1 \right\}^{*} $, the set of all 
binary strings, and $\mathcal{B}_k$ to denote 
$\left\{ 0, 1 \right\}^{k} $ the set 
of all binary strings of length $k$.

A keyed function is a function $f : \mathcal{B}^2 \to \mathcal{B}$ where the
first argument is called the key and is denoted by $k$. We say $f$ is
efficient if there is a polynomial-time algorithm that computes $f(k, x)$ for
any $k, x \in \mathcal{B}$. Typically, we treat $k$ as fixed and we sometimes
write $f_k(x)$. The security parameter $n$ dictates 
the key length, input length, and output length.
That is, three functions $\ell_{key}, \ell_{in}, \ell_{out}$
are associated with $f$: for any $k \in \mathcal{B}_{\ell_{key}(n)}$,
the functoin $f_k$ is only defined for inputs 
$x \in \mathcal{B}_{\ell_{in}(n)}$. For simplicity,
we assume each $f_k$ is length-preserving, i.e. the 
lengths of input, output and key are equal and of 
length $n$.

Since each keyed function $f$ induces a $k$-indexed family 
of functions $\mathcal{f}_k = \left\{ (k, f_k) : k \in \mathbb{N} \right\} $, we may uniformly choose $k \in \mathcal{B}_n$ and consider 
only the resulting $f_k$. No adversary can distinguish 
whether it is interacting with a uniform choice of $k$
producing $f_k$ or with a uniform choice of $f$ from 
the set of all keyed functions.

One may recall that there are $|B|^{|A|}$ functions of 
the form $g : A \to B$. Since a keyed, length-preserving function
is of the form $f : \mathcal{B}_n^2 \to \mathcal{B}_n$,
and $|\mathcal{B}_n| = 2^n$, there are $( 2^{n}2^{n} )^{2^n} = 2^{4n^2}$
keyed functions for any security parameter $n$. Similar reasoning 
gives $2^{n 2^n}$ functions in $\mathcal{f}_k$.

From now on, we define:

\begin{equation*}
    \mathcal{F}_n := \left\{ f \mid f : \mathcal{B}_n \to \mathcal{B}_n \right\} 
\end{equation*}

The intuition behind a pseudorandom function is this. A polynomial-time
distinguisher $D$ is given an oracle $\mathcal{O}$ which is either equal 
to $f_k$ (for uniform $k$) or equal to $f$ uniformly chosen from $\mathcal{F}_n$.
We assume the oracle is deterministic, and observe that $D$ can 
ask only polynomially many queries to $\mathcal{O}$ if it is to remain 
a polynomial-time algorithm.


\begin{definition}
    Let $f : \mathcal{B}^2 \to \mathcal{B}$ be an efficient, length 
    preserving, keyed function. $F$ is a pseudorandom 
    function if for all probabilistic polynomial-time distinguisher 
    $D$, there is some $\varphi \in \mathcal{N}$ such that: 

    \begin{equation*}
        \left|   P\left[ D^{F_k(\dot)}(1^n) = 1 \right] - P \left[ D^{f(\dot)}(1^n) = 1 \right]     \right| \leq \varphi(n)
    \end{equation*}

    where the first probability is taken over uniform choice of $k \in \mathcal{B}_n$ 
    and the randomness of $D$, and the second probability is taken over uniform 
    choice of $f \in \mathcal{F}_n$ and the randomness of $D$.
\end{definition}

Informally, $f$ is a pseudorandom function if an efficient distinguisher cannot
distinguish it from an arbitrary function uniformly sampled from the function
space with more than negligible probability. We must observe that the pseudorandom 
function $f_k$ is determined by the key, and hence the inability to distinguish $f_k$,
with $k$ sampled uniformly from $\mathcal{B}_n$, from $f$ sampled uniformly from 
$\mathcal{F}_n$, corresponds to an inability of inferring the key.

\begin{example}
    Let $f_k(x) = k + x$. If $k$ is uniform then the output of $f_k(x)$ is uniform 
    for any $x$. However, $f$ is not pseudorandom because its values on any two 
    points are correlated. 

    Concretely, consider $D$ querying $\mathcal{O}$ on $x_1, x_2$ to obtain 
    $\mathcal{O}(x_1) = y_1, \mathcal{O}(x_2) = y_2$, outputting $1$
    iff $y_1 + y_2 = x_1 + x_2$.

    If $\mathcal{O} = f_k$, for any $k$, then $D$ outputs 1. If $\mathcal{O} = g$,
    with $g$ chosen uniformly from $\mathcal{F}_n$, then the probability 
    that $y_1 + y_2 = x_1 + x_2$ is the probability that 
    $y_2 = x_1 + x_2 + y_1$, which is $2^{-n}$. 

    So, 

    \begin{equation*}
        \left|   P\left[ D^{F_k(\dot)}(1^n) = 1 \right] - P \left[ D^{f(\dot)}(1^n) = 1 \right]     \right| = \left| 1 - 2^{-n} \right| 
    \end{equation*}

    which is not negligible.
\end{example}

Let $\mathcal{P}_n$ denote the set of all permutations (bijections)
on $\mathcal{B}_n$. The cardinality of $\mathcal{P}$ is 
$( 2^n )!$.

We say $F$ is a keyed permutation if $\ell_{in} = \ell_{out}$ and 
for all $k \in \mathcal{B}_{\ell_{key}(n)}$, the function
$f_k : \mathcal{B}_{\ell_{in}(n)} \to  \mathcal{B}_{\ell_{out}(n)}$
is one-to-one (i.e. a permutation). We call 
$\ell_{in}$ the block length of $f$, and again we assume 
the input, output and key lengths are equal unless 
stated otherwise. 

A keyed permutation $f_k$ is efficient if there is a polynomial-time 
algorithm for computing $f_k(x) = y$ and $x = f^{-1}_k(y)$.

\begin{theorem}
    If $f$ is a pseudorandom permutation and $\ell_{in}(n) \geq n$,
    then $f$ is a pseudorandom function.
\end{theorem}

A pseudorandom permutation is said to be \textbf{strong} if 
it is indistinguishable even when a distinguisher is given 
oracle access to its inverse.

\begin{definition}
    A block cipher is an instantiation of a strong pseudorandom 
    permutation with some fixed key length and block length.
\end{definition}

A pseudorandom function/block cipher $f$ can be used to build 
pseudorandom generators by evaluating $f$ on a series of different 
inputs. In other words, $G(s) := f_s(1) \overline{+} f_s(2) \overline{+} \ldots \overline{+} f_s(\ell)$
pseudorandomly generates a binary string of length $\ell$
from a seed $s$, where $\overline{+}$ denotes the bitwise or.

More generally, a stream cipher $(Init, GetBits)$ with input vector
$v$ can be built from a pseudorandom function $f$ as follows:

\begin{itemize}
    \item Let $Init$ with input $s, v \in \mathcal{B}_n$ set $q_0 := (s, v)$.
    \item On input $q_i = (s, v_i)$ let $GetBits$ compute 
        $v_{i+1} = v_{i} + 1$, and set $y := f(v_{i+1})$. Let 
        $q_{i+1} := (s, v_{i+1})$ and output $(y, q_{i+1})$.
\end{itemize}

Thus, the encryption is

\begin{equation*}
    f_s(v_0 + 1), f_s(v_0 + 2),\ldots
\end{equation*}

\subsection{CPA-secure scheme from pseudorandom functions}

We cannot let $\mathcal{E}_k(m) = f_k(m)$ because $f_k$ is deterministic, and
thus the scheme could not be CPA-secure. However, we may encrypt by applying
$f_k$ to a random value $r$ instead of the message, and taking
$\mathcal{E}_k(m) = f_k(r) + m$. This in fact constitutes a pseudorandom pad
with the plaintext, but the pad $f_k(r)$ is different each time 
(assuming the same $r$ is not obtained repeatedly). 

\textbf{Construction.} Let $f$ a pseudorandom function. Define 

\begin{itemize}
    \item $\mathcal{G}$ as sampling $k \in \mathcal{B}_n$.
    \item $\mathcal{E}$ as computing $c:= \left< r, f_k(r) + m \right>$ 
        on input $k, m$, with $r$ uniformly sampled from $\mathcal{B}_n$.
    \item $\mathcal{D}$ as takingi nput $k, c = \left< r, s \right>$ and 
        outputting $m := f_k(r) + s$.
\end{itemize}

\begin{theorem}
    If $F$ is a pseudorandom function, then the above construction is  a
    CPA-secure private key encryption scheme for messages of length $n$.
\end{theorem}



















\end{document}



