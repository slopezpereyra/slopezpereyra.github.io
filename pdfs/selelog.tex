\documentclass[a4paper, 12pt]{article}

\usepackage[utf8]{inputenc}
\usepackage[T1]{fontenc}
\usepackage{textcomp}
\usepackage{amssymb}
\usepackage{newtxtext} \usepackage{newtxmath}
\usepackage{amsmath, amssymb}
\usepackage{forest}
\newtheorem{problem}{Problem}
\newtheorem{example}{Example}
\newtheorem{lemma}{Lemma}
\newtheorem{theorem}{Theorem}
\newtheorem{problem}{Problem}
\newtheorem{example}{Example} \newtheorem{definition}{Definition}
\newtheorem{lemma}{Lemma}
\newtheorem{theorem}{Theorem}


\begin{document}

\section{Introduction}

I will give you two kinds of problems. The first kind is the one you face in
your exam: finding the logical structure of non-formal arguments. The second
kind is the reverse of the first, and thus constitutes a good practice for it:
writing formal arguments in a non-formal way. 

For instance, a problem of the first kind could consist in detecting that the 
phrase 

\begin{quote}
    Sin ruedas, ningún automóvil puede moverse.
\end{quote}

resolves to $$\forall a \in A : \neg R(a) \Rightarrow \neg M(a)$$ where $A$ is
the set of all automoviles, $R$ the property of having wheels, and $M$ the
property of being able to move. In essence, this means detecting ($a$) that it
is a \textit{universal} argument (it verses about all elements in a set),
that it involves two predicates ($R$ and $M$), and that it states that 
one is a \textit{necessary condition} of the other.

A problem of the second kind could consist of reading a formal deductive argument such as 

\begin{align*}
    &\varphi \\ 
    &\varphi \Rightarrow \psi \\ 
    &\therefore \psi
\end{align*}

and inventing a scenario for it substituting its symbols (in this case $\varphi, \psi$) with 
real phrases. For example, if $\varphi$ where "Pigs fly" and $\psi$ where "I love 
my husband", the argument would be translated to any of the following forms:

\begin{quote}
    \begin{itemize}
        \item Pigs fly, and if pigs could fly I'd love my husband. Therefore, I love my husband. 
        \item If pigs could fly I'd love my husband, and pigs indeed fly. It follows that I love 
            my husband.
        \item I love my husband, because pig's fly, and that is all my love for him needs to 
            exist.
        \item etc.
    \end{itemize}
\end{quote}

But before moving on, let me remind you the meaning of logical symbols, and how to use them.

\subsection{Logic operators}

We use $\varphi, \psi, \phi$, and other such symbols to mean propositions. Propositions are 
statements susceptible of being true or false, and are the atoms which make up 
an argument. The symbol $\therefore $ means "therefore", and is usually used to 
mark the conclusion of an argument. 


Logical connectors are operators which link propositions.

\begin{itemize}
    \item $\varphi \equiv \psi$ means both formulas are entirely equivalent,
        and what makes one true/false makes the other one true/false. In any
        phrase, one may be replaced with the other without altering the meaning
        of the phrase.
    \item $\varphi \land  \psi$ is true if and only if both $\varphi$ and
        $\psi$ are true, and it reads "$\varphi$ and $\psi$".
    \item $\varphi \lor  \phi$ is true if and only if at least one of the propositions is true, 
        and it reads "$\varphi$ or $\phi$".
    \item $\varphi \Rightarrow \psi$ means that if $\varphi$ is true, then 
        $\psi$ is true as well, and it reads "$\varphi$ implies/entails $\psi$".
     
    \item $\varphi \iff \psi$ means that if $\varphi$ is true then $\psi$ is true, 
    and if $\psi$ is true then $\varphi$ is true. It reads "$\varphi$ if and only if $\psi$".
    \item $\neg \psi$ is true if $\psi$ is false, and viceversa, and it reads "not $\psi$".
\end{itemize}

\begin{problem}
    Is it true that $\varphi \Rightarrow \psi$ is the same than "$\varphi$ is a sufficient condition for $\psi$"?
\end{problem}

\begin{problem}
    Is it true that $\varphi \Rightarrow \psi$ is the same than "$\varphi$ is a necessary 
    condition for $\psi$"?
\end{problem}

\begin{problem}
    Is it true that $\varphi \iff \psi$ is the same than "$\varphi$ is a necessary and sufficient 
    condition for $\psi$"?
\end{problem}

\begin{problem}
    How would you write "$\varphi$ is a condition for $\psi$" in logical terms, without specifying 
    necessity nor sufficiency?
\end{problem}

Quantifications are important when propositions verse about elements in a
collection. For example, a proposition may speak of "all men" or "some men".
Expressions of the first kind are termed universal. Expressions of the second
kind are termed existential. 

In terms of logic, universal statements state that all elements in a set satisfy 
some predicate. If that set is called $S$, and the property they satisify is called $P$,
we write 

\begin{equation*}
    \forall x \in S : P(x)
\end{equation*}

to say any element $x$ of $S$ satisfies the property $P$. The classic example is 
to have $S$ be the set of all men and $P$ the property of being mortal, where 
the formula above means "all men are mortal". 

To say some element(s) of $S$ satisfies the property $P$, we write 

\begin{equation*}
    \exists x \in S : P(x)
\end{equation*}

For example, some (but not all) men are white. So if $P$ is the property of 
being white, the statement above is true. 

Observe that if $\forall x \in S : P(x)$ is true, then $\exists x \in S : P(x)$
is true as well. If all elements satisfy $P$, obviously some members of $S$
satisfy $P$. This means the universal quantification is, in a sense, stronger
than the existential quantification.

An important fact of logic is that to negate an existential quantification is
to say \textit{no} element in the set satisfies the property, and to negate a
universal quantification is to say \textit{at least} one element does not
satisfy the property. In logical terms,

\begin{align*}
    &\neg \left( \forall x : P(x) \right) \equiv \exists x : \neg P(x) \\ 
    &\neg(\exists x : P(x)) \equiv \forall x : \neg P(x)
\end{align*}

Think about the equivalences above and why they make sense. 

\begin{problem}
    How could you negate the statement: "All dogs are brown"
\end{problem}

\begin{problem}
    What negates the statement: "There are some houses in Britain which reach a
    hundred meters in altitude"? Write the negation in logical terms.
\end{problem}

\begin{problem}
    A laboratory at Yale cultivates transgenic worms of the \textit{C. elegans} species. The particular 
    mutation of the lab is kept secret, so that the species of \textit{C. elegans} in the lab 
    is found nowhere else in the world. It should be obvious, however, that the world 
    is filled with \textit{C. elegans} worms of the usual (non-mutant) kind. 

    (1) The lab claims to have proved that all \textit{C. elegans} worms 
    posses a particular gen which encodes the protein $K_9B\alpha$. What 
    finding could refute this claim?

    (2) A second lab has their own transgenic mutant species of \textit{C.
    elegans}. Yale's lab claims that all of their \textit{C. elegans} worms are
    genetically identical to all \textit{C. elegans} at the second lab.
    What would constitue conclusive evidence against the claim?

    (3) The lab claims that some naturally occuring \textit{C. elegans} in the
    wild are genetically identical to their mutant species. In other words,
    that the mutation they have induced occurs naturally as well. What would be
    the minimal amount of evidence required to make their claim true? What would 
    be the least amount of evidence required to make their claim false? Which one 
    is easier to obtain?

    (4) Write the claims in point (1), (2), (3) using formal language.
\end{problem}

A proposition may be simple or complex. A simple proposition has no operators
and no quantifications. For instance, "coffe is brown" has no "and", "or", no
implication, nothing. A complex proposition involves operators. For instance,
"if Coffe were brown, all cars would have both wheels and the ability to move"
has implications, quantifications, and logical ands: 

\begin{align*}
    &\text{Coffe is brown}&\text{implies} &~ ~ ~ ~ \text{All cars}&\text{Have wheels and move}\\
    &\varphi &\Rightarrow  &~ ~ ~ ~ \forall a \in A : &R(a) \land M(a)
\end{align*}

Thus, whenever you see a symbol such as $\varphi$ standing for a proposition,
do not assume $\varphi$ is a simple proposition. $\varphi$ might be composed of
a hundred propositions connected in intricate ways!

\section{Types of deductive arguments}

We say two arguments are of the same type when their logical structure is the
same. Understanding and being acquainted with the different types of deductive
arguments is crucial in rapidly detecting flaws or hidden assumptions.

\subsection{Modus ponens}

\textit{Modus ponens} is the paradigmatic example of deductive argument. A \textit{modus ponens}
argument is of the form 

\begin{align*}
    &\varphi \Rightarrow \phi \\ 
    &\varphi \\ 
   \therefore  & \psi
\end{align*}

For example, "If it rains, I use an umbrella. It is raining. Therefore, I am using an umbrella". 

\begin{problem}
    Give a sophisticated example of \textit{modus ponens} where $P$ and $Q$ are complex propositions 
    (i.e. they involve quantifications, logical operators, etc). The argument must not 
    necessarily correspond to reality.
\end{problem}

\subsection{Modus tollens}

\textit{Modus tollens} is the reverse of \textit{modus ponens}. It is of the form 

\begin{align*}
    &\varphi \Rightarrow \phi \\ 
    &\neg \phi \\ 
    \therefore &\neg \varphi
\end{align*}

For example: "If it rains, I use an umbrella. I am not using an umbrella. Hence, it cannot be raining."

\begin{problem}
    Same as with modus tollens.
\end{problem}

\begin{problem}
    Find the hidden assumption in the argument below:
\end{problem}

\begin{quote}
    Professor: It is easy to prove that $n$ is not prime. There are two ways in which this can 
    be done: the first involves the fact that prime numbers conform a ring in $\mathbb{Z}$. But 
    simpler is to observe that there is a number between $2$ and $n - 10$ which divides $n$, 
    from which our conclusion follows directly. 
\end{quote}

\begin{problem}
    Consider the argument above, including the hidden premise (i.e. take it as a complete argument). 
    Is it correct to say that, under a certain perspective, it is a modus ponens, and under another 
    perspective, it is a modus tollens? 
\end{problem}

\begin{problem}
    Why is the argument below \textbf{not} a modus ponens?
\end{problem}

\begin{quote}
    By definition, a kilometer is a thousand meters. The distance from $A$ to $B$ is 
    a thousand meters. Therefore, the distance from $A$ to $B$ is a kilometer.
\end{quote}

\subsection{Hypothetical syllogism (important for the test!)}

A hypothetical syllogism is an argument of the following form: 

\begin{align*}
    &\varphi \Rightarrow \psi \\ 
    &\psi \Rightarrow \phi \\ 
    \therefore &\varphi \Rightarrow \phi
\end{align*}

Why is this important for the test? Because it is the kind of argument which
appeared in the "what is the hidden assumption" problems you got wrong! The
essence of this syllogism is that the relationship of being "a consequence of"
is transitive; i.e. if $b$ is a consequence of $a$, and $c$ a consequence of
$b$, then $c$ is a consequence of $a$.

Your test typically adds a trap: instead of presenting an event $A$ as a consequence 
of an event $B$, it presents an event $A$ as a condition of an event $B$. This is,
it presents arguments of the following kind:

\begin{align*}
    &\neg \varphi \Rightarrow \neg \psi \\ 
    &\neg \psi \Rightarrow \neg \phi \\ 
    \therefore &\neg \varphi \Rightarrow \neg \phi
\end{align*}

which reads "$\varphi$ is a condition for $\psi$", etc. It should be obvious
that \textit{the form} of the argument is the same, since $\neg \varphi, \neg
\psi, \neg \phi$ are propositions just as much as $\varphi, \psi, \phi$ 
are propositions. It should also be obvious that the "chain" of 
implications could have as many terms as desired; i.e. that 

\begin{align*}
    &\varphi_1 \Rightarrow \varphi_2 \\ 
    &\varphi_2 \Rightarrow \varphi_3 \\ 
    &\vdots \\ 
    &\varphi_{n-1} \Rightarrow \varphi_n\\
    &\varphi_{n} \Rightarrow \psi\\
    \therefore & \varphi_{1} \Rightarrow \psi
\end{align*}

is of the same form. Two of the "find the hidden assumption" problems you had a
problem with were of the form 

\begin{align*}
    &\neg \varphi_1 \Rightarrow \neg \varphi_2 \\ 
    &\neg \varphi_2 \Rightarrow \neg \varphi_3 \\ 
    &(?) \\ 
    \therefore &\neg \varphi_1 \Rightarrow \psi
\end{align*}

\begin{problem}
    Find the hidden assumption.
\end{problem}

\begin{quote}
    Crossing the river requires a boat. The building of a boat requires buying wood of 
    the kind $\zeta$. Crossing the river is impossible with our budget.
\end{quote}

\begin{problem}
    Find the hidden assumption.
\end{problem}

\begin{quote}
    Politician: Reducing the fiscal deficit will, in the long run, produce a 
    more equitable society. Indeed, reducing the deficit is impossible
    impossible if the evil state, which I know you all detest as much as I do,
    doesn't shrink to its bare minimum. Such reduction must be carried out in
    multiple spheres, but in none as strongly as in public spending, which
    cannot be achieved without virtually shutting public infrastructure down. 
    This, accompanied by the right economical policies, will lure private investment.
\end{quote}

\begin{problem}
    Hidden assumption.
\end{problem}

\begin{quote}
    For a function to be a homotopy, it must be a continuous mapping between 
    two topological spaces with domain $[0, 1]$. But this is true only if 
    the function is differentiable in $\mathbb{R}$. Hence, for a function 
    to be a homotopy, it must have a non-complex range.
\end{quote}















\pagebreak
\subsection{Proof by cases}

A proposition $\tau$ is a tautology if it is always true. For example, $\tau = \varphi \lor \neg \varphi$
is a tautology: it is always true that either $\varphi$ is true, or $\neg \varphi$ 
is true, making $\tau$ true in all cases.

The tautology $\varphi \lor \neg \varphi$ is used in a special kind of 
argument, called proof by cases. It has the following 
form: 

\begin{align*}
    &\varphi \Rightarrow \psi \\ 
    &\neg \varphi \Rightarrow \psi \\ 
    \therefore  & \psi
\end{align*}

You probably use this argument in everyday life without noticing. For example,
"if I get a job, I have to move to Buenos Aires (to work there); if I don't get
a job, I have to move to Buenos Aires (to look for one). Therefore, I must move
to Buenos Aires". 

\begin{problem}
    Find hidden assumption.
\end{problem}

\begin{quote}
    If we do not deal with the infestation this month, we'll have to do it next month.
    Doing it next month requires aliens to come to earth in order to slave humankind,
    for only if humankind is slaved by aliens can the infestation be stopped next month.
    Which is indeed sad. Stopping the infestation this month is not easier, for, alas,
    such thing can only be accomplished with the coming of Christ. It seems reasonable to 
    say, then, that stopping the infestation is unlikely.
\end{quote}


\pagebreak 

\subsection{Proof by contradiction (reductio ad absurdum)}

Perhaps the most famous type of argument in logics. In logics, contradictions
(i.e. propositions of the form $\varphi \land \neg \varphi$) are impossible:
all contradictions are false, and anything entailing a contradiction is false.

To prove $\varphi$ by contradiction, you simply disregard $\varphi$ and examine
what happens if you assume $\neg \varphi$ is true (i.e. if you assume that
$\varphi$ is false). If you prove $\neg \varphi$ leads to a contradiction,
you have proved it must be false, and hence $\varphi$ must be true.

For example, a famous proof by contradiction is the proof that $\sqrt{2} $ is
an irrational number. It starts by assuming $\sqrt{2} $ is rational, i.e. it is
expressable as a ratio $\frac{a}{b}$. It then goes on to show that assuming
that this ratio exists leads to a contradictory situation. Hence $\sqrt{2} $
cannot be rational. Hippasus of Metapontum found this proof and was murdered by
the Pithagoreans for doing so, because the Pythagorean dogma was that all
numbers were rational.




















\end{document}



