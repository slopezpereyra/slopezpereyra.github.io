\documentclass[a4paper, 12pt]{article}

\usepackage[utf8]{inputenc}
\usepackage[T1]{fontenc}
\usepackage{textcomp}
\usepackage{amssymb}
\usepackage{newtxtext} \usepackage{newtxmath}
\usepackage{amsmath, amssymb}
\newtheorem{problem}{Problem}
\newtheorem{example}{Example}
\newtheorem{lemma}{Lemma}
\newtheorem{theorem}{Theorem}
\newtheorem{problem}{Problem}
\newtheorem{example}{Example} \newtheorem{definition}{Definition}
\newtheorem{lemma}{Lemma}
\newtheorem{theorem}{Theorem}
\usepackage{ebgaramond}
\usepackage{parskip}

\begin{document}

    
About five years ago, prying on my father's library, I came across a book
entitled *Fez, City of Islam*. The back cover informed that the author, Titus
Burckhard, was the grandnephew of Jacob Burckhardt, the author of *The
civilization of the Renaissance in Italy*. 

My reading of *Fez, City of Islam* was sweet as only very few can be. Its prose
was delicious; its content, of the highest culture. I felt as if, by virtue of
a prodigy, I'd been induced to dream and walk myself the labyrinthine Fez. When
I read it I was not at home, nor in my very country, not even in the
present time, nor even was myself, but lived sequestered in a century and a
place I never knew, and enjoyed in peace the life of a happier—though to me
quite exotic—man.

Whether the book is truly as remarkable as I recall, it's difficult to say. It
is often not its content what makes memorable a work of art, but the alchemy
incurred upon its opposition to our peculiar state of conscience. I content
myself to say that, in memory, the book is exquisite.

I recently acquired perchance another work of the author: *Cosmology and modern
science*. It is a general critique of scientific knowledge as a whole, which
emphasizes the invalidity of scientific models. The book contains an extensive
section combating what then was modern psychology—namely, psychoanalysis—with
particular stress over the work of Jung. The work, however, was a total
disappointment. Only the most general comments were somewhat satisfactory or
true; whenever a particular point was brought about, its justification was foul
and arbitrary, and its claim backwards and medieval.

The foulness of the book is of a common category: *dull fanaticism*—or, as it
is more commonly known, theology. This pointless discilpine has value only when
excercised with a mix of creativity and rigour. That this is possible is proven
by the examples of Scotus Eriugena, Origen of Alexandria, Pascal, Spinoza and
Saint Agustine. That it is very rare is also proven by the lack of more
examples, at least as far as my inquiries have gone. But even then, whatever 
value it has, is literary. (Origen's exegesis are simply exquisite 
to read, as are Augustine's \textit{Confessions} and Pascal's
\textit{Thoughts}.) One may argue that literary value is enough to make these
authors great, and this I do not object. But in what comes to philosophical
matters, at least if philosophy is to have something to do with truth, however
remotely, theology generally offers very little. 

The above exceptions aside, all my readings of theology were \textit{at best}
bearable. Their arguments never followed a straight path, but like a serpent
meandered in search of a conclusion in accordance with the Scriptures. Every
conclusion was beforehand an assumption, and problematic questions were solved,
with a slight of hand, by finding some convoluted sophistry that lead to the
presupposed conclusion. I discovered the most arbitrary ways of thought, the
most peculiar attempts at proof, the most extravagant arguments that human mind
is capable of bearing. Metaphysics is often like this, but theology has the
added component of a dogma, which is distasteful to anyone possessing at least
a spark of sensitivity.

The work *Cosmology and modern science* was no exception to the sins I
described above. For instance, at a certain point, the following chain of
thought is exposed: Firstly, the notion of form, as understood by Greek
philosophy, is defined. By form, the author correctly says, we understand the
aggregation or association of the qualities of a thing or a being—its immutable
essence. This is so that by form we understand a principle of individuation
that, while in itself being nothing but an archetype, is to become manifest in
connection to a matter or substance. (So far, Burckhardt only follows
Dionysious the Aeropagite.) After providing and commenting on such definition,
and arguing against some critiques of the notion, the author states:

> From what was just said follows that a species is itself an immutable "form"; it
> could not evolve and transform into another species, although it may contain
> variations, which are diverse "projections" of a unique essential form of which
> they will never be separated—like the branches are never separated from the
> trunk.

The whole thesis of Darwin, he says, is based on a confusion (*sic*) between the
concepts of species and variation. He soon after adds:

> One may certainly find some fishes that use their fines to crawl in the sand,
> but in vain one would expect to find the least beginning of a joint, which is
> the only thing that would make the forming of a leg or an arm possible.

Yes, indeed: This is a refutation of Darwin based on the metaphysical concept
of form, as understood by an Athenian convert that lived in the first century!

Only in metaphysics one can expect to find such bold attempts at arm-chair
reasoning. None of it, needless to say, even hints at an interest for reality.
First, a purely abstract concept is assumed to capture the true nature of
things; then the desired conclusion is drawn from such assumption. It would be
equally valid to assume the Pythagorean notion that all numbers are rational to
prove that π is a rational number. (In fact, even this wouldn't be as flawed as
Burckhardt's logic, in that the premise *all numbers are rational* is
refutable, while the platonic assumptions of our author are almost a mater of
taste.) 

This, if any, is the method of theology: Take an arbitrary assumption,
preferably based on ancient knowledge or primitive speculation, and use it only
to derive a ridicule conclusion. The enterprise is to be judged depending on
whether that conclusion, however obscene or stupid, agrees with whatever faith
you happen to profess.

Burckhardt was also dissatisfied with Jungian psychoanalysis, which
desacralizes religious experience and reduces religion to psychology. But every
one of his arguments against the Jungian thesis boiled down to the *assumption*
that the thesis is wrong; this is, to the assumption that the Abrahamic god is
the true god, and the consequential proposition that religious experience is a
supernatural phenomenon. At a certain point, Burckhardt claims that if the mind were
an epiphenomenon of matter, it could not be termed to be *intelligent*—why
this is so, nor why does it matter, he fails to say. He also states that,
because the mind is capable of perceiving the essence of things (whatever that
means), then there must be something essential and immutable to it—and that
such eternal faculties of the mind are what Jung confuses with evolutionary
endowments.

Burckhardt was a very sophisticated man in possession of an outstanding
education. If anything, he's an example of the way in
which fanaticism corrupts even potent intellects, rendering them sterile.
Intelligence and education are in general not sufficient to acquire even the
most elementary scientific intuition. Perhaps the greatest example of this is
Aristotle, whose non-natural works are of the deepest intelligence, but whose
natural philosophy and treatment of the Φύσις were flawed in their entirety. It
would have sufficed to him, as well as to most of the so-called natural
philosophers, to have even the most primitive embryo of an experimental method,
to find that most of their claims were wrong. 

These philosophers are excused because they cared to explained the word with
the tools they had at their disposal. Little doubt remains that, if the
scientific method would have existed in his time, they would have certainly
been the first to appreciate its power. Burckhardt, as well as the rest of
modern theologians, have everything at their disposal to inquire about the
world. They simply find it more important to agree with the barbaric doctrines
of some ancient desert-wanderers.







\end{document}



