\documentclass[a4paper, 12pt]{article}

\usepackage[utf8]{inputenc}
\usepackage[T1]{fontenc}
\usepackage{textcomp}
\usepackage{amssymb}
\usepackage{newtxtext} \usepackage{newtxmath}
\usepackage{amsmath, amssymb}
\usepackage{graphicx}
\usepackage{parskip}
\usepackage{hyperref}

\begin{document}

En \textit{Las metamorfosis}, Ovidio presenta con quince breves libros la
inagotable profundidad del \textit{ethos} grecolatino. A través de las
exploraciones derivadas de su lectura, que son en verdad infinitas, descubrí a
Hendrick Goltzius, un artista holandés del manierismo. Un interesante repaso de
la trayectoria de Goltzius puede leerse
\href{https://www.metmuseum.org/press-releases/first-major-retrospective-of-dutch-master-hendrick-goltzius-to-open-at-metropolitan-museum-june-26-2003-exhibitions}{aquí};
por lo pronto deseo compartir sus hermosas representaciones de \textit{Las
metamorfosis}. 

Una de las primeras historias de \textit{Las metamorfosis} es la de Apolo y
Dafne. Tras el diluvio, la tierra se puebla de monstruos y criaturas, uno de los
cuales es la gigantesca serpiente Pitón. Según las \textit{Fábulas} de Higino,
Pitón «acostumbraba a dar las respuestas de parte del oráculo en el monte Parnaso
antes que Apolo», y «su destino era que había de morir a raíz del parto de
Latona», madre de Apolo y Diana. Por esta razón, al descubrir que Latona estaba
embarazada, Pitón la persiguió para matarla, aunque sin éxito. Apolo venga a su
madre asesinando a Pitón con miles de flechas, quedando luego embriagado por su
propia \textit{hubris}. Esta escena es representada en el grabado siguiente, que
lleva la inscripción

\begin{quote}
Immensum certis strauit Pythona Sagittis / Nec meruit minimum Cynthius arte
deus, Latonae matri monstrum Iunonis ob iram / Et terrae infestum dum necat
atq[ue] mari»
\end{quote}

\begin{quote}
El dios Cintio (Apolo) abatió con flechas certeras a la inmensa Pitón, y no
mereció poco crédito por su arte, mientras mataba al monstruo, infesto para la
tierra y el mar, enviado por la ira de Juno contra su madre Latona.
\end{quote}


\begin{figure}[h] % El parámetro [h] intenta colocar la imagen "aquí" (here)
    \centering
    \includegraphics[width=1\textwidth]{../Images/ApoloMataPiton.jpg}
    \caption{Apolo matando a Pitón}
    \label{fig:goltzsiusio}
\end{figure}

Poco después, Apolo encuentra a Cupido y, lleno de arrogancia, se burla de que
un dios tan débil lleve un arco. Para vengarse, Cupido lo enamora de la ninfa
Dafne, clavándole una flecha con punta de oro en el corazón. A Dafne, por otra
parte, le clava una flecha con punta de plomo, garantizando que nunca pueda
corresponder a Apolo, quien la persigue incansablemente herido de amor. Al verse
acorralada en la orilla del Peneo, su padre, Dafne suplica por su ayuda.
Goltzius representa el proceso de metamorfosis subsiguiente: los pies de Dafne se
hunden en la tierra como raíces, su cuerpo es envuelto por corteza y sus manos
se ramifican en hojas de laurel. Al fondo a la derecha, se observa una escena
previa que muestra el inicio de la persecución en el paisaje, con Cupido
arrojando su flecha.


\begin{figure}[h] % El parámetro [h] intenta colocar la imagen "aquí" (here)
    \centering
    \includegraphics[width=0.8\textwidth]{../Images/DafneArbol.jpg}
    \caption{Dafne convirtiéndose en laurel en brazos de Apolo}
    \label{fig:goltzsiusio}
\end{figure}

El texto en el grabado dice:

\begin{quote}
Ardebat flagrans Titan Peneïda Daphnen / Illa thorum vitat
deuia lustra petens. Vitat, et in laurum cita vertitur, at sua semper / Dilecta
Phoebus tempora fronde tegit.
\end{quote}

\begin{quote}
El ardiente Titán (Apolo) estaba inflamado de amor por Dafne,
la hija de Peneo; ella evita el lecho nupcial buscando los bosques remotos. Lo
evita, y es convertida rápidamente en laurel, pero Apolo siempre cubre sus
sienes con el follaje de su amada.
\end{quote}

Otra hermosa representación de Goltzius es la de la violación y metamorfosis de
Ío. Júpiter, atraído a la doncella Ío, envuelve la tierra en una densa niebla
para ocultar su encuentro con ella (derecha del grabado). Su esposa Juno, al ver
el bosque en tinieblas, sospecha lo que está pasando y desciende a inspeccionar,
tras lo cual Júpiter convierte a Ío en vaca (centro del grabado) a fin de
ocultar su aventura. Al ver un novillo tan bello, y aún sospechando algo, Juno
se lo pide a su marido como obsequio, quien se ve forzado a aceptar para no
levantar más sospechas (izquierda del grabado).

\begin{figure}[h] % El parámetro [h] intenta colocar la imagen "aquí" (here)
    \centering
    \includegraphics[width=0.8\textwidth]{../Images/GoltziusIo.jpg}
    \caption{Júpiter rapta a Ío, la convierte en novillo y la regala Juno}
    \label{fig:goltzsiusio}
\end{figure}

El texto en latín dice 

\begin{quote}
Iuppiter Inachiden densa caligine stuprat, / Iuno sui
sensit furta petulca viri. Quod deus aduertens, Io sub imagine vaccae /
Occuluit, quae post de boue facta dea est.
\end{quote}

\begin{quote}
Júpiter viola a la hija de Ínaco bajo una densa niebla; Juno
percibió los engaños petulantes de su marido. Al notar esto el dios, ocultó a Ío
bajo la imagen de una vaca, la cual, después de haber sido vaca, fue convertida
en diosa.
\end{quote}

Otra famosa escena representada por Goltzius es la caída de Faetón, hijo de
Apolo. Incauto y arrogante, Fateón logró que su padre le conceda el carro solar
con el que recorre el universo, llevando luz al mundo, pero rápidamente pierde
el control de los potros flamígeros que lo empujan. El carro descontrolado
recorre el cosmos incendiando las constelaciones celestiales así como secando
los ríos, quemando los bosques, y volviendo negra la piel de los etíopes (sic).
Para detener la destrucción, Júpiter, padre de Apolo y por ende abuelo de
Faetón, lo derriba con un rayo, provocando su caída.

\begin{figure}[h] % El parámetro [h] intenta colocar la imagen "aquí" (here)
    \centering
    \includegraphics[width=0.8\textwidth]{../Images/FaetonCayendo.jpg}
    \caption{Caída de Faetón}
    \label{fig:goltzsiusio}
\end{figure}

El texto en latín dice

\begin{quote}
Sic phaetontevs nimivm temeraria lapsvs vota docet tandem fine carere bono. non
ambire probat sapiens sed lavdat honores, lavdat contingant si tamen illa
probis.
\end{quote}

\begin{quote}
Así, la caída de Faetón enseña que los deseos excesivamente temerarios carecen
al final de un buen desenlace. El sabio no aprueba la ambición, sino que alaba
los honores, siempre que estos recaigan en los hombres íntegros.
\end{quote}

Faetón murió al caer sobre el río Erídano. Clímene y las Helíades, madre y
hermanas de Faetón, respectivamente, lamentaron su muerte durante meses en el
río, hasta que sus lágrimas se convirtieron en ámbar y sus cuerpos en álamos.
Esta escena también es representada por Goltzius con perfecta hermosura.


\begin{figure}[h] % El parámetro [h] intenta colocar la imagen "aquí" (here)
    \centering
    \includegraphics[width=0.8\textwidth]{../Images/FaetonHermanas.jpg}
    \caption{Clímene y las Helíades llorando la muerte de Faetón y
    convirtiéndose en álamos; a la derecha, Cicno convirtiéndose en cisne.}
    \label{fig:goltzsiusio}
\end{figure}

\begin{quote}
Excipit Eridanus Phaetontia tepentibus undis / Fulmineo excussum celitus igne
Iouis. Deplorant tumulum Clymene, Heliadesq[ue] sorores, / Quas Dy popula fronde
repente tegunt.
\end{quote}

\begin{quote}
El Erídano recibe con aguas tibias a Faetón, expulsado por el fuego fulmíneo de
Júpiter desde el cielo. Clímene y las hermanas Helíades lloran ante el túmulo, a
quienes los dioses cubren de repente con follaje de álamo.
\end{quote}

El cisne en el grabado es Cicno, amigo de Faetón, cuyo duelo fue tan hondo que,
al entrar en el Erídano, se convirtió en cisne.

Goltzius también representó el secuestro de Europa y las subsiguientes aventuras
de Cadmo. Europa y Cadmo eran hijos de Agénor, rey de Fenicia. La hermosura de
Europa instigó a Júpiter a convertirse en un hermoso toro blanco de apariencia
mansa, a fin de seducir y raptar a la atractiva joven. Consigue ganarse la
confianza de Europa a tal punto que ella se aferra a sus cuernos y lo monta,
tras lo cual él se adentra en el mar llevándola hacia Creta. 

\begin{figure}[h] % El parámetro [h] intenta colocar la imagen "aquí" (here)
    \centering
    \includegraphics[width=1\textwidth]{../Images/EuropaJupiter.jpg}
    \caption{Júpiter raptando a Europa en forma de toro. Detrás, las compañeras
    de Europa lamentan su pérdida.}
    \label{fig:goltzsiusio}
\end{figure}


\begin{quote}
Europam asportat freta per Neptunia Taurus / Iuppiter, ad comites respicit illa suas. Illa merentes extrema in littoris ora / Cum coeli implorant Numina glauca maris.
\end{quote}

\begin{quote}
Júpiter, convertido en toro, transporta a Europa por los mares de Neptuno; ella
mira hacia sus compañeras. Aquellas, lamentándose en el extremo de la orilla del
litoral, imploran a las deidades del cielo y del mar glauco.
\end{quote}

Agénor envía a Cadmo a rescatar a su hermana. Al no encontrarla, consulta al
oráculo de Apolo, quien le ordena seguir a una vaca que nunca haya llevado yugo
y fundar una ciudad donde ésta se eche a descansar.

\begin{figure}[h] % El parámetro [h] intenta colocar la imagen "aquí" (here)
    \centering
    \includegraphics[width=0.8\textwidth]{../Images/CadmoOraculo.jpg}
    \caption{Cadmo consultando el oráculo de Apolo.}
    \label{fig:goltzsiusio}
\end{figure}

\begin{quote}
Europam toto frustra dum quaerit orbe / Impia Iussa Patris fugiens, Patriamq[ue] Patremq[ue]. Vitat Agenorides, Phoebiq[ue] oracula supplex / Consulit, et quae sit tellus habitanda requirit.
\end{quote}

\begin{quote}
Mientras busca a Europa en vano por todo el mundo, huyendo de las órdenes impías
de su padre, el hijo de Agenor (Cadmo) evita su patria y a su padre; suplica y
consulta el oráculo de Febo (Apolo) y pregunta qué tierra debe habitar.
\end{quote}


Al llegar al lugar indicado por la vaca, los compañeros de Cadmo se aproximan a
una cueva donde pretenden encontrar agua. Allí son devorados por un dragón
sagrado de Marte que custodiaba la fuente. Cadmo enfrenta y mata a la bestia,
y luego, por consejo de Atenea, siembra los dientes del dragón, de donde nacen
guerreros (espartos, o \textit{Spartoí}).


\begin{figure}[h] % El parámetro [h] intenta colocar la imagen "aquí" (here)
    \centering
    \includegraphics[width=0.8\textwidth]{../Images/CadmoDragonMata.jpg}
    \caption{El dragón sagrado de Marte matando a los compañeros de Cadmo.}
    \label{fig:goltzsiusio}
\end{figure}



\begin{figure}[h] % El parámetro [h] intenta colocar la imagen "aquí" (here)
    \centering
    \includegraphics[width=0.8\textwidth]{../Images/CadmoMata.jpg}
    \caption{Cadmo matando al dragón.}
    \label{fig:goltzsiusio}
\end{figure}


\begin{figure}[h] % El parámetro [h] intenta colocar la imagen "aquí" (here)
    \centering
    \includegraphics[width=0.8\textwidth]{../Images/CadmoMataDragonOleo.jpg}
    \caption{Cadmo matando al dragón, pintura.}
    \label{fig:goltzsiusio}
\end{figure}

\begin{quote}
Ultor Agenorides saevum procurrit in
hostem, / In fauces ferrum fumumq[ue] ignemq[ue] vomentes. Ferrum adigit,
lethumq[ue] fuga vitare parantem / Figit in adversam connixus corpore quercum.
\end{quote}

\begin{quote}
El vengador hijo de Agenor corre hacia el feroz enemigo, hacia
las fauces que vomitan humo y fuego. Clava el hierro y, al que intentaba evitar
la muerte con la huida, lo atraviesa contra una encina opuesta apoyándose con
todo su cuerpo.
\end{quote}

\end{document}



