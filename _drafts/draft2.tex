\documentclass[a4paper]{article}

\usepackage[utf8]{inputenc}
\usepackage[T1]{fontenc}
\usepackage{textcomp}
\usepackage{amsmath, amssymb}
\edef\restoreparindent{\parindent=\the\parindent\relax}
\DeclareSymbolFont{letters}{OML}{ztmcm}{m}{it}
\usepackage{parskip}
\restoreparindent
\newtheorem{problem}{Problem}
\newtheorem{example}{Example}
\newtheorem{lemma}{Lemma}
\newtheorem{theorem}{Theorem}
\newtheorem{problem}{Problem}
\newtheorem{example}{Example}
\newtheorem{definition}{Definition}
\newtheorem{lemma}{Lemma}
\newtheorem{theorem}{Theorem}

\begin{document}

In a paper entitled \textit{Empiricism must, but cannot, presuppose real
causation}, Hans Radder makes a case against $a.$ empiricism as a general
philosophical stance, but more particularly $b.$ empiricism as the
epistemological foundation of science. It is focused on the notion of causation,
to which the empiricist tradition denies a real ontology. The argument goes as
follows.

$i.$ Experimentation involves the \textit{material realization} of a process.
The idea underlying the generation of empirical knowledge is that the
realization of an interaction between a certain object ($O$) and an apparatus
($A$) concludes at a final state where correlations between $O$ and $A$ are made
apparent.

$ii.$ When evaluating the information revealed by the final state of $A$, we have to
take into account interactions between the $( O+A )$-system and its environment.
In general, great care is taken into suppressing any influence from external,
environmental factors into the system. This is required in order to make
interpretations of the final state epistemologically sound, and the material
realization of such state reproducible.

$iii.$ The epistemic requirement in $ii.$ shows that the notion of real
causation (in the form of a disturbance) plays an important role in the
production of empirical knowledge.

Note that the argument is made at three different levels. Scientific inquiry can
only succeed if $a.$ we ascribe a real \textit{ontology} to causation, in the
sense that we assume that there may be real external events that causally
disturb the material realization of the final state; $b.$ if we succeed
\textit{epistemologically} at knowing what this events are; and $c.$ if we
realize the \textit{methodological} achievement of suppressing their influence.

However, it is not entirely clear that a real ontology is being ascribed to
causation even by an agent that behaves \textit{as if} causation were real. The
empiricist's claim is that we cannot \textit{ascertain} that causation exists
because \textit{the necessity} of the cause-effect relationship is unobservable
and unprovable. And yet, I am quite certain every single empiricist lives or has
lived as if causation is real ---a disposition to which he is compelled by
entirely practical reasons: namely, that it is impossible to sustain any form of
life otherwise. The case is similar to every behavioral disposition which
assumes the existence of that which is not observed ---from an object in another
room to a country on the opposite pole---: their existence is simply $a.$ more
likely than alternative explanations, according to that vague form of
evidence which every-day life conforms, and $b.$ indispensable to the material
continuation of our lives.

The pragmatist would claim that there is nothing but a linguistic difference
between the propositions: \textit{I behave as if causation were real} and
\textit{I believe causation is real}. But this is not entirely true. Imagine a
traveler that, walking towards, say, Mandawa, finds himself at a crossroads. On
each side, two broken and incomplete signs read: \textit{To Man**wa} and
\textit{To *an**wa}, respectively. There is more evidence to say the first road
takes to Mandawa (one letter more of evidence, or one free variable less, to be
precise), so he would do well in taking it. But if this were all the information
he had, can he be \textit{sure} that that is the correct road? What is his
belief: that the road takes to Mandawa, or that it is more likely that the road
takes to Mandawa? His behavior fits them both, but they are inequivalent.

Going back to our case, then, there is nothing in behaving \textit{as if}
causation were real that contradicts the epistemological apprehension about
ascertaining its reality. Furthermore, for every observable correct prediction
of science, the evidence in favor of it increases ---and with it, so does that
in favor of the consistency of its assumptions. In short, while it is true that
science behaves \textit{as if} causation is real ---which every single one of us
do---, this fits entirely with the inferential basis of empiricism, so long as
the existence of causation is the best explanation to observed phenomena ---even
when the \textit{necessity} of the cause-effect relation is unprovable and
unobservable. In short, empiricism, far from outright denying causation, holds
it as the most plausible \textit{hypothesis}. And in this non of us are any
different from science, insofar as it is impossible to live as if causation were
not real and at the same time impossible to prove that it is.

In short, I do not believe the point sustained by Radder is particularly strong.
I find myself to be more worried about different matters when it comes to
science and causation. For example, how exactly does a given evidence provide
support for a hypothesis? What exactly does it mean to say that causation is
\textit{the most likely} explanation, even when we can not prove reasonably
prove it nor empirically observe necessity? These, and
more like these, are the problems of philosophy with regards to science. But I
cannot dwell upon them here, for this entry pertained to Radder's paper and,
besides, I have no satisfactory answers to give.


\end{document}
