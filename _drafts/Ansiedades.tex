\documentclass[a4paper, 12pt]{article}

\usepackage[utf8]{inputenc}
\usepackage[T1]{fontenc}
\usepackage{textcomp}
\usepackage{amssymb}
\usepackage{newtxtext} \usepackage{newtxmath}
\usepackage{amsmath, amssymb}
\newtheorem{problem}{Problem}
\newtheorem{example}{Example}
\newtheorem{lemma}{Lemma}
\newtheorem{theorem}{Theorem}
\newtheorem{problem}{Problem}
\newtheorem{example}{Example} \newtheorem{definition}{Definition}
\newtheorem{lemma}{Lemma}
\newtheorem{theorem}{Theorem}
\DeclareMathAlphabet{\mathcal}{OMS}{cmsy}{m}{n}

\usepackage{parskip}

\begin{document}

El temor es madre de la crueldad, y en pocas cosas sentimos tantas ansiedades
como en el amor. Nuestra educación sentimental enseña, más o menos
explícitamente, que la respuesta más violenta es usualmente la correcta. Si
tememos que la persona amada se enamore de otra, debemos limitarla; si nos da
miedo su abandono, debemos retenerla; si nos inquieta mostrarnos vulnerables
ante ella, debemos endurecernos. Pienso que todos sentimos, en un nivel
fundamental, un deseo por amar y ser amados, y que en particular el miedo a
perder el amor es la causa principal de muchos de nuestros vicios.

Séneca, en sus \textit{Cartas a Lucilio}, enseña que «ningún bien puede darnos
placer salvo aquél para cuya pérdida estamos preparados», puesto que son iguales
el dolor por la cosa perdida y el temor de perderla. En general, una postura
defensiva atrae el ataque, así como una impenetrable cerradura es más atractiva
para un ladrón que una puerta llanamente abierta. López de Gómara, en su
\textit{Historia general de las Indias}, dice que cierta nación cerca sus
huertos con hilos de algodón, resultando mucho más seguros que los fosos y los
alambres espinosos. Del mismo modo, por regla general, cuanto más libre,
compasiva y tolerante la ética que gobierna nuestro amor, menos admite la
proliferación del sufrimiento. 

Si añadimos a esto que, por regla general, acciones semejantes producen efectos
diferentes, estando nuestras vidas constantemente atadas a los caprichos del
azar y de la suerte,

\begin{quote}
    \textit{Nescia mens hominum fati sortisque futurae} \\
    ~ ~ ~ ~ [La mente de los hombres desconoce el destino y la suerte futura]\\ 
    ---Virgilio, \textit{Eneida}, Libro X
\end{quote}

se vuelve evidente que no debemos pretender jamás control alguno en el amor, y
que todo control que creamos poder tener es ilusorio.

Las personas celosas o controladoras generalmente reconocen que es el miedo lo
que induce sus comportamientos destructivos. Sin embargo, rara vez admiten que
ninguna de las acciones que toman evitan realmente aquello que temen. Las
personas monogámicas suelen engañarse al pensar que el anillo protector de su
contrato evitará que sus parejas se enamoren de otros, cuando, si nos atenemos a
los hechos, ni siquiera parece desalentar que esto suceda. Quienes, en vez de
procurarlo sanamente, demandan afecto o atención, rara vez se percatan de que
solo hacen menos probable su llegada, y no siempre apreciamos que la lealtad
ganada es más firme y duradera que la exigida. 

Cuando no es el miedo lo que nos hace obrar con violencia, suele ser el afán de
conquistar. Del mismo modo que, para conquistar, primero debemos dividir, las personas
usualmente buscan limitar y fragmentar la personalidad y libertad del ser amado
con el fin de tomar posesión de él. Se ama, pero \textit{en la medida que}; se
quiere, pero \textit{siempre y cuando}... 

Toda filosofía, pero en particular una filosofía concerniente a las experiencias
afectivas de los hombres, debe fundarse en la libertad, la compansión y la
tolerancia. Utilizo estos tres términos, que sin duda han sido gastados, en su
sentido \textit{radical}. Una ética superadora debe aceptar completa y
absolutamente la individualidad de las personas amadas, sin imponerles ninguna
restricción. Debe fundarse en el amor a los \textit{individuos}---en el sentido
etimológico, es decir, \textit{no-divididos} o \textit{no-fragmentados}---. No
debe imponer tales o cuales condiciones para el amor, porque esto admite que no
se ama a la persona sino a cierta expresión limitada de su personalidad. Debe
fundar los vínculos en el deseo de promover y estimular el desarrollo íntegro y
completo de la personalidad de la persona amada así como de la propia. En última
instancia, requiere un cariño y compromiso sinceros con la personalidad total,
no fragmentada, de aquellos a quien deseamos entregar nuestro corazón.

Cuando se presenta esta filosofía, las personas suelen reaccionar con
escepticismo o incluso rechazo. La principal razón es que nunca han
experimentado relaciones fundadas en los tres principios nombrados arriba,
imaginándolas como algo caótico, desordenado, o carente de compromiso. Lo mejor
que puede hacerse es no predicar, sino mostrar con el ejemplo que un amor
fundado en esta ética no sólo puede ser comprometido, responsable y duradero,
sino que el respeto por la individualidad y libertad ajenas lo dota de una
dulzura y una ternura superiores. La gente sana y de buen corazón, por regla
general, reconoce el amor y sus virtudes cuando lo tienen enfrente, y por lo
tanto basta con dar el ejemplo para que puedan convencerse.
















\end{document}



