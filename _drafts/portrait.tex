\documentclass[a4paper, 12pt]{article}

\usepackage[utf8]{inputenc}
\usepackage[T1]{fontenc}
\usepackage{textcomp}
\usepackage{amssymb}
\usepackage{newtxtext} \usepackage{newtxmath}
\usepackage{amsmath, amssymb}
\newtheorem{problem}{Problem}
\newtheorem{example}{Example}
\newtheorem{lemma}{Lemma}
\newtheorem{theorem}{Theorem}
\newtheorem{problem}{Problem}
\newtheorem{example}{Example} \newtheorem{definition}{Definition}
\newtheorem{lemma}{Lemma}
\newtheorem{theorem}{Theorem}
\DeclareMathAlphabet{\mathcal}{OMS}{cmsy}{m}{n}

\begin{document}

We find ourselves before a man who thinks himself to be a great man. I should
wish to study this man, to comprehend him, and to acquire a conception of him
that is equally truthful and compassionate. We shall examine him both in terms
of the values \textit{he} holds dear, so as to determine whether he fulfills his
own idea of a good, or even a great man, as well as in terms of the values I
hold dear, so as to see whether I can love him purely. I should confess that,
ultimately, such is my only goal---to love him purely, at best---to accept him
radically, at worst. Whatever path is to be taken must be decided only by
elucidating what is mean by «\textit{him}».

~ 

Our man's infancy is rather obscure and little can be elucidated from it. The
youngest of two brothers, his father passed away when he was four years old.
While his brother grew distant with their mother and showed all the signs of a
troubled youth---abuse of alcohol, aggressiveness and a tendency to react
violently---our man seems to have been a very timid young boy who knit a very
close relationship with his mother. One account describes the older brother as
being severely violent against our man during his infancy, causing his shyness
to exacerbate, but for obvious reason this cannot be verified. 

~

It is safe to say he idealized his mother, who was a refined but melancholic
woman. Several accounts paint this woman as someone who instigated conflict
between the two brothers, telling each the shortcomings of the other, and this
divisive attitude of sowing discord seems to have lasted throughout her entire
life. One report claims that, at around forty years old, our man finally learnt
of all the terrible things his mother would say behind his back, shattering his
illusion of a wonderful relationship and causing him tremendous distress and
instability. Our man has confirmed on several occasion that a conflict with his
mother emerged during this time, but was dismissive of its magnitude and
suggested it was a minor quarrel. 

~ 

Our man describes both of his parents excellently, though of course he knows
very little of his father. I was able to uncover a diary of his mother from the
mid sixties, when our man of interest was less than four years old. In this
diary, his mother paints a profoundly sordid account of her husband, who seems
to have been an alcoholic, and declares her profound distaste for him and for
her situation. She speaks of a «shadow» or a darker side of the husband which
only she knows exists, but which she must carry to her grave «for the sake of
the children».

~

The accounts of our man's adolescence are varied. He moved several times due to
his mother financial instability. He had two friends with whom he seemed to have
cultivated a healthy relationship. One woman describes him as being so «terribly
cruel» in high-school that she feared him, and it is not unthinkable---given
what we shall learn of his future life---that he might have enjoyed humiliating
others. He seemed to have a very hard time making new friends, a trait which
would accompany him throughout his entire life.

~

It is safe to say our man is a man of intelligence. He masters the French
language, reading Chateaubriand or Montaigne directly and teaching at
prestigious French institutions. He has a decent domain of the Italian, English,
Russian and (less extensively) Guaraní languages. He developed a successful
career both privately as a lawyer, achieving a good financial position, as well
as a remarkable career in academia, specifically legal anthropology. Our man is
sophisticated, with a subtle and refined understanding of the widest array of
cultural domains--- the Renaissance to the Nuremberg trials, Homer and the
history of Argentina, \textit{The Tale of Genji} and \textit{Anna Karenina}---.
On these and infinitely many other topics he could verse spontaneously as one
who speaks of the weather.

~ 

The grimmer side of this refinement, however, is revealed in the fact that his
relationship with culture---and intellectual affairs in general---is profoundly
mediated by self-esteem regulation. A piece of knowledge, to our man, is a
silver coin, a sort of token to be greedily collected. He reads only partially
to understand the world, since the world is ultimately secondary to him, and he
is mostly guided by the almost masturbatory pleasure derived from the
superiority which knowing makes him feel. To eyes deprived of vanity, a book on
Greek culture verses on the Greeks, a treatise of epistemology, on epistemology.
To him, however, these are all about himself. The world and its understanding
are accidental—the central pursuit is the onanistic ecstasy derived from feeling
superior to the average man. He is pejorative of everyone that serves him no
purpose or grants him no admiration. He never has anything good or positive to
say about anyone else. When confronted with anyone, intelligent or not, he will
attempt to establish his superiority with narcissistic vanity, and to
delegitimize whatever education or knowledge, or even uneducated insight, the
other person may have. For this reason, it is impossible to for him to form new,
genuine relations,  and he lives a very lonely life. He can’t stand anyone else,
where everyone else is ignorant and vile, and nobody can stand him, where he is
narcissistic and vain.

~ 

Perhaps our man has also said, not without a hint of vanity, that he believed
in Kantian ethics---perhaps, believing himself honest, he elaborated on how so
many of his sorrows followed from his unyielding commitment to the categorical
imperative. This is very curious, for our man is deemed by almost everyone as capable
of violence, as someone of virulent hate and purulent rancor, and as absolutely
incapable of self-criticism. In general, he lacks all prosocial sentiments,
loving himself above all others, feigning but never granting forgiveness. 

~ 

Whether or not our man is capable of love remains an open question. To many an
astute observer, it has seemed that people are to him nothing but a mirror on
which to enjoy his own reflection. He has only been observed to sustain
long-standing relations with those who are subordinate to him or those who
admire him and feed his ego. He has a clear pattern of dating younger women who
are usually their employees or students. When a person involved in this form of
relationship, for one or another reason, disentangles itself, and either wishes
to break free from its subordination or manages to break the spell and reveal a
sincere, non-idealized vision of him, he reacts with bitter anger and fury. He
never lets it on transparently, but he is controlling and exerts his
influence---preferring manipulation to direct violence--- so that people do what
he wishes. 

~ 

Let us take for instance his second long-standing relationship, and one of the
very few he has had. Very little is known by me of his partner before the time
they met. It is safe to say, however, that in entering this relationship she
sacrificed all sense of self and mimetized completely with our man of interest.
She has few or no friends anymore, and all of her time and attention is directed
to looking after our man devotedly. The caring is carried out in highly pompous
displays of concern and alleviation by both parties. It is all his perfect man,
his idol, a man who---in her own words, according to one account---«never did
anything wrong». Some see in this an expression of extreme or intense love. More
likely it is to say that this individual has no true knowledge of our man---that
she lives with a certain man, sleeps with a certain man, but that man is not
\textit{our} man. It is an abstract ideal which exists in both of their
minds---a phantom fostered by the vanity of one and the vulnerability of the
other---a great man, indeed---but not our man. In fact, and in this all accounts
agree, there is only \textit{one} person who loves our man---this is, one who
knows him exactly for what he is and still fosters love for him---but that is
too sad a case to recollect... In general, he is despised for what he is or
adored for what he is not.

~ 

His first marriage, which lasted around twenty years, was not in essence too
different. His first wife was a woman who grew under the claws of a profoundly
narcissistic mother and without a father. Parentification was the hallmark of
her youth, having been placed in the role of the understanding and non-opposing
daughter---in stark contrast with a volatile, conflictive, probably bipolar, but
ultimately kind and loving sister. In short, she was deprived of one of the most
precious experiences in life, which is the rebellion against our parent figures.
As in most of these cases, this rebellion came much later, and with much less
potency and effect. 

~ 


~

Our man is a vengeful man. he rationalizes his vengeful missions, always finding
the way to paint them as carrying out justice. He claimed for some time to want
to write a short treatise or essay on vengeance and its virtues, but to my
knowledge he never truly carried this project out. He was obsessed, and
passionately so, with evening the score—particularly, the score of a divorce of
which he felt the sole and only victim.

~

It is quite probable that, in our man's experience, the outer world does not
exist. I mean to say that our man is a corroboration of Schopenhauer’s
doctrine—in a tragic and perhaps pathetic sense. The primordial forces in his
life are will and emotion. The outer world does not stand to correct his
judgment on any matter—He fabricates this judgment based on the Dionysiac forces
of his soul—Reason exists only and solely to justify this fabrication. Gently
put, one could say he is all a poet and nothing of a science man—Realistically
put, he is blind to the fact that an obscure, and at times inextricable passion
drives the generation of his world, writes the legend and mythology of his
life—a force which he protects from all external examination, be it fact or
opinion—and weaves the story of a tragedy, much like the Chrstian passion, that
renders him a blameless and penitent soul before the evils of this world. A
passage from Jung’s Aion struck me as such an accurate description of this that
I actually see fit to reproduce it:

\begin{quote}
It is often tragic to see how blatantly a man bungles his own life and the lives
of others yet remains totally incapable of seeing how much the whole tragedy
originates in himself, and how he continually feeds it and keeps it going. Not
consciously, of course—for consciously he is engaged in bewailing and cursing a
faithless world that recedes further and further into the distance. Rather, it
is an unconscious factor which spins the illusions that veil his world. And what
is being spun is a cocoon, which in the end will completely envelope him.
\end{quote}







\end{document}



