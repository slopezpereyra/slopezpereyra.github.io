\documentclass[a4paper, 12pt]{article}

\usepackage[utf8]{inputenc}
\usepackage[T1]{fontenc}
\usepackage{textcomp}
\usepackage{amssymb}
\usepackage{newtxtext} \usepackage{newtxmath}
\usepackage{amsmath, amssymb}
\newtheorem{problem}{Problem}
\newtheorem{example}{Example}
\newtheorem{lemma}{Lemma}
\newtheorem{theorem}{Theorem}
\newtheorem{problem}{Problem}
\newtheorem{example}{Example} \newtheorem{definition}{Definition}
\newtheorem{lemma}{Lemma}
\newtheorem{theorem}{Theorem}
\usepackage{parskip}

\begin{document}

Una de las cosas más tristes de la condición humana es que, cuando tratamos de
lidiar con nuestros defectos, es muy difícil no incurrir en otros peores. Por
un lado está la culpa, que es degradante e inútil; por otro la
autocomplacencia, o simplemente la pereza que nos da tratar de
perfeccionarnos. 

Antes que nada, debo aclarar que cualquier cosa que diga no parte de una falsa
sensación de superioridad o sabiduría. En verdad, conozco pocas personas tan
llenas de rasgos infelices como yo. Por cada virtud encuentro al menos una
falencia, y en mi memoria atesoro ciertas cosas que me enorgullecen y otras
tantas que me avergüenzan. Una de mis memorias más felices es un día en que
salvé cientos de peces en una sequía, pero en mi adolescencia, por simple
brutalidad, asesiné a una hermosa lechuza. Con muchas dificultades, salvé a un
cachorro de morir de frío; a otro, que vi siendo maltratado por un bruto, lo
olvidé. A mis amigos les he ofrecido un hombro para llorar; en cierta ocasión,
lleno de ira, también quise ofrendarles mi castigo y mi violencia. Soy, en
suma, un hombre común y corriente: la impresión de la huella divina y el pecado
original vibran en mí con igual potencia.

Mi axioma fundamental es que debemos procurar la proliferación de la alegría.
Que debamos tratar de volvernos mejores es sólo un corolario. Pienso que el
hombre debe ser, en un buen sentido, intransigente consigo mismo, y no
despreocuparse de procurar la virtud. El concepto de virtud es anticuado, y al
pronunciarlo no quiero hacerme pasar por un griego. La virtud es algo
verdadero, y todos conocemos no sólo personas virtuosas, sino el delicioso amor
que nos inspiran. Sólo un necio no ha suspirado alguna vez pensando en la
compasión, la capacidad de perdonar, o la rectitud de alguna persona que la
vida puso en su camino. De \textit{esa} virtud estoy hablando, no de un ideal
abstracto.

Platón cometió el error de asociar la educación con la virtud y el vicio con la
ignorancia. Esto es, sin duda, un ideal aristocrático, y mi experiencia lo
desprueba. La educación puede mejorar a una buena persona tanto como emeporar a
una mala, y por sí sola nos dice muy poco acerca de su calidad humana. Una de
las personas más virtuosas que he conocido, cuya vida pareciera transcurrir en
incesante dicha y amor, y no por falta de las más serias dificultades, es una
hermosa mujer que aprendió a leer en su edad adulta, llevó por mucho tiempo una
vida campesina, y apenas recibió educación formal. Por otra parte, conozco
personas ruines con una educación excelente, y me asombra cómo sus lecturas
sólo las vuelve más sutiles.

En cuanto a mí, el primer obstáculo que encuentro al tratar de examinarme, o simplemente al
lidiar conmigo mismo cada día---que no es fácil---, es la culpa que me generan
mis propios vicios y defectos. Me desagrado fácilmente: mi rutina no es lo
suficientemente buena, no leo tanto como quisiera, y no estudio tantas
matemáticas como me gustaría. Incluso cuando mis días son excelentes, y son
dedicados por completo al estudio, me pesa la sensación---que mi historial
familiar justifica---de que no viviré demasiado y, por lo tanto, de que mi
tiempo no es suficiente. 

Si estudiamos el asunto con algo de frialdad, es evidente que la culpa que nos
generan nuestros vicios es inconducente. No contribuyen ni a expiar un error
cometido ni a reparar un daño, y si examinamos el estado general de las cosas,
\textit{ceteris paribus}, un mundo en que una persona está sintiendo culpa es
peor que uno en que no la está sintiendo. En la medida de lo posible, debemos
olvidar todo lo que hagamos mal, a no ser que sea para aprender a actuar mejor.

> El olvido es la única venganza y el único perdón

Esto no es fácil si una persona es sensible, porque de todos los sentimientos
oscuros, la culpa es el único que se nos inculca falsamente como una virtud.
Miles de jóvenes todos los días son forzados a repetir, mientras golpean
rítmicamente sus pechos:

> Por mi culpa, por mi culpa,
> por mi gran culpa.

Pero la culpa es un sentimiento verdaderamente terrible. César Vallejo, en
\textit{Los heraldos negros}, describe al hombre azotado por todas las
tragedias como uno cuyos ojos traicionan que su vida se enpoza en un
\textit{charco de culpa}. Todos conocemos la historia de Judas, que lloró:
\textit{I have sinned in that I have betrayed the innocent blood}, y se ahorcó.
Dante, poco después de entrar en el cuarto círculo, exclama: "\textit{¿Por qué
dejamos que nuestra culpa nos consuma de este modo?}". Y Edipo, al descubrir su
propio crimen, se quitó la vista. Cualquier educación sentimental que se precie
debería prevenirnos de este mal. No es un fin en sí mismo ni un medio para un
fin feliz; ¿qué argumento podría convencerme de que aumenta la felicidad de
nadie?

Algunas personas, incluso por cosas menores, como el despertarse tarde o el
beber unos tragos de más, se sumen tan hondamente en su culpa que son incapaces
de continuar su vida con normalidad. Los azotan sentimientos de vergüenza y
autodesprecio. Se ocupan tan ardua y esmeradamente del castigo de sí mismos,
que nos les queda tiempo de nada más; y así, al final del día, a la desdicha
original le agregan también la de sentir que el tiempo se les fue de las manos
sin haberlo logrado que rinda frutos.

Si bien somos educados para sentir vergüenza por estas desviaciones, y hasta
cierto punto ella es indicativa de una sensibilidad sana, pienso que podemos
despojarnos de ese sentimiento si lo examinamos racionalmente. Es falso que
seamos del todo racionales, y en general no controlamos lo que sentimos, pero
es verdadera la máxima estoica según la cual no nos afectan las cosas, sino
nuestras ideas de las cosas. Si las personas se entregaran a un examen genuino
de la culpa, creo que verían fácilmente cuán inconducente es, y pienso que esto
sí reduce nuestra propensión a sentirla. Mi propio caso es evidencia, puesto
que siempre fui de sentirme culpable incluso por cosas menores, y logré
liberarme hasta cierto punto de ese defecto a través de su análsis.

En el otro extremo, suele asombrarme la facilidad con que algunas personas se eximen
de la responsabilidad de lidiar con sus vicios. En algunos casos extremos incluso los
trastocan en virtudes. Un hombre que hablaba pestes a sus hijos de la madre que
les dio la vida se justificó diciendo que la causa de su error era su amplia
inteligencia. Según dijo, las personas superdotadas tienden a tratar a los
menores como si fueran adultos. ¿Cuántos hombres vengativos, con la vanidad
herida, cobraron la culpa de sus enemigos en nombre de la justicia? ¿Cuántos 
antisociales y soberbios justifican su desprecio de la gente con una supuesta
superioridad moral, intelectual, estética o social? Leen a Antonio Machado, que 
nos dice de los hombres felices:

> donde hay vino, beben vino;
> donde no hay vino, agua fresca

y asienten solemnemente, pero son incapaces de disfrutar de la fraternidad
humana. La culpa, que es la contracara de esta facilidad innata para la
autoabsolución, al menos revela un atisbo de sensibilidad para con nuestros
errores.

Pienso que estamos obligados a por lo menos intentar disminuir nuestros
defectos y nuestros vicios. Desde el momento en que algo de nosotros empeora el
estado de cosas, para nosotros mismos o alguien más, resolverlo se transforma
en un imperativo ético. Parece que estoy diciendo algo trivial, y en verdad a
mí también me lo parece, pero cumplir con ese imperativo es extremadamente
difícil. Nuestros defectos se arraigan con más facilidad que nuestras virtudes,
y la tentación de hacer la vista gorda con ellos es muy grande.




















\end{document}



