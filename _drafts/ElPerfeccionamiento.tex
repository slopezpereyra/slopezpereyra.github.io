\documentclass[a4paper, 12pt]{article}

\usepackage[utf8]{inputenc}
\usepackage[T1]{fontenc}
\usepackage{textcomp}
\usepackage{amssymb}
\usepackage{amsmath, amssymb}
\newtheorem{problem}{Problem}
\newtheorem{example}{Example}
\newtheorem{lemma}{Lemma}
\newtheorem{theorem}{Theorem}
\newtheorem{problem}{Problem}
\newtheorem{example}{Example} \newtheorem{definition}{Definition}
\newtheorem{lemma}{Lemma}
\newtheorem{theorem}{Theorem}
\usepackage{parskip}
\usepackage{ebgaramond}

\begin{document}

Una cosa muy triste de la condición humana es que, cuando tratamos de lidiar
con nuestros defectos, es muy difícil no incurrir en otros peores. Por un lado
está la culpa, que es degradante e inútil; por otro la autocomplacencia, o
simplemente la pereza que nos da tratar de perfeccionarnos. 

Sin embargo, pienso que debemos ser intransigentes con nosotros mismos, y no
despreocuparnos de procurar la virtud. La palabra *virtud* es anticuada, y al
decirla no quiero hacerme pasar por un griego. La virtud es algo verdadero, y
todos conocemos no sólo personas virtuosas, sino el delicioso amor que nos
inspiran. Sólo un necio no ha suspirado alguna vez pensando en la compasión, la
capacidad de perdonar, o la rectitud de alguna persona que la vida puso en su
camino. De *esa* virtud estoy hablando, y no de un ideal abstracto.

Platón cometió el error de asociar la educación con la virtud y el vicio con la
ignorancia. Esto es, sin duda, un ideal aristocrático, y mi experiencia lo
desprueba. Una de las personas más virtuosas que he conocido, cuya vida
pareciera transcurrir en incesante piedad, es una hermosa mujer que aprendió a
leer en su edad adulta, llevó por mucho tiempo una vida campesina, y apenas
recibió educación formal. Por otra parte, conozco personas ruines con una
educación excelente, cuyas lecturas y saberes sólo las vuelven más sutiles.

Por mi parte, no hablo desde una falsa sensación de superioridad o sabiduría.
Conozco pocas personas tan llenas de rasgos infelices como yo. Por cada virtud
encuentro al menos una falencia, y en mi memoria atesoro ciertas cosas que me
enorgullecen y otras tantas que me avergüenzan. Una de mis memorias más felices
es un día en que salvé cientos de peces en una sequía, pero en mi adolescencia,
por simple brutalidad, asesiné a una hermosa lechuza. Con muchas dificultades,
salvé a un cachorro de morir de frío; a otro, que vi siendo maltratado por un
bruto, lo olvidé. A mis amigos les he ofrecido un hombro para llorar; en cierta
ocasión, lleno de ira, también quise ofrendarles mi castigo y mi violencia.
Soy, en suma, un hombre común y corriente: la impresión de la huella divina y
el pecado original vibran en mí con igual potencia. 

El primer obstáculo que encuentro al tratar de examinarme, o simplemente al
lidiar conmigo mismo cada día—que no es fácil—, es la culpa que me generan
mis propios vicios y defectos. Me desagrado fácilmente: mi rutina no es lo
suficientemente buena, no leo tanto como quisiera, y no estudio tantas
matemáticas como me gustaría. Incluso cuando mis días son excelentes, y son
dedicados por completo al estudio y al trabajo, me pesa la sensación—que mi
historial familiar justifica—de que no viviré demasiado y, por lo tanto, de
que mi tiempo no es suficiente. 

Si estudiamos el asunto con algo de frialdad, es evidente que la culpa que
nos generan nuestros vicios es inconducente. No contribuyen ni a expiar un
error cometido ni a reparar un daño, y si examinamos el estado general de las
cosas, *ceteris paribus*, un mundo en que una persona está sintiendo
culpa es peor que uno en que no la está sintiendo. En la medida de lo posible,
debemos olvidar todo lo que hagamos mal, a no ser que sea para aprender a
actuar mejor.

> El olvido es la única venganza y el único perdón

Esto no es fácil si una persona es sensible, porque de todos los sentimientos
oscuros, la culpa es el único que se nos inculca falsamente como una virtud.
Miles de jóvenes todos los días son forzados a repetir, mientras golpean
rítmicamente sus pechos:

> Por mi culpa, por mi culpa,
> por mi gran culpa.

Pero la culpa es un sentimiento verdaderamente terrible. César Vallejo, en
*Los heraldos negros*, describe al hombre azotado por todas las
tragedias como uno cuyos ojos traicionan que su vida se enpoza en un
*charco de culpa*. Todos conocemos la historia de Judas, que lloró:
*I have sinned in that I have betrayed the innocent blood*, y se ahorcó.
Dante, poco después de entrar en el cuarto círculo, exclama: "*¿Por qué
dejamos que nuestra culpa nos consuma de este modo?*". Y Edipo, al descubrir su
propio crimen, se quitó la vista. Cualquier educación sentimental que se precie
debería prevenirnos de este mal, que no es un fin en sí mismo ni un medio para un
fin feliz.

Algunas personas, incluso por cosas menores, como el despertarse tarde o el
beber unos tragos de más, se sumen tan hondamente en su culpa que son incapaces
de continuar su vida con normalidad. Los azotan sentimientos de vergüenza y
autodesprecio. Se ocupan tan ardua y esmeradamente del castigo de sí mismos,
que nos les queda tiempo de nada más; y así, al final del día, a la desdicha
original le agregan también la de sentir que el tiempo se les fue de las manos
sin haber gozado de su vendimia.

Si bien somos educados para sentir vergüenza por nuestras desviaciones, y hasta
cierto punto esto es indicativo de cierta sensibilidad, pienso que podemos
despojarnos de ese sentimiento si lo examinamos racionalmente. 
Aunque en general no controlamos lo que sentimos, es verdadera la máxima
estoica según la cual no nos afectan las cosas, sino nuestras ideas de las
cosas. Si las personas se entregaran a un examen genuino de la culpa, creo que
verían fácilmente cuán inconducente es, y pienso que esto sí reduciría su
propensión a sentirla. Mi propio caso es evidencia, puesto que siempre tendí a
a la culpa incluso por cosas menores, y logré liberarme hasta cierto punto de
ese defecto a través de su análsis.

Por otro lado, algunas personas se eximen demasiado fácilmente de la
responsabilidad de lidiar con sus vicios. En algunos casos incluso los
trastocan en virtudes. ¿Cuántos hombres vengativos, con la vanidad herida,
cobraron la culpa de sus enemigos en nombre de la justicia? ¿Cuántos
antisociales y soberbios justifican su desprecio de la gente con una supuesta
superioridad moral o intelectual? Leen a Antonio Machado, que nos dice de los
hombres felices:

> donde hay vino, beben vino;<br>
> donde no hay vino, agua fresca

y asienten solemnemente, pero son incapaces de disfrutar de la fraternidad
humana. La culpa, que es la contracara de esta facilidad innata para la
autoabsolución, al menos revela un atisbo de sensibilidad para con nuestros
errores; y de todos los malos modos en que podemos lidiar con nuestros
defectos, no hacerlo en absoluto es el peor. 

Existe gente que sería capaz de corregirse si pudiera verse con un poco más de
claridad. Cualquiera que haya convivido con alguien puede ver cuánto las
personas chocan una y otra vez con la misma piedra. Cuando sufren por algo,
simplemente sufren: no se detienen a examinar de qué manera contribuyen a su
propio sufrimiento. Por regla general, nos es más fácil ver estas flaquezas en
otros, pero pienso que todos somos al menos un poco así. Necesitamos una cierta 
dosis de autorevisionismo e introspección; la evolución es imposible de otro modo.

Una total falta del hábito introspectivo es el medio más seguro para la
constancia. Quien no se examina no se cuestiona; quien no se cuestiona no
cambia, y quien no cambia no mejora. No es suficiente carecer de la vanidad que
nos impide reconocer nuestros defectos: primero hay que contar con la capacidad
de verlos. Por si esto no fuera poco, una vez que los vemos hay decidir
correctamente qué hacer con ellos: diagnóstico y remedio son cosas diferentes,
y el mundo está repleto de aprendices de brujo que, queriendo volverse
dragones, se convirtieron en sapos.

Aunque la recomiendo, la introspección es un arte peligroso que fácilmente se
transforma en solipsismo y estupidez. No es un fin en sí mismo, y que debería
perpetrarse con el propósito de alcanzar una resolución; es decir, con el fin
de cambiar un hábito o dirigir nuestras acciones. En cuanto una decisión es
alcanzada, debería abandonarse: nuestro tiempo está mejor gastado lidiando con
las hermosuras que nos ofrece el mundo, que no son pocas, antes que en la
exploración de nuestras propias almas.

En suma, encuentro que son tres los caminos que pueden ayudarnos a volvernos
mejores: la introspección orientada a formar reglas para la acción, la
aniquilación de toda culpa, y la falta de vanidad. Si logramos vernos
francamente, con todas nuestras desgracias, y aún así evitar la culpa y la
inacción, podemos darnos por satisfecho. No hay nadie tan malo que no pueda
volverse al menos un poco mejor.
















\end{document}



