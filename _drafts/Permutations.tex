\documentclass[a4paper, 12pt]{article}

\usepackage[utf8]{inputenc}
\usepackage[T1]{fontenc}
\usepackage{textcomp}
\usepackage{amssymb}
\usepackage{newtxtext} \usepackage{newtxmath}
\usepackage{amsmath, amssymb}
\newtheorem{problem}{Problem}
\newtheorem{example}{Example}
\newtheorem{lemma}{Lemma}
\newtheorem{theorem}{Theorem}
\newtheorem{problem}{Problem}
\newtheorem{example}{Example} \newtheorem{definition}{Definition}
\newtheorem{lemma}{Lemma}
\newtheorem{theorem}{Theorem}
\usepackage{parskip}

\begin{document}

Hill climbing is a mathematical optimization algorithm. Given a function $f :
\mathbb{R}^n \to \mathbb{R}$, the algorithm takes an arbitrary
$\overrightarrow{x} \in \mathcal{D}_f$ and compares $f(\overrightarrow{x}),
f(\Delta \overrightarrow{x})$. If the sfhit increases the
function, it lets $\overrightarrow{x} := \Delta\overrightarrow{x}$; otherwise 
it lets $\overrightarrow{x} := -\Delta\overrightarrow{x}$. The process 
is repeated until some criteria is met.

Hill climbing can be used to optimize the greedy coloring of a graph. The use
of this algorithm for greedy coloring is justified by a nice theorem.

Given a graph $G = (V, E)$, the parameter of the Greedy coloring algorithm
$\mathcal{G}$ is an order $\mathcal{O} = v_{i_1}, \ldots, v_{i_n}$ of the
vertices, where the $i_j$ define the coloring order of $V = \left\{ v_1,
\ldots, v_n \right\} $. Thus, shiftes in the parameter space correspond to
changes in the coloring order. 

Such permutations lack a nice property of real functions; namely, that their
parameters can only be shifted in two directions: positive or negative. Instead
of these two choices, an order $\mathcal{O}$ of $n$ vertices has $n!$
permutations. We cannot try them out and chose the one that maximizes our
function.

However, there is a property which points to permutations that may not
necessarily improve the coloring, but are guaranteed not to worsen it. Such
property reduces drastically the space of candidate permutations. In fact,
given an initial ordering $\mathcal{O}$, this space of alternative permutations
is often so small that none of them improve the coloring. Thus, it is common to
generate $k$ random initial orderings $\mathcal{O}_1, \ldots, \mathcal{O}_k$,
so as to explore their (reduced) permutation spaces, in hope that one of their
permutations will in fact improve the coloring. 

\begin{quote}
    \textbf{Theorem.} Let $G = (V, E)$ a graph with a proper coloring of $r$
    colors $C = \left\{ c_1, \ldots, c_r \right\} $. Let $V_{c_i} := \left\{ x\in V
    : c(x) =c_i \right\} $. Let $P : C \to C $ a permutation of $c_1, \ldots, c_r$;
    i.e. $P(c_i) = c_j$ entails the $i$th color becomes the $j$th color.
    
    $\mathcal{G}$ with the order $V_{P(c_1)}, \ldots, V_{P(c_r)}$ colors 
    $G$ with at most $r$ colors.
\end{quote}

The proof of this theorem is simple to do inductively over $r$. If $r = 1$ the
case is trivial. Assume the theorem holds for $r = k$. Let $x_0 \in
\bigcup_{i=0}^{k+1} V_{P(c_{i})}$. If $x_0$ belongs to any of the first $k$
sets, $\mathcal{G}$ colors it with one out of $k$ colors by hypothesis. If $x_0
\in V_{P(c_{i+1})}$, the case where $\mathcal{G}$ colors it with a color less
than $k+2$ makes our statement true. Assume this is not the case. Then there is
some $y_0 \in \Gamma(x_0)$ such that $c(y_0) = k + 1$. But then $y_0 \in
V_{P(c_{i+1})}$. And since $x_0$ is also in this set, we should have $c(x_0) =
c(y_0)$. Then the coloring is not proper. $(\bot)$

Informally, if $\mathcal{G}$ colored $V$ into groups $V_{c_1}, \ldots,
V_{c_r}$, then coloring first the vertices with color $P(c_1), P(c_2), \ldots,
P(c_r)$ is at least as good as the original coloring. The permutation $P$ that
we must use is arbitrary: the theorem states nothing special about it. In
general, one uses permutations that put vertices with highest colors or highest
degrees first, since these are the problematic ones. We will define a few such
permutations.

\textbf{Permutation 0.} The first permutation will order the vertices from last
to first color. Thus, assuming $\mathcal{G}$ used $r$ colors, the order will be
$V_{c_r}, \ldots, V_{c_1}$. 

A pseudo-code implementation:

\begin{align*}
    &\mathcal{O} = \textbf{int } Array[n]\\
    &\mathcal{D} := \textbf{Queue } Array[r] \\
    &\textbf{for } 1 \leq i \leq n \textbf{ do} \\ 
    &\qquad push!(i, \mathcal{D}[c(v_i)])\\
    &\textbf{end}\\
    &\textbf{int } i := 0\\
    &\textbf{for } queue \textbf{ in } \mathcal{D} \textbf{ do}\\
    &\qquad\textbf{while } queue \neq \emptyset \textbf{ do}\\
    &\qquad \qquad O[i] = pop(queue) \\ 
    &\qquad \qquad i = i + 1\\
    &\qquad\textbf{end}\\
    &\textbf{end}\\
    &\textbf{return } \mathcal{O}
\end{align*}

The first for loop is $O(n)$. Since $\sum_{i}^r |\mathcal{D}_i| = n$, the
second for loop with its inner while is $O(n)$. $\therefore $ The algorithm is
$O(n)$.

\textbf{Permutation 1.} The second permutation we will write is a simple
cardinality permutation. The color with the most number of vertices goes first,
the color with the least number of vertices goes last. There are at most $O(n)$
sets $V_{c_i}$. Using QSort we have $O(n \log n)$ complexity.


\textbf{Permutation 2.} We will order the colors using their divisibility. First will come 
all colors divisible by four, then all colors divisible by 2, and then all other 
colors. In pseudo-code, 

\begin{align*}
    &x = | \left\{ c \in C : c \equiv 0 \mod 4 \right\}  |\\
    &y = | \left\{ c \in C : c \equiv 0 \mod 2 \land c \not\equiv 0 \mod 4 \right\}  |\\
    &\mathcal{O} = \textbf{int } Array[n] \\ 
    &\textbf{int }u = v = w = 0 \\
    &\textbf{int } index\\
    &\textbf{for } i \in \left\{ 1, \ldots, r \right\}  \textbf{ do }\\ 
    &\qquad\textbf{while } \mathcal{D}[i] \neq \emptyset \textbf{ do}\\
    &\qquad\qquad v = pop(\mathcal{D}[i]) \\ 
    &\qquad\qquad \textbf{if } i \equiv 0 \mod 4 \textbf{ do} \\ 
    &\qquad \qquad \qquad index = u \\ 
    &\qquad\qquad\qquad u = u + 1 \\ 
    &\qquad\qquad\textbf{else if } i \equiv 0 \mod 2 \textbf{ do } \\ 
    &\qquad\qquad\qquad index = x + v\\ 
    &\qquad \qquad \qquad v = v + 1 \\ 
    &\qquad\qquad\textbf{else do}  \\ 
    &\qquad\qquad\qquad index = x + y + w\\ 
    &\qquad \qquad \qquad w = w + 1 \\ 
    &\qquad\qquad\textbf{fi}\\
    &\qquad\qquad \mathcal{O}[index] = v \\ 
    &\qquad\textbf{od}\\
    &\textbf{od}
\end{align*}

which is $O(n)$ of course.

I generated a random graph of $n = 30$ vertices and $m = 32$ edges. I tested
the permutations and found that the last permutation coloured the graph with
$1$ less color than the other ones. To the left, the Greedy coloring over the 
natural order of the vertices; to the right, the coloring over the order
of \textbf{Permutation 2}.













\end{document}



