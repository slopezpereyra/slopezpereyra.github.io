\documentclass[a4paper, 12pt]{article}

\usepackage[utf8]{inputenc}
\usepackage[T1]{fontenc}
\usepackage{textcomp}
\usepackage{amssymb}
\usepackage{newtxtext} \usepackage{newtxmath}
\usepackage{amsmath, amssymb}
\newtheorem{problem}{Problem}
\newtheorem{example}{Example}
\newtheorem{lemma}{Lemma}
\newtheorem{theorem}{Theorem}
\newtheorem{problem}{Problem}
\newtheorem{example}{Example} \newtheorem{definition}{Definition}
\newtheorem{lemma}{Lemma}
\newtheorem{theorem}{Theorem}
\usepackage{parskip}

\begin{document}

Hill climbing is a mathematical optimization algorithm. Given a function $f :
\mathbb{R}^n \to \mathbb{R}$, the algorithm takes an arbitrary
$\overrightarrow{x} \in \mathcal{D}_f$ and compares $f(\overrightarrow{x}),
f(\Delta \overrightarrow{x})$. If shifting $\overrightarrow{x}$ increases the
function, it lets $\overrightarrow{x} := \Delta\overrightarrow{x}$ and repeats
the process, and thus iteratively until some criteria is met.

Hill climbing can in fact be used to optimize the greedy coloring of a graph.
The use of this algorithm for greedy coloring is justified by a nice theorem.

Given a graph $G = (V, E)$, the parameter of the Greedy algorithm $\mathcal{G}$
is an order $\mathcal{O} = v_{i_1}, \ldots, v_{i_n}$ of the vertices, where the
$i_j$ define the coloring order of $V = \left\{ v_1, \ldots, v_n \right\} $.
Thus, shiftes in the parameter space correspond to changes
in the coloring order. 

Such permutations lack a nice property of real functions; namely, that their
parameters can only be shifted in two directions: positive or negative. Instead
of these two choices, an order $\mathcal{O}$ of $n$ vertices has $n!$
permutations. We cannot try them out and chose the one that maximizes our
function.

There is a property, however, which reveals permutations that may not
necessarily improve the coloring, but are guaranteed not to worsen it. Such
property reduces drastically the space of alternative permutations. In fact,
given an initial ordering $\mathcal{O}$, this space of alternative permutations
is often so small that none of them improve the coloring. Thus, often $k$
initial orderings $\mathcal{O}_1, \ldots, \mathcal{O}_k$ are randomly defined,
and their (reduced) permutation spaces are explored, in hope that one of their
permutations will in fact improve the coloring. 

The property I'm refering to is the following.

> \textbf{Theorem.} Let $G = (V, E)$ a graph with a proper coloring of $r$
colors $C = \left\{ c_1, \ldots, c_r \right\} $. Let $V_{c_i} := \left\{ x\in V
: c(x) =c_i \right\} $. Let $P : C \to C $ a permutation of $c_1, \ldots, c_r$;
i.e. $P(c_i) = c_j$ entails the $i$th color becomes the $j$th color.

Then $\mathcal{G}$ with the order $V_{P(c_1)}, \ldots, V_{P(c_r)}$ colors 
$G$ with at most $r$ colors.


The proof of this theorem is simple to do inductively over $r$. If $r = 1$ the
case is trivial. Assume the theorem holds for $r = k$. Let $x_0 \in
\bigcup_{i=0}^{k+1} V_{P(c_{i})}$. If $x_0$ belongs to any of the first $k$
sets, $\mathcal{G}$ colors it with one out of $k$ colors by hypothesis.
If $x_0 \in V_{P(c_{i+1})}$, the case where $\mathcal{G}$
colors it with a color less than $k+2$ makes our statement true.
Assume this is not the case. Then there is some $y_0 \in \Gamma(x_0)$
such that $c(y_0) = k + 1$. But then $y_0 \in V_{P(c_{i+1})}$.
And since $x_0$ is also in this set, we should have $c(x_0) = c(y_0)$.
Then the coloring is not proper. $(\bot)$

Informally, if $\mathcal{G}$ colored $V$ into groups $V_{c_1}, \ldots,
V_{c_r}$, then coloring first the vertices with color $P(c_1), P(c_2), \ldots,
P(c_r)$ is at least as good as the original coloring. The permutation $P$ that
we must use is arbitrary: the theorem states nothing special about it. In
general, one uses permutations that put vertices with highest colors or highest
degrees first, since these are the problematic ones. We will define two such
permutations.

\textbf{Permutation 1.} Let 

\begin{align*}
    m(x) &:= \min \left\{ d(y) : y \in V_{c_x} \right\} \\
    M(x) &:= \max \left\{ d(y) : y \in V_{c_x} \right\} \\
\end{align*}

where $d(x)$ is the degree of a vertex $x \in V$.

If $\mathcal{G}$ colored $G$ with colors $1, \ldots, r$, we shall produce the
following permutation: First, all colors divisible by $4$, ordered decreasingly
y $M(x)$; then all even colors not divisible by $4$, ordered decreasingly by
$M(x) + m(x)$; finally, all remaining colors ordered decreasingly by $m(x)$.

This permutation combines high-degree and high-color to account for how "difficult"
the vertices in a given color were.

\textbf{Permutation 2.} Order the vertices like so:

\begin{align*}
    x_1, x_2, \ldots, x_{r-1} = 2, x_{r} = 1
\end{align*}

where $x_1$ is the color with least cardinality, excluding $1$ and $2$; $x_2$
is the one with least cardinality excluding $1, 2, x_1$, and so on. The
assumption here is that colors with just a few vertices are the most
problematic.

--- 

We shall provide code implementations for these permutations and test them with
$\mathcal{G}$ over a good number of graphs. Both permutations can be
implemented with $O(n)$ complexity. We shall run $\mathcal{G}$ over a fixed
number of permutations, so the total complexity of the testing algorithm will
be $O(m) + O(n)$.

\end{document}



