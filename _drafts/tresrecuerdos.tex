\documentclass[a4paper, 12pt]{article}

\usepackage[utf8]{inputenc}
\usepackage[T1]{fontenc}
\usepackage{textcomp}
\usepackage{newtxtext} 
\usepackage{newtxmath}
\usepackage{amsmath, amssymb}
\usepackage{parskip}

% Definición de entornos (Eliminadas las duplicaciones)
\newtheorem{problem}{Problem}
\newtheorem{example}{Example}
\newtheorem{lemma}{Lemma}
\newtheorem{theorem}{Theorem}
\newtheorem{definition}{Definition}

% Corrección en la declaración del alfabeto matemático
\DeclareMathAlphabet{\mathcal}{OMS}{cmsy}{m}{n}

\title{Tres recuerdos de Natasha \\[1ex] \large (y otras entradas de diario)}
\date{}

\begin{document}

\maketitle

\section*{Recuerdo primero}

Cuando todavía era un niño enamoradizo, me entregué —como a tantas otras cosas
que la gente gustaba de llamar «pérdidas de tiempo»— a la lectura de Matsuo
Bashō y Kobayashi Issa. Parece mentira, pero aquella tarde había recordado el
dulce haiku que promete la incipiencia del amor:

\vspace{1em}

{\small
\begin{quote}
Under the cherry blossoms\\
strangers are not\\
really strangers.
\end{quote}
}

\vspace{1em}

Aunque el calor era insoportable y el día era clarísimo, quienes conocíamos el
inclemente clima de esta región del mundo sabíamos que se avecinaba una tormenta
tropical. Estábamos reunidos en un antiguo patio donde la gente de la ciudad se
amontona a escuchar música bajo la amorosa sombra de los mangos y los urunday,
meciendo nuestros sentimientos al son de unas canciones con un dejo folklórico y
feliz, como se mece a un niño febril que quiere llevarse a las serenas regiones
del sueño.

Lentamente los tintes carmesíes del arrebol fueron invadidos por una oscuridad
de plata. Los cuerpos, los cuerpos de las gentes y las cosas, aparecían ante mí
como fantasmas nativos de un mundo mágico y perfecto. Una profunda soledad
invadió mi corazón, tal vez porque la música se había vuelto melancólica, tal
vez porque en mis puños cerrados todavía me aferraba a un puñado de ceniza y la
sentía escaparse de mis manos lentamente, como el polvo fino de un reloj de
arena pasa de una parte a otra. «Así me escondo en este mundo»—pensé—«con esta
arena, con este recuerdo apretado entre mis manos».

Salí a sentarme solo en un banco de madera, cerca de un árbol de naranjos, con
el alma oprimida por la añoranza de una mujer que ha muerto, del deseo de verla
bailar en la hojarasca con aquellas personas que no la conocieron, del ansia de
volver a sentir sus manos y su voz llamándome, pero esta vez sin lágrimas ni
pena. ¿Qué puede doler más que el pensamiento desesperado de querer hundir los
labios en la tierra para llamar un nombre que, día a día, es recordado por menos
y menos personas? Lentamente, como un yaguareté que acecha en los juncales, la
tormenta iba anunciándose en la voz del viento.

Sentí aproximarse, más que un cuerpo, una mirada—una que algunos días después,
en otro patio nocturno, llamaría una «mirada de cinco siglos»—. No sé cómo pero
supo recordarme que había otro mundo dentro, un mundo de personas vivas que
bailaban vivamente, besándose los labios o tocándose las manos, amándose en
secreto aunque fuera por sólo unos instantes, bajo el influjo de una música a la
vez citadina y de interior. Aquellos ojos negros eran ojos vivos y, en cierto
modo, me llamaron al mundo de los vivos. Y entonces abandoné mi pensamiento,
abandoné a mi amada muerta, y mirando a la muchacha que ya se levantaba de mi
lado pensé:

\begin{quote}
Bajo el árbol de naranjos\\
los extraños no son\\
verdaderamente extraños
\end{quote}

Poco después la tormenta desató su furia, como una bendición, sobre nosotros.

\pagebreak

\section*{Recuerdo segundo}

Bajo las tenues y pequeñas luces que colgaban sobre el aire nocturno del patio,
su rostro dejaba traslucir los vestigios guaraníticos, africanos y europeos que
conjugaban en él una hermosura sin nombre. El piso de ladrillo era cercado por
estrechas franjas de tierra fértil, alfombrada de perladas piedrecitas blancas,
desde las cuales decenas de plantas tropicales conjuraban una especie secreta de
encanto similar al de sus ojos negros. En efecto, había en aquellos ojos todo lo
que en América creemos —o fingimos creer— haber enterrado, otorgándole lo que
alguien describió correctamente como una «mirada de cinco siglos». Sentada en
silencio, descalza y con un cigarrillo en la mano, la salvaje hojarasca de un
jazmín brasilero abierta a poca distancia de su oscuro cabello, como infinitos
dedos que sueñan dar sonido a un instrumento delicioso y primitivo, ella también
se preguntaba quiénes fueron los hombres, los infinitos hombres, que la habían
llevado allí. Ella no lo sospechaba, pero ella era —como todas las cosas— apenas
algo más que un eco. Y yo la contemplaba con el silencio respetuoso de los que
invaden un cementerio, disimulando mi distante asombro, aunque todo en mí
sintiera que una reverberancia de voces milenarias iba aflorar, en cualquier
momento, desde sus labios.

—¿Qué hace?—me pregunté, temiendo que lograra desnudar mi corazón, y una voz
interior me dijo «está esperando». Sus pies descalzos se unieron el uno al otro
como dos trenzas juveniles; sus dedos se entrelazaron y recorrieron su cuerpo
como enredaderas silvestres; el patio se deshizo y dio lugar a una hierba
milenaria, inundando mis sentidos cruel o tiernamente. Nada quedó a mi
alrededor. Solo y confundido, miré la hermosa flor que había nacido ante mí, que
ya no me increpaba con cinco siglos de mirada. Y repetí en silencio: «está
esperando».

\pagebreak


\section*{Recuerdo último}

Aunque lentamente, las nubes de la tormenta ya empezaban a disiparse en el cielo
gris que plateaba las aguas del Paraná. Durante los primeros minutos de
conversación, como es usual en estas ocasiones, mis palabras zozobraban un poco
y me sentía levemente vulnerable. Con toda certeza, ella sabía la verdad de mi
corazón, y lo tenía ante sí desnudo y sin tinieblas que escondieran sus
secretos. Sentí que la fuerza del río, como un lazo espiritual, nos unía
misteriosamente. Es posible que fuéramos peces en un acuario íntimo y secreto, o
juncos que arriman sus cabezas o enlazan sus raíces en el lecho del río, o las
dos torres de un frágil castillo de arena. Sus ojos me parecían otra vez estar
pletóricos de tiempo, llenos de siglos, eras y estaciones. Su rostro pérsico,
perfilado sobre las aguas argentadas del río, era alumbrado por la poca luz
pálida que rompía las densas nubes de la tormenta menguante.

Hablamos de los dos milenios de Roma, de cierta anécdota divertida que recuerdo
de mi estadía en Cuba, de mis viajes a la India y sus viajes a Europa, de Twin
Peaks, de la trama de una novela de Conrad, y del minimalismo de las letras en
las canciones que había escrito en su álbum. Una pinta de cerveza acompañó estas
digresiones, y un observador incauto tal vez se apresuraría a decir que mi
incipiente estado de ebriedad era la explicación de mi adquirido
desenvolvimiento, de mi relajación, de que mis palabras no zozobraran más sino
expresaran, con el distendido auxilio de mis manos, la alegría que se gestaba en
mi interior. Pero esto sería equivocado. Toda mi paulatina calma era abrigada
por una sola causa: la creciente sospecha de que no estaba solo, de que ella
sentía, si no lo mismo que yo, algo parecido; sospecha que era apenas algo más
que una esperanza, pero que se encendía más y más gracias a al nosequé que ardía
en su mirada.

En este tipo de circunstancias, suele suceder que el mundo exterior parece
emular nuestro fuero interno. Tal vez por eso, a medida que la angustia que se
había formado en mí, pensándome perdido, se trastocaba en esperanza, las nubes
densas se disipaban más y más, llenando el horizonte de rayos carmesíes que
penetraban la superficie del río como dagas celestiales. Yo estaba de espaldas a
la puesta del sol, de manera que contemplaba el arrebol del horizonte
derramándose no en el agua, sino en la piel oscura de su exótico rostro.
Comprendí, al ver sus ojos teñidos de un púrpura exquisito, que el horizonte
gris a mis espaldas se había convertido ya en un mundo de celajes ardientes. El
arrebol intenso de mi tierra se diluía sobre aquellos ojos antiguos, como tinta
escarlata echada en el agua de un estanque japonés, causando en mí una suerte de
asombro primitivo.

Muy pronto el agua se oscureció y la crueldad de la noche se alzó sobre nosotros
como un vengativo hechizo guaraní. Solo entonces comprendimos, con claridad
total, lo que ocurría. Después de besarla sentí en mis labios un sabor extraño y
desconocido. Era el sabor que sentiría un hombre si
besara la cara oscura de la luna. Estuvimos quietos un momento, un instante en
que sólo el agua continuó su curso milenario. Parecíamos estatuas inmóviles en
esta región del mundo llena de huellas ocultas. Me pregunté si aquel beso
también dejaría una huella en las arenas, o una marca singular en el río, como
una moneda de plata que, arrojada en él, brillara secreta y milenariamente en su
profundidad. Pero no abrigué esperanza: era posible que mañana todo se
desvaneciera, que la moneda de plata se hundiera en el lecho fangoso, o fuera
tragada por el pez más viejo del río.

Fuimos —ahora lo comprendo— lo que tantos otros han sido: fantasmas aparecidos e
idos en el curso de unas horas, de modo tal que nadie, ni siquiera nosotros, ha
de saber con certeza sin en verdad estuvimos allí, o si no seguimos allí ahora
mismo, como dos espectros, en el atardecer eterno de un sol que nunca acaba de
ponerse. La moneda de plata, recién arrojada al río, todavía se hunde y no ha
tocado el fondo lúgubre. Es imposible decir si brillará como una estrella
subfluvial, o sucumbirá en un fondo lleno de peces primitivos y recuerdos que,
siniestramente, yacen sepultados.

\pagebreak 

\section*{Con el diario del lunes}

Cuando pienso, habiéndolo comprendido todo, en las perfectas y precisas
coincidencias que hicieron posible las cosas que he escrito, las cosas que
sentí, este verano, no puedo evitar pensar que el destino se está cobrando una
deuda, o que un hechizo vengativo entretejió los acontecimientos como una bruja
teje su vestido negro. Jamás, en cuatro años, alguien que no sea mi pareja me
quitó el sueño. Por lo general, soy yo quien es querido sentimentalmente,
incluso amado, y es el otro quien es sólo deseado o apenas digno de interés.

Repasemos los acontecimientos con la frialdad que tengo que forzarme a adquirir
ahora que la suerte ya está echada. Primera anomalía: por primera vez en años,
conozco alguien que me gusta más allá de un simple deseo pasajero. Me coso los
labios porque tiene pareja, nadie sabe lo que siento, y el sentimiento no es lo
suficientemente fuerte como para causarme dolor. A lo sumo, siento una pasiva
resignación que no llega siquiera a ser amarga.

Segunda anomalía: justo en este verano, justo cuando vivo este sentimiento, esta
persona abre su relación. (Se empieza a ver la mano del diablo.) Así y todo, no
reservo ninguna esperanza. No gusta de mí. Acepto estoicamente esta realidad y
ni siquiera trabajo por cambiarla. 

Tercera anomalía: actúo un tanto impulsivamente, mostrándole lo que escribí en
mi diario ---en este mismo diario---, y para mi sorpresa es receptiva. (Yo
esperaba una inmediata parada de carro.) La dinámica comunicacional cambia.
Percibo interés, pero sigue siendo ambigüo. 

Finalmente vamos a una cita, la cita que describo en el último recuerdo, y todo
sale «bien». Las comillas son enormes. Embriagado de pensar que, aunque solo
unos días atrás me creía perdido, ahora parecía interesarse en mí, me quemo como
un campeón, hablo de más. Esta es la anomalía final: yo, que pongo muros a
diestra y siniestra, y no dejo que nadie se me acerque---que cuido mis espaldas
como si me persiguiera el diablo---hablo apresuradamente, me exhibo, no me
escondo. En resumen, \textit{I don't play it cool}, le hago saber que me gusta.
(Primera cita, \textit{what the fuck man}?) Y soy tan estúpido que ni
siquiera me percaté del error que cometía. Porque, sin que yo me diera cuenta,
ella me estaba mirando como yo miro a las pobres pibas que, muy equivocadamente,
se enganchan conmigo, o incluso se enamoran---con esa misma cruel
condescendencia que se mezcla un poco con el rechazo, la saturación y el cringe. 

Cuando pienso en esas muchachas, a veces me digo a mí mismo: «pobre piba, qué
duro gustar de alguien que no gusta de vos». ¿No es esta la soberbia horrible
que estoy pagando ahora? ¿Por qué las beso igual si sé que, por más claro que yo
sea, eso enciende su ilusión? ¿Y por qué me besó ella viéndome tan perdido? El
pobre idiota---yo---vuelve a su casa y escribe acerca de una moneda de plata que
se hunde en el río. Así anda por este mundo el pobre boludo, romantizando.
Se merece el bullying que le estoy haciendo.

Ahora, la misma desilusión apagó mis sentimientos. Aunque sea un poco. La quiero
y deseo ser su amigo. La ansiedad de temer que el vínculo ---en cualquier forma
que pudiera tomar--- se perdiera ya no existe. Ahora me queda una ensalada de
emociones y pensamientos intrusivos: la bronca de que justo la única vez en
media década que alguien me gusta sea también la única vez que no gustan de mí,
la ironía con que yo mismo observo mi propia estupidez, la vergüenza de haberme
quemado como un gil cualquiera ---me mandaron al lobby de toque---, la alegría
de igual haberla podido besar, la recontra alegría de que la amistad sigue
intacta, la melancolía de que la voy a extrañar ---a ella y a todos mis
amigos---, la inmensa confusión acerca de qué significó esto ---¿por qué me
gustó \textit{ella}, de todas las personas con que me relaciono, si somos tan
distintos?---y un inmenso etc. Estoy tan enredado que ni siquiera puedo escribir
bien ---esta entrada de diario parece la catarsis de una minita adolescente
después de mirar \textit{Crepúsculo} y pensar en «su Edward»---. Comparo lo que
escribía hace una semana ---esmerado, prolijo, literario--- con este mamarracho
y digo «wow, así de fuerte me pegó».

Realmente, es para reírse. La conclusión a la que siempre llego en esta vida es
que soy un boludo. Y en serio que parece que las cosas hubieran estado armadas
para enseñarme una lección. Para bajarme del metafórico pony en el que vivo
montado. \textit{Well, mission accomplished!}



\pagebreak

\section*{Epílogo: Dos poemas a Natasha}

\vspace{5mm}

\textbf{I}

Three riverine cats were witness, nothing more,\\
of our obscure affair. The river still\\
repeats the words we whispered on the shore,\\
without a doubt against the river's will.\\
I hear the wicked waters ask: —What for?\\
Why did those feet offend the coast? To kill?\\
To perjure and corrupt? Or to restore\\
something we took and offered to the nil...\\
Only the waters know that day I learned\\
that afterglows could wane in human eyes\\
like purple ink that's bleeding in a pond.\\
Only the waters now remain concerned,\\
while we try to forget or to disguise.\\
Only the water speaks—we won't respond.

\textbf{II}

Desde un oriente yermo la tarde se derrama\\
sobre la anciana orilla de un agua que delira\\
con ser la viva sangre de un Cristo que nos llama\\
con algo de tristeza y con algo de mentira...!

El agua enrojecida, nocturna, te proclama.\\
(Yo quise, vida, verte tal como Dios te mira:\\
llena de luz y sombra, como una diurna trama\\
tejida con los hilos de una luna que expira.)

Recuerdo la asombrosa navaja de tu aliento,\\
que destejió las hebras de mi carne florida\\
cuando sentí en tu boca la muerte de una estrella.

El río, antigua lágrima de Cristo, con el viento\\
se lleva esas memorias, y en ellas nuestra vida, \\
fundiéndose en la sombra distante de tu huella.





\end{document}




