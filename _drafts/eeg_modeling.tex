\documentclass[a4paper, 12pt]{article}

\usepackage[utf8]{inputenc}
\usepackage[T1]{fontenc}
\usepackage{textcomp}
\usepackage{amssymb}
\usepackage{newtxtext} \usepackage{newtxmath}
\usepackage{amsmath, amssymb}
\newtheorem{problem}{Problem}
\newtheorem{example}{Example}
\newtheorem{lemma}{Lemma}
\newtheorem{theorem}{Theorem}
\newtheorem{problem}{Problem}
\newtheorem{example}{Example} \newtheorem{definition}{Definition}
\newtheorem{lemma}{Lemma}
\newtheorem{theorem}{Theorem}


\begin{document}

    
\begin{definition}
    An EEG is a $3$-uple $E = (S, f_s, \Omega)$ where $S \in \mathbb{N}$ is the
    number of samples in each signal, $f_s \in \mathbb{N}$ is the sampling rate
    of the signals, and $\Omegam \in S \times k$ is the signal matrix.
\end{definition}

In general, EEG signals are interpreted by epochs of length $L_e \in \mathbb{N}$
seconds, which are sometimes in its turn divided in $m$ subepochs.
Of course,

\begin{align*}
    S = L_e f_s N_e
\end{align*}

where $N_e$ is the number of epochs in the EEG. From which follows that 

\begin{align*}
    N_e = \frac{S}{L_e f_s}
\end{align*}

Since an epoch refers to a certain portion of the EEG in the temporal
dimension, it is useful to model an epoch $e$ as follows.

\begin{definition}
    Let $E = (S, f_s, \Omega)$ an EEG. An epoch $e : \mathbb{N} \to \mathbb{N}^2$ is defined as 

    \begin{align*}
        e(n) =  \Big( (n-1) L_e f_s + 1, ~ n L_e f_s \Big) 
    \end{align*}
\end{definition}

This definition is such that if $e(n) = (l, u)$ then every row $\Omega_{i *}$
with $l \leq i \leq u$ corresponds to a record within the $n$th epoch, or
rather that the $n$th epoch consists of the signals $\Omega_{l*},
\Omega_{(l+1)*}, \ldots, \Omega_{u*} $.

We may overload this convention and let $e(n, m) = \left( \left(n-1 \right)L_e
f_s, mL_e f_s  \right) $ and take $e(n)$ to be simply the case $e(n, n)$. Thus,
$e(n, m)$ is the $2$-uple with the lower and upper bounds of all rows in
$\Omega$ corresponding to epochs $n, n+1, \ldots, m$.




\end{document}



