\documentclass[a4paper, 12pt]{article}

\usepackage[utf8]{inputenc}
\usepackage[T1]{fontenc}
\usepackage{textcomp}
\usepackage{amssymb}
\usepackage{newtxmath}
\usepackage{amsmath, amssymb}
\newtheorem{problem}{Problem}
\newtheorem{example}{Example}
\newtheorem{lemma}{Lemma}
\newtheorem{theorem}{Theorem}
\newtheorem{problem}{Problem}
\newtheorem{example}{Example} \newtheorem{definition}{Definition}
\newtheorem{lemma}{Lemma}
\newtheorem{theorem}{Theorem}

\begin{document}

En el sistema de Pitágoras, entre los diez pares de opuestos que constituían el
universo, estaba lo uno y lo múltiple. Más aún, todo lo singular era expresable
como una relación de proporción entre otras dos cosas singulares; es decir, la
multiplicidad era determinada por la proporcionalidad de las unidades. En este
sentido, unidad significaba \textit{número primo}. (Este dogma estaba tan
establecido que, según dice la leyenda, los pitagóricos asesinaron a Hípaso de
Metaponto por haber revelado que $\sqrt{2}$ no era la proporción
entre ningún otro par de números.)

Trazar una analogía entre la legislación, por un lado, y la eniantoméresis
\textit{uno-múltiple}, por otro, es más difícil. Pero aquí van algunos
intentos.

Una relación de equivalencia es una relación que satisface tres condiciones:
todo elemento está en esa relación consigo mismo; si un elemento está en
relación con otro, este último está en la misma relación con el primero; y, por
último, si un elemento está en relación con otro, y este otro con un tercero,
el primero y el tercero están en la misma relación. Por ejemplo, si asumimos
que toda persona es pariente de sí misma, la relación de parentezco es de
equivalencia: quienquiera que sean $x, y$ y $z$, se cumple que: $x$ es pariente
de $x$, si $x$ es pariente de $y$ entonces $y$ es pariente de $x$, y si $x$ es
pariente de $y$ e $y$ es pariente de $z$, $x$ es pariente de $z$. 

La relación de conciudadano, en el conjunto de habitantes de una nación, es una
relación de equivalencia. En matemáticas, lesgilar es un concepto funcional:
las únicas leyes que se imponen sobre un conjunto son las funciones cuyo
dominio es ese conjunto.





\end{document}



