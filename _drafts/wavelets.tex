% ===================================================================
% A Primer on Wavelet Analysis: From Intuition to Application
% This document is a self-contained LaTeX file.
% ===================================================================

% --- 1. PREAMBLE ---

% --- Principle 1: Document Class ---
% We use the 'article' class as this is a standard academic-style document.
\documentclass[11pt, a4paper]{article}

% --- Principle 2: Universal Preamble Block ---
% This block sets up page geometry, fontspec (for Noto fonts), and babel (for language).

% Geometry: Standard margins for an A4 page.
\usepackage[a4paper, top=2.5cm, bottom=2.5cm, left=2cm, right=2cm]{geometry}

% Fontspec: The modern engine for font selection.
\usepackage{fontspec}

% Babel: Manages language-specific typesetting.
% We set 'english' as the main language.
\usepackage[english, bidi=basic, provide=*]{babel}

% Babelprovide: Import language settings.
\babelprovide[import, onchar=ids fonts]{english}

% Babelfont: Set the document fonts to the Noto family.
% We use Noto Sans as the main (rm) font for a clean, accessible look.
\babelfont{rm}{Noto Sans}
% We also set the sans-serif (sf) and mono (tt) fonts.
\babelfont{sf}{Noto Sans}
\babelfont{tt}{Noto Sans Mono}

% --- Principle 3: Smart Package Loading ---
% We load only the packages necessary for the document's content.

% amsmath: For all mathematical equations and environments.
\usepackage{amsmath}

% booktabs: For professional-quality tables.
\usepackage{booktabs}

% listings: To format and display code blocks.
\usepackage{listings}

% xcolor: Provides colors, needed for 'listings' syntax highlighting.
\usepackage[dvipsnames]{xcolor}

% caption: For customizing captions for figures and tables.
\usepackage{caption}
\captionsetup{font=small, labelfont=bf}

% --- Configuration for 'listings' (Python Code) ---
\definecolor{codegreen}{rgb}{0,0.6,0}
\definecolor{codegray}{rgb}{0.5,0.5,0.5}
\definecolor{codeblue}{rgb}{0.0,0.0,0.6}
\definecolor{codebg}{rgb}{0.98,0.98,0.98}

\lstdefinestyle{pythonstyle}{
    backgroundcolor=\color{codebg},   
    commentstyle=\color{codegreen},
    keywordstyle=\color{blue},
    stringstyle=\color{codeblue},
    basicstyle=\ttfamily\footnotesize,
    breakatwhitespace=false,         
    breaklines=true,                 
    captionpos=b,                    
    keepspaces=true,                 
    numbers=left,                    
    numbersep=5pt,                  
    showspaces=false,                
    showstringspaces=false,
    showtabs=false,                  
    tabsize=2,
    language=Python
}
\lstset{style=pythonstyle}
% --- End of listings configuration ---

% hyperref: For links and document references.
% This MUST be the last package loaded.
\usepackage[colorlinks, linkcolor=blue, urlcolor=blue, citecolor=blue]{hyperref}

% --- 2. DOCUMENT METADATA ---
\title{A Primer on Wavelet Analysis: \\ From Intuition to Application}
\author{A Guided Introduction}
\date{\today}

% --- 3. DOCUMENT BODY ---
\begin{document}

\maketitle

\begin{abstract}
This document provides a detailed, mathematically-grounded, and accessible introduction to Wavelet Analysis. We explore the limitations of the traditional Fourier Transform and motivate the development of wavelets for time-frequency analysis. We cover the theory and intuition behind the Continuous Wavelet Transform (CWT) and the Discrete Wavelet Transform (DWT), supplemented by figures and practical Python code examples.
\end{abstract}

\tableofcontents
\newpage

% --- SECTION 1: THE PROBLEM WITH FOURIER ---
\section{The Problem with Fourier: The "Why"}

\subsection{The Fourier Transform: A "What," Not a "When"}
For decades, the cornerstone of signal analysis has been the Fourier Transform. Its core idea is to decompose a signal $f(t)$ from the time domain into its constituent frequencies in the frequency domain $F(\omega)$. For a signal $f(t)$, its Fourier Transform $\hat{f}(\omega)$ is given by:

$$
\hat{f}(\omega) = \int_{-\infty}^{\infty} f(t) e^{-i\omega t} dt
$$

This integral essentially computes the "inner product" or "similarity" of the signal $f(t)$ with a complex exponential $e^{-i\omega t}$ (a sine/cosine wave) at \textit{every} frequency $\omega$. The result, $\hat{f}(\omega)$, is the spectrum of the signal, showing which frequencies are present and their relative amplitudes.

\subsection{The Localization Limitation}
The Fourier Transform is exceptionally powerful, but it has a fundamental flaw: it loses all time-based information. The basis functions, $e^{-i\omega t}$, are sine and cosine waves that extend infinitely in time. When we calculate $\hat{f}(\omega)$, we are asking, "How much of this frequency $\omega$ exists in the signal \textit{over its entire duration}?"

\textbf{Intuition:} Imagine a song that plays a C note, followed by a G note.
\begin{itemize}
    \item \textbf{What you hear:} C, then G.
    \item \textbf{What Fourier sees:} A spectrum showing both C and G, but with no indication of which note was played first.
\end{itemize}

This is the \textbf{time-frequency localization problem}. The Fourier Transform gives perfect frequency resolution but zero time resolution. At the other extreme, the original signal $f(t)$ gives perfect time resolution but zero frequency resolution.

% --- Directive A: Placeholder Figure ---
\begin{figure}[htbp]
  \centering
  \framebox{\parbox{0.9\textwidth}{\centering
    \vspace{4cm}
    \textbf{Figure 5: Scaleogram of the Chirp Signal} \\
    A 2D plot with Time on the x-axis and Frequency on the y-axis.
    A bright, diagonal ridge is visible, starting at a low frequency
    and moving to a high frequency as time increases.
    \vspace{4cm}
  }}
  \caption{The CWT result for the chirp signal. The diagonal ridge perfectly localizes the signal's frequency content in time.}
  \label{fig:scaleogram}
\end{figure}

The scaleogram (Fig. \ref{fig:scaleogram}) clearly shows the frequency sweeping upwards, demonstrating the CWT's power in analyzing non-stationary signals.

% --- SECTION 4: CWT IN PRACTICE (CODE) ---
\section{CWT in Practice: Python Example}

We can perform CWT in Python using the \texttt{PyWavelets} library. This code generates the chirp signal (Fig. \ref{fig:chirp}) and its scaleogram (Fig. \ref{fig:scaleogram}).

\begin{lstlisting}[language=Python, caption=Python code for CWT of a chirp signal., label=list:cwt]
import numpy as np
import matplotlib.pyplot as plt
import pywt

# 1. Generate the Chirp Signal
sampling_period = 0.01
t = np.arange(0, 10, sampling_period)
# f(t) = sin(2 * pi * f(t) * t), where f(t) = t
f_start = 1
f_end = 10
f_t = np.linspace(f_start, f_end, len(t))
signal = np.sin(2 * np.pi * f_t * t)

# 2. Perform Continuous Wavelet Transform (CWT)
# We use the 'cmor' (Complex Morlet) wavelet
# We need to define the scales to check.
# Scales correspond to pseudo-frequencies
total_scales = 128
# f = pywt.scale2freq('cmor1.5-1.0', scale) / sampling_period
# We can find scales corresponding to our frequencies
scales = np.logspace(np.log10(f_end/f_end), np.log10(f_end/f_start), total_scales)
scales = scales * pywt.scale2freq('cmor1.5-1.0', 1) / f_end

coefficients, frequencies = pywt.cwt(signal, scales, 
                                     'cmor1.5-1.0', 
                                     sampling_period=sampling_period)
power = (np.abs(coefficients)) ** 2

# 3. Visualization
# (This code would generate Fig 4 and 5)

# Plot the signal (Fig 4)
plt.figure(figsize=(10, 6))
plt.subplot(2, 1, 1)
plt.plot(t, signal)
plt.title("Chirp Signal (Frequency 1Hz to 10Hz)")
plt.xlabel("Time (s)")
plt.grid(True)

# Plot the Scaleogram (Fig 5)
plt.subplot(2, 1, 2)
# Frequencies are returned by cwt
plt.imshow(power, extent=[t.min(), t.max(), frequencies.min(), frequencies.max()],
           cmap='jet', aspect='auto', interpolation='bilinear')
plt.title("Wavelet Scaleogram (CWT)")
plt.ylabel("Frequency (Hz)")
plt.xlabel("Time (s)")
plt.tight_layout()
plt.show()
\end{lstlisting}

% --- SECTION 5: THE DISCRETE WAVELET TRANSFORM (DWT) ---
\section{The Discrete Wavelet Transform (DWT)}

\subsection{Theory: Multi-Resolution Analysis}
The CWT is computationally expensive and highly redundant (the coefficients are correlated). For many applications like data compression (JPEG 2000) and denoising, we don't need such high resolution.

The \textbf{Discrete Wavelet Transform (DWT)} provides an efficient and compact representation. It is based on \textbf{Multi-Resolution Analysis (MRA)}, which analyzes the signal at different scales by using a \textbf{filter bank}.

The DWT works by passing the signal through two complementary filters and then downsampling:
\begin{enumerate}
    \item \textbf{Low-Pass Filter (g):} This filter averages the signal, removing high-frequency details. The output is called the \textbf{Approximation Coefficients (cA)}. This is the "trend" or "essence" of the signal.
    \item \textbf{High-Pass Filter (h):} This filter highlights differences, capturing the high-frequency components. The output is called the \textbf{Detail Coefficients (cD)}. This is the "noise" or "nuance."
\end{enumerate}
After filtering, the outputs are \textbf{downsampled by 2} (we keep only every other sample). This is possible without loss of information (due to the Nyquist theorem) and makes the transform efficient.

% --- Directive A: Placeholder Figure ---
\begin{figure}[htbp]
  \centering
  \framebox{\parbox{0.9\textwidth}{\centering
    \vspace{4cm}
    \textbf{Figure 6: DWT Filter Bank (1 Level)} \\
    A diagram showing the input signal $f(t)$ splitting into two branches. \\
    - Top branch: Low-Pass Filter $[g]$ -> Downsample [$\downarrow 2$] -> Output [cA] \\
    - Bottom branch: High-Pass Filter $[h]$ -> Downsample [$\downarrow 2$] -> Output [cD]
    \vspace{4cm}
  }}
  \caption{A single-level Discrete Wavelet Transform decomposition.}
  \label{fig:dwt_filter_bank}
\end{figure}

This process is hierarchical. To get the next "level" of decomposition, we take the Approximation coefficients (cA) and pass \textit{them} through the same filter bank.

% --- Directive A: Placeholder Figure ---
\begin{figure}[htbp]
  \centering
  \framebox{\parbox{0.9\textwidth}{\centering
    \vspace{6cm}
    \textbf{Figure 7: Multi-Level DWT (Wavelet Tree)} \\
    A tree diagram: \\
    Level 0: Signal \\
    Level 1: -> [cA1] and [cD1] \\
    Level 2: [cA1] splits -> [cA2] and [cD2] \\
    Level 3: [cA2] splits -> [cA3] and [cD3] \\
    The final coefficients are [cA3, cD3, cD2, cD1]
    \vspace{6cm}
  }}
  \caption{A 3-level DWT decomposition. The signal is broken down into a low-resolution approximation (cA3) and details at various scales (cD1, cD2, cD3).}
  \label{fig:dwt_tree}
\end{figure}

The \textbf{Inverse DWT (IDWT)} perfectly reconstructs the original signal by reversing the process: upsampling and passing through synthesis filters.

% --- SECTION 6: DWT IN PRACTICE (DENOISING) ---
\section{DWT in Practice: Signal Denoising}

A powerful application of the DWT is denoising. The intuition is that:
\begin{itemize}
    \item \textbf{Real signal:} Most of its energy is concentrated in a few large Approximation coefficients.
    \item \textbf{Noise:} Its energy is spread out across all Detail coefficients at all scales.
\end{itemize}
Therefore, we can "denoise" a signal by:
\begin{enumerate}
    \item \textbf{Decompose:} Perform a DWT on the noisy signal.
    \item \textbf{Threshold:} Set all the small Detail coefficients (cD) to zero. This removes the noise while keeping the main signal energy intact. This is called "thresholding."
    \item \textbf{Reconstruct:} Perform an IDWT on the modified coefficients.
\end{enumerate}

% --- Directive A: Placeholder Figure ---
\begin{figure}[htbp]
  \centering
  \framebox{\parbox{0.9\textwidth}{\centering
    \vspace{4cm}
    \textbf{Figure 8: Noisy Signal} \\
    A plot of a "step" signal (e.g., 0 for a while, then 1)
    with significant random noise added.
    \vspace{4cm}
  }}
  \caption{The input signal: a clean step function corrupted by noise.}
  \label{fig:noisy_signal}
\end{figure}

% --- Directive A: Placeholder Figure ---
\begin{figure}[htbp]
  \centering
  \framebox{\parbox{0.9\textwidth}{\centering
    \vspace{4cm}
    \textbf{Figure 9: Denoised Signal} \\
    A plot showing the original noisy signal (faded) and the
    reconstructed signal (bold). The step is preserved, but
    the noise is dramatically reduced.
    \vspace{4cm}
  }}
  \caption{The output of the DWT denoising process.}
  \label{fig:denoised_signal}
\end{figure}

This process is shown in the Python code below.

\begin{lstlisting}[language=Python, caption=Python code for DWT denoising., label=list:dwt]
import numpy as np
import matplotlib.pyplot as plt
import pywt

# 1. Create a noisy signal (Fig 8)
true_signal = np.concatenate((np.zeros(500), np.ones(500)))
noise = np.random.normal(0, 0.2, 1000)
noisy_signal = true_signal + noise

# 2. Decompose using DWT
# We use 'db4' (Daubechies 4 wavelet) at level 5
wavelet = 'db4'
level = 5
coeffs = pywt.wavedec(noisy_signal, wavelet, level=level)
# coeffs = [cA5, cD5, cD4, cD3, cD2, cD1]

# 3. Threshold the Detail coefficients
# We find a universal threshold
sigma = np.median(np.abs(coeffs[-1])) / 0.6745
threshold = sigma * np.sqrt(2 * np.log(len(noisy_signal)))

# Apply "soft" thresholding
# (set to 0 if below thresh, shrink if above)
thresholded_coeffs = []
for c in coeffs:
    thresholded_coeffs.append(pywt.threshold(c, threshold, mode='soft'))

# 4. Reconstruct the signal
reconstructed_signal = pywt.waverec(thresholded_coeffs, wavelet)

# 5. Visualization (Fig 9)
plt.figure(figsize=(10, 5))
plt.plot(noisy_signal, color='gray', alpha=0.5, label='Noisy Signal')
plt.plot(reconstructed_signal, color='red', linewidth=2, label='Denoised Signal')
plt.plot(true_signal, color='black', linestyle='--', label='True Signal')
plt.legend()
plt.title("Wavelet Denoising using DWT")
plt.show()
\end{stlisting}

% --- SECTION 7: CHOOSING A MOTHER WAVELET ---
\section{Choosing a Mother Wavelet}

The choice of mother wavelet $\psi(t)$ is crucial and depends on the application.
\begin{itemize}
    \item \textbf{Haar wavelet:} The simplest wavelet, a square wave. Good for detecting sharp edges or steps.
    \item \textbf{Daubechies (db) wavelets:} A family (db2, db4, ...). They are compact and good for general-purpose signal processing and compression.
    \item \textbf{Morlet wavelet:} A complex-valued wavelet (a sine wave in a Gaussian envelope). Very good for CWT and feature extraction in time-frequency, as it has a clear frequency interpretation.
    \item \textbf{Symlets (sym):} Modifications of Daubechies to be more symmetric, which can be useful in image processing.
\end{itemize}
The choice depends on properties like orthogonality (DWT), symmetry, and a "vanishing moments" (how well it represents polynomials).

% --- Directive A: Placeholder Figure ---
\begin{figure}[htbp]
  \centering
  \framebox{\parbox{0.9\textwidth}{\centering
    \vspace{6cm}
    \textbf{Figure 10: A Zoo of Wavelets} \\
    Four small plots showing the shapes of: \\
    1. Haar (a square step) \\
    2. Daubechies 4 (db4) (an asymmetric, fractal-like shape) \\
    3. Morlet (a wave in a Gaussian envelope) \\
    4. Symlet 8 (sym8) (a more symmetric version of db)
    \vspace{6cm}
  }}
  \caption{A selection of common mother wavelets, each with different properties suited for different tasks.}
  \label{fig:wavelet_zoo}
\end{figure}

% --- SECTION 8: CONCLUSION ---
\section{Conclusion}
Wavelet analysis fundamentally extends the power of signal processing beyond the static, non-temporal world of the Fourier Transform. By using a "small wave" as a basis, it allows us to create a multi-resolution "microscope" to examine a signal in both time and frequency simultaneously.

\begin{itemize}
    \item The \textbf{CWT} is a powerful tool for \textit{analysis} and \textit{visualization}, producing intuitive scaleograms that reveal the time-frequency DNA of a signal.
    \item The \textbf{DWT} is a powerful tool for \textit{processing} and \textit{representation}, enabling efficient compression and denoising by separating a signal into its core components (approximations) and its details.
\end{itemize}

From analyzing non-stationary financial data and denoising astronomical images to compressing JPEGs and detecting transients in gravitational waves, wavelets provide a flexible and powerful framework for understanding a world that is, by its very nature, localized in time.

\end{document}
