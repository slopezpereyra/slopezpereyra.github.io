\documentclass[a4paper]{article}


\begin{document}

Not long ago I found online a person I met in my teenage years. Although I did
not know her well, she struck me at the time as a calm and quiet soul. I came to
find many sad things about her. 

She had joined a group of women who, as she put it, \lq\lq gifted each other
money\rq\rq{}---an alternative economy, which was feminist \lq\lq because the
group consisted of women only\rq\rq{} and defied the mainstream capitalist
dynamics. On entering the group, a woman was of the fire element, and had to
give her money away to another who had reached the element of water. At some
point, through some mechanism she made not clear, the one that was fire
was to become water and receive \lq\lq that great abundance\rq\rq{} too. In
short, in what seemed to me a desperate need for belonging, she had fallen for a
pyramid scheme. I later came to find she had sold most of her things in order to
meet the initial payment required for entering the group, and that two years
later she still awaits her time to become water---and although no fire lasts
that long, she trusts her time will come, and that abundance awaits.

She was also very caught up in what I can only term \textit{pensamiento mágico}.
Every being is the vibration of a spiritual substance---A ritual was to be held
involving the burning of pieces of paper with our fears written on
them---Everything she believed was a bizarre mixture of a conspiracy theory,
a self-help book for bored women over fifty, and \textit{duendology}
(duendología). In short, she seemed to have lost track of what was real, and to
be very lonely and credulous indeed. And I could not help but to feel a
tremendous compassion for her.

I was reminded of a very dark period of my life, which transpired about three
years ago. The trigger is too private to note here---suffices to say a
tragedy in the most inner circle of my family provoked a profound depression in
me. Perhaps for the worst, at the time I was studying the work of C.G. Jung,
which was so suggestive in a mind as vulnerable as mine was at the time that I
secluded myself to the examination of my dreams. I came to learn that
self-absorption is the finest ally of anguish. It is safe  to say my only
happiness at the time came from the study of mathematics and chess, to both of which
I devoted myself almost obsessively. But one must be very lost indeed when chess
and mathematics, which are everywhere held to be the precursors of madness, are
one's anchors. Whatever the case, I hardly remember anything of that period in
my life. When I think of them, two things only stand out: a profound solitude,
which nothing could ever break, and a sadness as deep as I have ever felt.
Everything else is blurry and almost irrecoverable in my memory.

I was reminded, I say, that at this time I came quite close to Christianity. A
stranger may think it stupid, perhaps even insulting, to take Christianity as
similar to the magic ideas that seized the mind of this old acquaintance I have
spoken about. But whoever knows me personally will understand the comparison is
not far from true. I have never received any religious education. I have not
been baptized. I had only read the bible in my twenties, out of shear cultural
and literary curiosity. And not only was I not religious, but after years of
acquainting myself with the most prominent philosophers, particularly those of
the XIX century, I was a committed atheist. 

At the time, I spoke with two friends of mine. One was a devoted Christian, who
had studied Spanish literature in New York, specializing in the Catholic
writers of the \textit{Siglo de oro}. The other was a former Christian that
converted to Islam and led a highly religious life. I came to both of them with
a single, unique question: how was faith brought about. I distinctly remember
\textit{wanting} to believe---so desperate was I---but I couldn't find a way to
do so. I will avoid any detailed discussion on this: suffices to say their
answers did not convince me. Luckily for me, I eventually regained my strength
and became my normal self, and these religious concerns became once more a matter
of cultural diversion only.

All of this to say: no one is so blessed and protected from misfortune that he
may not fall prey to desperate answers for desperate questions. It is a repeated
stoic idea that fortune is the master of us all---but it is also a true one. A
few weeks ago, I found the word \textit{weird} in an English ballad from the
XII century---but it was used in a strange sense. Upon consulting the
etymology of the word, I found that \textit{weird}, from the Anglo-saxon
\textit{Wyrd}, meant \textit{faith, fatum, destino}. The \textit{Weird sisters}
were the Norns of Scandinavian mythology, as accounted I think by Snorri
Sturluson. These three goddesses spin the threads of fate in a manner inexorable
to men. Who is to say what is being woven for him? I may find that old
acquaintance strange, her ideas perhaps even ridiculous---but how many
misfortunes split the distance from her fantastic confusion and my sober
tranquility? 
    












\end{document}
