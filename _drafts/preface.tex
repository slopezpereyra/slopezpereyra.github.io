
En las \textit{Confesiones} de Agustín se enseña que todo es bueno; en la
\textit{Ética} de Spinoza, que todo es necesario. Quiero pensar que, al menos
en lo que concierne a estos poemas, estas enseñanzas son ciertas; que cuanto
hay de innecesario en ellos es bueno, y cuanto hay de malo es necesario. Solo 
bajo ese consuelo, contradictorio y absurdo, puedo arrojarlos a la luz.

