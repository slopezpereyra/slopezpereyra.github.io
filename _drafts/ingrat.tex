\documentclass[a4paper, 12pt]{article}

\usepackage[utf8]{inputenc}
\usepackage[T1]{fontenc}
\usepackage{textcomp}
\usepackage{amssymb}
\usepackage{newtxtext} \usepackage{newtxmath}
\usepackage{amsmath, amssymb}
\newtheorem{problem}{Problem}
\newtheorem{example}{Example}
\newtheorem{lemma}{Lemma}
\newtheorem{theorem}{Theorem}
\newtheorem{problem}{Problem}
\newtheorem{example}{Example} \newtheorem{definition}{Definition}
\newtheorem{lemma}{Lemma}
\newtheorem{theorem}{Theorem}
\DeclareMathAlphabet{\mathcal}{OMS}{cmsy}{m}{n}

\begin{document}


Nothing proofs more strongly that reality is a fragile illusion than the image
we make of our own parents. No people in this world are closer nor more distant
to us, none so impenetrable and yet familiar. When this more or less expected
detachment from reality takes a pathological or bizarre form, it is equally
common to find people idealizing and vilifying their parents beyond the grounds
of reason. They either fail to apply to their parents the standards set for
everyone else, or lack for them the compassion of which they may otherwise be
capable of. And who has not find in life the tragic figure of a parent, of a
human---this is, obscure and loving, contradictory and present---parent, whose
children have deprived of all compassion and kindness? Who has not seen an
otherwise loving parent pay the price of a sin committed once or twice, of
mistakes committed under pressure or lack of character, for all his from that
moment solitary and unjust life? Who has not seen children forgive themselves a
thousand times excesses and mistakes they do not pardon their parents for
committing once? 








\end{document}



