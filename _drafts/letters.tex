\documentclass[a4paper]{article}

\usepackage[utf8]{inputenc}
\usepackage[T1]{fontenc}
\usepackage{textcomp}
\usepackage{amsmath, amssymb}
\edef\restoreparindent{\parindent=\the\parindent\relax}
\DeclareSymbolFont{letters}{OML}{ztmcm}{m}{it}
\usepackage{parskip}
\restoreparindent
\newtheorem{problem}{Problem}
\newtheorem{example}{Example}
\newtheorem{lemma}{Lemma}
\newtheorem{theorem}{Theorem}
\newtheorem{problem}{Problem}
\newtheorem{example}{Example}
\newtheorem{definition}{Definition}
\newtheorem{lemma}{Lemma}
\newtheorem{theorem}{Theorem}

\begin{document}
    

Hola amigo,

Te escribo una respuesta muy tardía, espero que no te moleste. Los
últimos tres meses han sido de una intensidad casi sin precedentes. Tus noticias
me pusieron muy feliz, se ve que estás cómodo y contento con la vida allá.
Evidentemente, Kathy es un personaje total y seguro se entretienen mucho. Me
hizo reír un montón el video, no solo por el boxeo de la señorita sino porque tu
risa es demasiado contagiosa jajaja. Las fotos me encantaron por cierto! 

Acá las noticias personales son excelentes pero la situación general del país es
pésima por donde se la mire. Pero vamos a lo positivo primero. Me preguntaste
por el doctor de Penn y el paper; el señor trabaja hace muchos años en el cruce
entre neurociencia e inteligencia artificial. No creo que recuerdes pero mi
paper (en resumen) presenta dos funciones matemáticas que, en principio,
deberían servir para determinar el grado de neuroplasticidad existente en
regiones corticales del cerebro. El testeo empírico corrobora que efectivamente
las funciones capturan aspectos importantes de la dinámicas sinápitcas. Por
ejemplo, permiten que un *random forest* model prediga si una persona
es depresiva o no (entre otras cosas) con una precisión >90%. En fin, al señor le pareció
interesante, me dio muchos consejos sobre cómo escribir el paper de mejor
forma ---porque yo le daba demasiada atención a la parte matemática y él
aconsejó que haga más énfasis en las implicancias clínicas --- y lo
publicaremos juntos. De hecho, en Febrero vamos a enviar el abstract a una
conferencia que se hará en Houston, con suerte lo aceptarán.

Por otro lado, no sé si vos recordarás que en mi laboratorio colaboramos con un
lab de la Universidad de Washington (el estado, no la ciudad). El director de
ese laboratorio me invito a ir a Washington un par de semanas el año entrante,
todo pago, para que conozca a su equipo y me sume a trabajar con ellos. También
me invitó a hacer el doctorado con él cuando me reciba. Si bien no creo que
termine haciendo el doctorado allá, porque la idea con Selene es otra, es muy
bueno saber que al menos hay una universidad que ya me aceptaría. Además este
señor es una bestia, es un doctor en astrofísica *y* en neurociencias --- no sé
cómo hizo para tener el doble doctorado. En fin, una persona muy formada. 

En lo negativo, parece que el CONICET no durará mucho tiempo más, como te habrás
enterado por tu papá. Estuve ayudando a Silvina con trámites de su beca en
España; todos en la ciencia estamos apuntando a otros países. Es una desgracia,
pero una carrera científica en Argentina es inviable. Lo que más
duele es que la gente no valora la ciencia. Ayer escuché a alguien referirse al
CONICET como una "institución deficitaria" (sic). Este discurso se ha viralizado
y hecho moneda corriente. La gente no entiende cómo funciona la ciencia ni la
valora particularmente; el gobierno de turno, menos todavía. Dedicar una vida al
estudio y el progreso científico para que la gente te considere un parásito...
Es una tristeza.

Hay muchas más otras cosas negativas. Pero no quiero hablar demasiado de
ellas, imagino que te enterás leyendo las noticias y que no tiene punto
repetir lo que ya sabés. 

Me da curiosidad en qué consiste tu práctica diaria en el doctorado, cómo es el
día a día. Contame un poco de eso. Qué es lo que más y lo que menos te gusta?
Dijiste que sentís que algunas cosas te superan intelectualmente. Supongo que
eso es de lo más normal en el mundo de la ciencia; al menos yo lo siento
frecuentemente en el lab, ni hablar de que seguro lo sentiré en el doctorado.
Pero no creo que sea nada que eventualmente no hayas superado o puedas superar.

Por otro lado, ¿cómo estás con el tema de la concentración y la atención? La
última vez que hablamos me contaste que habías estado batallando mucho con eso
antes de irte y que estabas un poco mejor. Espero que hayas seguido bien. 

Bueno, no tengo mucho más que contar. Esta noche es navidad así que feliz
navidad supongo. Cristo no es mi Dios pero es encarna una ética que es cada vez
más necesaria. Si querés ver algo hermoso de Argentina, mirá una película que se
llama "El ambulante". Está en la plataforma Mubi, aunque seguro la conseguís
pirata. Es una oda a la humanidad, hermosa, y (sin decirlo) es una expresión de
lo más lindo de la ética cristiana. Siento que es una recomendación navideña
apropiada. 

Un abrazo!!





\end{document}
