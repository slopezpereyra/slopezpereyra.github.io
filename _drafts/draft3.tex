\documentclass[a4paper]{article}
\usepackage{etoolbox}
\AtBeginEnvironment{quote}{\par\singlespacing\small}



\begin{document}

The conservative tradition generally draws its own roots to the philosophers of
the Enlightenment. Conversely, much of the mainstream left ---both in its
post-modern identitarian form as well as its more classical branches---
disregards the Enlightenment as either little more than the intellectual
stream of a primitive bourgeoisie ---a claim sustained sometimes by sophisticated
reasons--- or as the now hegemonic ideology of white and western men. But I have
come to suspect both conservatives and progressives alike are wrong in this
regard. A sincere examination of Enlightened thought leads to an unambiguous
conclusion on the matter: namely, that the very men who are esteemed to be the
intellectual guardians of liberal order would be petrified of horror if alive to
witness the modern development of capitalism. Furthermore, so many common
principles become apparent between these philosophers and those of the most
various socialist traditions ---from Marxist to anarchist thought--- that it is
impossible to disregard the hypothesis that they are, taken abstractly, almost
identical, and that whichever differences exist result merely of the application
of unvarying principles to distinct realities.



For example, common discourse puts classical liberalism as opposing the
autocratic power of the European monarchies of the early modern period. But this
is only partially true. In truth, the philosophers of the Enlightenment were
opposed to \textit{concentrated power}. It is a historical contingency that such
concentration occurred at the time in the court, but it is short-sighted---to
say the least---to abandon such position from the moment monarchies fell and
onwards. Furthermore, it would be a double mistake to believe not only that they
were opposed to concentrated power only as it was expressed in monarchies:
concentration of power, instead in some exceptional cases, is a consequence of
concentration of wealth---an obvious observation they were keen to make. In the
spirit of all these virtuous thinkers was, almost with no exception, a profound
repulsion for the concentration of wealth.

Take, for example, Rousseau's \textit{Discourse on the origin and basis of
Inequality among Men}. Should anyone had access to the content of the book while
being somehow unaware of who his author was, he would surely guess to say the treatise
was the work of a utopian socialist or an early libertarian thinker. Such guess
would be justified for many different reasons, of which only a few I should wish
to discuss as I advance this exposition. For instance, one may recall the
passage:

\begin{quote}
    Yo habría querido que nadie en el Estado pudisese considerarse como superior
    o por encima de la ley, ni que nadie que estuviera fuera de ella pudiese
    imponer que el Estado lo reconociese, porque cualquiera pueda ser la
    constitución de un gobierno, si se encuentra en él un solo hombre que no sea
    sumiso a la ley, todos los demás quedan necesariamente a la discreción de él
    ($\ldots$).
\end{quote}

In \textit{The social contract}, a footnote that is seldom recalled by
conservative writers reads:

\begin{quote}
    Bajo los malos gobiernos, esta igualdad [la que establece el pacto social y
    es base de todo sistema social] no es más que aparente e ilusoria: sólo
    sirve para mantener al pobre en su miseria y al rico en su usurpación. En
    realidad, las leyes son siempre útiles a los que poseen y perjudiciales a
    los que no tienen nada. De esto se sigue que el estado social no es
    ventajoso a los hombres sino en tanto que todos ellos poseen algo y ninguno
    demasiado.
\end{quote}

Furthermore, take Rousseau's conception of the nature of men. Such conception is
usually said to be that men is good by nature, but although this is---in fact
textually---his claim, it misses to convey the subtlety of the whole idea which
the phrase attempts to summarize. The natural men Rousseau is talking about is a
pure individual that has not yet entered a social arrangement, that is good
because, lacking any moral capacity, it wishes nor does anything else but that
which secures his existence or pleasure. It is good in an entirely negative sense.






    
\end{document}
