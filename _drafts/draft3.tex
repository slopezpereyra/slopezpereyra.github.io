\documentclass[a4paper]{article}


\begin{document}


The conservative tradition generally draws its own roots to the philosophers of
the Enlightenment. Conversely, much of the mainstream left ---both in its
post-modern identitarian form as well as its more classical branches---
disregards the Enlightenment as either little more than the intellectual
stream of a primitive bourgeoisie ---a claim sustained sometimes by sophisticated
reasons--- or as the now hegemonic ideology of white and western men. I suspect both
conservatives and leftists alike are wrong in this regard. A sincere inspection
of Enlightened thought leads to an unambiguous conclusion on the matter. Namely,
that the very men who are esteemed to be the intellectual guardians of liberal
order would be petrified of horror if alive to see the conditions under which
modern capitalism develops. Furthermore, such common principles become apparent
between these philosophers and those of the most various socialist traditions
---from Marxist to anarchist thought--- that it is impossible to disregard the
hypothesis that they are, taken abstractly, almost identical, and that whichever
differences exist result merely of the application of unvarying principles to
distinct realities.




    
\end{document}
