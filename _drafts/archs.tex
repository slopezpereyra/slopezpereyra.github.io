\documentclass[a4paper]{article}

\usepackage[utf8]{inputenc}
\usepackage[T1]{fontenc}
\usepackage{textcomp}
\usepackage{amsmath, amssymb}
\edef\restoreparindent{\parindent=\the\parindent\relax}
\DeclareSymbolFont{letters}{OML}{ztmcm}{m}{it}
\usepackage{parskip}
\restoreparindent

\usepackage{etoolbox}
\AtBeginEnvironment{quote}{\par\singlespacing\small}

\begin{document}

\section{Foreword}

I wish to put in writing some observations on an intellectual matter I deem
of special interest. One could perhaps dare to reduce it to the study of
brains. In this sense, the matter is circumscribed to the inquiry of what may be
termed a \lq\lq representative instinct\rq\rq{} in mankind. Such \lq\lq
instinct\rq\rq{}---I use the term loosely for now---is plausibly the
byproduct of the emergence of symbolic faculties in instinctive beings.
This may not be too precise, but it is also not blatantly wrong, and thus I
accept it as a decent preliminary formulation. I am speaking of what a
peculiar branch of psychoanalysis termed \textit{archetypes}.

The object of this inquiry, I must say, was not originally conceived to be of
biological nature, and I do wish to attend to the earlier formulations of the
concept of \textit{archetype}. Firstly, because in the genealogy of any
scientific idea there are large branches of---in general quite
illuminating---pre-scientific antecedents. Secondly, because the hypothesis that
concerns us goes far beyond the scope of scientific inquiry, raising questions
about human nature itself.

As a last preliminary comment, I should wish to say the following: although the
concept, as we postulate it, has a tighter link with scientific understanding
than its original formulations, and although a considerable amount of empirical
work has been set forth by affective neuroscience on its favor, the existence
of archetypes is still merely a hypothesis. Only the scientific study of brain
functioning can ultimately provide satisfactory evidence in favour or against
it. It is probably the case that mind-problems are of the kind where
propositions are to be measured only in that way---by the weight of existing
evidence, and not by conclusive proof. A love for sharp edges and rigor compels
me to limit the scope of my speculation to what I deem to be in line with this
evidence and our current scientific understanding of the brain. It is utterly
short-sighted to claim that reason can only delve into matters where certain
knowledge exists (or could exist). The advancement of science itself must always
begin with educated conjectures, and a philosophical attitude towards uncertain
matters can \textit{at least}, when executed correctly, reveal the right and the
wrong questions to be raised.

\section{Archetypes}

The term archetype dates back at least to Dionysius the
Pseudo-Areopagite, who suggested an echo (\textit{apekhémata}) of the divine
essence exists on every sensible object, by virtue of which they may elevate
to the immaterial \textit{arkhitypía}. Here, the word expresses the perfect
platonic ideal of which each sensible object is an imperfect realization.
This is not the sense of the term that interests us---although the notion of
\textit{echo} (or \textit{imprint, trace}, etc.) will be relevant. In fact,
our issue is partly the different meanings the term ascribes to across
traditions and contexts. Even the work of Jung, that elevated the term to an
unprecedented intellectual dimension, lacks an unequivocal definition,
sometimes confronting us with the suspicion that the very author
bestows it with different meanings depending on the period of his
intellectual life or the object of his exposition. 

It is my opinion that this linguistic issue is not as daunting as it may
appear at first. A simple reason is that many of the meanings commonly
attributed to the term can be readily disregarded as nonsensical. Secondly,
some advances in the field of neuroscience—particularly in the line of
research set forth by Panksepp—have contributed a great amount of empirical
material to the question. This rich set of facts lays out more plausible—and
potentially falsifiable—notions of what may be meant by “archetype”. In
other words, though the riddle is far from answered, we have at our disposal
a whole domain of reality that the ancient philosophers—or early
psychoanalysts—lacked. Thus, the contending formulations, to our surprise,
are in the end not very numerous. 

It then becomes the question how to penetrate into the essence of the
formulations at hand. For this purpose, aside from the consideration of
empirical evidence, I intend to follow two elementary principles. Firstly, to
assess the amount of presuppositions implied in each of them. Secondly, to think
of their meaning as determined by the sensible effects the concept is
represented to produce. In the latter lies whatever is to be meant by the
concept; in the first, its intellectual economy. 

\section{Jung}

Jung had an explicitly dualistic outlook on the psyche. He endorsed
the philosophical stance according to which there is no evidence in favor of
the hypothesis that links psychological phenomena to physical and chemical
processes, that there is no reason to regard the mind as an epiphenomenon of
matter, and that it should be treated as a \textit{sui generis} factor—at least
until the artificial creation of a mind can be established as an achievable
endeavor. This is explicitly held in the work \textit{Archetypes of the collective
unconscious}, written around 1932. The state of evidence at the time may
perhaps make this claim understandable. However, at least to my
knowledge, he never explicitly relinquished it. 

Jung frequently associated the notion of archetype to that of primordial image.
This association is particularly present whenever he was interested in drawing
the parallelism between mythological motifs and the archetypal phenomena he
allegedly witnessed in his clinical work. He also ties the concept of archetype
to the notion of pattern of functioning: 

\begin{quote}
Like every animal, he [the man] possesses
a preformed psyche which breeds true to his species and which (...) reveals
distinct features traceable to family antecedents. (...) We are unable to
form any idea of what those dispositions or aptitudes are which make
instinctive actions in animals possible. And it is just as impossible for us
to know the nature of the preconscious psychic disposition that enables a
child to react in a human manner. We can only suppose that his behavior
results from patterns of functioning, which I have described as images. The
term “image” is intended to express not only the form of the activity taking
place, but the typical situation in which the activity is released” — 1959,
pag. 78. 
\end{quote}

This evolutionary speculation is not trivial. It is in line with
more recent expositions, such as those elaborated in the work of Campbell.
However, it is a rather strange twist of the logical chain to propose an
evolutionary basis for an immaterial phenomenon. The process of natural
selection affects the course of biological species made up of material
elements, all the way down to a rather peculiar acid, whose material nature—I
should hope—is uncontested. Putting that aside, to claim we do not know the
mechanisms that make instinctive actions possible is false under the present
state of science. The neuroscientific findings seem to support, to a
certain extent, the Jungian hypothesis of “archetypal” behavioral patterns.
We will come to discuss these findings later on. 

To Jung, primordial images are contentless, structural patterns—just like an
instinct, taken by itself, is also strictly formal. This is an important
point to make wherever we find those delirious babblings mumbled by the
sadly numerous followers of the new age philosophy, who distort Jung's
theories to fantasize about a world of actual images, where this or that
archetype “appears as” this or that other figure and contains some form of
sacred message. Granted, at times the author made it easy for such
misinterpretations to be drawn out of his work. But it is also true that
Jung asserted above all $a.$ the affective tone of these “primordial images”,
and $b.$ that they should be understood not by virtue of an essential content
but by their teleology—\textit{id est}, the specific behavioral disposition induced
by them. 

\begin{quote}
But we only arrive at the meaning of a physical organ when we begin to ask
teleological questions. Hence the query arises: What is the biological
purpose of the archetype? —1959, pag. 161. 
\end{quote}


It continues to be unclear in what way the biological conception
of the archetype and the immaterial notion of the psyche may theoretically
harmonize. But we shall leave this question aside for the moment to discuss
a bit more deeply about these primordial images, or patterns of behavior.

With regards to their teleology, one mustn't too hastily convince himself
that it exists. A fair number of biological traits are nothing but the
byproduct of others. A general idea is that the phenomenology of
archetypes---the specific behavioral dispositions that are understood to be
archetypal---is the induction of an affective state by stimuli to which we
were once selected to respond affectively, or
stimuli resembling other to which we were selected to respond affectively,
even when at the present state of history such response is unwarranted.
To rephrase: In the same way the fact that sucrose was once scarce makes us feel a
compulsive appetite for it even today, other stimuli are also imbued with,
so to speak, archaic affect. In this regard, we find a clarifying and
mundane example in \textit{The Masks of God: Primitive Mythology}, by
Campbell. 

\begin{quote}
    Chicks with their eggshells still adhering to their tails dart for
    cover when a hawk flies overhead, but not when the bird is a gull or
    duck, heron or pigeon. Furthermore, if the wooden model of a hawk is
    drawn over their coop on a wire, they react as though it were
    alive---unless it be drawn backward, when there is no response.

    Here we have an extremely precise image---never seen before, yet
    recognized with reference not merely to its form but to its from in
    motion, and linked, furthermore, to an immediate, unplanned, unlearned,
    and even unintended system of appropriate action: flight, to cover.
    (\ldots) Furthermore, even if all the hawks in the world were to
    vanish, their image would still sleep in the soul of the chick---never
    to be roused, however, unless by some accidente of art ($\ldots$). With
    that the obsolete reaction of the flight to cover would recur; and,
    unless we knew about the earlier danger of hawks to chicks, we should
    find the sudden eruption difficult to explain. \lq Whence\rq, we may
    ask, \lq this abrupt seizure by an image to which there is no
    counterpart in the chicken's world? ($\ldots$)\rq.
\end{quote}

It is not difficult to observe what is being suggested in this passage.
Namely, that there were now non-present stimuli imbued with affect by
virtue of evolutionary archaic systems; that the responses elicited by
these stimuli were arguably propitious to survival; that these archaic
responses can be evoked by \lq\lq accidents of art\rq\rq, provoking seemingly
unwarranted responses in people. The psychologist that risks to say this
draws his attention, of course, to all those affective motifs deemed
universal in human experience: patterns suspected to be ubiquitous even
across different religious contexts and times, appearing with equal
strength in uncommunicated cultures as in the spontaneous production of
individual people. 

The scope of this alleged universality is not altogether clear. Some have
proposed general practices but not particular images to fall within it.
Shamanism may be a good example. Others have drawn their attention to particular
imaginations: for example, that of the universal flood. I do not wish to discuss
this point, for I cannot see a way to circumscribe the speculation within a
limiting frame. And where there is place for unbounded speculation there is no
place for truth. So, having presented a general overview of the more or less
pre-scientific conception of \textit{archetypes}, I should wish to proceed with
the equally interesting---but more epistemologically promising---attempts of
neuroscience at tackling this question.

\section{Archaic affect}

Allow me to advance that I shall not give too much detail on the neurobiology of
the findings here commented. Doing so would make this writing much
longer and technical than I intend. I should rather wish to summarize what I
understand to be the principal conclusions and postulates of affective
neuroscience and their relationship to the concept of \textit{archetype}. Before
discussing these contributions, however, it is pertinent to say a few words
concerning both their ontology and their epistemology. This is important because
discussions concerning the mind are famously problematic, and it would be
imprudent to simply delve into the neuroscience before addressing a few
concerns.

With regards to ontology, it is a common mistake to believe that Descartes
committed an intellectual sin when postulating his \textit{res cogitans}.
In fact, when doing so, he acted in accord with standard scientific practice. It
is obvious that certain human faculties cannot be explained by mechanical laws.
If matter was to be entirely described by the laws of mechanics, as was the
stance at Descarte's time, then some other substance had to be postulated in
order to explain these non-mechanistic faculties. When Newton and Galileo
postulated \textit{forces}, they were doing so in similar conditions. Rather
than a positive scientific achievement, such postulate was a problematic, but
inescapable necessity.

The standard philosophical stance today is a commitment to a materialist
monism. I share this stance---but the conundrum which forced Descartes to
reject monism remains. It seems that, rather than postulating a new substance,
the appropriate position is a form of Spinozian \textit{dual-aspect monism}, as
Panksepp described it. It is a fact that thought emerges from matter---it is
happening now as I write---but it is plausible that properties emergent from
matter are better understood in non-material terms. This is in no sense an
ontological claim; I simply state that a complex system may be not be properly
described by the same laws which describe its constitutive elements. So, it
seems the missing piece which Descartes' \textit{res cogitans} aimed to account
for were, in fact, complex neural dynamics. \textit{How} complex neural dynamics
may produce thought, emotion, etc., pertains to scientific inquiry rather than
to philosophical speculation. Affective neuroscience, in particular the work of
Panksepp and Damasio, has provided an immense corpus of evidence in favor of the
notion that conciousness is not only an emergence of complex neural
organization, but particularly of those deep regions of the brain whose function
seems to be the realization of emotion and affect.

With regards to epistemology, the principal question is: How can we study
emotions scientifically? The answer to this question is given by three
observations which are, to me, scientifically impeccable. \textit{(1)} Animals
exhibit outward indicators of emotional states. The clearest example are the
separation distress calls. \textit{(2)} We can inquire on the neurobiology
regulating such expressions using standard scientific methods. \textit{(3)} If
artificial exposition to the neurobiological regulators of such emotions
produces corresponding \textit{feelings} in humans, the weight of the evidence
favors the conclusion that animals also experience these feelings---at least
when the brain systems involved are homologous. For
example, in a paper entitled \textit{Opioid blockade and social comfort in
chicks} (1980), Panksepp showed that opioid blockade with naloxone reduced the
imprinting effect in chicks. More generally, his findings at the time suggested
that opioids mediate social interaction and bonding. More recent papers of his
added a great deal of evidence to this claim. We know certain opioids, such as
oxytocin, mediate human bonding and partially regulate the feelings of warmness
and care present, for instance, between a newborn and a mother. Since the
neurobiological circuitry involved is homologous in both species, it is not
unreasonable to claim that similar feelings permeate the bond between a little
chick and its mother.

As a last note, and to put this preamble aside, I should say that I follow
Panksepp's terminology. By \textit{emotion}, I mean any affective,
cognitive, behavioral or physiological change in the organism. In this sense,
the word is more tightly linked to its etymology than to its conversational
meaning. By \textit{affect}, I mean quite broadly any subjective and
experiential feelings, whereas an \textit{emotional affect} denotes the
experiential component of an internal brain state.

Emotional affects are generally associated to external events. When compared to
external inductors of affect, internal ones (e.g. the memory of a deceased loved
one) are rather few, and perhaps limited to our own species. Even so, every
affect is an internal function of the brain. Scientists have used the terms
\textit{valence, arousal} and \textit{surgency} to describe different aspects of
the affective experience, without ever daring to speak of emotions. The present
state of science, so I believe, conclusively leads to the fact that such terms
would make no sense if not denoting affect.

The essential thesis of affective neuroscience as a field, as far as I
understand it, is the following: that affective experience \lq\lq reflects a
primitive form of consciousness which was the evolutionary platform for the
emergence of more complex forms of consciousness\rq\rq{} (Panksepp,
\textit{Affective consciousness: Core emotional feelings in animals and humans},
2005). Here, \textit{consciousness} refers to \lq\lq brain states that have an
experiential feel to them, and it is envisioned as a multi-tiered process that
needs to be viewed in evolutionary terms, with multiple layers of
emergence.\rq\rq{} (\textit{idem}). 

The evidence in favour of this thesis is close to overwhelming and, to my eyes,
at least in essence, little doubt remains about their truth. There are, of
course, disputes concerning technicalities---the manner in which different
ways of consciousness evolved, the neurobiology modulating certain aspects of
consciousness---but the essential claim remains untouched. Not only is the
experiential component of emotion common to, at least, all mammals, but it
represents a primitive form of consciousness that is still present in the human
brain. This presence is attested by the fact that the systems from which such
form of consciousness emerges remain very well preserved and highly homologous
across mammal species. It is this evolutionary insight what draws a link between
the conclusions of affective neuroscience and pre-scientific speculations on
psychological archetypes.

Panksepp used the term \textit{equalia} (\textit{e} from evolutionary,
\textit{qualia} from its traditional philosophical sense) to denote the
varieties of positive or negative feelings not simply mediated by our perceptual
interfaces, but by the orchestration of inherited emotional systems in the
brain. The raw affects engendered by such subcortical systems are \lq\lq
ancestral memories (instincts) that promote survival\rq\rq\ (the reader is
surely reminded of Campbell's example of the chick and the eagle).  

It is not hard to see that these \lq\lq ancestral memories\rq\rq\ are what Jung
termed \textit{primordial images}. Needless to say, a highly complex, and in
many senses still impenetrable interplay exists between the archaic emotional
systems of the mammalian brain and the cortical regions to which our own species
owns its so-called superior faculties. Notwithstanding, these cortical
regions are entirely inessential to primary forms of consciousness; such forms
subsist even in extreme cases of virtually non-existing cortical development.
This was proven not only in Panksepp's work, but quite famously by Damasio as
well. 

Indeed, it seems that consciousness is like a curl or (in Spanish)
\textit{rizo}; the superior aspects of it can be removed while preserving
pre-conscious affect, but whenever its primitive components are lost, nothing of
it subsists. The different \lq\lq levels\rq\rq \ of consciousness have been
described in the neuroscientific literature as \textit{primary, secondary} and
\textit{tertiary} consciousness; or as \textit{core consciousness} and
\textit{extended consciousness}. Regardless of the terminology, lower levels of
consciousness have been demonstrated to depend on the primitive,
value-encoding neurocircuitry described before. In short, consciousness can
exist without cognition---but it cannot be without affect.

The relationship between the neurocircuitry of affective experience and
cognitively complex phenomena, such as spirituality or dreams, has not been
explored so far. However, in so far as affect is, virtually by definition,
value-encoding, it seems necessary that spiritual experience is some kind of
cognitive sublimation of these archaic components of human experience.
This would at least account for $a.$ the striking structural similarities across
the wide range of so-called archetypal stories and beliefs---the hero story,
animism, shamanism, etc.---\textit{b.} the profound affect with which such
stories and beliefs are imbued; and $c$. the absolute resilience of such
narratives in a scientific society. The same can be said of dreams. 

There is one peculiarly interesting paper, (The affective Core of the
Self)[https://www.frontiersin.org/articles/10.3389/fpsyg.2017.01424/full], which
surveys the insight provided by affective neuroscience into one of the specific
archetypes proposed by Jung. I will let the reader judge the conclusions of this
paper on his own; they do not differ from what I have described thus far but in
scope and technical precision.

\section{Summary}

Let us return to Campbell's quote:

\begin{quote}
    Chicks with their eggshells still adhering to their tails dart for
    cover when a hawk flies overhead, but not when the bird is a gull or
    duck, heron or pigeon. Furthermore, if the wooden model of a hawk is
    drawn over their coop on a wire, they react as though it were
    alive---unless it be drawn backward, when there is no response.

    Here we have an extremely precise image---never seen before, yet
    recognized with reference not merely to its form but to its from in
    motion, and linked, furthermore, to an immediate, unplanned, unlearned,
    and even unintended system of appropriate action: flight, to cover.
    (\ldots) Furthermore, even if all the hawks in the world were to
    vanish, their image would still sleep in the soul of the chick---never
    to be roused, however, unless by some accidente of art ($\ldots$). With
    that the obsolete reaction of the flight to cover would recur; and,
    unless we knew about the earlier danger of hawks to chicks, we should
    find the sudden eruption difficult to explain. \lq Whence\rq, we may
    ask, \lq this abrupt seizure by an image to which there is no
    counterpart in the chicken's world? ($\ldots$)\rq.
\end{quote}

This is indeed the mysterious fact unraveled by affective neuroscience. For all
practical purposes, it is correct to say an \textit{ancestral memory} exists in
the chick; one deeply embedded in ancient subcortical systems that mediate
affective experience. Campbell presented his example to represent in practical
terms the meaning of \textit{archetype} in Jungian psychology. I think it is
correct to say the existence of archetypes, \textit{precisely} as understood by
Jung, counts today with a substantial amount of evidence, and fits entirely with
the advancements of neuroscience. Somewhere in his complete works---I remember
not where---Jung states that we are, so to speak, \textit{thousands of years
old}. This is sensibly true. To speak of the value-encoding circuitry which lays
well beyond cognitive control, in the deepest regions of our brain, as a
\textit{personality}, as Jung did, is merely a verbal matter. 

Pseudo-Areopagite, I repeat, suggested a celestial echo (\textit{apekhémata})
exists in every sensible thing; and that by virtue of such imprint they may
reach the immaterial \textit{arkhitypía}. The \textit{ancient memories} revealed
by neuroscience, quite curiously, have something of both of these Greek
concepts. They are \textit{apekhémata} insofar as they are an ancestral trace, an
evolutionary imprint that, so to speak, echoes through
time by means of millennial evolution. They are \textit{arkhitypía} in that they
exist deep beyond the cognitive regions of our brain, evoking typical
patterns of behavior across a range of biological species, while being perhaps
impossible to represent by means of pure cognition or imagination.
































\end{document}
