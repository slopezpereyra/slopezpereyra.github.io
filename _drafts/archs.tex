\documentclass[a4paper]{article}

\usepackage[utf8]{inputenc}
\usepackage[T1]{fontenc}
\usepackage{textcomp}
\usepackage{amsmath, amssymb}
\edef\restoreparindent{\parindent=\the\parindent\relax}
\DeclareSymbolFont{letters}{OML}{ztmcm}{m}{it}
\usepackage{parskip}
\restoreparindent

\usepackage{etoolbox}
\AtBeginEnvironment{quote}{\par\singlespacing\small}

\begin{document}

    \section{Foreword}

    I wish to put in writing some observations on an intellectual matter I deem
    of special importance. One could, perhaps, dare to reduce it to the study of
    brains. In this sense, the matter is circumscribed to the inquiry of what may be
    termed a \lq\lq representative instinct\rq\rq{} in mankind. Such \lq\lq
    instinct\rq\rq{}---I use the term loosely for now---is plausibly the
    byproduct of the emergence of symbolic faculties in instinctive beings.
    This may not be too precise, but it is also not blatantly wrong, and thus I
    accept it as a decent preliminary conception. I am speaking of what a
    peculiar branch of psychoanalysis termed \textit{archetypes}.

    The object of this inquiry, I must say, was not originally conceived to be of
    biological nature, and I do wish to attend to its earlier formulations.
    Firstly, because in the genealogy of any scientific idea there are large
    branches of---in general quite illuminating---pre-scientific antecedents.
    Secondly, because the hypothesis that concerns us goes far beyond the scope of
    scientific inquiry, insofar as it raises questions with regards to
    the nature of the mind and, in consequence, our conception of human nature
    itself.

    As a last preliminary comment, I should wish to say the following: although
    the psychoanalytic conceptualization of archetypes is partially disregarded,
    although a considerable amount of empirical work has been set forth by
    affective neuroscience on this matter, the existence of archetypes is still
    merely a hypothesis. Only the scientific study of brain functioning can
    ultimately provide satisfactory evidence in favour or against it. It is
    probably the case that mind-problems are of the kind where propositions are
    to be measured only in that way, by the weight of existing evidence, and not
    by conclusive proof. A love for sharp edges and rigor compels me to limit
    the scope of my speculation to what I deem to be in line with this
    evidence and our current scientific understanding of the brain. It is
    utterly short-sighted to claim that reason can only delve into matters where
    certain knowledge exists (or could exist). The advancement of science itself
    must always begin with educated conjectures, and a philosophical attitude
    towards uncertain matters can \textit{at least}, when executed correctly,
    reveal the right and the wrong questions to be raised.

    \section{Archetypes}

    The term archetype dates back at least to Dionysius the
    Pseudo-Areopagite, who suggested an echo (\textit{apekhémata}) of the divine
    essence exists on every sensible object, by virtue of which they may elevate
    to the immaterial \textit{arkhitypía}. Here, the word expresses the perfect
    platonic ideal of which each sensible object is an imperfect realization.
    This is not the sense of the term that interests us---although the notion of
    \textit{echo} (or \textit{imprint, trace}, etc.) will be relevant. In fact,
    our issue is partly the different meanings the term ascribes to across
    traditions and contexts. Even the work of Jung, that elevated the term to an
    unprecedented intellectual dimension, lacks an unequivocal definition,
    sometimes confronting us with the suspicion that the very author
    bestows it with different meanings depending on the period of his
    intellectual life or the object of his exposition. 

    It is my opinion that this linguistic issue is not as daunting as it may
    appear at first. A simple reason is that many of the meanings commonly
    attributed to the term can be readily disregarded as nonsensical. Secondly,
    some advances in the field of neuroscience—particularly in the line of
    research set forth by Panksepp—have contributed a great amount of empirical
    material to the question. This rich set of facts lays out more plausible
    —and potentially falsifiable—notions of what may be meant by “archetype”. In
    other words, though the riddle is far from answered, we have at our disposal
    a whole domain of reality that the ancient philosophers—or early
    psychoanalysts—lacked. Thus, the contending definitions, to our surprise,
    are in the end not very numerous. 

    It then becomes the question how to penetrate into the
    essence of the definitions at hand. I intend to follow two general
    principles for this purpose. First, to assess the amount of presuppositions
    implied in them. Second, to think of their meaning only as the sensible
    effects the concept is represented to produce under certain specific
    conditions. In the latter lies whatever is to be meant by the concept; in
    the first, its intellectual economy. 
    
    \section{Jung}

    Jung had an explicitly dualistic outlook on the psyche. He endorsed
    the philosophical stance according to which there is no evidence in favor of
    the hypothesis that links psychological phenomena to physical and chemical
    processes, that there is no reason to regard the mind as an epiphenomenon of
    matter, and that it should be treated as a \textit{sui generis} factor—at least
    until the artificial creation of a mind can be established as an achievable
    endeavor. This is explicitly held in the work \textit{Archetypes of the collective
    unconscious}, written around 1932. The state of evidence at the time may
    perhaps make this claim understandable. However, at least to my
    knowledge, he never explicitly relinquished it. 

    Jung frequently associated
    the notion of archetype to that of primordial image. This association is
    particularly present whenever he was interested in drawing the parallelism
    between mythological motifs and the archetypal phenomena he allegedly
    witnessed in his clinical work. He also ties the concept of archetype to the
    notion of pattern of functioning: 

    \begin{quote}
    Like every animal, he [the man] possesses
    a preformed psyche which breeds true to his species and which (...) reveals
    distinct features traceable to family antecedents. (...) We are unable to
    form any idea of what those dispositions or aptitudes are which make
    instinctive actions in animals possible. And it is just as impossible for us
    to know the nature of the preconscious psychic disposition that enables a
    child to react in a human manner. We can only suppose that his behavior
    results from patterns of functioning, which I have described as images. The
    term “image” is intended to express not only the form of the activity taking
    place, but the typical situation in which the activity is released” — 1959,
    pag. 78. 
    \end{quote}

    This evolutionary speculation is not trivial. It is in line with
    more recent expositions, such as those elaborated in the work of Campbell.
    However, it is a rather strange twist of the logical chain to propose an
    evolutionary basis for an immaterial phenomenon. The process of natural
    selection affects the course of biological species made up of material
    elements, all the way down to a rather peculiar acid whose material nature—I
    should hope—is uncontested. Putting that aside, to claim we do not know the
    mechanisms that make instinctive actions possible is false under the present
    state of science. The neuroscientific findings seem to support, to a
    certain extent, the Jungian hypothesis of “archetypal” behavioral patterns.
    We will come to discuss these findings later on. 

    To Jung, primordial images are contentless, structural patterns—just like an
    instinct, taken by itself, is also strictly formal. This is an important
    point to make wherever we find those delirious babblings mumbled by the
    sadly numerous followers of the new age philosophy, who distort Jung's
    theories to fantasize about a world of actual images, where this or that
    archetype “appears as” this or that other figure and contains some form of
    sacred message. Granted, at times the author made it easy for such
    misinterpretations to be drawn out of his work. But it is also true that
    Jung asserted above all $a.$ the affective tone of these “primordial images”,
    and $b.$ that they should be understood not by virtue of an essential content
    but by their teleology—\textit{id est}, the specific behavioral disposition induced
    by them. 

    \begin{quote}
    But we only arrive at the meaning of a physical organ when we begin to ask
    teleological questions. Hence the query arises: What is the biological
    purpose of the archetype? —1959, pag. 161. 
    \end{quote}


    It continues to be unclear in what way the biological conception
    of the archetype and the immaterial notion of the psyche may theoretically
    harmonize. But we shall leave this question aside for the moment to discuss
    a bit more deeply about these primordial images, or patterns of behavior.

    With regards to their teleology, one mustn't too hastily convince himself
    that it exists. A fair number of biological traits are nothing but the
    byproduct of other traits. In other words, there are traits that arouse
    even when no selective pressure favoured their existence. A general idea is
    that the phenomenology of archetypes---the specific behavioral dispositions
    that are understood to be archetypal---is the induction of an affective
    state by stimuli to which we were once selected to respond affectively, or
    stimuli resembling other to which we were selected to respond affectively,
    even when at the present state of history such response is unwarranted.
    To rephrase: In the same way the fact that sucrose was once scarce makes us feel a
    compulsive appetite for it even today, other stimuli are also imbued with,
    so to speak, archaic affect. In this regard, we find a clarifying and
    mundane example in \textit{The Masks of God: Primitive Mythology}, by
    Campbell. 

    \begin{quote}
        Chicks with their eggshells still adhering to their tails dart for
        cover when a hawk flies overhead, but not when the bird is a gull or
        duck, heron or pigeon. Furthermore, if the wooden model of a hawk is
        drawn over their coop on a wire, they react as though it were
        alive---unless it be drawn backward, when there is no response.

        Here we have an extremely precise image---never seen before, yet
        recognized with reference not merely to its form but to its from in
        motion, and linked, furthermore, to an immediate, unplanned, unlearned,
        and even unintended system of appropriate action: flight, to cover.
        (\ldots) Furthermore, even if all the hawks in the world were to
        vanish, their image would still sleep in the soul of the chick---never
        to be roused, however, unless by some accidente of art ($\ldots$). With
        that the obsolete reaction of the flight to cover would recur; and,
        unless we knew about the earlier danger of hawks to chicks, we should
        find the sudden eruption difficult to explain. \lq Whence\rq, we may
        ask, \lq this abrupt seizure by an image to which there is no
        counterpart in the chicken's world? ($\ldots$)\rq.
    \end{quote}

    It is not difficult to observe what is being suggested in this passage.
    Namely, that there were now non-present stimuli imbued with affect by
    virtue of evolutionary archaic systems; that the responses elicited by
    these stimuli were arguably propitious to survival; that these archaic
    responses can be evoked by \lq\lq accidents of art\rq\rq, evoking seemingly
    unwarranted responses in people. The psychologist that risks to say this
    draws his attention, of course, to all those affective motifs deemed
    universal in human experience: patterns suspected to be ubiquitous even
    across different religious contexts and times, appearing with equal
    strength in uncommunicated cultures as in the spontaneous production of
    individual people. 

    The scope of this alleged universality is not altogether clear. Some have
    proposed general practices but not particular images to fall within it.
    Shamanism may be a good example. Others have drawn their attention to
    particular imaginations: for example, that of the universal flood. I do not
    wish to discuss this point, for I cannot see a way to circumscribe the
    speculation within a limiting frame. And where there is place for unbounded
    speculation there is no place for truth. So, having presented a general
    overview of the more or less pre-scientific conception of
    \textit{archetypes}, I should wish to proceed with the equally
    interesting---but more epistemologically promising---attempts of
    neuroscience at tackling this question.

    \section{Archaic affect}

    We follow here Panksepp's terminology. By \textit{emotion}, we mean any
    affective, cognitive, behavioral or physiological change in the organism. In
    this sense, the word is more tightly linked to its etymology than to its
    conversational meaning. By \textit{affect}, we mean any subjective and
    experiential feeling. In this sense, there can be emotion without
    affect---or non-affective emotional processes---as well as non-emotional
    affects---for example, taste, touch, etc.

    Emotional affects---this is, the experiential component of internal organic
    changes, such as the that which accompanies thirst and hunger, fear, the
    pleasure associated to sucrose consumption---is generally associated to
    external events. When compared to external inductors of affect, internal
    ones are rather few. Even so, every affect is an internal function of the
    brain. Here, the traditional concepts of \textit{valence, arousal} and
    \textit{surgency} are the tools used by psychologists to describe different
    aspects of the affective experience.

    The essential thesis of affective neuroscience as a field, to my view, is
    the following: that affective experience \lq\lq reflects a primitive form of
    consciousness which was the evolutionary platform for the emergence of more
    complex forms of consciousness\rq\rq{} (Panksepp, \textit{Affective consciousness:
    Core emotional feelings in animals and humans}, 2005). Here,
    \textit{consciousness} refers to \lq\lq brain states that have an
    experiential feel to them, and it is envisioned as a multi-tiered process
    that needs to be viewed in evolutionary terms, with multiple layers of
    emergence.\rq\rq{} (\textit{idem}).

    The evidence in favour of this thesis is close to overwhelming and, to my
    eyes, at least in essence, little doubt remains about its truth. I refer the
    reader primarily to the work of Panksepp and Damasio. There are, of couse,
    disputes concerning technical matters---the manner in which different ways
    of consciousness evolved, the neurobiology modulating certain aspects of
    consciousness\footnote{Compare, for example, the work of Damasio with that
    of Panksepp}---but the essential claim remains untouched. Not only is the
    experiential component of emotion common to, at least, all mammals, but it
    represents a primitive form of consciousness that is still present in the
    human brain. This presence is attested by the fact that the systems
    from which such form of consciousness emerges remain very well preserved
    and highly homologous across mammal species.

\end{document}
