% Options for packages loaded elsewhere
\PassOptionsToPackage{unicode}{hyperref}
\PassOptionsToPackage{hyphens}{url}
%
\documentclass[
]{article}
\usepackage{amsmath,amssymb}
\usepackage{lmodern}
\usepackage{iftex}
\ifPDFTeX
  \usepackage[T1]{fontenc}
  \usepackage[utf8]{inputenc}
  \usepackage{textcomp} % provide euro and other symbols
\else % if luatex or xetex
  \usepackage{unicode-math}
  \defaultfontfeatures{Scale=MatchLowercase}
  \defaultfontfeatures[\rmfamily]{Ligatures=TeX,Scale=1}
\fi
% Use upquote if available, for straight quotes in verbatim environments
\IfFileExists{upquote.sty}{\usepackage{upquote}}{}
\IfFileExists{microtype.sty}{% use microtype if available
  \usepackage[]{microtype}
  \UseMicrotypeSet[protrusion]{basicmath} % disable protrusion for tt fonts
}{}
\makeatletter
\@ifundefined{KOMAClassName}{% if non-KOMA class
  \IfFileExists{parskip.sty}{%
    \usepackage{parskip}
  }{% else
    \setlength{\parindent}{0pt}
    \setlength{\parskip}{6pt plus 2pt minus 1pt}}
}{% if KOMA class
  \KOMAoptions{parskip=half}}
\makeatother
\usepackage{xcolor}
\IfFileExists{xurl.sty}{\usepackage{xurl}}{} % add URL line breaks if available
\IfFileExists{bookmark.sty}{\usepackage{bookmark}}{\usepackage{hyperref}}
\hypersetup{
  pdftitle={Random graph generation},
  hidelinks,
  pdfcreator={LaTeX via pandoc}}
\urlstyle{same} % disable monospaced font for URLs
\setlength{\emergencystretch}{3em} % prevent overfull lines
\providecommand{\tightlist}{%
  \setlength{\itemsep}{0pt}\setlength{\parskip}{0pt}}
\setcounter{secnumdepth}{-\maxdimen} % remove section numbering
\ifLuaTeX
  \usepackage{selnolig}  % disable illegal ligatures
\fi

\usepackage{ebgaramond}

\title{Random graph generation}
\author{}
\date{}

\begin{document}
\maketitle

The generation of connected random graphs is non-trivial and important
to many applications. In particular, it is not easy to sample a
connected random graph from the space \(\mathcal{G}\) of all connected
graphs with uniformity; i.e. without a bias for graphs of a special
kind. This entry contains three algorithms for sampling random connected
graphs. The first two are simple and of reasonable complexity, but
biased. The third algorithm is less efficient, but it can be proven that
it samples graphs without bias.

\hypertarget{bottom-up-approach}{%
\subsection{Bottom-up approach}\label{bottom-up-approach}}

The bottom-up approach consists in generating a simple graph and
constructing a random graph from it. We shall use spanning trees to
(surprise) span graphs from them. Some definitions are in order.

\begin{quote}
\begin{itemize}
\item
  Let \(\mathcal{T}_n\) the set of all trees of \(n\) vertices,
  \(\mathcal{G}_n\) the set of all graphs with \(n\) vertices, and
  \(\mathcal{G}\_{n, m}\) the set of all graphs with \(n\) vertices and
  \(m\) edges. We shall assume the vertices of these graphs are labeled
  \(1, \ldots, n\).
\item
  For any \(T \in \mathcal{T}_n\), we define
  \(\mathcal{U}_T := \left\\{ G  \in \mathcal{G}_n : T \subseteq G \right\\}\)
  and refer to it as \emph{the universe} of \(T\). Thus, the universe of
  \(T \in \mathcal{T}_n\) is the set of graphs spanned by \(T\).
\item
  Let
  \(\Lambda(n) = \\{ \\{ x, y \\} : x, y \in \\{ 1, \ldots, n\\} \\}\),
  the set of all possible edges in a graph of \(n\) vertices.
\end{itemize}
\end{quote}

Before discussing any graph generation, I must review an important
concept relating to trees: the Prüfer sequence.

\hypertarget{pruxfcfer-sequence}{%
\subsubsection{Prüfer sequence}\label{pruxfcfer-sequence}}

The Prüfer sequence of \(T \in \mathcal{T}_n\) is a word over the
alphabet \(\Sigma = \\{0, \ldots, n - 1\\}\) of length \(n - 2\). Prüfer
proved that there is a bijection between \(\mathcal{T}_n\) and
\(\Sigma^{n-2}\) (which incidentally provided a nice proof of the fact
that there are \(n^{n-2}\) distinct trees of \(n\) vertices).

The algorithms hereby presented generate random trees by generating
random Prüfer sequences. It is easy to construct algorithmically the
tree corresponding to a Prüfer sequence, and thus a random tree is
obtained.

Let \(T_2\) denote the unique tree of two vertices. Let \(L\) be the
label-set, i.e.~the set of natural numbers which label the vertices of
our graphs. Then, a recursive algorithm for constructing the
corresponding to a Prüfer sequence \(p = p_1\ldots p_{n-2}\) over a
label set \(L\) is the following:

\[
\\begin{align*}
&\\textbf{ begin } \\textbf{f}(p, L) \\\\
&\\quad\\quad\\textbf{if } |p| = 0 \\textbf{ do} \\\\ 
&\\quad\\quad\\quad \\quad  \\textbf{return } T_2 \\text{ with labels in } L\\\\ 
&\\quad\\quad\\textbf{else} \\\\ 
&\\quad\\quad\\quad\\quad k := \\min_j \\{ j \\notin p\\} \\\\ 
&\\quad\\quad\\quad\\quad L' := L - \\{k\\}\\\\
&\\quad\\quad\\quad\\quad p' := p_2\\ldots p_{n-2} \\\\ 
&\\quad\\quad\\quad\\quad T := \\textbf{f}(p', L') \\\\ 
&\\quad\\quad\\quad\\quad V(T) := V(T) \\cup \\{k\\} \\\\ 
&\\quad\\quad\\quad\\quad E(T) := E(T) \\cup \\{\\{k, p_1\\}\\} \\\\ 
&\\quad\\quad\\quad\\quad \\textbf{return } T\\\\ 
&\\quad\\quad\\textbf{fi} \\\\ 
&\\textbf{end}
\\end{align*}
\]

Here is an illustration of the algorithm ran on \(p = 17375\),
\(L = \mathbb{N}_7\). The illustration is taken from
\href{https://www.cs.tufts.edu/comp/150GT/documents/Prufer\%20sequences\%20-\%20from\%20\%5B\%20Gross,\%20Yellen\%20\%5D\%20\%20Graph\%20Theory\%20and\%20Its\%20Applications,\%203e.pdf}{this
source}.

\hypertarget{random-graph-generation}{%
\subsubsection{Random graph generation}\label{random-graph-generation}}

Let \(T_w\) denote the graph corresponding to the Prüfer sequence \(w\).
Given a tree \(T\) we define a special family of edges:

\[ 
S_T = \\{e \in \Lambda(n) : e \notin E(T)\\} = \Lambda(n) - E(T)
\]

These are called the \emph{spanning edges}, since these are the edges
required to span connected graphs from the spanning tree \(T\). Observe
that \(\emptyset \in S_T\) and is the set of edges required to span
\(T\) out of \(T\).

Then, for any fixed \(n\), we let the language
\(\left\\{ 1, \ldots, n \right\\}^{n-2}\) be the index set of an indexed
family of functions \(\mathcal{F}\) defined as:

\[\begin{align*}
    \mathcal{F}(w) : \mathcal{U}_{T_w} &\to S_T  \\\\
    \mathcal{F}(w)(G) &= E(G) - E(T_w)\end{align*}\]

\begin{quote}
\textbf{\(\mathcal{F}(w)\) is a bijection.} Let \(S \in S_{T_w}\) for an
arbitrary \(T_w\). Define \(G = (V(T), S \cup E(T_w)\). Then
\(T_w \subseteq G\) and \(G\) belongs to the domain of
\(\mathcal{F}(w)\). \(E(T_w) \cap S_{T_w} = \emptyset\) by definition.
Then \(S = E(G) - E(T_w)\).

\(\therefore\) \(\mathcal{F}(w)\) is surjective.

Let \(G, G' \in \mathcal{U}_{T_w}\) for an arbitrary \(T_w\). Assume
\(\mathcal{F}(w)(G) = \mathcal{F}(w)(G')\). Assume \(G \neq G'\). Since
\(T_w\) spans both, all edges in \(T_w\) are in \(G, G'\) and their
difference must lie in an edge outside of \(T_w\). But all edges outside
of \(T_w\) are in \(\mathcal{F}(w)(G)\) and \(\mathcal{F}(w)(G')\)
respectively, which are the same. This is a contradiction.
\(\therefore\) \(G = G'\).

\(\therefore\) \(\mathcal{F}(w)\) is injective.

\(\therefore\) \(\mathcal{F}(w)\) is bijective.
\end{quote}

To summarize the relations we have established, we see that there is a
nice generative path

\[
\text{Prüfer sequence} \to \text{Tree} \to S \in S_T \to G \in \mathcal{U}_T
\]

which inspires effective procedures for random graph generation. The
most obvious procedure is the following. Given an desired number of
vertices \(n\),

\begin{quote}
\emph{(1)} Generate randomly \(p = p_1\ldots p_{n-2} \in \Sigma^{n-2}\).

\emph{(2)} Span the tree \(T = (V, E)\) of the Prüfer sequence \(p\).

\emph{(3)} Let
\(k \in_R \left\\{ 0, \ldots, \frac{ n(n-1) }{2} \right\\}\).

\emph{(4)} Let \(\ell_1, \ldots, \ell_k \in_R S_T\), all distinct.

\emph{(5)} Let \(E = E \cup \left\\{ \ell_1,\ldots, \ell_k \right\\}\)
\end{quote}

Because all trees of \(n\) vertices correspond to a sequence, all trees
can be sampled. And all connected graphs can be derived from the set of
all spanning trees. Then this procedure generates all graphs in
\(\mathcal{G}_n\).

The question is whether it is equally likely to generate any two graphs
in \(\mathcal{G}_n\). It is obvious that it is equally likely to
generate any tree. And the probability that a given graph is generated
depends entirely on the number of spanning trees it contains. Not all
graphs have the same number of spanning trees. \(\therefore\) It is more
likely to generate a graph with many spanning trees than a graph with
few spanning trees.

\begin{center}\rule{0.5\linewidth}{0.5pt}\end{center}

The second effective procedure extends the input from only \(n\), the
number of vertices, to \(m\), the number of edges. Thus, it specifies
the problem further into the question of how to generate more or less
dense graphs of \(n\) vertices. The effective procedure consists simply
in generating \(T_w\) and sampling \(\ell \in_R S_T\) repeatedly until
\(|E_T| = m\).

\[
\begin{align*}
    &\\textbf{Input: } n, m\\\\
    &T := (V, E) = \\textbf{genRandomTree}(n)\\\\
    &S_T := [\\Gamma^c(v_1), \\ldots, \\Gamma^c(v_n)]  \\\\
    &C := [ |S_T[1]|, \\ldots, |S_T|[n]|]  \\\\
    &V := [1, \\ldots, n]   \\\\
    &n' := n\\\\
    &\\textbf{while } |E(T)| < m \\textbf{ do } \\\\ 
    &\\qquad i := \\textbf{random}(1, n') \\\\ 
    &\\qquad v := V[i]\\\\
    &\\qquad \\textbf{if }  d(v) == n - 1  \\textbf{ do } \\\\ 
    &\\qquad \\qquad \\textbf{deleteAt}(V, i)\\\\ 
    &\\qquad\\qquad n' := n' - 1 \\\\ 
    &\\qquad\\textbf{else } \\\\ 
    &\\qquad\\qquad j = \\textbf{random}(1, C[v])\\\\
    &\\qquad\\qquad w := S_T[v][j] \\\\ 
    &\\qquad\\qquad E(T) := E(T) \\cup  \\left\\{ v, w \\right\\} \\\\
    &\\qquad\\qquad C[v] := C[v] - 1 \\\\ 
    &\\qquad\\qquad \\textbf{deleteAt}(S_T[v], j)  \\\\ 
    &\\qquad\\qquad\\textbf{deleteElement}(S_T[w], v)\\\\ 
    &\\qquad\\textbf{fi}\\\\
    &\\textbf{od}\\\\
    &\\textbf{return }
\end{align*}
\]

Generating a tree from a random Prüfer sequence is \(O(n^2)\). Forming
\(S_T\) is also \(O(n^2)\). Within the while loop there are simply index
manipulations, so the complexity of the loop is
\(\varphi \times O(1) = \varphi\), with \(\varphi\) the complexity of
the number of iterations.

All iterations add an edge except those where a saturated vertex is
chosen. A saturated vertex may be chosen at most once. \(\therefore\)
There are \(O(n)\) iterations where a saturated vertex is chosen. Since
the rest of the iterations add an edge, their number is fixed:
\(m - (n-1)\), where \((n-1)\) is the number of edges in the spanned
tree. \(\therefore\) There are exactly \(m -n + 1\) edge-adding
iterations.

\(\therefore \varphi = O(n) + O(m - n + 1) = O(n) + O(m) = O(m)\)

\(\therefore\) The complexity of the algorithm is
\(O(n^2) + O(m) = O(n^2)\).

I showcase here random graphs with their source random trees as
generated by this algorithm. The left-most graph is the random tree
which spanned the right-most graph. The algorithm was implemented in C
but the generated graphs were plotted using the \texttt{networkx} Python
package.

\hypertarget{top-down-approach}{%
\subsection{Top-down approach}\label{top-down-approach}}

A top-down approach analogous to the previous algorithms would in
principle consist in the generation of a \(K_n\) and the random removal
of edges in the graph. This involves some extra complexity: while adding
edges to a connected graph cannot disconnect it, removing edges from a
graph may do so. Thus, on each iteration, some algorithmic process
should determine if an edge can be deleted or not without violating the
connectivity invariant.

Thus, an effective procedure goes as follows. Given \(n\) and a desired
number of edges \(m\),

\begin{quote}
\emph{(1)} Generate \(K_n = (V, E)\).

\begin{enumerate}
\def\labelenumi{(\arabic{enumi})}
\setcounter{enumi}{1}
\tightlist
\item
  Let \(V_c = V\) the list of vertices which can be pruned.
\end{enumerate}

\emph{(3)} Let \(v \in_R V_c\), \(w \in_R \Gamma(v)\).

\begin{enumerate}
\def\labelenumi{(\arabic{enumi})}
\setcounter{enumi}{3}
\item
  Check if removing \(\\{v, w\\}\) preserves connectivity; if so, remove
  it from \(E\).
\item
  Define \(\Gamma(v) = \Gamma(v) - \\{w\\}\),
  \(\Gamma(w) = \Gamma(w) - \\{v\\}\), regardless of whether the edge
  was removed.
\item
  If \(d(z) = 1\) or \(\Gamma(z) = \emptyset\), remove \(z\) from
  \(V_c\), where \(z \in \\{v, w\\}\).
\end{enumerate}

\emph{(7)} If \(|E| \neq m\), go to (3).
\end{quote}

Assume \(\textbf{BFSFind}(E, x, y)\) is an algorithm that checks if
there is a path from \(x, y\) in a given edge set. Then a pseudo-code
implementation of the effective procedure above is:

\[
\begin{align*}
&(V, E) := \\textbf{genKn}(n)\\\\
&\\Gamma := [\\Gamma(v_1), \\ldots, \\Gamma(v_n)]\\\\
&V_c := [v_1, \\ldots, v_n]\\\\ 
&\\textbf{while } |E| \\neq m \\textbf{do } \\\\ 
&\\quad\\quad v = \\textbf{randFrom}(V_c) \\\\ 
&\\quad\\quad w = \\textbf{randomFrom}(\\Gamma[v]) \\\\ 
&\\quad\\quad \\Gamma[v] = \\Gamma[v] - \\{w\\} \\\\ 
&\\quad\\quad \\Gamma[w] = \\Gamma[w] - \\{v\\} \\\\ 
&\\quad\\quad \\textbf{if BFSFind}(E - \\{v, w\\}, v, w) \\textbf{ do}  \\\\ 
&\\quad\\quad\\quad\\quad E := E - \\{v, w\\}\\\\
&\\quad\\quad \\textbf{fi}  \\\\ 
&\\quad\\quad \\textbf{if } d(v) = 1 \\lor \\Gamma[v] = \\emptyset \\textbf{ do } \\\\
&\\quad\\quad\\quad\\quad V_c := V_c - \\{v\\} \\\\ 
&\\quad\\quad\\textbf{fi}  \\\\ 
&\\quad\\quad \\textbf{if } d(w) = 1 \\lor \\Gamma[w] = \\emptyset \\textbf{ do } \\\\
&\\quad\\quad\\quad\\quad V_c := V_c - \\{w\\} \\\\ 
&\\quad\\quad\\textbf{fi}  \\\\ 
&\\textbf{od}
\end{align*}
\]

Generating a \(K_n\) is \(O(n^2)\). The \(\textbf{while}\) selects a
random vertex from \(V_c\) and attempts to erase one of its edges. There
is only one case one an edge is not removed; namely, when the sampled
edge is a bridge. This happens at most once per bridge. There are at
most \(n - 1\) bridges in a graph. \(\therefore\) There are \(O(n)\)
iterations which do not remove an edge.

The remaining iterations will remove an edge and there will be exactly
\(\frac{n(n-1)}{2} - m\) of them.

\(\therefore\) There are \(O(n) + O(\frac{n(n-1)}{2} - m) = O(n^2 - m)\)
iterations.

The operations in each iteration are \(O(1)\) except BFS. The complexity
of BFS grows linearly with the number of edges. \(\therefore\) BFS is
\(O(n^2)\).

\(\therefore\) The algorithm is
\(O(n^2) + O(n^2 - m)O(n^2) = O(n^4 - n^2m)\).

In practice the algorithm will perform better than this. BFS is stops
whenever \(w\) is find starting from \(v\). This still is asymptotically
\(O(|E|)\), but in practice the bound will seldom be reached.
Furthermore, BFS is ran on increasingly sparser graphs. Its asymptotic
complexity is given by the number of edges in the initial \(K_n\), but
it decreases with each edge-removing iteration.

Below, I display a \(K_{100}\) and the random tree generated from it.

To prove that our algorithm is correct and unbiased requires more
formalization. Let \(\mathcal{C}\_{n,m} \subset \mathcal{G}\_{n,m}\) be
the set of connected graphs of \(n\) vertices, \(m\) edges. Let
\(\mathbb{C}(n,m) = |\mathcal{C}\_{n,m}|\). Let \(\mathcal{E}\_{n, m}\)
be the class of edges which, if removed from a \(K_n\), produce a graph
in \(\mathcal{C}\_{n, m}\). Since any \(G \in \mathcal{C}\_{n, m}\)
corresponds univocally to a set of edges \(W \in \mathcal{E}\_{n,m}\),

\[
|\mathcal{E}_{n, m}| = \mathbb{G}(n, m)
\]

and there is a bijection
\(f\_{n,m} : \mathcal{E}\_{n,m} \mapsto \mathcal{C}_{n, m}\) mapping a
set of edges to the connected graph generated by removing those edges
from \(K_n\).

We shall prove that \emph{(1)} our algorithm effectively constructs a
\(W \in \mathcal{E}\_{n,m}\) and computes \(f(W)\) and \emph{(2)} that
any \(W \in \mathcal{E}\_{n,m}\) has an equal probability of being
constructed.

\emph{(1)} The algorithm iteratively removes edges ensuring that the
connectivity invariant is preserved. It is trivial to see that it
removes \(k := \binom{n}{2} - m\) edges. Let
\(S = \\{e_1, \ldots, e_{k}\\}\) be the set of randomly sampled edges,
where \(e_i\) was sampled at the \(i\)th edge-removing iteration.

Preserving the invariant entails that, across iterations, the sampling
spaces \(E_1, \ldots, E_r\) from which edges to remove are sampled are:

\begin{itemize}
\tightlist
\item
  \(E_1 = \\{ e \in W : W \in \mathcal{E}\_{n,m} \\}\)
\item
  \(E_2 = \\{ e \in W : W \in \mathcal{E}\_{n,m} \land \\{ e_1 \\} \subseteq W \\}\)
\item
  \(E_3 = \\{ e \in W : W \in \mathcal{E}\_{n,m} \land \\{ e_1, e_2 \\} \subseteq W \\}\)
\end{itemize}

etc. Thus, the general form is \$E\_i = \textbackslash\{ e \in W : W
\in \mathcal{E}\_\{n,m\} \land \textbackslash\{ e\_1, \ldots,
e\_\{i-1\}\textbackslash\} \subseteq W \textbackslash\} \$.

It follows that \(S = \\{ e_1, \ldots, e_k \\} \subseteq W\) for some
\(W \in \mathcal{E}\_{n,m}\). But \(|S| = |W| = k\). Then \(S = W\) and
\(S \in \mathcal{E}\_{n,m}\). And since \(S\) is the set of removed
edges, the algorithm computes \(f(S)\).

\emph{(2)} Since there is a bijection between \(\mathcal{C}\_{n,m}\) and
\(\mathcal{E}\_{n,m}\), a graph is more probable than others if and only
if there is a set \(S \in \mathcal{E}\_{n,m}\) that is more probably
constructed than others. This could only be true for two cases:
\emph{(1)} An edge or set of edges in \(S\) is more likely to be chosen,
or \emph{(2)} \(S\) contains more elements than other members of
\(\mathcal{E}\_{n,m}\). But \emph{(1)} is impossible if the selection is
random, and \emph{(2)} contradicts that \(|S| = \binom{n}{2} - m\) for
every \(S \in \mathcal{E}\_{n,m}\).

\(\therefore\) The algorithm is correct and unbiased.

\begin{center}\rule{0.5\linewidth}{0.5pt}\end{center}

As an appendix, let us derive \(\mathbb{C}(n,m)\). It is straightforward
to reason that

\[
\begin{align*}
    |\mathcal{G}_{n,m}| = \binom{\frac{n(n-1)}{2}}{m}
\end{align*}
\]

This inspires the definition of a generating function for the class
\(\mathcal{G}_{n, m}\) of graphs of \(n\) vertices, \(m\) edges:

\[
\begin{align*}
    A(z, u) &= \sum_{k=0}^{\infty}\left(\sum_{r = 0}^{\infty} \binom{\frac{k(k-1)}{2}}{r}  u^r\right) \frac{z^k}{k!}\\\\
                                                               &=\sum_{k=0}^{\infty}(1 + u)^{\frac{k(n-1)}{2}} \frac{z^k}{k!} \\\\
                                                               &= 1 + \sum_{k=1}^{\infty} (1+u)^{k(k-1)/2} \frac{z^k}{k!}
\end{align*}
\]

An important observation is that any \(G \in \mathcal{G}\_{n,m}\) is a
set of elements \(G_1, \ldots, G_r \in \mathcal{C}\_{n, m}\).
Informally, any graph is a set of connected graphs. Since the
relationship of the class \(\mathcal{G}\_{n, m}\) and the class
\(\mathcal{C}\_{n, m}\) is the set-of relation, the generating function
\(C(z, u)\) of \(\mathcal{C}\_{n,m}\) is satisfies

\[A(x) = e^{C(z, u)}    \]

Then

\[
\begin{align*}
    C(z, u) &= \ln \left[1 + \sum_{k=1}^{\infty} (1+u)^{k(k-1)/2} \frac{z^k}{k!}\right] \\\\
    &= \sum_{q=1}^{\infty} \frac{ (-1)^{q+1} }{q} \left[ \sum_{k=1}^{\infty}\left( 1+u \right)^{k (k-1) / 2} \frac{z^k}{k!}  \right]^{q} 
\end{align*}
\]

Thus, \(C(z, u)\) is the generating function of the composite sequence
\(\\{\\{|\mathcal{C}\_{n, m}|\\}\_{n\geq 1}\\}\_{m\geq 0}\). The
resulting expression for the number of connected graphs of \(n\)
vertices, \(m\) edges is:

\[
\begin{align*}
    \mathbb{C}(n, m) = n! ~ [ z^n ][ u^m ] \sum_{q=1}^{\infty} \frac{ (-1)^{q+1} }{q} \left[ \sum_{k=1}^{\infty}\left( 1+u \right)^{k (k-1) / 2} \frac{z^k}{k!}  \right]^{q} 
\end{align*}
\]

where \([z^n][u^m]C(z, u)\) are the \(n\)th and \(m\)th coefficients of
the generating functions for the polynomials dependent on \(z\) and
\(u\), respectively.

\begin{quote}
\textbf{Example}. For \(m = n - 1\) (i.e.~connected trees) across
\(n = 2, 3, \ldots\), \(\mathbb{C}\) effectively produces the sequence

\[    1, 1, 3, 16, 125, 1296, \ldots  = \{n^{n-2}\}_{n\geq2 }\]

which we know is correct because there are \(n^{n-2}\) Prufer sequences
of length \(n\).
\end{quote}

\end{document}
