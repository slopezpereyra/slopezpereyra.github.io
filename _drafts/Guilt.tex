\documentclass[a4paper, 12pt]{article}

\usepackage[utf8]{inputenc}
\usepackage[T1]{fontenc}
\usepackage{textcomp}
\usepackage{amssymb}
\usepackage{newtxtext} \usepackage{newtxmath}
\usepackage{amsmath, amssymb}
\newtheorem{problem}{Problem}
\newtheorem{example}{Example}
\newtheorem{lemma}{Lemma}
\newtheorem{theorem}{Theorem}
\newtheorem{problem}{Problem}
\newtheorem{example}{Example} \newtheorem{definition}{Definition}
\newtheorem{lemma}{Lemma}
\newtheorem{theorem}{Theorem}
\usepackage{ebgaramond}

\begin{document}

Of all negative sentiments, guilt is perhaps the only one to be actively
encouraged by customary education. Thousands of children everyday are forced to
beat beat their chests and whisper:

\begin{quote}
    through my fault, through my fault, \linebreak
    through my most grievous fault;
\end{quote}

César Vallejo, in \textit{Los heraldos negros}, describes the man riddled by
tragedy as one who's eyes betray his entire life as \textit{un charco de culpa}
(a puddle of guilt). Everybody knows of Judas' faith, who cried: \textit{I have
sinned in that I have betrayed the innocent blood}, and hanged himself. Dante,
shortly after entering the fourth circle, asks horrified: \textit{Why do we let
our guilt consume us so?} And everybody knows the story of Oedipus, which
psychoanalysis popularized, who realizing the man he had killed was his father,
and the woman he married his mother—who hanged herself—blinds himself out
of remorse.

Yet, though guilt is familiar to everyone, it is
intriguing to see how varied are its causes among people. I personally make an
interest of the matters a person feels guilty for, as they are often quite
revealing of their personality.

For instance, I once knew a man who carried a truly sinister weight, having
destroyed his family through acts of virulent hate. When I asked what he felt
guilty for, he gravely said a single thing still haunted him to that day: That
he wasn't all that nice to certain girlfriend of his youth. In one of Guevara's
diaries, he tells that a little dog which accompanied his troops was about to
get them killed: in a sudden burst of barking, it was giving away their
position. He and another soldier had to silently asphyxiate the poor creature,
which apparently caused him such sorrow that he could never forget. 

Bertrand Russell, in \textit{The conquest of happiness}, makes a long and
well-argued case against guilt. He points out, indisputably to me, that guilt
induces no positive effect in the world. The victim of a wrongdoing is in no
way benefited if the author of it feels guilty, nor the latter incurs in any
reparation by drowning in remorse. We should strive to amend the mistakes we
can amend and forget those we cannot. It is evident that a world where this
ethics is practiced is happier and more harmonious than its alternative.

If someone has ever hurt me, or treated me unjustly, I sincerely wish that he
and I forget about it. If the evil is too serious, for there are things one
cannot forget, what I seek is restoration—and no amount of guilty feelings can
have a restorative effect. Inversely, though one should in general strive for
kindness, it is inevitable that we shall be unfair or wicked to some people. As
long as our wrongdoings aren't so great that even a compassionate soul may be
incapable of forgetting them, we can forget them ourselves if amendment isn't
possible. If that limit is crossed, I have nothing to advice, and perhaps there
is nothing to be said but to recall Plauto's words: \textit{Nihil est miserius
quam animus hominis conscius.} [Nothing is more miserable than the soul of a
man conscious of guilt.]








\end{document}



