\documentclass[a4paper, 12pt]{article}

\usepackage[utf8]{inputenc}
\usepackage[T1]{fontenc}
\usepackage{textcomp}
\usepackage{amssymb}
\usepackage{newtxtext} \usepackage{newtxmath}
\usepackage{amsmath, amssymb}
\newtheorem{problem}{Problem}
\newtheorem{example}{Example}
\newtheorem{lemma}{Lemma}
\newtheorem{theorem}{Theorem}
\newtheorem{problem}{Problem}
\newtheorem{example}{Example} \newtheorem{definition}{Definition}
\newtheorem{lemma}{Lemma}
\newtheorem{theorem}{Theorem}


\begin{document}



What makes a man dangerous is not the strength of his arms, but the
content of his heart. The weakest men, resented or enraged, will hurt
a thousand more than any gentle giant. Similarly, what makes beautiful
a person is not the power, but the purity of their thoughts. 

Gregariousness is the among the most beautiful of virtues. The world is filled
with kind and beautiful people, some as simple and immediate as a flower, some
as complex and mysterious as an abyss or the moon. With the exception of
serious flaws---for violence and wickedness are real---one need only practice a
radical acceptance of the other for their virtues to become almost instantly
apparent, and to learn how very much alike we all are. (If one is not willing
nor capable of practicing such acceptance, one has very little right to any
claim of justice, for the world is constantly accepting our many flaws.)

For instance, a few days ago I spent a calm and sweet afternoon with a girl who
(so I believed) shared very little with me. Whilst my childhood was stable and
comfortable, she was the daughter of an adolescent mother and a criminal father
who ended up going to jail. The city I was born in is far from being a great
city, but it is the capital of my province---she had been born in the deepest
interior of a poor province, which (on top of it all) she fled to escape her
father, winding up in the austral barrens of Usuahia. We could not talk about
politics---she wasn't interested---nor books---she wasn't learned---nor any of
the things which I'd been taught, in my childhood, made a person worth
conversing with. 

All of this, to my fortune, left only place to talk about the content of
our hearts: of the struggles and joys which decorated the landscapes of
our minds. In that conversation, which perhaps was just a whisper, I saw
the clearest sparks of human kindness. I learned that we were ongoing
rather similar psychological processes; that, in all our differences, the
sufferings which plagued our families (and our minds) were not as
different as first glances gave away; that it wasn't an interest nor
a passion what was bringing us together, but something else indescribable and
true. And in her kiss I felt we were unsullied, freed from the hindrances which
set us apart from others.


I see it now: Aloofness is not a crown. If there is any such thing as wisdom in
this world, it must be this: that, at least \textit{in potentia}, a perfect
form of kindness resides in every living thing. If one assumes this is true
(and I believe one must, even if only to find a reason a live), a simple
imperative arises: that one ought to strive to find it and, if possible, to
nurture it in oneself and others. It is not a revolution, but a world under
such ethic is superior to its alternative.



\end{document}



