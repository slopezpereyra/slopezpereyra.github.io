\documentclass[a4paper, 12pt]{article}

\usepackage[utf8]{inputenc}
\usepackage[T1]{fontenc}
\usepackage{textcomp}
\usepackage{amssymb}
\usepackage{newtxtext} \usepackage{newtxmath}
\usepackage{amsmath, amssymb}
\newtheorem{problem}{Problem}
\newtheorem{example}{Example}
\newtheorem{lemma}{Lemma}
\newtheorem{theorem}{Theorem}
\newtheorem{problem}{Problem}
\newtheorem{example}{Example} \newtheorem{definition}{Definition}
\newtheorem{lemma}{Lemma}
\newtheorem{theorem}{Theorem}
\DeclareMathAlphabet{\mathcal}{OMS}{cmsy}{m}{n}

\begin{document}

\title{Against social media}
\maketitle

Among the many conventional practices of which a modern person may participate,
the use of social media presents two characteristics which are relatively
peculiar. The first is that people are typically not uninclined to question it.
The second is that, though certainly embraced by most, it is also typically
recognized \textit{at least} as problematic.

It would be rushed to deduce from this that convincing people to abandon social
media is easier than, say, convincing them that free love is preferable to
monogamy. Contrarily, the fact that people usually recognize the evils of social
media usage and yet fail to abstain from it indicates how deeply pervaded and
needful of it they have become. In this, and in many other ways, the use of
social media resembles more the epidemic emergence of an addictive substance
than merely a conventional practice. 

Here, I do not understand social media as an abstraction or as something that is
\textit{in principle} pernicious. I am rather speaking of social media
networks as designed and implemented by mega-corporations such as Meta and X
Corp. The social media networks owned by these companies compress \textit{by
far} the great majority of social media usage, so it is a useful simplification
to speak of their use with the more general phrase \textit{social media usage}.

To begin with, social media networks possess all the typical evils of technology
as delivered by Silicon Valley's mega-corporations. Most of these are public
knowledge and [many descriptions](https://www.jstor.org/stable/26609097) of them
exist. These technologies not only foster tyranny but are tyrannical themselves.
A person scrolling through social media is constantly subjected to stimuli it
has never chosen, needed, and perhaps even wanted. What is more, it is not only
turned into a product, but also into an experimental subject. Every interaction
with the platform is used as information to more precisely exploit their
weaknesses, sophisticating the mechanisms designed to capture their attention.
This is what is euphemistically called «personalizing user experience».

Changes to the network's settings are designed to provide the illusion of
control, and it is hard not to imagine, when considering their range of
possibilities, a slave which is free to choose the color of its clothes or a
prisoner free to change its cell's wallpaper patterns. The reason why any sort
of control is illusory is because these networks consist \textit{by design} in
platforms of surrendering personal information. This means that using them in a
way that discloses no personal data defeats their very purpose and is thus not
likely to occur. Herein lies the irony: though data is sometimes stolen without
permission, this, in most cases, is unnecessarily violent. It suffices to
\textit{seduce} people into giving away their information.

The means of this seduction are the eternal flaws and tragedies of human nature.
Vanity, a desire to be seen, loneliness, greed, the longing for a sense of
community: all these are exploited so as to convince people to surrender
themselves into a distant, almost absolutely abstract master. In return for your
public and, more often than not, also private life—in exchange for accepting
to be bombarded by targeted advertising—in appreciation for displaying each and
every one of your softest spots, so that they may be used against you, you are
given spiritual pennies. The illusion of connection, stupid and ephemeral
content so that you may forget yourself for a while, and the miserable joy of
being seen. This is the Promethean exchange which the article I cited above
describes, where you get «gifts of enlightenment and ease in exchange for some
measure of awe, gratitude, and deference to the technocratic elite that
manufactures them».

Any structure of power which bases itself in tyranny, and whose subsistence
depends on fostering the lesser side of human nature, is repugnant and vile.
Instead of widening the horizons of human expression, social media networks 
narrow our points of view, radicalize us into smaller and smaller communities of
like-minded individuals, and make us forget the human touch. Could better social
networks exist? Are these in principle flawed?

As is usually case with technology, social networks when considered generally
are morally neutral. It is their capitalist design what perverts them. There are
social media networks that are $(a)$ not centrally owned by any private or state
entity and $(b)$ which does not subject their users to any form of tyranny. It
goes without saying that the satisfaction of $(b)$ requires the satisfaction of
all the conditions that make a software free (as in \textit{libre}, not as in
\textit{gratis}).
[Mastodon](https://en.wikipedia.org/wiki/Mastodon_(social_network)) 

is a famous
such type of network. Security concerns have been raised, since decentralization
induces particular problems, but speaking broadly these concerns may be
compensated by the fact that, by design, these networks do not require their
users to provide personal information.

As is typically the case, these decentralized endeavors lack the features
provided by technological giants, whose extraordinary economical power allows
them to surpass all competition in what comes to the «quality» of their
products. This of course assumes that «quality» is unconcerned with freedom and
ethics, which is not indisputable. Whatever the case, in general, people do not
care enough so as to endure great discomfort so as to utilize a purely free
software. I include myself in this description: like everyone else, and though
I've tried to minimize it, my computer runs a great deal of proprietary
programs.
















\end{document}



