\documentclass[a4paper, 12pt]{article}

\usepackage[utf8]{inputenc}
\usepackage[T1]{fontenc}
\usepackage{textcomp}
\usepackage{amssymb}
\usepackage{newtxtext} \usepackage{newtxmath}
\usepackage{amsmath, amssymb}
\newtheorem{problem}{Problem}
\newtheorem{example}{Example}
\newtheorem{lemma}{Lemma}
\newtheorem{theorem}{Theorem}
\newtheorem{problem}{Problem}
\newtheorem{example}{Example} \newtheorem{definition}{Definition}
\newtheorem{lemma}{Lemma}
\newtheorem{theorem}{Theorem}
\DeclareMathAlphabet{\mathcal}{OMS}{cmsy}{m}{n}
\usepackage{parskip}

\begin{document}

\title{Against social media}
\maketitle

Among modern conventional practices, the use of social media presents two
characteristics which, taken together, are somewhat peculiar. The first is that
people are typically not uninclined to question it, with many even recognizing
its perniciousness. The second is that, despite being widely recognized at least
as problematic, people find it almost unthinkable to abstain from it, and even
consider people who do so to be eccentric at best, perhaps even suspicious.

Mainstream social media platforms are frequently challenged. To take a single
example, TikTok is under constant scrutiny by government and human rights
organizations. The European Comission found it to have an addictive design and
promote compulsive behavior [1], and the Algorithmic Transparency Institute, in
collaboration with Amnesty International's digital forensics team, showed how
TikTok amplifies harmful and even suicidal content to adolescent users [2].
Similar formal investigations have been carried out on all major social media
platforms; and beyond official inquiries, public opinion is in general acceptant
of the fact that these networks are profoundly problematic.

It would be rushed to deduce from this that convincing people to abandon social
media is easier than, say, convincing them that free love is preferable to
monogamy. Contrarily, the fact that people usually recognize the evils of social
media usage and yet fail to abstain from it indicates how deeply pervaded and
needful of it they have become. In this, and in many other ways, the use of
social media resembles more the epidemic emergence of an addictive substance
than merely a conventional practice.

Here, I do not understand social media as an abstraction or as something that is
*in principle* pernicious. I am rather speaking of social media
networks as designed and implemented by mega-corporations such as Meta and X
Corp. The social media networks owned by these companies compress *by
far* the great majority of social media usage, so it is a useful simplification
to speak of their use with the more general phrase *social media usage*.

Social media networks possess all the typical evils of technology as delivered
by Silicon Valley's mega-corporations. Most of these are public knowledge and
[many descriptions](https://www.jstor.org/stable/26609097) of them exist.
These technologies not only foster tyranny but are tyrannical themselves. A
person scrolling through social media is constantly subjected to stimuli they
never chose, needed, and perhaps even wanted. What is more, people are not only
turned into a product in the form ad consumers, but also into an object of
constant surveillance. Every interaction with the platform is used as
information to more precisely exploit their softer underbelly, sophisticating
the mechanisms designed to keep their attention captive. This is what is
euphemistically called «personalizing user experience».

Furthermore, these network's are designed to provide the illusion of control. A
certain degree of customization is allowed, and some privacy settings exist.
However, (a) the properietary nature of these networks makes it impossible to
verify how privacy is in fact protected, and (b) these settings are typically
limited—as famously shown be Facebook's policy of keeping data even from deleted
accounts. Here, the user is put in a position analogous to that of a slave which
is free to choose the color of his clothes or a prisoner free to change his
cell's wallpaper patterns.

It should be clear that any form of user control *must* be illusory,
because these networks consist *by design* in platforms of surrendering
one's identity to public and private scrutiny. This means that any use of them
which discloses no personal data defeats its very purpose and is thus not likely
to occur. The strategy of these networks is henceforth to avoid unnecessary
violence: instead of stealing data, they aim at *seducing* people into
giving away their information, an art which they have mastered to perfection.

The means of this seduction are the eternal flaws and tragedies of us all.
Vanity, a desire to be seen, loneliness, greed, the longing for a sense of
community or belonging: all these are exploited so as to convince people to surrender
themselves into a distant, almost absolutely abstract master. In return for your
public and, more often than not, also private life—in exchange for accepting
to be bombarded by targeted advertising—in appreciation for displaying each and
every one of your softest spots, so that they may be used against you, you are
given spiritual pennies: the illusion of connection, stupid and ephemeral
content so that you may forget yourself for a while, and the miserable joy of
being seen. This is the Promethean interchange upon which social networks are
founded: you get «gifts of enlightenment and ease in exchange for some measure
of awe, gratitude, and deference to the technocratic elite that manufactures
them».

Any structure of power which bases itself in tyranny, and whose subsistence
depends on fostering the lesser side of human condition, is repugnant and vile.
Instead of widening the horizons of human expression, social media networks
narrow our points of view, radicalize us into smaller and smaller communities of
like-minded individuals, and make us forget the human touch.

Could better social networks exist? As is usually case with technology, social
networks when considered generally are morally neutral. It is their capitalist
design what perverts them. There are social media networks that are (a) not
centrally owned by any private or state entity and (b) which do not subject
their users to any form of control or surveillance. It goes without saying that the satisfaction
of (b) requires the satisfaction of all the conditions that make a software
free (as in *libre*, not as in *gratis*).
[Mastodon](https://en.wikipedia.org/wiki/Mastodon_(social_network)) is a famous
such type of network. Are these solutions to the issues raised above?

The answer is yes and no. On one hand, decentralized and free social networks
are not tyrannical by design, and thus do not possess the inherent evils of
networks such as X or Instagram. However, they do not elude the
problems of lack of substance, induction of compulsive behavior, and
superficiality. In comparison with tyranny and surveillance, these issues are
trivial and thus less worthy of attention. But it must still be asked:
What purpose does a social network account serve, even assuming it doesn't
violate your freedom?

Most people, I think dishonestly, answer that social networks serve the purpose
of keeping them informed. We pay little attention to how profoundly sad this
statement is. The hidden assumption of such statement is that we are too lazy or
unconcerned to actively seek information ourselves, perhaps through more
substantial means. On the contrary, we prefer to be spoon-fed with offensively
small, simplified chunks of information, which may certainly contain a link to
more substantial articles, but which in all honesty we are unlikely to read.
Furthermore, the propaganda model (as described by Chomsky and Herman) is
essentially unaffected by social networks, insofar as these are platforms for
news outlets to disseminate their content, without the networks producing news
themselves. So it should be needless to explain, at least to those familiar with
the model, how profoundly corrupt the attitude of informing oneself through
social media is.

In any case, I do not think people seriously believe that social media is a
source of information. I rather suppose that this is a more or less socially
acceptable justification for using them. The real reasons, I believe, are more
superficial. They draw our minds away from real life. We all seem to have a
profound desire, perhaps even a need, to be *distracted*. Regardless of
the degree to which this urge is felt, it should be clear that this urge must be
resisted. Furthermore, I would dare to venture that resisting it becomes easier
and easier the less one indulges in it, and vice versa.

On the other hand, I have notihing against leisure and distraction per se. What
is more, I believe most people should have even more spare, free time than they
are allowed to. As Jung put it, we are a question addressed to the world, and we
are to produce an answer lest the world produce one for us. Our spare time is
the opportunity we have to connect with the creative, enriching facets of our
own personal life, as well as the time we have to connect with each other. Thus,
it is clear that the problem is not that we wish to be distracted, but where do we direct our distraction. Since social media networks are by designed addictive
and inducing of compulsive behavior, it is no surprise that most people find
waste ther lesiure scrolling. They should feel every such minute as a
*theft*, for no instant of our life ever returns.

We underestimate how profoundly promising boredom is. The opportunity to be
bored is indeed a privilege. It is when we are bored that we feel inclined to
seek more rewarding experiences from life. The compulsion induced by social
networks of course exploits this natural inclination by providing us a rapid,
albeit shallow satisfaction. It is when I am bored that I decide to read, that I
decide to learn a music sheet, that I decide to write, that I decide to spend
time with a lover or a friend. Assuming a decent
standard of living, life is never short of opportunities to explore creative,
intellectual or affective satisfaction. Boredom is the catalyst that triggers
our pursuit of them.

Yet less and less we are able to endure boredom, and the quality of our lives
degrades in proportion. The shallow, rapid satisfaction which social networks
provide, in depriving us from the opportunity to learn, to explore and to
grow—in robbing us of our spare time, I mean—produces intellectual and moral
stagnation. I do not claim it makes us more stupid—though an argument can be
made—but at least it leaves us equally ignorant as before and hinders our
potential. The use of social media is perhaps not a root cause of superficiality
—be it intellectual or moral—but it certainly is a factor that not only makes it
more permanent, but also tinges it with a rather distinctive tone.

What is there to be done? The answer is so simple that I am amazed it's not
generally followed. We should abstain from using social media in any significant
way. This can be understood as a boycott, but I view it more generally as simply
an abstention to partake in a practice that is tyrannical at worst, superficial
and stupefying at best. We must only come to a simple, straightforward
realization: we do not need social media. At least assuming you don't depend on
them professionally, there is no need they satisfy which cannot be satisfied in
a more healthy, meaningful and enriching way.



[1] https://ec.europa.eu/commission/presscorner/detail/en/ip_26_312
[2] https://www.amnesty.org/en/latest/news/2023/11/tiktok-risks-pushing-children-towards-harmful-content/





\end{document}



