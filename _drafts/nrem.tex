\documentclass[a4paper, 12pt]{article}

\usepackage[utf8]{inputenc}
\usepackage[T1]{fontenc}
\usepackage{textcomp}
\usepackage{amssymb}
\usepackage{newtxtext} \usepackage{newtxmath}
\usepackage{amsmath, amssymb}
\newtheorem{problem}{Problem}
\newtheorem{example}{Example}
\newtheorem{lemma}{Lemma}
\newtheorem{theorem}{Theorem}
\newtheorem{problem}{Problem}
\newtheorem{example}{Example} \newtheorem{definition}{Definition}
\newtheorem{lemma}{Lemma}
\newtheorem{theorem}{Theorem}
\DeclareMathAlphabet{\mathcal}{OMS}{cmsy}{m}{n}

\begin{document}


\section{Formal Language Theoretic Definition of NREM Periods}

We give a rigorous formulation of the notion of an NREM period using
formal language theory. The goal is to encode sequences of sleep stages
as words over a finite alphabet and define, precisely and unambiguously,
which substrings correspond to valid NREM periods.

\subsection{Alphabet and Basic Definitions}

Let the finite alphabet
\[
\Sigma = \{1,2,3,4,5,6\}
\]
represent sleep stages, where:

\begin{itemize}
\item $\mathcal{N} = \{2,3,4\}$ denotes NREM stages,
\item $R = \{5\}$ denotes REM sleep,
\item $W = \{6\}$ denotes wakefulness,
\item $O = \Sigma \setminus \mathcal{N}$ denotes non--NREM stages.
\end{itemize}

A full night of sleep is represented by a word
\[
x = x_1 x_2 \dots x_n \in \Sigma^*.
\]

We assume each symbol represents one epoch of fixed duration
(e.g. 30 seconds). Let

\[
m \in \mathbb{N}
\quad\text{and}\quad
n \in \mathbb{N}
\]

denote thresholds corresponding to:

\begin{itemize}
\item $m$: the minimum number of consecutive epochs defining
a terminating sequence (e.g. $m=10$ epochs for $5$ minutes),
\item $n$: the minimum total number of NREM epochs required
for a valid NREM period (e.g. $n=30$ epochs for $15$ minutes).
\end{itemize}

\subsection{Runs and Substrings}

Given a word $x \in \Sigma^*$, a substring
\[
y = x_i x_{i+1} \dots x_j
\]
is called a \emph{run} of a set $A \subseteq \Sigma$
if $y \in A^+$ and either:

\begin{itemize}
\item $i=1$ or $x_{i-1} \notin A$, and
\item $j=|x|$ or $x_{j+1} \notin A$.
\end{itemize}

Thus a run is a maximal contiguous block of symbols from $A$.

\subsection{Terminating Sequences}

A substring $t$ is called a \emph{terminating sequence} if either

\[
t \in R^{m}R^*
\qquad \text{or} \qquad
t \in W^{m}W^*.
\]

In other words, $t$ is a maximal run of REM or wakefulness
containing at least $m$ consecutive symbols.

\subsection{Admissible Interruptions}

Let
\[
S = O^{<m}
\]
denote the set of non--NREM substrings whose length is strictly
less than $m$. These represent interruptions that do not terminate
an NREM period.

Intuitively, interruptions shorter than $m$ epochs are ignored
for the purposes of continuity.

\subsection{Definition of an NREM Period}

Let $y \in \Sigma^*$ be a substring. We say that $y$ is a valid
\emph{NREM period} if and only if all of the following conditions hold:

\begin{enumerate}

\item \textbf{Structure with admissible interruptions.}

The substring $y$ can be written as

\[
y = z\,t,
\]

where

\[
z \in (S^*\mathcal{N})^+
\]

and $t$ is a terminating sequence.

Equivalently, $z$ consists of NREM stages possibly interspersed
with interruptions shorter than $m$ epochs.

\item \textbf{No internal terminating sequences.}

Within $z$ there is no substring belonging to
$R^{m}R^*$ or $W^{m}W^*$.
That is, no internal run of REM or wakefulness reaches
the terminating threshold.

\item \textbf{Minimum total NREM duration.}

Let
\[
|z|_{\mathcal{N}}
\]
denote the number of symbols of $z$ belonging to $\mathcal{N}$.
Then
\[
|z|_{\mathcal{N}} \ge n.
\]

\item \textbf{Termination.}

The suffix $t$ is a terminating sequence, meaning that
the NREM period ends upon the first occurrence of
at least $m$ consecutive REM epochs or at least
$m$ consecutive wake epochs.

\end{enumerate}

\subsection{Intuition}

This definition formalizes the following clinical criteria:

\begin{itemize}
\item NREM sleep need not be continuous; short interruptions
are allowed.
\item Interruptions shorter than $m$ epochs are ignored.
\item Long runs of REM or wakefulness terminate the period.
\item A valid period must contain at least $n$ epochs of
NREM sleep in total.
\end{itemize}

\subsection{Language-Theoretic Characterization}

Let $\mathcal{T}$ denote the set of terminating sequences.
Define the language

\[
\mathcal{L}_{\mathrm{NREM}}
=
\Bigl\{
y \in \Sigma^* \;\Big|\;
y = zt,\;
z \in (S^*\mathcal{N})^+,\;
|z|_{\mathcal{N}} \ge n,\;
t \in \mathcal{T}
\Bigr\}.
\]

Then $\mathcal{L}_{\mathrm{NREM}}$ is a regular language.
Indeed, recognition requires only:

\begin{itemize}
\item counting consecutive runs up to length $m$,
\item counting total NREM symbols up to threshold $n$,
\item finite memory of the current run type.
\end{itemize}

Hence there exists a deterministic finite automaton (DFA)
recognizing $\mathcal{L}_{\mathrm{NREM}}$.








\end{document}



