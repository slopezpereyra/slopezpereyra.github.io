\documentclass[a4paper, 12pt]{article}

\usepackage[utf8]{inputenc}
\usepackage[T1]{fontenc}
\usepackage{textcomp}
\usepackage{amssymb}
\usepackage{newtxtext} \usepackage{newtxmath}
\usepackage{amsmath, amssymb}
\newtheorem{problem}{Problem}
\newtheorem{example}{Example}
\newtheorem{lemma}{Lemma}
\newtheorem{theorem}{Theorem}
\newtheorem{problem}{Problem}
\newtheorem{example}{Example} \newtheorem{definition}{Definition}
\newtheorem{lemma}{Lemma}
\newtheorem{theorem}{Theorem}


\begin{document}


\normalsize
(1) Question all authority.

\small
\begin{quote}
    Every form of authority is to be justified. Some forms of authority can be 
    justified; all forms which cannot be justified ought to be dismantled. 
\end{quote}

\normalsize
(2) If you can work with dignity, you should work.

\small
\begin{quote}
    Creative and productive activity, carried out with liberty and freedom, is
    meaningful and contributes to the improvement of individual and collective
    life. We are constantly taking things from society: we must strive to give 
    something back to it. 
\end{quote}

\normalsize
(3) Do not be puritan, prudish, nor overly strict in moral matters.


\small
\begin{quote}

We are all capable of spontaneous as well as premeditated acts of generosity
and wickedness. I reject all essentialism on this matter, and am inclined to
believe we vary on that spectrum as naturally and as much as on the spectrum of any
other human characteristic.

Some doctrines, among which we may count Christianity, strive to drive us towards the 
charitable side of the spectrum by means of strict puritanism. I have observed this fails
to correct people who have more chaotic tendencies; if anything, strict rigour
and the guilt which it induces worsens the degree of entropy which already
exists in their moral life.  

Regardless of this anecdotal comment, we should not strive for puritanism in
principle. Borges accurately characterized the claim that one should never lie
as a \textit{pedantería ética}. Give yourself the liberty to be incorrect,
sarcastic, caustic, or even blunt, as well as the liberty to follow a path
which common opinion would deem immoral if you believe it to be justified and
right. Learn to put on your strict ethical shoes when facing important issues,
and relaxing with matters which are minor or cannot significantly affect your
life and the life of others. People who are overly strict and inflexible,
specially with irrelevant matters, produce an effect opposite to that which
they desire, and are naturally more prone to authoritarianism, obstinacy, and
blunder.

\end{quote}
\normalsize

(4) Be an extrovert.


\small
\begin{quote}

Despondency and dejection, melancholy and anguish, are either directly caused
or worsened by solipsism. Do not take yourself for too interesting a matter of
inquiry: every star you see has gazed upon the entirety of your
ancestry. Direct your mental energy towards the natural world, to politics, to
science, and the arts. Love the world as well as others more than you love
yourself.

\end{quote}
\normalsize

(5) Remember the words of Bertrand Russell: Of all forms of caution, caution in love is perhaps the most fatal to true happiness.

~

(6) Distrust of unfriendly people.


\small
\begin{quote}

The romantic tradition painted, to my view, an incorrect picture of loneliness.
People who lack the capacity to befriend others most likely have one of the
following two flaws: lacking the capacity to appreciate the virtues of others;
exacerbating the flaws of others or simply failing to tolerate them. 


\end{quote}
\normalsize





\end{document}



