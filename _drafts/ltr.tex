\documentclass[a4paper, 12pt]{article}

\usepackage[utf8]{inputenc}
\usepackage[T1]{fontenc}
\usepackage{textcomp}
\usepackage{amssymb}
\usepackage{newtxtext} \usepackage{newtxmath}
\usepackage{amsmath, amssymb}
\newtheorem{problem}{Problem}
\newtheorem{example}{Example}
\newtheorem{lemma}{Lemma}
\newtheorem{theorem}{Theorem}
\newtheorem{problem}{Problem}
\newtheorem{example}{Example} \newtheorem{definition}{Definition}
\newtheorem{lemma}{Lemma}
\newtheorem{theorem}{Theorem}


\begin{document}


Rocío,

~

Espero que esta carta te encuentre bien y no te cause malestar escuchar de mí.
Estimo (probablemente mal, las fechas no son lo mío) que han pasado cuatro o
cinco años desde la última vez que nos vimos. Ignoro si me recordás o no, y si
es que lo hacés no sé decir si será con la indiferencia con que conjuramos las
cosas grises del pasado, con rencor o resentimiento, o (aunque lo dudo mucho)
con alguna que otra alegría o remembranza feliz. 

En mi caso, pese a que terminamos mal, y que eso fue completamente mi culpa, te
sigo recordando con cariño: cada vez que hago una masa casera, tal como vos me
enseñaste, o me encuentro con Juan Cruz, a quien ahora veo poco, elijo recordar
lo mejor de la corta-pero intensa-relación que tuvimos.

Últimamente, estuve reflexionando sobre mi pasado, y en particular sobre
aquellos a quienes lastimé. Es inevitable pensar en vos. 

Te escribo esta pequeña carta para decirte dos cosas. La primera es que te pido
perdón: mi manejo cuando nos separamos fue realmente pésimo, te causé mucho
dolor innecesario, no fui claro ni reponsable con vos, y lo siento mucho. No te
merecías eso. 

La segunda es que muchas de las cosas que, en tu dolor, viste y
dijiste de mí, eran verdad. Yo sé que no necesitás que te lo diga, que vos
debés tenerlo muy claro por vos misma, pero supongo que lo que quiero decirte
es que tu dolor y tu visión de mí no eran cosas que estaban en tu cabeza ni
confusiones surgidas de la separación, sino hechos innegables acerca de la
persona que fui en ese momento: frío, calculador, y manipulador. Lo cierto es
que no me importó que pudiera lastimarte. Me avergüenza, y si pudiera volver
atrás y obrar de otra manera lo haría.

Espero que estés bien Ro. No es un decir: realmente espero que, sea donde 
sea que estés ahora, cualquiera sea el momento que estés transitando, te 
encuentres bien. Quiero pensar (aunque sé que tal vez es un exceso de optimismo)
que, mientras nuestra relación duró y fue buena, supe darte cosas lindas que 
tal vez perduren en vos: algunos libros, el ajedrez, las madrugadas jugando 
al \textit{For the king} con Juancho, las noches viendo \textit{Pasión de Gavilanes} y 
comiendo empanadas. Yo busco recordar esas cosas cuando me acuerdo de vos, 
ojalá pese al dolor que te causé tampoco las hayas olvidado.

~

Santiago









\end{document}



