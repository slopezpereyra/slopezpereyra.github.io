\documentclass[a4paper, 12pt]{article}

\usepackage[utf8]{inputenc}
\usepackage[T1]{fontenc}
\usepackage{textcomp}
\usepackage{amssymb}
\usepackage{newtxtext} \usepackage{newtxmath}
\usepackage{amsmath, amssymb}
\newtheorem{problem}{Problem}
\newtheorem{example}{Example}
\newtheorem{lemma}{Lemma}
\newtheorem{theorem}{Theorem}
\newtheorem{problem}{Problem}
\newtheorem{example}{Example} \newtheorem{definition}{Definition}
\newtheorem{lemma}{Lemma}
\newtheorem{theorem}{Theorem}


\begin{document}

Whatever there is to be gained in the free exercise of love is a matter of
depth, not of abundance. The people we come to know who truly have something to
offer—to the world, I mean, not only to ourselves—are very few. Yet usually
their hearts guard secrets captivating, undecipherable... 

I met a girl some time ago who seems to me rather strange. Her soul seems to
peculiarly combine the mysterious qualities of an archetype—something of a
witch, something of a nymph, something else which words cannot profess—with the
sweet transparency of a child. What I mean to say is this: everything
in her is \textit{cloaked}. Some things so extraordinarily well that I cannot
but wonder what lies in their depths; others so poorly that, just like we play
along with the fantasies of a child, i.e. with tender condescension, I only
pretend not to see them.

I very well know that it is arrogant to presume we know anything about anyone,
including ourselves. And yet I cannot help but sense that this girl was afraid
of \textit{something}. What was it? To love, to be loved? To be vulnerable,
to shiver naked before me, my hand holding her neck, my lips writing a sinful
cross in her skin and heart? We are all afraid, after all, but she acted as
if all her doubts and fears (which I could only glimpse) belonged originally
to her, as if she was alone in all her fears, as if all the terrors of the
night and all the wicked omens of the moon were apparent only to her dark eyes.

I have often wished to speak to her, to \textit{truly} speak to her, removing
the nocturnal cloak from within which she has observed me. I have wished to
say: \textit{I know you are afraid}. I have wished to whisper into her ears, as
a father would: \textit{You don't have to be afraid}. I have wished to offer her
things which we both know I cannot offer.

In general, I do not like people who put up a tough front. Sensible souls
should let their sensibility show. There is selfishness in this preference: I
find it incredibly difficult to hide my feelings, to act as if I had a thick
skin. It doesn't take much to have me write a poem or a song. Yet, with this
girl, I played my part to perfection. I did not want to, but she was playing a
part herself, so I had no choice. Perhaps we were both fools to act as if we
did not care that much. Perhaps she truly did not care that much and I am
indeed a fool.

I thought of writing a letter to her, of explaining to her that we could mean
more to each other than what we allow for. Yet there is something obscure and
attractive in the way we treat each other: we hide from one another, we let so
little of ourselves show, to then rip us apart and let our sins in flesh and
bone, the violent secrets of our minds, flourish in a frenzy. Like actors who
fake their cry and suddenly, unexpectedly, though just for a brief moment, find
themselves actually crying. 

On top of that, if I were to suggest that there could be anything more to
us—even if it were only a friendship—she would either calmly play her tough girl
part and cast me aside, as tough people do when dealing with sentimentals, or
she would lose her cool and flee.

Perhaps I will, eventually, \textit{uncloak} her. Perhaps someday she will
reveal herself to me vulnerable and true. I would adore her so much more. But I fear
I'd also lose her.











\end{document}



