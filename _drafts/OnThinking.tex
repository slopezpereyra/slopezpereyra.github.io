\documentclass[a4paper, 12pt]{article}

\usepackage[utf8]{inputenc}
\usepackage[T1]{fontenc}
\usepackage{textcomp}
\usepackage{amssymb}
\usepackage{newtxtext} \usepackage{newtxmath}
\usepackage{amsmath, amssymb}
\newtheorem{problem}{Problem}
\newtheorem{example}{Example}
\newtheorem{lemma}{Lemma}
\newtheorem{theorem}{Theorem}
\newtheorem{problem}{Problem}
\newtheorem{example}{Example} \newtheorem{definition}{Definition}
\newtheorem{lemma}{Lemma}
\newtheorem{theorem}{Theorem}


\begin{document}

Writing is not to me entirely voluntary. I write not about the things I know,
for these could hardly fill a page or two, but of the things I happen to be
pondering about. More often than not, the query is not \textit{what is it} but
\textit{who am I}, and whatever I say is not as much a statement about the
world as a declaration of sentiment.

On mathematics, I lack any form of special talent, except a powerful tendency
to be gripped and become obsessed by certain problems; and anyone who reads
what I have written on it will find that I was merely taking notes---and not
profound ones. The truest of the things I write is poetry, which is the single
mode of expression, among those which I practice, that is honest about the fact
that it pretends nothing but to release the soul from its burdens. In fact, the
force which drives me to consider any matter, and to write any single page, is
the same force which drives the artist: Catharsis.

For instance, a few days ago I happened to ask myself: What is the number of
connected graphs of $n$ vertices and $m$ edges. The question was asked with
innocence: the matter was not innocent at all. A state of restlessness took
possession of me. I did not choose to pursue the question: I desired to restore
spiritual equilibrium. And thus I thought about it, and thus I wrote about
it, until satisfaction was reached.

Similarly, an intuition may seize control of my heart. I do not know exactly
what it is, so I begin the quest for words which could describe it. A poem
comes to be, but not because I wanted it. It feels as if he claimed himself
through me.

I read out of pleasure and without much of a commitment to retain what I read,
so the opinions of men greater than myself are often lost to me. If a book is
not compelling to me I disregard it, regardless of the authority of his author,
and I do not mind the fact that I make my limitations evident in failing to
appreciate a man of greater fame than me. I often recall the lesser rather than
the central aspects of a work, and an occasional curiosity can captivate me
more than the solid gist of a deep argument. Above all, I despise grandiloquent
or convoluted works, and feel no shame in saying that I saw nothing to
enjoy---nor even to comprehend--- in the likes of a Hegel or a Heidgger,
however deep they are claimed to be. I pursue matters with candour and joy, and
thus clarity and honesty compel me above all things.

The authors which I read are friends to me, more than authorities. I tend to
admire their intelligence just as much as their personal qualities. I'm
interested in their lives and in their works, and find it hard to appreciate an
author who I deem to be a questionable individual. I converse with them in my
mind and, rather often, in my dreams. I have dreamt myself discussing poems
with Miguel Hernández, walking Borges across a sublime library, or drinking a
cold beer with Bertrand Russell. On each occasion, our conversations were rich
and I could recollect them after waking up.

In everything I seek love, and in almost everything I find it. The pursuit of
knowledge is, to me, romantic---as much as the pursuit of artistic expression
and human affection---. Though, on intellectual matters, I strive for clarity
and precision, I do not confuse these qualities with the sterile and
passionless attitude which characterizes academic writing. Bewilderment is
the precursor of science---in any matter the essential question is \textit{what
are the facts}---but thought is not and cannot be merely \textit{fact
enumeration}. In the pursuit of knowledge there is a sense of mystery which can
only be equated to the fathomless perplexities of human
passions. If one writes as one thinks, all these---fact and mystery, precision
and passion---intertwine.

If I confuse myself or write equivocally on some matter, I pay little attention
to it. I treat my trains of thought as axiomatically dubious, and see myself as
neither wise nor expert, so what shame could I feel in not attaining truth,
when I deem myself so ill-prepared? If I happen to find a truth, or to
demonstrate a property, I feel unspeakable joy. But if I fail to do so, which
is most often the case, and like Tantalus cannot attain the fruit that I
unceasingly envision, I feel no frustration nor consider this a curse: the
struggle itself is a source of pleasure, and failure not a source of anguish.
And so, as one can see, I am so disposed that I stand with nothing to loss and
much to win.

And so my writings come to be: eclectic and unorganized, dubious and
superficial, more a testament to how close or far I can see than to what there
is to see. If anything, aside from serving me to put certain things in order,
and to reach a cathartic equilibrium, the only question they can answer is not
\textit{what did he know} but \textit{what did he think about}?






\end{document}



