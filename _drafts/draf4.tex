\documentclass[a4paper, 12pt]{article}

\usepackage[utf8]{inputenc}
\usepackage[T1]{fontenc}
\usepackage{textcomp}
\usepackage{amssymb}
\usepackage{newtxtext} \usepackage{newtxmath}
\usepackage{amsmath, amssymb}
\newtheorem{problem}{Problem}
\newtheorem{example}{Example}
\newtheorem{lemma}{Lemma}
\newtheorem{theorem}{Theorem}
\newtheorem{problem}{Problem}
\newtheorem{example}{Example} \newtheorem{definition}{Definition}
\newtheorem{lemma}{Lemma}
\newtheorem{theorem}{Theorem}
\DeclareMathAlphabet{\mathcal}{OMS}{cmsy}{m}{n}

\begin{document}


Todos, a toda hora, somos avasallados por infinitas fuerzas que nos llaman a la
quietud, al conformismo y la idiotez. En cada asunto que pensamos el
apresuramiento conforma opiniones sin mérito; en cada conversación, palabras
vacías, palabras que no son de nuestro corazón, quieren decirse; la mano
seductora de la pereza intelectual nos llama sensualmente. Esta es la verdadera
lucha. Debemos practicar, *toda la vida*, una sana intransigencia. Debemos
alejarnos de las conversaciones vanas. Debemos escapar de la frivolidad. Cuando
quiera que sintamos que perdemos nuestro tiempo, debemos recordarnos: en este
paraíso existen cosas que pueden asombrarme y que yo debo descubrir. En el
asombro está la vida. 

La curva de la vida comienza a dibujarse duramente sobre el rostro de mis
contemporáneos. En sus caras sospecho la abyecta o adorable huella
que, desde su encierro, imprimen las almas. En tantos veo una auto-complaciencia
extraordinaria, un obsceno desinterés por nutrirse de las cosas nobles de la
vida. Muchos empiezan a beber en exceso, a darse licencias. Dejan de leer. Dejan
de pensar. Pienso, que eventualmente, dejarán de sentir. Hay en mis pesadillas
un tono verde pálido que debe parecerse al de esas vidas que se mueven de una
frivolidad a otra, que ocultan el fangoso fondo de la angustia donde el deseo y
el amor se han enterrado. Son como los ríos cuya superficie turbulenta esconde
un fondo quieto y pantanoso. (Yo, en mi juvenil encierro, parezco lento y
estatuario, pero recorro mundos milearios y siembro en mi lodoso fondo
hermosas algas y serpientes.)








\end{document}



