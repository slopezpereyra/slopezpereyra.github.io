\documentclass[a4paper]{article}


\usepackage[utf8]{inputenc}
\usepackage[T1]{fontenc}
\usepackage{textcomp}
\usepackage{amsmath, amssymb}
\edef\restoreparindent{\parindent=\the\parindent\relax}
\DeclareSymbolFont{letters}{OML}{ztmcm}{m}{it}
\usepackage{parskip}
\restoreparindent

\begin{document}

Kropotkin saw in every human creation the vast horizon of our lineage.
Everything of value, from agricultural technique to astronomy, to him was but
the blossom of a capital inherited---material and intellectual---a relic
tracing back to the common ancestry of mankind. Rousseau, in the
\textit{Discourse}, speaking of an imaginary founder of private property,
wonders:

\begin{quote}
    How many crimes, wars, murders and how much misery and horror the human
    race might have been spared if someone had pulled up the stakes or filled
    in the ditch, and cried out to his fellows: ‘Beware of listening to this
    charlatan. You are lost if you forget that the fruits of the earth belong
    to all and that the earth itself belongs to no one’.
\end{quote}

Smith warns against  the "vile masters of mankind", whose guiding maxim is "all
for ourselves, and nothing for other people", and states:

\begin{quote}
    As soon as the land of any country has all become private property, the
    landlords, like all other men, love to reap where they never sowed, and
    demand a rent even for its natural produce.
\end{quote}

It is true, in the case of Smith, that he was concerned with describing things
as they were, rather than suggesting how they ought to be. \textit{The Wealth
of Nations} provides a model for the workings of a dynamical system, much like
the Hodgkin-Huxley model, and not ethical imperatives. In this, it is similar
to \textit{Das Kapital} but different from most other works of Marx.
Notwithstanding, that which does originate from a moral attitude in it, sprouts
from the ideals of the Enlightenment---in which it differs from \textit{Das
Kapital} but is strikingly similar to, say, the \textit{Manuscripts of 1848}.

A society in which individuals are free to exercise creative and critical
thought in association with others---not bounded by fetters nor preoccupied by
tyranny---a society were people are educated enough so as to be capable of
independence, and powerful enough so as to be capable of influence: Such was
the liberal ideal. Economical considerations were subordinated to these, which
are philosophical in nature. So, the query arises: Are these ideals realized in
state capitalism? No comments. The second question is: Do liberals today truly
draw from these ideals? No comments. The third question, after a negative of
the second, is: Who aspires still to these ideals? The anarchist tradition,
ignoring radically individualist deviations, did preserve them---the communist
tradition, ignoring right-wing deviations of Marxism, did preserve them.

The world today is far, at least as far as it ever was, from realizing them.
Rather than a reason to disregard them, this is a reason to retrieve them.
So-called liberals in Argentina suddenly appear as lovers of the exercise of
violence and force, apologists of tyrannies, and subordinates to concentrated
private power. This should not be tolerated. The left, at least the libertarian
left---I used the word \textit{libertarian} in its original sense---should
persist in drawing from the spirit of these ideals and decisively defeat the
claim that \textit{liberal} means ultra-conservative. 

I'm not making a point on terminology---it is meaningless to simply say oneself
ore someone else is a liberal or a socialist---. I'm making an ideological
point; namely, that liberal ideals have been mostly forgotten, and that which
has not been forgotten has been distorted. The recovery of these ideals can
only come from the left. It was, as a matter of fact, the left which drew
inspiration from them, and one could make the case that socialism, in all its
variations, was but an application of the liberal principle of opposition to
concentrated power in societies where such concentration fell on private
capital and not on the court or the state. 

Both Soviet and anti-Soviet propaganda, though for different reasons, somehow
convinced the masses that a state as tyrannical as the USSR was the socialist
ideal. This is equally ridiculous than the belief, also enforced by propaganda,
that societies where private capital is concentrated to an abysmal degree, and
where such concentration is secured by directing violence---whose monopoly is
in hands of the State---towards the people, are democratic, free, and somehow
in accordance to the aspirations of the founders of classical liberalism. (So
one can see that indoctrination goes both ways, and how deep it penetrates
public opinion; for, as the going says, \textit{el muerto se asusta del
degollado}.) 










\end{document}
