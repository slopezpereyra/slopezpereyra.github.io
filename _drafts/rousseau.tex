\documentclass[a4paper]{article}


\usepackage[utf8]{inputenc}
\usepackage[T1]{fontenc}
\usepackage{textcomp}
\usepackage{amsmath, amssymb}
\edef\restoreparindent{\parindent=\the\parindent\relax}
\DeclareSymbolFont{letters}{OML}{ztmcm}{m}{it}
\usepackage{parskip}
\restoreparindent

\begin{document}

$\S$ Kropotkin saw in every human creation the vast horizon of our lineage.
Everything of value, from agricultural technique to astronomy, is but the
blossom of a capital inherited---material and intellectual---a relic tracing
back to the common ancestry of mankind. Rousseau, in the \textit{Discourse},
speaking of an imaginary founder of private property, wonders:

\begin{quote}
    How many crimes, wars, murders and how much misery and horror the human
    race might have been spared if someone had pulled up the stakes or filled
    in the ditch, and cried out to his fellows: ‘Beware of listening to this
    charlatan. You are lost if you forget that the fruits of the earth belong
    to all and that the earth itself belongs to no one’.
\end{quote}

Smith warns against  the "vile masters of mankind", whose guiding maxim is "all
for ourselves, and nothing for other people", and states:

\begin{quote}
    As soon as the land of any country has all become private property, the
    landlords, like all other men, love to reap where they never sowed, and
    demand a rent even for its natural produce.
\end{quote}

It is true, in the case of Smith, that he was concerned with describing things
as they were, rather than suggesting how they ought to be. \textit{The Wealth
of Nations} provides a model for the workings of a dynamical system, much like
the Hodgkin-Huxley model. Notwithstanding, that which
does originate from a moral attitude in it, sprouts from the ideals of
Enlightenment---mainly, but not solely, an opposition to concentrated power.

A society in which individuals are free to exercise creative and critical
thought in association with others---not bounded by fetters nor preoccupied by
tyranny---a society were people is educated enough so as to be capable of
independence, and powerful enough so as to be capable of influence: Such was
the liberal ideal. Economical considerations were subordinated to these, which
are philosophical in nature.

The first question is: Are these ideals realized in state capitalism? No
comments. The second question is: Do liberals today truly draw from these
ideals? No comments. The third question, after a negative of the second, is:
Who aspires still to these ideals? The anarchist tradition, ignoring radically
individualist deviations, did preserve them---the communist tradition, ignoring
right-wing deviations of Marxism, did preserve them.

The world today is far, at least as far as it ever was, from realizing them.
Rather than a reason to disregard them, this is a reason to retrieve them.
So-called liberals in Argentina suddenly appear as lovers of the exercise of
violence and force, apologists of tyrannies, and subordinates to concentrated
private power. This should not be tolerated. 







\end{document}
