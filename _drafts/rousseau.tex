\documentclass[a4paper]{article}


\usepackage[utf8]{inputenc}
\usepackage[T1]{fontenc}
\usepackage{textcomp}
\edef\restoreparindent{\parindent=\the\parindent\relax}
\DeclareSymbolFont{letters}{OML}{ztmcm}{m}{it}
\usepackage{parskip}
\usepackage{ebgaramond}
\restoreparindent

\begin{document}

Kropotkin saw in every human creation the vast horizon of our lineage.
Everything of value, from agricultural technique to astronomy, to him was but
the blossom of a capital inherited---material and intellectual---a relic
tracing back to the common ancestry of mankind. Rousseau, in the
\textit{Discourse}, speaking of an imaginary founder of private property,
wonders:

\begin{quote}
    How many crimes, wars, murders and how much misery and horror the human
    race might have been spared if someone had pulled up the stakes or filled
    in the ditch, and cried out to his fellows: ‘Beware of listening to this
    charlatan. You are lost if you forget that the fruits of the earth belong
    to all and that the earth itself belongs to no one’.
\end{quote}

Smith warns against  the "vile masters of mankind", whose guiding maxim is "all
for ourselves, and nothing for other people", and states:

\begin{quote}
    As soon as the land of any country has all become private property, the
    landlords, like all other men, love to reap where they never sowed, and
    demand a rent even for its natural produce.
\end{quote}

But Smith was concerned with describing things as they were rather than
suggesting how they ought to be. \textit{The Wealth of Nations} provides a
model for the workings of a complex system, not ethical imperatives. In this it
is, at least in principle, a scientific work. \textit{Das Kapital} shares this
property, unlike most other works of Marx.

Yet, what in \textit{The Wealth of Nations} does originate from a moral
attitude, sprouts from the same ideals which drove French and German
romantic traditions---which include the young Marx. It would not be surprising that
even educated readers would confuse a random passage from \textit{The Wealth of
Nations} with, say, the \textit{Manuscripts of 1848}. From an ideological
perspective, the similarities are unsurprising. It \textit{should} be a
triviality to identify Enlightened ideals with the primitive utopians--- e.g. the
likes of Étienne Cabet---as well as modern socialists. The fact that this
identification is not perceived, even by public intellectuals, is (pun
intended) illuminating.

But what was the liberal ideal? It is of course impossible to reduce a complex
and rich tradition to a few sentences. Yet certain key features are common
across its spectrum. These, are for instance, that freedom to exercise critical
thought and creative activity, in solitude or in association with others, is
fundamental. That people ought to be educated enough so as to be capable of
independence, and powerful enough so as to be capable of influence. That
selling or renting oneself is degrading, as it's degrading that work is so
alien to a man's inclinations that he must be bribed or coerced to do
it by fear of inanition. And a long etc. 

Economical considerations were subordinated to these, which are philosophical
in nature. So, the query arises: Are these ideals realized in state capitalism?
No comments. The second question is: Do liberals today truly draw from these
ideals? No comments. The third question, after a negative of the second, is:
Who aspires still to these ideals? 

The world today is far, at least as far as it ever was, from realizing them.
Rather than a reason to disregard them, this is a reason to resucitate them.
So-called liberals suddenly appear as lovers of the exercise of violence and
force, apologists of tyrannies, and subordinates to concentrated private power.
This should not be tolerated. The left, at least the libertarian left---I used
the word \textit{libertarian} in its original sense---should persist in drawing
from the Enlightened ideals. Since they have been forgotten, since what little
is remembered is distorted, the left should resurrect them. Socialism, in
all its variations, is but an application of the fundamental liberal
principle---opposition to concentrated power---to societies where such
concentration falls on private capital and not on the court or the state.









\end{document}
