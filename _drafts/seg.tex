\documentclass[a4paper, 12pt]{article}

\usepackage[utf8]{inputenc}
\usepackage[T1]{fontenc}
\usepackage{textcomp}
\usepackage{amssymb}
\usepackage{newtxtext} \usepackage{newtxmath}
\usepackage{amsmath, amssymb}
\newtheorem{problem}{Problem}
\newtheorem{example}{Example}
\newtheorem{lemma}{Lemma}
\newtheorem{theorem}{Theorem}
\newtheorem{problem}{Problem}
\newtheorem{example}{Example} \newtheorem{definition}{Definition}
\newtheorem{lemma}{Lemma}
\newtheorem{theorem}{Theorem}
\DeclareMathAlphabet{\mathcal}{OMS}{cmsy}{m}{n}

\begin{document}

Seguís en mí, Paulina. A todas horas

las manchas gangrenosas en el techo

forman tu rostro muerto. 

~ ~ ~ ~ ~ ~~~ ~ ~ ~ ~ ~ ~ ~ ~ ~ ~ ~ ~ (<<¿Qué te has hecho,

qué tienes en la mano, por qué lloras?>>

Me sueño hablándote de esta manera.

Mi sueño, que es mi vida, así te espera.)

\pagebreak 

Mi cuerpo: luz sin consuelo.

Surca mis venas azules

la fragata de los muertos.

~ 

Tu cuerpo: niebla sin vida.

En el agua de tus huesos

un náufrago no te olvida.



\pagebreak 

A vos nomás te escribo, pues nadie más conoce 

el lenguaje del silencio.

El símbolo preciso es la palabra vacía.

~

Las dos agujas hablan: dicen que son las doce.

Entonces te presencio.

(Estoy soñando. Vuelves. No has muerto todavía.)











\end{document}



