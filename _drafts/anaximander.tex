\documentclass[a4paper]{article}

\begin{document}


\section{Preliminary comment}

The purpose of this entry is to serve as record of a particular period in my
intellectual life. It contains what originally were the scattered notes of a
rereading of the presocratic philosophers. At the time, I was simultaneously
studying the ninth volume of the \textit{Complete works} of C.G. Jung. The
coincidence of these separate studies conferred the notes with a rather peculiar
tone---so much, in fact, that their content appears to me strange and almost
pointless.

Even so, the core intuition that propelled me to write them, and to organize
them into a single study, remains kindled and strong within me. I am speaking of
the suspicion that human experience, even across its wide range of cultural and
spatio-temporal differences, tends to structurally identical patterns of
representation. To put it differently, I mean the idea that there exists a
symbolical counterpart of instinct, that there are innate tendencies of symbolic
representation---perhaps the byproduct of the emergence of symbolic faculties in
instinctive animals. This is what I loosely mean by \textit{archetypical}
images. Instinct and archetype are here understood to be enantiomeric.

Archetypes are not eternal symbols floating in the soul of man. They are, if
anything, a biological tendency. Every human faculty varies within a limited
range---nothing distinctly human could otherwise exist. I am currently putting
in writing a survey of the existing evidence in favour of this \lq\lq
collective unconscious\rq\rq{}. I shall refer any discussion on this matter to
that upcoming work and disregard it for the time being.

In less general terms, the notes focused on the figure of Anaximander. I suspect
the philologist will find these writing unbearable. In every symbolic or
mythical aspect of presocratic philosophy, he sees a persian or egipcian
influence, an oriental breath infusing some exotic tone into otherwise natural
and empirical claims. Needless to say not all philologists believe
this---Nietzsche and Colli, for example, disregard it---but even those who
don't fail to explain these mysterious elements in a satisfactory manner. I did
provide in these notes, I think, sufficient reasons to justify why these
approaches are insufficient. But I must confess to now doubt of the value that a
psychological approach to presocratic philosophy may produce. I fear all these
notes ended up being nothing but a literary exercise.

\section{Introduction}

Presocratic thought, in spite of tremendous efforts over the last two centuries,
remains in almost total obscurity. Aristotelian exegesis, which sees in it
nothing but a rational attempt at describing the Φύσις, casts its shadow over
practically all philology. Granted, part of the presocratic tradition was a
genuine effort for discerning the natural properties of the world. But the fact
that, in every philosopher, the most natural postulates were derived, as a
general rule, from some mythical ground which was the corner stone of his
doctrine, suggests that this part was also the most superficial. Every ἀρχή is
ultimately arbitrary and irrational, and none of them can be explained as a
strictly scientific \textit{intellectio}. They are cosmogonic images of
speculative nature, whose resemblance with mythological expressions make them
worthy of the epithet of \textit{mythical}. In Anaximander's τὸ ἄπειρον and
Anaxagoras' νοῦς, for example, this is particularly clear. Philologists have
tried to explain this mythical aspects in three different ways.

$ a. $ The first consisted in trying, sometimes with admirable intelligence, to
show that these mythical expressions were in truth strictly rational---or simply
assuming this was the case. Thus, for example, Hesiod's χάος is the logical
regression from complex to simpler elements (Gigon, 1971), Anaximander puts the
earth at the center of the universe compelled by a geometric intuition (Jaeger,
1933), and Aristotle teaches that Thales chooses water as the ἀρχή by virtue of
its transmutability. But it is illicit to project a naturalist spirit to
postulates that, taken by themselves, are closer to myth than to empirical
observation. The case of the presocratic philosophers is not all that different
from that of alchemy, sometimes described merely as a pre-scientific study of
matter, when undoubtedly it is also a symbolic universe (Jung, 1944; Roob,
2014).

$ b. $ In other cases, the mythical grounding of presocratic philosophy was
recognized, but nothing was seen in it but a proper aristocratic diversion of
the Dionysian and drunk Greek spirit. That was the stance of Nietzsche and
Colli, for example. This hypothesis, however, has two problems. First, it misses
the tremendous resemblance between the symbolic expressions that appear in
presocratic philosophers and those of many other myths. It is not the time to
point this resemblances out, but we shall deal with some of them in a moment.
They show that, in what comes to its irrational aspect, presocratic thought is
not all that original, and that what is attributed to a properly Greek amusement
might turn out to be universal representations. Secondly, the symbolic
expressions that are thus explained have, in truth, very little in common with
with the typical worldview of ancient Greeks. This will become clearer when
describing point $ c. $ 

$ c. $ The third explanation consists in the appeal to some eastern
influence---be it Persian, Babylonian, or Egyptian. This hypothesis deserves our
attention the most, because it is the most plausible. It is a fact that
presocratic Greece was in continuous contact with other peoples, so much so that
many presocratic philosophers were colonial agents or travelers themselves.
But it is imperative to observe that, for the Greek, nature was the total
expression of the divine. The gods were not agents of miracles, but the force
that kept the wheel of natural order turning. Religious feeling was engrained in
objective reality and not in the portentous. This is why Otto (1929) claimed
that the Greek worldview had, as a basic character, a \lq\lq natural ideality or
ideal naturality\rq\rq{}, and Gigon (1971) made the clever observation that the
characters of the \textit{Theogony} were not brought about because they were
gods, but because one cannot obviate the regions represented by them in a
wholesome picture of reality.

The eastern sentiment, quite contrarily, was that objective reality was not the
background of the divine, but the illusory veil that separates us from it. The
divine delve on an immaterial realm that only coincided with sensible reality by
means of occult correspondences. The immaterial soul, a matter of little
importance to the Greek, had a preponderant role. Thus, while the Greek
conceived himself to be a member of an objective reality where the divine was
clear and manifest, the eastern man conceived reality to be the dream that
distorts the divine essence and from which he was to free himself. Where the
Greek saw the organic and manifest, the eastern man guessed but a symbol and a
mystery. The first looked at the sun and saw the warm glow of Apollo, who lived
beyond the sea. The latter saw in it a light that existed only because it had been
created by his own sense of sight.

Granted, it is vain to describe the religious spirit of two peoples in such a
brief note. However, the synthesis above should suffice to show that the Greek
conception of reality was radically opposed to that of eastern peoples, and that
it provided to them a complete perspective of the world, a sufficient framework
of experience. Whence, then, would the Greek spirit be so susceptible to eastern
influence? For, indeed, if an individual is, so to speak, touched by an idea,
there must be something in him susceptible to that touch, and the same can be
said of a people. If the phenomena that concerns us were reduced only to the
peculiarities of Greek culture, we should expect the latter to be quite
impenetrable to eastern religious sentiments---in a manner analogous to the way
in which we are deaf to a foreign language. The matter is even more problematic
when we consider, as we said before, that at the moment in which presocratic
philosophy was flourishing, Greece, far from being weakened or in crisis, found
itself in an intense period of colonial expansion and power (Lane Fox, 2007). It
was besotted with pride and without any need of incorporating foreign myths that
were, on top of everything, radically incompatible with those of its own. If
eastern peoples played a part, they did so incidentally, but not sufficiently
nor, ultimately, necessarily.

A hypothesis that may fill this voids and avoid this flaws must not project over
symbolic expressions a rational or empirical character, on top of explaining its
anti-idiosyncratic nature and resemblance with other myths. My conjecture is
that the different cosmogonies exposed by presocratic philosophers are
archetypal images, whose expression in other traditions has been pointed out,
but whose influence in presocratic philosophy was overlooked by a hermeneutic
tradition not accustomed to such psychological concepts. I do not deny that the
discovery of the Φύσις was a genuine enterprise---I only claim it was propitious
to the projection of parallel psychological developments by virtue of the
mysterious and complicated nature of the cosmogonic and astronomic phenomena in
question, on one hand, and by the lack of a scientific method that protected, so
to speak, the inquiry from psychological contamination.

In relation to the meaning of \textit{archetype}, or archetypal image, I must
once more refer the reader to the entry \textit{On archetypes}. For the
psychoanalytic conception, which I do not follow entirely, I refer him also to
both tomes of the ninth volume in Jung's \textit{Complete works}. As a quick
summary, I should only say that by \textit{archetype} I understand
phylogenetically determined, evolutionary archaic, and affective patterns of
representation. That we share a common affective substrate with, at least, all
mammals, is an established scientific fact (Panksepp, 2000; Panksepp and Gordon,
2003; Panksepp, 2005; Panksepp, 2011). Furthermore, neuroscience has gathered
some evidence in favor of the existence of some of the specific archetypes
theorized by Jung (Alcara et. al., 2017). Of course, such evidence is
inconclusive and should be carefully interpreted. 

Archetypes are not a metaphysical idea, a mystical intuition, nor an
abstraction. They are---so I believe---a psychological fact. The advancement of
neuroscience in the production of evidence must necessarily be accompanied by
the study of their concrete cultural and behavioral manifestations. They are
relevant for individual and social life because the structures mediating
primitive, affective behavior are, as far as we can tell, also those modulating
the construction of meaning---as it is evidence by the obstinate appearance of
archetypal images in myths and cosmogonies. They are, in all probability, the
product of \lq\lq value-encoding neural systems\rq\rq{} (Panksepp y Burgdof,
2003) associated to the articulation of meaning.

I will focus on Anaximander for two reasons. Firstly, a study of this sort over
the whole corpus of presocratic thought is a colosal endeavor, far beyond my
capacity and time. (I should comment to my reader: I write only a very limited
number of hours per week, and devote almost all of my time to my scientific
endeavors.) Secondly, Anaximander wrote one of the most famous fragments of
ancient philosophy; namely, the fragment D-K 12 A 10, whose cosmogonic character
makes its comparative analysis somewhat easier. 

\section{Anaximander}

The fragment of interest comes from Plutarch^1:

>«Anaximandro … dice que lo infinito es la causa de la generación y destrucción
de todo, a partir de lo cual —dice— se segregan los cielos y en general todos
los mundos, que son infinitos. Declara que su destrucción y, mucho antes, su
nacimiento se producen por el movimiento cíclico de su eternidad infinita… Dice
también que, en la generación de este cosmos, lo que de lo eterno es capaz de
generar lo caliente y lo frío fue segregado, y que, a raíz de ello, una esfera
de llamas surgió en torno al aire que circunda a la tierra, tal como una corteza
[rodea] al árbol; al romperse la [esfera] y quedar encerradas [sus llamas] en
algunos círculos, se originaron el sol, la luna y los astros.»

Anaximander teaches that the origin of all things is \lq\lq the
infinite\rq\rq{}, also translated as the Unlimited or the Undetermined. In this
principle, all pairs of opposites are contained, and so are all worlds, which
are released from it. The Unlimited produces the world in a continuous
fashion---it is not the creator at a sole and distant point in time, but the
perennial and continual agent of the generation and extinction of
things---it is \textit{el fondo del movimiento cósmico}. 

From the Unlimited, a \lq\lq seed\rq\rq{} (γóνιμον, \lq\lq that which is $\ldots$
capable of generation\rq\rq{}) of light and night was segregated. The word
γóνιμον is an important subtlety. From the Unlimited did not sprout light and
darkness, but that which engenders them. Light and night thus
constitute the original opposition, begotten by the seed that segregated from
the Unlimited. From a psychological perspective, one is tempted to say this is a
projection of the immemorial separation of consciousness from primordial
unconscious existence, similarly to the words of God: \lq\lq \textit{let there be
light}\rq\rq{}. It is said that light and night covered the Earth like the cortex
would a tree and in a double layer. Light was the interior layer---night was the
exterior one. But the husk of light was teared appart---we are not told how nor
why---and the fire that once was total was dispersed in small spherical shapes.
Those are the stars and \textit{astrums} we observe.

The world to which the seed is thrown, the cosmic state in which this primordial
segregation occurs, is the empty space, the mythical χάος that is first proposed
in Hesiod's \textit{Theogony}. The word χάος does not express, as Ovidio
thought, an aleatory mess. It means something along the line of \textit{cavity}
or \textit{cleft}. It was used, for example, to denote the wide opening of a
mouth or the aperture of a cave. Its formulation as primordial state is a
widespread and well documented archetypal motif. \textit{Gensis 1:2} speaks of
the abyss that was shadow only: \lq\lq \textit{And the earth was without form, and void;
and darkness was upon the face of the deep}\rq\rq. The primordial Hunger in the
\textit{Vedas}, insofar as hunger is a state of inner emptiness, or the
universal symbol of the cave as a transformation place---this is, of generation
by means of prior destruction---are also related to Hesiodic χάος. It is worth
mentioning, although briefly, the role of Mercury in alchemy, which
Paracelsus said to be \lq\lq \textit{the concealer of the rest [of things]---their
corporeal vessel} ($\ldots$)---\rq\rq{} (Paracelsus & Ed, 2018). But what
concerns us now is that, in Anaximander, it is the seed of light and shadow what
alters that primordial state. That seed only is what induces content and shape
in the original void, covering the world with the double layer of its fruit.

It must be said that, from a psychological perspective, the emergence of
consciousness is both an individual and a collective phenomena---it is a 
phylogenetic and an ontogenenetic development. It is clear that unconscious
life precedes consciousness, and that the latter emerges from the latter in a
continuous fashion---this is, by changes of \textit{degree} and not of
nature. But awareness of thought came much later than thought itself, and
the emergence of consciousness is matricidal, insofar as the
flourishing of understanding comes with the relegation of an entire form of life
to a realm of absolute obscurity. Thus, in Anaximander, from the Undetermined
sprouts the initial contradiction of day and night, conscious and unconscious
life. This emergence, though expressing form- and content-aquisition by
means of resolving the original χάος and, quite literally, producing a
world---this is, though being a creative act---also creates a tension inherent to the
separation of a unity into irreconcilable opposites. This tension seems to be
what the tearing of the light sheet expresses, which the philosopher never
explains and is, to my eyes, the result of an irrational intuition. Thus, the
primordial unity of light is also destroyed---and this immediately after being
conformed.

This dissemination of the original light into smaller spheres can be, on its
turn, associated to a different psychological fact. The \textit{self} archetype,
expressed in the divine source of the Unlimited, is not only an archetype but,
insofar as it expresses a totality, union and source of all the rest. Indeed,
what psychoanalysis has termed \textit{individuation} can be thought of
symbolically as the union of dispersed lights into a single great luminary.
These are two opposite forms of totality: the primordial unconscious where
everything is, so to speak, undivided---the final integration of opposite poles.
The first is the primordial origin, the latter the ultimate end. In this sense,
they imply each other as logical necessities. This might be one of the most
peculiar aspects of the \textit{self} archetype, if it truly exists---that is,
that it expresses as a return to a pristine state of harmony and, at the same
time, a positive and forward-looking synthesis.

The image of the dispersed luminosities finds another parallel in the
\textit{scintillae} of medieval alchemy. These are subtle flashes of light
present in the \lq\lq substance of transformation\rq\rq{}, associated to the
\textit{anima mundi} and the Holy Spirit. These two notions are different
modulations of a unique intuition; namely, that of the hidden and numinous force
that drives the world. It is no mystery then that the concept of
\textit{scintillae} was associated to them, insofar as it conduces---and in some
sense spiritualizes---the process of alchemical transformation. Kunrath calls
these luminosities \lq\lq \textit{mundi futuri seminarium}\rq\rq{}. They are \lq\lq
semillas de luz diseminadas en el caos\rq\rq{}.

The \textit{self} archetype is abundant in modulations of various nature.
Insofar as it points to the totality of the psyche, it is, in general, the
intuition expressed by every form of monism or monotheism. Diogenes of
Apollonia, for example, postulated either the Undetermined or the \textit{air}
as principle, according to different sources. The philologic discussion
interests us not, for if he spoke of air he did so not as a material
principle---this can only be an Aristotelian mistake---but as engendering
breath, as that air of life that was insufflated into every thing existent. It
is, then, quite identical to Anaximander's principle, at least from a
psychological perspective. Simplicius of Cicilia quotes Diogenes in a passage of
his \textit{Physics} (151, 20-153, 5):

> «Me parece (...) que todas las cosas que existen son alteraciones de lo mismo,
y que son lo mismo. (...) Pero todas estas cosas se generan a partir de lo
mismo, como alteraciones diversas en diversos momentos, y vuelven hacia lo
mismo».

But we can even abandon the Greek world, where the philologists contents himself
with tracing a more or less clear chain of influences. In the
\textit{Bṛhádāraṇyaka Upaniṣad}, the primordial Death, which is Hunger, creates
a mind. The new-born mind conceives the following thought: \lq\lq \textit{que yo
tenga un }ātman\rq\rq{}. The \textit{ātman} is a clear modulation of the self
archetype, insofar as it points to the psychological self in its absolute
totality, the \lq\lq transcendent\rq\rq and \lq\lq non-transcendent\rq\rq{}
planes integrated. It is the breath that vivifies all things. As such it relates
to the \textit{scintillae} of medieval alchemy and Diogenes' air. The first wish
formulated by this primordial mind is, then, a total and unified identity. We
are then told of the emergence of the first man, the \textit{Puruṣa}: \lq\lq En
el principio sólo era el ātman. Y no habiendo otro salvo él mismo, pensó y se
dijo: “Soy yo”. De ahí que su nombre sea \lq yo\rq{}\rq\rq{}.

This \lq\lq yo\rq\rq{}, this \lq\lq I\rq\rq{}, refers not to the self but to the
ego, the subject of consciousness. Thus, with this thought, from the primordial
\textit{ātman} the \textit{Puruṣa} is formed. We find once again that the
formation of consciousness, expressed in the creation of the first man, implies
a lost of unity in the psyche, which is teared into \textit{ātman}=self y
\textit{aham}=ego. This intuition is the one projected by Anaximander when he
speaks of the production of light and night from the γóνιμον. This disruptive
and disintegrating character is what is expressed in the tearing of the light
cortex into smaller and disseminated spheres. 

Of the \textit{Puruṣa} we are also told that it has \lq\lq a thousand
eyes\rq\rq{}. Ignacio de Loyola speaks of a vision that frequently became to
him: a glow that sometimes took the shape of a snake, and seemed filled of
shining eyes. Monoimo the Arab teaches that the primitive man possessed \lq\lq
many faces and many eyes\rq\rq{}. Caesarius of Heisterbach says of the
\textit{Anthropos}, the first man, that it was like a sphere and had eyes
everywhere (\textit{ex omni parte oculata}). Angela of Foligno, en one ocassion,
saw in the host \lq\lq dos ojos esplendidísimos tan grandes que parecía que de
la hostia solamente quedaban los bordes\rq\rq{}. She also recalls: \lq\lq
($\ldots$) una vez no ante la hostia, sino en la celda, se me aparecieron ojos
con tan gran belleza y tan deleitables que ciertamente no creo que pierda nunca
la alegría\rq\rq{}.

The association between the emergence of consciousness and phenomena involving
multiple lights is not, as we see, unusual. Its vinculation with the
\textit{self} seems clear insofar as it occurs in experiences of connection with
the divine (be it in the Christian host or the Vedic \textit{ātman}). It is
worth mentioning here that, once more, we observe the double character of the
archetype---its tone of primordial phenomenon and of ultimate truth. As a last
comment, this small luminosities, expressed in polyoftalmic or astronomical
visions, probably refer the fragmentary phenomena of consciousness, while the
greater light (the divine eye or the sun as the eye of God, the
\textit{scintilla una} Kunrath spoke of, etc.) is a modulation of the
\textit{self}. That these luminosities have in general a spherical shape
corresponds to the unifying and integrating sense of the archetype, and we find
this in Anaximander too, since from the huks of light, small fires are dispersed
\lq\lq in some circles\rq\rq{}. Hippolytus of Athens also tells that, according
to Anaximander, \lq\lq the \textit{astrums} we see are generated as a circle of
fire, separating from the fire of the world, each surrounded by air\rq\rq{}.

What I mean to record here, then, after so many examples, is that Anaximander's
doctrine is not at all original. It fits perfectly with a mythological motif of
extensive documentation. Even in the idea that the principle governs all things,
which appears in Diogenes also, the parallel with the eye motif is evident,
insofar as the capacity to see it all is the hyperbolic form of government. So,
God has eyes that \lq\lq están sobre el camino del hombre\rq\rq{} and \lq\lq is
always watching everything we do\rq\rq{} (\textit{Job, 34:21}), and Chronos is
\lq\lq the one that sees it all\rq\rq to Sophocles and \lq\lq the devil that
everything sees\rq\rq in certain Greek funerary inscription. Argos Panoptes, the
thousand-eyed giant, is a guardian and protector.

The archetype, naturally, is not confined to ancient mysticism. Emerson, for
example, spoke of a mysterious sphere that is and is not always the same.

> Genius studies the causal thought, and, far back in the womb of things, sees
the rays parting from one orb, that diverge ere they fall by infinite diameters.
Genius watches the monad through all his masks as he performs the metempsychosis
of nature. Genius detects through the flies, through the caterpillar, through
the egg, the constant individual (...).

The primordial sphere, the thought that is the cause of all that is existent, is
the \textit{principia}---it is the Pythagoric monad in relation to which every
death is but a transmigration---it is the hidden Individual of the phenomenical
world. To be and not to be always the same is synthesis of opposites, a common
trait of the archetype, insofar as the \textit{self} is not a principle of
perfection, but of completeness. The totality of the psyche is precisely that: a
totality. Therein lay all shadow and all light, the chthonic and celestial
worlds.  That Anaximander considers the opposites as contained in the
Undetermined is, therefore, consistent with the general phenomenology of the
archetype. The symbol of Christ, to name a familiar one, has a distinct
enantiodromic nature. The coming of the Antichrist is not a mere prophetic
dream, but the product of a psychological law that took Christian mysticism to
the idea of a coming reign of shadow. This is particularly manifest in Gnostic
thought, among which we may point a particular passage of the \textit{Pistis
Sophia}. When Jesus was a child, a spirit that proclaimed himself to be his
brother, descended onto Mary. It looks just like Jesus, and she confounds the
spirit with him---but he comes from the inferior regions of Chaos. When
reminding Jesus of this episode, Mary tells him of the spirit: \lq\lq he
embraced thee and kissed thee, and thou also didst kiss him, and you became
one\rq\rq{}.

Similarly, Angela of Foligno says that, when she heard the Holy Spirit, she
wanted to see if she could forget this voice that spoke onto her, and tells:
\lq\lq I started looking at the vines, so as to forget those words ($\ldots$),
and wherever I would turn I would tell myself: \lq This is my creature\rq{}. And
I felt an ineffable divine joy\rq\rq{}.

The mystical experience was so strong that she felt the whole of creation
belonged to her as it originally belonged to God. But immediately after she
says: \lq\lq And then to my memory came all my sins and my vices, and I saw in
my nothing but sins and defects\rq\rq{}. 

To Angela of Foligno, the simultaneity of the sweet mystical experience and the
assault of her vices and sins onto her memory was shocking. We have mentioned the
Gnostic intuition of the inseparability of Christ and Antichrist, but here we
find ourselves with a spontaneous experience of this intuition in a catholic
christian, who probably never thought the voice of the Holy Spirit could come
accompanied by such horrendous darkness. But if we understood anything about the
\textit{self} archetype, such simultaneity cannot surprise us. The experience
was hardly interpretable from the framework of catholic dogma, but from a
psychological perspective it is impeccably consistent. In general, and in
accordance to what we have thus far exposed, the archetype manifests as affected
by \lq\lq equal yet opposite forces\rq\rq.

The exposition of so many examples is not an act of vain erudition. With some
luck, I have produced a decent record of the similarities between Anaximander's
doctrine and countless other myths. I have nothing else to say, insofar as the
whole matter is still obscure to me, and the more I advance with these notes the
more I mistrust my own conclusions. I should prefer to close my notes on
Anaximander here and resume then in another period of my life, when some of
these ideas have matured in my mind. I an contempt to have put in writing that
was happening in my mind as I restudied presocratic philosophy. 


\end{document}
