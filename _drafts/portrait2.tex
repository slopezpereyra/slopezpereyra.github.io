\documentclass[a4paper, 12pt]{article}

\usepackage[utf8]{inputenc}
\usepackage[T1]{fontenc}
\usepackage{textcomp}
\usepackage{amssymb}
\usepackage{newtxtext} \usepackage{newtxmath}
\usepackage{amsmath, amssymb}
\newtheorem{problem}{Problem}
\newtheorem{example}{Example}
\newtheorem{lemma}{Lemma}
\newtheorem{theorem}{Theorem}
\newtheorem{problem}{Problem}
\newtheorem{example}{Example} \newtheorem{definition}{Definition}
\newtheorem{lemma}{Lemma}
\newtheorem{theorem}{Theorem}
\DeclareMathAlphabet{\mathcal}{OMS}{cmsy}{m}{n}

\begin{document}

Bajo las tenues y pequeñas luces que colgaban sobre el aire nocturno del patio,
el rostro de aquella joven dejaba traslucir vestigios guaraníticos, africanos y
europeos, que conjugaban en él una hermosura sin nombre. El piso de ladrillo era
cercado por estrechas franjas de tierra fresca y piedra blanca, desde las cuales
decenas de plantas tropicales conjuraban una especie secreta de encanto similar
al de sus ojos negros. En efecto, había en ellos todo lo que en América creemos
—o fingimos creer— haber enterrado, otorgándole lo que alguien describió
correctamente como una «mirada de cinco siglos». Sentada en silencio, descalza y
con un cigarrillo en la mano, la salvaje hojarasca de un jazmín brasilero
abierta a poca distancia de su oscuro cabello, como infinitos dedos que sueñan
dar sonido a un instrumento delicioso y primitivo, ella también se preguntaba
cuáles fueron los hombres, los infinitos hombres, que la habían llevado allí.
Ella no lo sospechaba, pero ella era —como todas las cosas— poco más que un eco.
Y yo la contemplaba con el silencio respetuoso de los que invaden un cementerio,
disimulando mi distante asombro, aunque todo en mí sintiera que una
reverberancia de voces milenarias iba aflorar, en cualquier momento, desde sus
labios.









































\end{document}



