\documentclass[a4paper, 12pt]{article}

\usepackage[utf8]{inputenc}
\usepackage[T1]{fontenc}
\usepackage{textcomp}
\usepackage{amssymb}
\usepackage{newtxtext} \usepackage{newtxmath}
\usepackage{amsmath, amssymb}
\newtheorem{problem}{Problem}
\newtheorem{example}{Example}
\newtheorem{lemma}{Lemma}
\newtheorem{theorem}{Theorem}
\newtheorem{problem}{Problem}
\newtheorem{example}{Example} \newtheorem{definition}{Definition}
\newtheorem{lemma}{Lemma}
\newtheorem{theorem}{Theorem}
\DeclareMathAlphabet{\mathcal}{OMS}{cmsy}{m}{n}

\begin{document}

$\S$  Bajo las tenues y pequeñas luces que colgaban sobre el aire nocturno del
patio, su rostro dejaba traslucir los vestigios guaraníticos, africanos y
europeos que conjugaban en él una hermosura sin nombre. El piso de ladrillo era
cercado por estrechas franjas de tierra fértil, alfombrada de perladas
piedrecitas blancas, desde las cuales decenas de plantas tropicales conjuraban
una especie secreta de encanto similar al de sus ojos negros. En efecto, había
en aquellos ojos todo lo que en América creemos —o fingimos creer— haber
enterrado, otorgándole lo que alguien describió correctamente como una «mirada
de cinco siglos». Sentada en silencio, descalza y con un cigarrillo en la mano,
la salvaje hojarasca de un jazmín brasilero abierta a poca distancia de su
oscuro cabello, como infinitos dedos que sueñan dar sonido a un instrumento
delicioso y primitivo, ella también se preguntaba quiénes fueron los hombres, los
infinitos hombres, que la habían llevado allí. Ella no lo sospechaba, pero ella
era —como todas las cosas— apenas algo más que un eco. Y yo la contemplaba con
el silencio respetuoso de los que invaden un cementerio, disimulando mi distante
asombro, aunque todo en mí sintiera que una reverberancia de voces milenarias
iba aflorar, en cualquier momento, desde sus labios.

---¿Qué hace?---me pregunté, temiendo que lograra desnudar mi corazón, y una voz
interior me dijo «está esperando». Sus pies descalzos se unieron el uno al otro
como dos trenzas juveniles o dos lianas tropicales; sus dedos se entrelazaron y
recorrieron su cuerpo como enredaderas silvestres; el patio se deshizo y dio
lugar a una hierba milenaria, inundando mis sentidos cruel o tiernamente. Nada
quedó a mi alrededor. Solo y confundido, miré la hermosa flor que había nacido
ante mí, que ya no me increpaba con cinco siglos de mirada. Y repetí en
silencio: «está esperando».

\newpage

$\S$ Sobre la hierba prolija del cementerio, no sé si sus sandalias negras con
encajes brillantes ofendían o alegraban la solemnidad de los muertos. su remera
negra tenía un precioso y seductor escote lágrima, dentro del cual un suave
lunar perlaba una piel bronceada durante aburridas siestas de tomar sol. conversaba
con cierta impertinencia sobre el consumo de drogas, la vida después de la
muerte y la naturaleza del alma, sentada junto a la tumba de nuestra vieja amiga,
que se suicidó tras años de adicción y deterioro psicológico y moral. acomodó
las flores y limpió la tierra y el agua de lluvia que se acumulaban sobre la
lápida con una delicadeza y un cariño que me parecieron maternales. el amor que
sintió alguna vez por la amiga muerta permanecía intacto, y aunque otros
visitantes pudieran juzgar sus ropas hermosas, su conversación desafortunada y
un tanto ruidosa, su actitud despreocupada---en fin, su aparente impertinencia
en aquel silencioso cementerio---la bondad de su corazón, la sinceridad de sus
dolores íntimos, y su espontánea calidez eran igual de aparentes.

pocos meses antes, ella fue mi acompañante durante mi primera visita al
cementerio, una tarde lluviosa y melancólica en la que mi dolor todavía era
fresco y lacerante. esta vez, para mi sorpresa, no sentí nada. mi atención
estaba centrada en los vivos, en el hermoso rostro de mi acompañante, en su
extraña conversación, que yo escuchaba con cierta pasividad resignada. ella
vivía una vida distinta a la mía, en un universo distinto al mío. la hermosura de
sus ropas, la delicadeza de su maquillaje, el bronceado de su piel, el brillo de
sus sandalias, su escote lágrima---en resumen, todos los detalles superficiales que
podrían sugerir un alma superficial---no revelaban ni vanidad vulgar ni, al fin
y al cabo, siquiera sana coquetería. en su consciencia, cada instante era
signado por la duda de sí misma, por la abrasante convicción de que una mujer
sólo es tomada en serio si satisface cierto criterio perverso de perfección, que
nadie---tampoco yo---podría amarla si lo más mínimo de su fuero externo sufriera
la más pequeña alteración. 

desde su tierna infancia, padres y hermanos le enseñaron---sin mediar palabra
alguna---que la delgadez y la belleza garantizaban a las mujeres un trato
respetuoso y diferencial, humillándola por su intermitente gordura e inculcando
en ella la creencia, no del todo errada en muchos círculos, de que el amor
otorgado a una persona es proporcional a la gratificación estética que es capaz
de producir. yo la comprendía y soñaba mostrarle que el amor puede ser perfecto,
pero nos presentía a la vez separados por un río inabarcable, y al verla abrazar 
tan firmemente las mismas convicciones que destruían su posibilidad de
ser feliz, la consideraba---en cierta medida---una partícipe del crimen. 

yo no debía «despertarla»: sus ojos estaban bien abiertos en un mundo, en una
vida, donde todas aquellas cadenas que a mí me parecían tan lejanas eran
verdaderas, tenían un peso espiritual que dificultaba real, incluso físicamente
sus pasos. por eso la compadecía sin saber del todo cómo hablarle.

unas horas después nos despedimos, solos en la oscuridad de la noche, mirándonos
frente a frente en la calle de tierra que conduce a la quinta de mi padre,
rodeados de luciérnagas cuya aparición intermitente reflejaba la inconstante
llama de la felicidad a lo largo de su vida. «amo las luciérnagas», me dijo, sin
sospechar la triste ironía que inundó mi corazón. besé sus labios y nos
abrazamos en silencio. éramos irremediablemente diferentes---ella lo supo y yo
también---y sin embargo sabíamos comprender los duelos secretos del otro. y eso
también es una forma del amor.





































\end{document}



