\documentclass[a4paper, 12pt]{article}

\usepackage[utf8]{inputenc}
\usepackage[T1]{fontenc}
\usepackage{textcomp}
\usepackage{amssymb}
\usepackage{newtxtext} \usepackage{newtxmath}
\usepackage{amsmath, amssymb}
\newtheorem{problem}{Problem}
\newtheorem{example}{Example}
\newtheorem{lemma}{Lemma}
\newtheorem{theorem}{Theorem}
\newtheorem{problem}{Problem}
\newtheorem{example}{Example} \newtheorem{definition}{Definition}
\newtheorem{lemma}{Lemma}
\newtheorem{theorem}{Theorem}


\begin{document}

    
\section{El proceso como abstracción}

\subsection{Introducción}

Una de las abstracciones fundamentales en un sistema operativo es la del
\textit{proceso}. Si pensamos, antes de definirlo, qué es un proceso, vemos que
no es trivial llegar a una idea provechosa. En principio, todo lo que una
computadora hace es, precisamente, computar: ¿es un proceso una serie de
computaciones? Si es así, ¿toda serie de computaciones es un proceso? ¿No puede
un conjunto \textit{no consecutivo} de computaciones constituir un proceso? ¿Es
una única computación un proceso? Si no, ¿cuál es la cantidad de computaciones
que hace que algo sea un proceso?

No vamos a discutir estas ideas a fondo, pero es interesante plantearlas.
Por el momento, diremos que un proceso es simplemente un programa que se 
está ejecutando, y asumimos que todos entendmos qué queremos decir con un 
programa.

En sí mismo, el programa es algo inerte: es un conjunto de instrucciones en el
disco esperando que algo difunda en él el soplo de la vida, como el barro
originario en los mitos hebreos. Es precisamente el sistema operativo quien 
insufla vida en el programa, haciéndolo correr.

Sin embargo, el problema que enfrenta el sistema operativo es muy diferente al
que enfrentó Yavhé cuando quiso darle vida a la materia inerte. Dios es
todopoderoso, pero un procesador no. Si tenemos un solo procesador, ¿cómo
podemos darle vida a múltiples procesos? La respuesta simple es: \textit{no
podemos}. La respuesta interesante, sin embargo, es: podemos simular que lo
hacemos. Es decir, podemos olvidarnos de nuestra pretensión de ser Dios, y 
contentarnos con ser un practicante de ilusiones. Menos glorioso, sí, pero 
es lo que nos toca.

\subsection{Virtualización}

Cuando utilizando un único hardware para simular una multiplicidad simultánea
de procesos, decimos que estamos \textit{virtualizando}. La forma básica en que
el sistema operativo (SO) virtualiza la CPU es ejecutar un proceso durante un
quantum (corto) de tiempo $t$, deteniéndolo y corriendo otro, y así
sucesivamente. Esta técnica se llama \textit{time sharing}, y sacrifica
\textit{performance} a cambio de la ilusión de simultaneidad.  

\begin{quote}
    Imaginemos dos dos canicas situadas en los extremos opuestos de una mesa.
    Imaginemos que las muevo muy poco, primero una y luego la otra, hacia el
    centro de la mesa, hasta que se tocan. Si mis manos fueran lo
    suficientemente rápidas, creerías ver que la canicas se mueven al mismo
    tiempo.
\end{quote}


Para implementar la Virtualización de la CPU, y hacerlo bien, el SO necesita 
$(a)$ procedimientos de bajo nivel y $(b)$ inteligencia de alto nivel. A los 
procedimientos de bajo nivel se los llama \textit{mecanismos}, y consisten 
en protocolos que implementan una funcionalidad. La inteligencia de 
alto nivel se refiere a políticas que determinan criterios y decisiones 
en el sistema operativo.

\begin{quote}
    Por ejemplo, una política puede referirse a qué hacer cuando múltiples procesos 
    llegan a la CPU al mismo tiempo. ¿Cuál debe ejecutarse primero?
\end{quote}

\subsection{Machine state del proceso}

Para comprender mejor qué es un proceso, debemos entender su \textit{machine
state}; es decir, qué puede el programa leer o cambiar cuando está
ejecutándose.

Un componente obvio del \textit{machine state} es la memoria. Con
\textit{memoria} nos referimos tanto a las instrucciones del proceso, los datos
que el programa puede leer o escribir, etc. Otro componente son los
\textit{registros}, como el \textit{instruction pointer}, \textit{program
counter}, etc. Por último, la información de \textit{input/output} (I/O), como
una lista de los archivos que el proceso tiene abiertos, es parte de su
\textit{machine state}.

\subsection{API del proceso}

Una idea preliminar de la API de un proceso es dada por la siguiente 
lista de interfaces:

\begin{itemize}
    \item \textit{Crear:} Un SO debe poder crear procesos nuevos.
    \item \textit{Destruir:} Un SO debe poder destruir procesos que se están
        ejecutando.
    \item \textit{Esperar:} Un SO debe poder esperar que un proceso se 
        termine de ejecutar. 
    \item \textit{Control vario:} Además de destruir o esperar por un proceso, 
        otros mecanismos de control son posibles, como suspender 
        un proceso para resumirlo más tarde. 
    \item \textit{Status:} Un SO usualmente provee interfaces que 
        dan información acerca de un proceso, como hace cuánto 
        se está ejecutando y cuál es su estado.
\end{itemize}

\subsection{Creación de procesos}

¿Cómo es que el SO transforma programas en procesos? Es decir, ¿cómo 
da inicio a la ejecución de un programa?

El primer paso es cargar (\textit{load}) el código y cualquier dato estático
necesario en la memoria; particularmente, en el \textit{address space} del
programa. Este proceso consiste esencialmente en leer bytes del disco y
ubicarlos en alguna región de la memoria. Los SOs modernos realizan este
proceso \textit{lazily} (perezosamente), cargando los bytes de a poco y a
medida que se van necesitando.

Una vez el código estático ha sido ubicado en memoria, hay poco más 
que deba hacerse. Se aloca algo de memoria para el stack y el heap 
del programa, y se realizan algunas tareas de inicialización. Por 
ejemplo, en sistemas UNIX, cada proceso tiene tres \textit{file descriptors}:
standard input, standard output, y standard error.

\begin{quote}
    Un \textit{file descriptor} es una representación abstracta que el 
    sistema operativo hace para acceder y manejar archivos, sockets,
    y pipes. Cada vez que un proceso abre un archivo u otro recurso
    de I/O, el kernel le asigna a ese recurso un entero que lo identifica,
    Ese es su \textit{file descriptor}. De allí en más, el programa sólo
    usa ese identificador para operar con el recurso.  

    El \textit{standard input} es el \textit{file descriptor} 0, y 
    su propósito es recibir datos de input para un proceso,
    típicamente del teclado. En general, a menos que se especifique 
    lo contrario, el \textit{standard input} lee de la terminal.
    Cada vez que llamas un comando de terminal, el mismo se envía
    a un programa a través de standard input.
\end{quote}






























\end{document}



