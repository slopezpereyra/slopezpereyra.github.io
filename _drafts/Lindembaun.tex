\documentclass[a4paper, 12pt]{article}

\usepackage[utf8]{inputenc}
\usepackage[T1]{fontenc}
\usepackage{textcomp}
\usepackage{amssymb}
\usepackage{newtxtext} \usepackage{newtxmath}
\usepackage{amsmath, amssymb}
\newtheorem{problem}{Problem}
\newtheorem{example}{Example}
\newtheorem{lemma}{Lemma}
\newtheorem{theorem}{Theorem}
\newtheorem{problem}{Problem}
\newtheorem{example}{Example} \newtheorem{definition}{Definition}
\newtheorem{lemma}{Lemma}
\newtheorem{theorem}{Theorem}


\begin{document}

    
Given a first order type $\tau$, we use $S^\tau$ to denote the set of $\tau$-sentences,
i.e. formulas with no free variables. Let $T = (\Sigma, \tau)$ a theory. 
Then we can define 

\begin{equation*}
    \varphi \dashv \vdash_T \psi \iff T \vdash \left( \varphi \leftrightarrow \psi \right) 
\end{equation*}

or equivalently 

\begin{equation*}
    \dashv \vdash_T = \left\{ (\varphi, \psi) \in S^\tau : T \vdash \left( \varphi \leftrightarrow \psi \right)  \right\} 
\end{equation*}

It is easy to see through natural deduction that $\dashv \vdash_T$ is 
an equivalence relation.

We say that a sentence $\varphi \in S^\tau$ is refutable when $(\Sigma, \tau)
\vdash \neg \varphi$. It is also easy to prove via natural deduction that the
set of theorems and the set of refutable sentences form two distinct equivalent
classes under $\dashv \vdash $. In other words, any pair of theorems 
imply each other, and any pair of refutable sentences impliy each other.

Let $\left[ \varphi \right] $ denote the equivalent class of $\varphi$
with respect to $\dashv \vdash_T $. Define $\textbf{s}$ 
as

\begin{equation*}
    \left[ \varphi \right]  \textbf{ s } \left[ \psi \right] = \left[ \left( \varphi \lor  \psi \right)  \right] 
\end{equation*}

In short, let the supremum of the equivalent classes of two formulas be the
class of their disjunction. In particular, the supremum of (the classes of) two 
theorems is the class of theorems, and the supremum of (the classes of) a 
theorem and a refutable sentence still is the class of theorems. Only 
the supremum of (the classes of) two refutable sentences is the class of 
refutable sentences.

Define the corresponding operations for the infimum and complement of 
equivalent classes: 

\begin{align*}
    &\left[ \varphi \right] \textbf{ i } \left[ \psi \right] = \left[ \left( \varphi \land  \psi \right)  \right] \\ 
    &\left( \left[ \varphi \right]  \right)^c = \left[ \neg \varphi \right] 
\end{align*}

With $0^T$ as the set of theorems in $T$, $1^T$ as the set of refutable
sentences in $T$,

\begin{equation*}
    \mathcal{B} = \left( S^\tau / \dashv \vdash_T, \textbf{s}, \textbf{i}, ^{^T}, 0^T, 1^T \right) 
\end{equation*}

is a Boolean algebra. This algebra is called *Lindenbaum*'s algebra 
(under theory $T$). Observe that, by virtue of Dedekind's theorem,
this algebra has an associated partial order. This order is described 
as follows: 

\begin{equation*}
    \left[ \varphi \right] \leq \left[ \psi \right]  \iff T \vdash \left( \varphi \to \psi \right) 
\end{equation*}

A Lindenbaum's algebra is used to prove Godel's completeness theorem.

# Further description 

~

Let $\psi \in S^\tau$ s.t. $T \not\vdash \psi$. It is easy to see 
that natural deduction allows us to conclude, 
$$

















\end{document}



