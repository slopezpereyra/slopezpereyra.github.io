\documentclass[a4paper]{article}

\usepackage[utf8]{inputenc}
\usepackage[T1]{fontenc}
\usepackage{textcomp}
\usepackage{amsmath, amssymb}
\edef\restoreparindent{\parindent=\the\parindent\relax}
\DeclareSymbolFont{letters}{OML}{ztmcm}{m}{it}
\usepackage{parskip}
\restoreparindent
\newtheorem{problem}{Problem}
\newtheorem{example}{Example}
\newtheorem{lemma}{Lemma}
\newtheorem{theorem}{Theorem}
\newtheorem{problem}{Problem}
\newtheorem{example}{Example}
\newtheorem{definition}{Definition}
\newtheorem{lemma}{Lemma}
\newtheorem{theorem}{Theorem}

\begin{document}

\textbf{Anaximandro y sus símbolos}

\textbf{Introducción}

El pensamiento presocrático, pese a los esfuerzos de los últimos dos siglos,
permanece en una casi total oscuridad. La sombra de la exégesis aristotélica,
que ve en él sólo un esfuerzo racional por describir la Φύσις, se cierne sobre
prácticamente toda la tradición filológica. Parte de él, indudablemente, se
trata de un esfuerzo genuino por discernir las propiedades naturales del mundo.
Pero el hecho de que incluso sus más naturales postulados deriven, como regla
general, de un sustrato mítico, y que ese sustrato sea la piedra angular de sus
doctrinas, señala que esa parte acaso sea también la más superficial. Todo ἀρχή
es, en última instancia, arbitrario e irracional, y la elección de uno u otro no
puede explicarse como una intelección estrictamente científica. Son imágenes
cosmogónicas de carácter especulativo, cuya similitud con expresiones
mitológicas las hacen dignas del epíteto de míticas. En τὸ ἄπειρον de
Anaximandro y el νοῦς de Anaxágoras, por ejemplo, esto es particularmente claro.
El sustrato mítico de la filosofía presocrática es un problema, porque la
posiciona en directa oposición a lo que, desde Aristóteles, se ha querido ver en
ella; a saber, una filosofía natural, libre de cuestiones metafísicas. Los
exégetas lidiaron con este problema de tres maneras. a) La primera consistió en
proferir un colosal esfuerzo, a veces admirable, por demostrar que las
expresiones míticas eran estrictamente empíricas o racionales —o simplemente
asumir que ese era el caso—. Así, por ejemplo, el χάος de Hesíodo es un proceso
lógico de regresión de lo más complejo a lo más simple (Gigon, 1971),
Anaximandro pone la tierra en el centro del mundo obedeciendo a una intuición
geométrica (Jaeger, 1933), y Aristóteles enseña que Tales escoge el agua como
ἀρχή por su transmutabilidad. Pero es ilícito proyectar un espíritu naturalista
sobre postulados que, tomados por sí mismos, son más cercanos al mito que a la
observación empírica. El caso de los presocráticos no es distinto del de la
alquimia, descrita a veces sólo como un estudio pre-científico de la materia,
cuando se trata además de un claro universo simbólico (Jung, 1944; Roob, 2014). 

b) Otras veces,
se reconoció la urdimbre mitológica que teje las doctrinas presocráticas, pero
no se vio en ella más que un fenómeno estrictamente griego, una diversión
aristocrática propia de la dionisíaca embriaguez del alma griega. Así lo vieron,
por ejemplo, Nietzsche y Colli (1948). Esta hipótesis, sin embargo, adolece de
dos problemas. Primero, pasa por alto la inmensa similitud entre las expresiones
simbólicas de los presocráticos y las de muchos mitos. No es el momento de
señalar estas similitudes, pero vamos a tratar con un gran número de ellas más
adelante. Ellas enseñan que, en su aspecto simbólico, el fenómeno presocrático
tiene más bien poco de singular, y que lo que se ha confundido con diversiones
del espíritu griego acaso sean imágenes arquetípicas. En segundo lugar, las
expresiones que quieren interpretarse como una extravagancia eminentemente
griega están en completa oposición con la cosmovisión típica de los griegos.
Esto se hará claro al tratar la última hipótesis con que la tradición trató de
resolver este problema. c) Me refiero a aquella según la cual una influencia
oriental —ya babilónica, ya pérsica o egipcia— es el origen de esas expresiones.
Esta hipótesis es la que merece más nuestra atención, porque es la más
plausible. Es un hecho que la Grecia presocrática estaba en continuo contacto
con otros pueblos, e incluso que varios filósofos presocráticos fueron agentes
coloniales o viajeros en tierras orientales. Pero es preciso notar que, para el
griego, era la naturaleza la total expresión de lo divino. Los dioses no eran
agentes de fenómenos milagrosos, sino la fuerza que hacía girar la rueda del
orden natural del mundo. El sentimiento religioso se encontraba en el seno de la
realidad objetiva y no en lo portentoso. Por esta razón Otto (1929) ha dicho que
la cosmovisión griega tiene como carácter básico una «idealidad natural o una
naturalidad ideal», e hizo Gigon (1971) la clara observación de que las figuras
de la Teogonía no son traídas a colación porque sean dioses, sino porque no
pueden faltar las zonas por ellas representadas en el cuadro de conjunto del
todo.

El oriental$^{1}$, por el contrario, concebía la realidad objetiva no como el fondo
de la divinidad, sino como el velo ilusorio que nos separa de ella. Lo divino
yacía en un plano inmaterial que sólo coincidía con la realidad sensible a
través de correspondencias ocultas. El alma inmaterial, de casi nula relevancia
para el griego, ocupaba un papel preponderante. Así, mientras el griego se veía
a sí mismo como parte de una realidad objetiva en que lo divino es manifiesto,
el oriental concebía la realidad como el sueño que oculta la divina esencia y
del que debía liberarse. Donde el griego percibía lo orgánico y manifiesto, el
oriental intuía un misterio y un símbolo. El primero miraba el sol y sentía que
contemplaba la cálida luz de Apolo, que vive en la lejanía. El otro veía una luz
que sólo existe por haber sido creada por sus propios ojos. Es vano describir el
espíritu religioso de dos pueblos en tan poco espacio. Sin embargo, la síntesis
anterior debería bastar para mostrar que la cosmovisión griega proveía un marco
delimitado de experiencia, una perspectiva completa del mundo, radicalmente
opuestos a los orientales. ¿Cómo, entonces, sería tan susceptible de su
influencia? Porque, en efecto, si un individuo es, por así decirlo, «tocado» por
una noción cualquiera, debe existir en él algo susceptible de ser tocado por
ella, algo a lo que ella pueda apelar; y lo mismo puede decirse de un pueblo. Si
el fenómeno se redujera sólo a las peculiaridades culturales del pueblo griego,
lo esperable habría sido que éste fuera impenetrable para las imágenes
religiosas orientales, de modo análogo a como puede decirse que somos sordos a
una lengua extraña. El asunto es incluso más problemático cuando consideramos
que, tal como se dijo antes, en el momento en que el pensamiento presocrático
florecía, la cultura griega, lejos de encontrarse debilitada o en crisis, se
hallaba en un intenso período de expansión colonial y poderío (Lane Fox, 2007).
Era, en efecto, ebria de orgullo propio, y no le era de ningún modo necesario
incorporar mitos extranjeros del todo incompatibles con los propios. Si los
pueblos orientales participaron en esto lo hicieron como factor incidente, pero
no suficiente ni, en última instancia, necesario. Una hipótesis que pueda
completar estas lagunas y evitar estas falencias debe, en primer lugar, no
proyectar sobre expresiones simbólicas un carácter racional o empírico; además,
ha de explicar su carácter anti-idiosincrático y, por último, tomar en cuenta su similitud con otros mitos. Este
artículo sostiene que hay una correspondencia entre tales expresiones y
determinados arquetipos psicológicos, cuya manifestación en otros universos
simbólicos ha sido documentada, pero cuyo influjo en los presocráticos fue
pasado por alto por una tradición hermenéutica no familiarizada con la ciencia
psicológica. No se pretende negar que el descubrimiento de la Φύσις era una
empresa genuina, sino señalar que esta fue propicia a la expresión de procesos
psíquicos paralelos, cuya proyección era facilitada, por un lado, por la
misteriosa naturaleza de los procesos cosmogónicos y astronómicos que se
pretendían explicar; por otro, por la falta de un método científico que
protegiera, por así decirlo, al objeto de estudio de ser contaminado por tales
proyecciones. La naturaleza arquetípica de ciertos postulados presocráticos,
debe decirse, fue mencionada por otros (véase, por ejemplo, Campbell, 1949 y
Jung, 1969), pero nunca examinada ni establecida del todo. En relación al
significado de arquetipo psicológico, el lector puede consultar la definición
original en la obra de C.G. Jung, especialmente el noveno volumen de sus
Complete Works. Pero, si bien en términos generales nuestra terminología
psicológica se vale de la jerga de la psicología analítica, no debería pensarse
que suscribimos a una concepción psicoanalítica del inconsciente. Más bien, por
arquetipos entendemos patrones de representación de naturaleza afectiva,
determinados filogenéticamente y evolutivamente arcaicos. Que compartimos un
sustrato afectivo común con, al menos, todos los mamíferos es un hecho
neurocientífico establecido (Panksepp, 2000; Panskepp y Gordon, 2003; Panskepp,
2005; Panskepp, 2011). Más aún, la neurociencia ha ofrecido evidencia en favor
de la existencia de arquetipos psicológicos específicos, como el arquetipo del
self, y su posible localización en los circuitos cerebrales (Alcara et al.,
2017). Desde luego, tal evidencia no es conclusiva y debe tomarse con
precaución. Pero podemos poner a un lado este problema utilizando una
perspectiva pragmatista. En el contexto de esta investigación, la etiología
específica de los arquetipos es irrelevante, en tanto la idea refiere las mismas
consecuencias prácticas cualquiera sea el origen que se le atribuya. En lo que
toca a sus efectos, da igual que sean imágenes culturalmente transmitidas,
generación tras generación, a lo largo de los siglos, o modulaciones de ciertos
patrones de representación filogenéticamente heredados. Sea su origen endógeno o
exógeno, o cualquier


proporción de ambos, el concepto implica los mismos efectos. En otras palabras,
su valor es ortogonal a la hipótesis etiológica que se prefiera, y reside en su
existencia más que en la razón de su existencia. El estudio de los presocráticos
desde esta perspectiva nos enfrenta necesariamente con el problema de las
imágenes cosmogónicas de diversos mitos y su especificidad psicológica. Un
abordaje interdisciplinario del problema es, como puede verse, necesario. Si
bien la investigación se funda en la psicología, tanto la naturaleza de nuestro
objeto como el método comparativo a utilizar vuelven inevitables las
implicancias de la mitología, la religión comparada y la filología. El problema
es relevante en un doble sentido. En primer lugar, la perspectiva de este
artículo abre un campo lleno de potencial interpretativo en un terreno que ha
avanzado poco o nada en el curso de este siglo, y en cuya antigua sementera el
mundo occidental tiene raíces: el pensamiento griego arcaico. Confío en que el
modesto aporte de la psicología de los arquetipos contribuirá a la comprensión,
aunque sea parcial, de sus postulados menos aprehensibles. En segundo lugar, los
arquetipos no son una idea metafísica, una intuición mística ni una abstracción.
Son, por el contrario, un hecho psicológico. El avance de la neurociencia en la
suplementación de datos acerca de la naturaleza de lo arquetípico debe,
necesariamente, ser acompañado por el análisis concreto de sus manifestaciones
culturales y comportamentales. Así como es vano el estudio neuropsicológico de
las emociones sin una paralela inquisición en su valor conductual concreto, debe
existir una retroalimentación entre el estudio de los patrones neurales que
median las imágenes arquetípicas y el análisis de sus expresiones específicas en
la vida humana. Su relevancia en la vida individual y social es consecuencia de
que se trata de estructuras psicológicas arcaicas de carácter afectivo, cuyo rol
en la construcción de significado parece evidenciado por su obstinada aparición
en mitos y cosmogonías. Son, con toda probabilidad, el producto de «value-
encoding neural systems» (Panksepp y Burgdorf, 2003; véase además Panksepp,
2005, Panksepp, 2011 y Alcara et. al., 2017) asociados a la articulación de
sentido (meaning). Si pudiéramos concluir que las expresiones míticas del
pensamiento presocrático corresponden a determinados arquetipos,
independientemente de si provienen o no de

oriente, podríamos explicar que el espíritu griego haya podido producirlas por
sí solo o, en todo caso, ser susceptible a su influjo. De no existir esa
correspondencia, como se ha hecho notar, no habría habido en el griego
susceptibilidad alguna a la que tales elementos, en caso de ser extranjeros,
pudieran apelar, dada su irremediable incompatibilidad de estos con la
cosmovisión griega. Es claro que la pregunta que queda abierta es si tal
correspondencia de hecho existe. Demostrar esto sobre el conjunto del
pensamiento presocrático es una tarea colosal. Lo que me propongo es demostrarlo
sobre un caso concreto: Anaximandro. Particularmente, me centraré en el
fragmento D-K 12 A 10, en tanto su carácter cosmogónico vuelve más fácil su
análisis comparado y, a la vez, más clara su naturaleza arquetípica. II. CASO DE
ESTUDIO: ANAXIMANDRO Pretendo mostrar que la cosmogonía de Anaximandro, en
concreto como es descrita en el fragmento D-K 12 A 10, es una proyección del
arquetipo del self, y describe la experiencia inmemorial de la emergencia de la
consciencia del seno de lo inconsciente 2 . El fragmento en cuestión proviene de
Plutarco: «Anaximandro … dice que lo infinito es la causa de la generación y
destrucción de todo, a partir de lo cual —dice— se segregan los cielos y en
general todos los mundos, que son infinitos. Declara que su destrucción y, mucho
antes, su nacimiento se producen por el movimiento cíclico de su eternidad
infinita… Dice también que, en la generación de este cosmos, lo que de lo eterno
es capaz de generar lo caliente y lo frío fue segregado, y que, a raíz de ello,
una esfera de llamas surgió en torno al aire que circunda a la tierra, tal como
una corteza [rodea] al árbol; al romperse la [esfera] y quedar encerradas [sus
llamas] en algunos círculos, se originaron el sol, la luna y los astros.» En
este fragmento, Anaximandro enseña que el principio de todas las cosas es «lo
infinito», traducido también como lo Ilimitado o lo Indeterminado. En el seno de
este principio están contenidos los opuestos y los mundos, que se desprenden de
él. Lo Ilimitado produce el 2 Una vez más, no hago uso del término inconsciente
en su sentido psicoanalítico. Aunque tal sentido no es incompatible con el
contenido de este trabajo, prefiero la noción más terrenal y flexible de lo
simplemente no-consciente. Que la consciencia emerge de una organización neural
originalmente inconsciente es un hecho generalmente aceptado, aunque la
naturaleza de esa emergencia es discutida. Véase, por ejemplo, Damasio y
Carvalho The nature of feelings: evolutionary and biological origins (2013).

mundo de continuo. No actúa como creador en un único momento del pasado, sino
que es el agente continuo e inagotable de la engendración y extinción de las
cosas, el fondo del movimiento cósmico. De lo Ilimitado se segregó una «semilla»
(γóνιμον, «lo que … es capaz de generar») de la luz y la noche 3 . La palabra
γóνιμον es una sutileza importante. De lo Ilimitado no se separan los opuestos
de la luz y la oscuridad, sino aquello que es capaz de engendrarlos. La luz y la
noche constituyen la oposición originaria, engendrada a través de la semilla
segregada de lo Ilimitado. Desde una perspectiva psicológica, puede tratarse de
una proyección de la separación inmemorial de la consciencia del seno primordial
de lo inconsciente, análogamente a las palabras de Dios: «hágase la luz». Se
entiende que la luz y la noche cubrieron la Tierra como la corteza a un árbol,
en una doble capa. La noche era la interior, la luz la exterior. Pero la cáscara
de luz se rasga, aunque no se nos dice por qué, y el fuego que antes era una
totalidad se dispersa en pequeñas formas circulares. Son las estrellas y los
astros que vemos. El mundo en el que se desprende la semilla, el estado cósmico
en que se produce la segregación originaria, es el espacio vacío, el χάος mítico
que aparece primero en la Teogonía de Hesíodo. La palabra χάος no significa,
como quiso Ovidio, un desorden aleatorio. Expresa más bien una «hendidura» o
«cavidad». Se usaba, por ejemplo, para expresar el abrir mucho la boca o la
abertura de una cueva. Su formulación como estado primigenio es un motivo
arquetípico de sobrada documentación. Génesis 1:2 nos habla del abismo que solo
era poblado por tinieblas: «And the earth was without form, and void; and
darkness was upon the face of the deep». También el Hambre originaria védica, en
la medida en que el hambre es, en cierto modo, un vacío interior, o el símbolo
universal de la cueva como lugar de transformación —es decir, de generación a
partir de la destrucción—, guardan relación con la idea de χάος. Vale añadir,
aunque sea brevemente, el Mercurio de la alquimia, del que dice Paracelso que es
«the concealer of the rest [of things] —their corporeal vessel (...)—»
(Paracelsus &amp; Ed, 2018). Pero lo que por ahora nos importa del χάος es que,
en Anaximandro,

es la semilla de la luz y de la sombra lo que altera su estado originario. Ella
es quien induce contenido y forma en el vacío, cubriendo el mundo con el doble
manto de su fruto. Debe decirse que, en un sentido psicológico, el surgimiento
de la consciencia es un fenómeno a la vez individual y colectivo. Se produjo,
por un lado, en el plano evolutivo de la especie; por otro, acontece a cada
individuo en el curso de su propio desarrollo psicobiológico. Es claro, de
acuerdo a lo que podemos observar de la vida psíquica animal, que ella tiene a
lo inconsciente como forma primordial, y que la consciencia emerge de ese seno
de manera continua, es decir, distinguiéndose en una cuestión de grado y no de
naturaleza (Damasio, Ref). El ser consciente de pensar es posterior al acto de
pensar. Pero el surgimiento de la consciencia es matricida, en tanto, al
desprenderse del seno de la psique, la luz del entendimiento debe relegar un
mundo entero a la más ciega oscuridad. Así, en Anaximandro, del germen de lo
indeterminado surge la doble faz de la luz y la noche, lo consciente y lo
inconsciente. Este surgimiento expresa una adquisición de forma y contenido, en
tanto destruye el χάος originario y, literalmente, produce un mundo —es decir,
es un acto creador—; pero genera también la tensión inherente a la separación de
una unidad en opuestos irreconciliables. Esta tensión es lo expresado en el
rasgarse de la cáscara de la luz, que el filósofo no explica nunca 4 y es, a
todas luces, expresión de una intuición irracional. La unidad originaria de la
luz se destruye inmediatamente luego de ser formada. Esta diseminación de la luz
bien puede estar, a su vez, relacionada con otro hecho psicológico. El arquetipo
del self, expresado en el «divino» seno de lo Ilimitado, no es sólo un arquetipo
sino también, en tanto señala una totalidad, unión y fuente de todos los demás.
En efecto, parte del proceso de integración de los contenidos inconscientes
podría pensarse, en términos simbólicos, como la unión de luces dispersas en una
gran capa de luz. Que en Anaximandro la expresión del self esté asimilada a la
unidad originaria, anterior al surgimiento disruptivo de la consciencia, señala
una semejanza de esa unidad con la expresada por el arquetipo. Son dos formas de
totalidad del todo contrarias, la una el inconsciente primordial donde todo, por
así decirlo, es indiviso; la otra el

3 El fragmento habla de llamas, es decir de la pareja de opuestos calor-frío.
Sigo el criterio de Gigon, que considera que esto es una sustitución realizada
por el autor tardío, a quien esta pareja de contrarios debió de resultar más
familiar. Anaximandro debió hablar de luz y noche.

4 Los autores tardíos que refieren esta parte de su doctrina, y que
probablemente la conocieron de primera mano, no refieren explicación alguna
acerca de este rasgamiento. Lo más probable, entonces, es que Anaximandro
tampoco la haya explicado nunca, y no que haya una explicación perdida con la
obra original.

resultado de la integración de dos opuestos separados. En este sentido se
implican necesariamente: una es la consecuencia lógica de la otra, y este es un
hecho psicológico del todo significativo. No es improbable que éste sea el más
crucial de los rasgos enantiodrómicos del self. El arquetipo se manifiesta en la
imagen del retorno a un estado primigenio y, al mismo tiempo, como una síntesis
positiva. Esto explica el hecho curioso de que el self, aunque parece señalar
sólo un estado ulterior del desarrollo psicológico, sea tan presente en mitos
que describen el origen de las cosas, el estado más anterior y primigenio. La
imagen de las luminosidades dispersadas tras la capa de la noche encuentra,
además, un paralelo en las scintillae de la alquimia medieval. Son sutiles
destellos presentes en la “sustancia de transformación”, asociadas al anima
mundi y el Espíritu Santo. Estas dos nociones son diferentes modulaciones de una
misma intuición; a saber, la de la fuerza oculta y numinosa que conduce el
mundo. No es un misterio entonces que la idea de las scintillae fuera asociada a
ellas, en tanto conduce, y en cierto sentido espiritualiza, el proceso de
transformación alquímica. Kunrath llama a estas luminosidades mundi futuri
seminarium. Son “semillas de luz diseminadas en el caos” 5 . Ya ha sido
estudiada la relación entre el arquetipo del self y estas expresiones de la
alquimia. Que en Anaximandro encontremos imágenes del todo análogas sólo
fortalece nuestra hipótesis. El self abunda en otras modulaciones míticas. En
tanto señala la totalidad de la psique es, por lo general, la intuición
expresada en todo monismo o monoteísmo. Diógenes de Apolonia, por ejemplo,
postuló o lo Indeterminado o el aire como principio, según distintas fuentes. La
discusión filológica aquí no nos importa, porque si habló de aire no lo hizo
como principio material —esto sólo puede ser una mala lectura aristotélica— sino
como hálito engendrador, como aliento que es savia de todas las cosas. Es
entonces, incluso tomando esa traducción, idéntico al principio de Anaximandro,
al menos desde una perspectiva psicológica. Simplicio nos da en su Física (151,
20-153, 5) la siguiente cita de Diógenes: «Me parece (...) que todas las cosas
que existen son alteraciones de lo mismo, y que son lo mismo. (...) Pero todas
estas cosas se generan a partir de lo mismo, como alteraciones diversas en
diversos momentos, y vuelven hacia lo mismo».

Poco después agrega que «lo que tiene inteligencia es lo que los hombres llaman
aire, y que todos son gobernados por él y domina todas las cosas; pues me parece
precisamente que esto es dios, y llega a todo y dispone de todas las cosas y
está presente en todo. Y no existe ninguna cosa que no participe de esto». Pero
no tenemos que quedarnos en el mundo griego, donde el filólogo se contenta con
rastrear una más o menos clara cadena de influencias. En el Bṛhádāraṇyaka
Upaniṣad 6 , la Muerte primigenia, que es el Hambre, obra una mente. La recién
nacida mente concibe el siguiente pensamiento: «que yo tenga un ātman». El ātman
es una clara expresión del arquetipo del self, en tanto significa el sí-mismo
psicológico en su totalidad absoluta, planos «trascendente» e «intrascendente»
integrados. Es el aliento o hálito que vivifica todas las cosas. Como tal puede
asociarse, al igual que las scintillae y el aire de Diógenes, a la noción de
anima mundi. El primer deseo formulado por esta mente primigenia es, entonces,
una identidad total y unificada. Posteriormente se nos dice del surgimiento del
primer hombre, el Puruṣa: «En el principio sólo era el ātman. Y no habiendo otro
salvo él mismo, pensó y se dijo: “Soy yo”. De ahí que su nombre sea “yo”». Este
«yo» es la palabra aham, que se refiere no al sí mismo (ātman, self) sino a al
ego, al sujeto de la consciencia. Así, con este pensamiento, a partir del ātman
primigenio es formado el Puruṣa. Encontramos aquí otra vez que la formación de
la consciencia, expresada en la creación del primer hombre, implica una pérdida
de la unidad de la psique, que inmediatamente se rasga en ātman=self y aham=ego.
Esta intuición es la misma que proyecta Anaximandro cuando nos dice que, del
seno de lo Indeterminado, el γóνιμον produce la luz y la noche. Y este mismo
carácter disruptivo y desintegrador es lo que parece expresarse en el rasgarse
de la capa de la luz en pequeñas luminosidades, consecuencia de la
irreconciliable tensión de los opuestos. Del Puruṣa se nos dice además que tiene
«mil ojos». Ignacio de Loyola nos comenta una visión que lo asombraba con
frecuencia: un brillo que a veces tomaba forma de serpiente, y parecía lleno de
luminosos ojos. Monoimo el árabe enseñaba que el hombre primitivo fue poseedor
de «muchos rostros y muchos ojos».

5 Carl Jung, Arquetipos e inconsciente colectivo, pág.135, Paidós.

6 Las similitudes entre el pensamiento presocrático y el de las Upaniṣad fueron
advertidas antes. Véase, por ejemplo, Giorgio Colli, La naturaleza ama
esconderse. Por lo general causó desconcierto en el campo filológico, en tanto
son fenómenos contemporáneos y un contacto es del todo impensable.

Caesarius de Heisterbach dice del Anthropos, el primer hombre, que es como una
esfera y tiene ojos en todos lados (ex omni parte oculata). Ángela de Foligno,
en una ocasión, vio en la hostia «dos ojos esplendidísimos tan grandes que
parecía que de la hostia solamente quedaban los bordes». Y dice además: «(…) una
vez no ante la hostia, sino en la celda, se me aparecieron ojos con tan gran
belleza y tan deleitables que ciertamente no creo que pierda nunca la alegría».
La asociación del surgimiento de la consciencia con múltiples fenómenos de luz
no es, como vemos, inusual. Su vinculación con el self es clara en tanto las
mismas asociaciones ocurren en la experiencia del contacto con la divinidad (sea
en la hostia cristiana o en el ātman védico). Cabe mencionar que aquí de nuevo
se hace patente el doble carácter de fenómeno originario y verdad última del
arquetipo. Por lo demás, las pequeñas luminosidades, expresadas en imágenes
polioftálmicas o astronómicas, probablemente refieren los fragmentarios
fenómenos de consciencia, mientras que la luz mayor (el ojo divino o el sol como
ojo de Dios, la scintilla una de Kunrath, etc.) es una modulación del arquetipo.
Que estas luminosidades posean generalmente una figura esférica corresponde al
sentido unificador e integrador del self, y también lo hallamos en Anaximandro,
pues de la cáscara de luz se separan pequeñas luces «en algunos círculos».
Hipólito también nos dice que según Anaximandro «los astros se generan como un
círculo de fuego, separándose del fuego del mundo, circundado cada uno por aire»
(D-K 12 A 11). Recordemos que es más probable que Anaximandro hablase de un
círculo de luz y no de fuego. Aquí encontramos otra vez el doble motivo de la
luz mayor, en el fuego del mundo, y los fenómenos de luz dispersos, los círculos
astrales. A esta altura debe ser claro que Anaximandro no es en nada de esto
original. Su doctrina concuerda de manera perfecta con un motivo mitológico
sobradamente documentado. Incluso en la idea de que el principio gobierna todas
las cosas, «», que aparece también en Diógenes, el paralelo con el motivo de los
ojos es evidente, en tanto la capacidad de verlo todo constituye una forma de
gobernación. Así, Dios tiene ojos que «están sobre el camino del hombre» y «está
siempre vigilando todo lo que hacemos» (Job 34:21), y es Chronos «el que todo lo
ve» para Sófocles y «el demonio que todo lo mira» en cierta inscripción
funeraria griega. Argos Panoptes, el gigante de mil ojos, es un guardián y un
custodio.

El arquetipo, naturalmente, no está supeditado al misticismo antiguo. Emerson,
por ejemplo, intuyó una esfera misteriosa de la que todo proviene, y que es y no
es siempre lo mismo. Dice de esa intuición: «Genius studies the causal thought,
and, far back in the womb of things, sees the rays parting from one orb, that
diverge ere they fall by infinite diameters. Genius watches the monad through
all his masks as he performs the metempsychosis of nature. Genius detects
through the flies, through the caterpillar, through the egg, the constant
individual (...)». La esfera originaria, el pensamiento que es causa de todas
las cosas, es el principio; es la mónada pitagórica en relación a la cual toda
muerte es una mera transmigración; es el individuo oculto tras el transitorio
mundo fenoménico. El ser y no ser siempre lo mismo es una forma de síntesis de
los opuestos, elemento común del arquetipo, en tanto el self no es un principio
de perfección, sino de completitud. La totalidad de la psique es precisamente
eso: una totalidad. En ella están la luz y la sombra, los mundos ctónico y
celestial. Que Anaximandro considere que los opuestos están contenidos en lo
Ilimitado es, por lo tanto, consistente con la fenomenología general del
arquetipo tal como se manifiesta en otros mitos. El símbolo de Cristo, por
nombrar un ejemplo familiar, tiene una naturaleza marcadamente enantiodrómica.
La llegada del Anticristo no es sólo un sueño profético, sino el producto de un
principio psicológico que llevó a la mística cristiana a la certeza de un
venidero reinado de las sombras. Esto es particularmente manifiesto en el
pensamiento gnóstico, dentro del que cabe destacar un mito narrado en el Pistis
Sophia. Cuando Jesús era un niño, un espíritu, autodeclarado hermano suyo,
desciende ante María. Se ve tal como Jesús, y ella lo confunde con él, pero
proviene de las regiones inferiores del caos. Al recordar a Jesús este suceso,
María le dice: «he embraced thee and kissed thee, and thou also didst kiss him,
and you became one». Así también Ángela de Foligno dice que, al escuchar al
Espíritu Santo, quiso ver si podía olvidarse de la voz que le hablaba, y
comenta: «Y empecé a mirar las viñas, para olvidarme de aquellas palabras (...),
y donde quisiera que yo mirase me decía a mí misma: “Esta es mi criatura”. Y
sentía una dulzura divina inefable». El éxtasis místico es tal que siente que la
creación le pertenece como originariamente pertenece a Dios. Pero inmediatamente
agrega: «Y entonces me tornaban a la memoria todos mis pecados y mis vicios y,
por otra parte, no veía nada en mí sino pecados y defectos».

Para Ángela de Foligno, la simultaneidad de la experiencia extática y el asalto
de los vicios y pecados habrá sido del todo consternante. Hemos hablado de la
intuición gnóstica de la inseparabilidad de Cristo y el Anticristo, pero aquí
nos hallamos frente a la espontánea experiencia de esa intuición en una
cristiana católica, que probablemente jamás imaginó que la voz del Espíritu
Santo pudiera ser acompañada por tan hondas oscuridades. Pero si hemos
comprendido algo del arquetipo del self, esa simultaneidad no puede
sorprendernos. La experiencia era difícilmente interpretable en el marco de la
dogmática católica, pero desde una perspectiva psicológica es impecablemente
consistente. Por lo general, y en concordancia con lo anterior, el arquetipo se
manifiesta como afectado por «equal yet opposite forces» 7 . La exposición de
tantos ejemplos no es un acto de vana erudición. Con suerte, se habrá hecho
notar la semejanza de las expresiones expuestas con las de Anaximandro. Y no
sólo en lo Ilimitado como continente de todos los opuestos, sino también como
engendrador y destructor del cosmos y las cosas. A este respecto vale mencionar
otro fragmento de Anaximandro. Me refiero al D-K 12 A 9, de tono claramente
místico: «a partir de donde hay generación para las cosas, hacia allí también se
produce la destrucción, según la necesidad; en efecto, pagan la culpa unas a
otras y la reparación de la injusticia, de acuerdo con el ordenamiento del
tiempo». Lo Ilimitado es no sólo principio sino fin de cada cosa, como lo es el
Señor en Apocalipsis 1:8: «Yo soy el Alfa y la Omega, principio y fin». Pero
esto no es novedad, pues es referido en multitud de otros fragmentos. Lo
singular surge con las palabras «culpa» y «justicia». Gigon dice que es
pensamiento indio, no griego, el considerar la individuación como una culpa. El
filólogo otorga demasiada atención al dónde y al cuándo, e ignora que toda
expresión humana corresponde a un hecho universal. En este sentido, los árboles
le tapan el bosque. No es, en mi opinión, extraño que el pasaje refiera de hecho
una culpa implicada en la existencia, en tanto, como se ha dicho antes, el
verdadero acto creador, la formación de la consciencia, es matricida. En el
contexto simbólico descrito, en que la cosmogonía de Anaximandro es una
proyección del surgimiento de la consciencia, es del todo consistente otorgar a
este fragmento ese carácter «indio» que deja al filólogo perplejo. El
surgimiento de la consciencia es la creación del mundo. Ignoramos qué es la luna

más allá de la representación que hacemos de ella, porque ella es esa
representación, y nada puede existir objetivamente sin un sujeto al que
presentarse como objeto. La muerte no es el no- ser, en tanto la vida corpórea
sigue un largo curso de transformación y desintegración después de la muerte; es
la no-consciencia, de la cual, desde luego, no podemos formar representación
alguna. Es, en términos simbólicos, el retorno a lo Indeterminado, a partir del
cual se produjo la generación y hacia el cual la destrucción se mueve. Pero en
esa inimaginable noche debe existir justicia, porque la muerte de la consciencia
paga la culpa de su propio nacimiento. Lo que señalo aquí, debe decirse, no es
en absoluto nuevo. Gigon, además, ha señalado un posible paralelo mitológico del
fragmento D-K 12 A 9 en el mito de Cronos y la cadena de «culpas» y
«reparaciones» parricidas producida desde el dios hacia su descendencia. La
hipótesis es atractiva, pero no parece consistente con el significado
psicológico general que hallamos en la doctrina de Anaximandro. Podemos decir
con seguridad que, si el fragmento proviene de un mito griego, debe ser de uno
cuyo universo simbólico esté emparentado con el de la doctrina a la que
pertenece. El mito griego que más claramente satisface esta condición, a mis
ojos, es el de Prometeo, que por entregar a los hombres el don de la luz es
castigado. La dádiva de la luz, el surgimiento de la consciencia, conlleva una
culpa, una traición a lo divino que debe ser reparada. La filología no señaló
este paralelo nunca. Por lo contrario, optó por extrañarse del carácter «indio»
de esa ley cósmica o por tacharla de mística y concluir con eso el asunto. Esto
se debe a que él es claro sólo en virtud del significado psicológico de ambas
expresiones; significado que, a su vez, sale a la luz sólo a través del análisis
comparado de la doctrina de Anaximandro con otros mitos y expresiones
arquetípicas. Este no es el lugar para entrar en detalle acerca del mito de
Prometeo, pero, aunque sólo esta breve asociación será posible por ahora, esta
probable conjetura es, a mis ojos, digna de atención. Further… (!?)

III. CONCLUSIÓN (?!)

El método seguido hasta aquí corre el riesgo de confundirse con mera dispersión.
He recogido un considerable número de similitudes entre la cosmogonía de
Anaximandro y otras expresiones arquetípicas cuya vinculación al arquetipo del
self ha sido, o bien discutida antes de mí, o bien considerablemente más clara
que en el presocrático. Prioricé su cantidad antes que el grado de detalle con
que las he descrito;

7 Palabras de la paciente que es objeto del libro A study in the process of
individuation, Jung 1968.

mi esperanza es que las semejanzas hablen por sí solas. Si sabemos que una clase
de fenómeno se define por cierto conjunto de características, demostrar esas
características en un caso debe equivaler a demostrar su pertenencia a la clase.
Con suerte, la exposición anterior fue suficiente para mostrar que la cosmogonía
de Anaximandro tiene todas las características propias de una proyección
arquetípica. La extensión de tal demostración sobre el resto de la filosofía
presocrática es un trabajo que permanece abierto. Si puede admitirse que las
bases fundamentales de esa filosofía, al menos en el caso general, es más
cercana al pensamiento mítico que a nuestras intelecciones científicas, ya
habremos hecho la mitad del trabajo. La otra mitad consistirá en distinguir su
naturaleza arquetípica. Los fundamentos científicos que nos permiten suponer
esta naturaleza son sólidos, pero tal distinción sólo puede lograrse a través de
un estudio comparado. En última instancia, lo que se hará ver es cuán hondo en
la condición humana se hunden las raíces de la experiencia afectiva del mundo.

---

1 Me refiero aquí a los pueblos del oriente con que los griegos tuvieron
contacto en el período arcaico.


    
\end{document}
