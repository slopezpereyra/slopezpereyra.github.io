\documentclass[a4paper, 12pt]{article}

\usepackage[utf8]{inputenc}
\usepackage[T1]{fontenc}
\usepackage{textcomp}
\usepackage{amssymb}
\usepackage{newtxtext} \usepackage{newtxmath}
\usepackage{amsmath, amssymb}
\newtheorem{problem}{Problem}
\newtheorem{example}{Example}
\newtheorem{lemma}{Lemma}
\newtheorem{theorem}{Theorem}
\newtheorem{problem}{Problem}
\newtheorem{example}{Example} \newtheorem{definition}{Definition}
\newtheorem{lemma}{Lemma}
\newtheorem{theorem}{Theorem}
\usepackage{parskip}

\begin{document}

Some observations:

\begin{itemize}
    \item $K_n$ can span every $G \in \mathcal{G}_n$.
    \item For any $G \in \mathcal{G}_{n, m}$, there are $M = \binom{n}{2} - m$ edges that must be 
        removed to span it from a $K_n$.
    \item The order in which the edges are removed does not matter.
    \item The space of prunable edges $\mathcal{E}$ is not constant, since an edge may become 
        a bridge and disappear from $\mathcal{E}$.
    \item Following the previous statement: $\mathcal{E}$ initializes as $\Lambda(n)$ but 
        loses an element per generated bridge.
\end{itemize}

Sea $\Lambda(n) = \left\{ \left\{ x, y \right\} : x, y \in \left\{ 1,\ldots,n
\right\}  \right\} $ el conjunto de todos los lados posibles en un grafo
etiquetado de $n$ vértices. Sabemos que $|\Lambda(n)| = \binom{n}{2}$.

El problema es generar un grafo arbitrario de $n$ vértices, $m$ lados con un
algoritmo que borra lados de un $K_n$. El algoritmo es fácil de dar: generar un
$K_n$, seleccionar aleatoriamente un vértice que no sea un puente y borrarlo;
repetir hasta que el grafo resultante tenga $m$ lados.

Mi preocupación es si la probabilidad de generar cada posible grafo es la
misma; es decir, si existe o no un sesgo por algún tipo de grafo.

Hay dos problemas. Incluso si asumimos que el conjunto de lados que puede
borrarse permanece constante en cada iteración (no lo hace), es difícil
determinar cuántos grafos conexos pueden generarse. Sabemos que $|\Lambda(n)| =
\binom{n}{2}$, y por lo tanto generar un grafo cualquier de $m$ lados a a
partir del $K_n$ involucra seleccionar $\frac{ n(n-1) }{2} - m$ lados del
conjunto de lados posibles. Es decir, hay

\begin{align*}
    \binom{\frac{n(n-1)}{2}}{\frac{n(n-1)}{2} - m} = \binom{\frac{n(n-1)}{2}}{m} =: C_{n, m}
\end{align*}

tales grafos. El problema es que algunos son conexos y otros no. Y no logro
identificar una manera de contar cuántos grafos disconexos estamos contando,
excepto cuando la cantidad de lados deseados es $n-1$ (es decir, excepto cuando
queremos generar árboles).

Fuera de eso, el conjunto de lados que podemos seleccionar no permanece
constante. Cada vez que removemos un lado, es posible que algún otro se
convierta en un puente y por lo tanto no pueda seleccionarse en el futuro. Pero
tampoco parece haber una manera de determinar cuándo sucederá esto ni cuántas
veces.











\end{document}



