\documentclass[a4paper, 12pt]{article}

\usepackage[utf8]{inputenc}
\usepackage[T1]{fontenc}
\usepackage{textcomp}
\usepackage{amssymb}
\usepackage{newtxtext} \usepackage{newtxmath}
\usepackage{amsmath, amssymb}
\newtheorem{problem}{Problem}
\newtheorem{example}{Example}
\newtheorem{lemma}{Lemma}
\newtheorem{theorem}{Theorem}
\newtheorem{problem}{Problem}
\newtheorem{example}{Example} \newtheorem{definition}{Definition}
\newtheorem{lemma}{Lemma}
\newtheorem{theorem}{Theorem}
\usepackage{parskip}

\begin{document}

The generation of connected random graphs is non-trivial and important to many
applications. In particular, it is not easy to sample a connected random graph
from the space $\mathcal{G}$ of all connected graphs with uniformity; i.e.
without a bias for graphs of a special kind. This entry contains two algorithms
for sampling random connected graphs that do not overcome a special kind of
bias. A third algorithm which overcomes all bias, though at great computational
expense, is presented.

\section{Bottom-up approach}

Before proceeding, quick definitions.

\begin{itemize}
    \item Let $\mathcal{T}_n$ the set of all trees of $n$ vertices,
        $\mathcal{G}_n$ the set of all graphs with $n$ vertices. We shall
        assume the vertices of these graphs are labeled $1, \ldots, n$. 
    \item For any $T \in \mathcal{T}_n$, we define $\mathcal{U}_T := \left\{ G
        \in \mathcal{G}_n : T \subseteq G  \right\} $ and refer to it as
        \textit{the universe} of $T$.
    \item Let $\Gamma(n) = \left\{ \left\{ x, y \right\} : x, y \in \left\{ 1, \ldots, n\right\}    \right\} $. 
\end{itemize}

In general, whenever I write of a tree $T$, I mean an arbitrary $T \in \mathcal{T}_n$;
and whenever I write of a graph $G$, I mean an arbitrary $G \in \mathcal{G}_n$.

---

There are several properties of trees which make them attractive for graph
generation. On one hand, any finite $G$ contains a finite number of spanning
trees $T \subseteq G$; on the other, any tree spans a finite number of graphs.
It is trivial to observe that $T \subseteq G$ if and only if there is some $S
\in \Gamma(n)$ s.t. $E(G) = E(T) \cup S$. We can make use of this special
relationship between $\Gamma(n)$ and $\mathcal{U}_T$ to produce random
connected graphs.

Each tree is perfectly identified by its Prüfer sequence. Then, for a fixed
$n$, the language $\left\{ 1, \ldots, n \right\}^{n-2} $ indexes a family of
functions $\mathcal{F}$ defined as:

\begin{align*}
    \mathcal{F}(w) : \mathcal{U}_{T_w} &\to \Gamma(n)  \\ 
    \mathcal{F}(w)(G) &= E(G) - E(T_w)
\end{align*}

where $T_w$ is the tree corresponding to the Prüfer sequence $w$. We shall use
$\mathcal{F}_w$ to abbreviate $\mathcal{F}(w)$.

Since there is a perfect correspondence between any given $G \in \mathcal{U}_T$ and the
set of edges which, if aggregated to $T$, produce $G$, $\mathcal{F}_w$ is a
bijection. (This is easy to prove.) This entails that any $G \in \mathcal{G}_n$
which may be spanned from $T_w$ corresponds to a unique set in
$\Gamma(n)$---namely, that which extends $T$ into $G$.

Thus, we have established a series of generational relations. A Prüfer sequence
maps to a unique tree $T$, the tree maps to a universe of connected graphs
$\mathcal{U}_T$, and each graph in this universe is perfectly determined by a
set in $\Gamma(n)$. This inspires the following effective procedure for
generating random connected graphs of $n$ vertices.

\begin{quote}
    \textit{(1)} Generate randomly $p = p_1\ldots p_{n-2} \in \Sigma^{n-2}$.

    \textit{(2)} Span the tree $T = (V, E)$ of the Prüfer sequence $p$.

    \textit{(3)} Let $k \in_R \left\{ 0, \ldots, \frac{ n(n-1) }{2} \right\} $.

    \textit{(4)} Let $\ell_1, \ldots, \ell_k \in_R \Gamma(n) - E(T)$, all distinct.

    \textit{(5)} Let $E = E \cup \left\{ \ell_1,\ldots, \ell_k \right\} $
\end{quote}

Because all trees of $n$ vertices correspond to a sequence, all tres can be
sampled. And all connected graphs can be derived from the set of all spanning
trees. Then this procedure generates all graphes in $\mathcal{G}_n$.

The question is whether it is equally likely to generate any two graphs in
$\mathcal{G}_n$. It is obvious that it is equally likely to generate any tree.
And the probability that a given graph is generated depends entirely on the
number of spanning trees it contains. Not all graphs have the same number of
spanning trees. $\therefore $ It is more likely to generate a graph with many
spanning trees than a graph with few spanning trees.

The second algorithm extends the input from only $n$, the number of 
vertices, to $m$, the number of edges. Thus, it specifies the problem 
further into the question of how to generate denser or sparser connected graphs
of $n$ edges. The effective procedure to do this is:

\begin{quote}
    \textit{(1)} Generate randomly $p = p_1\ldots p_{n-2} \in \Sigma^{n-2}$.

    \textit{(2)} Span the tree $T = (V, E)$ of the Prüfer sequence $p$.

    \textit{(3)} If $V = \left\{ v_1, \ldots, v_n \right\} $, list all its non-neighbours; i.e. generate $\Gamma^c(v_1), \ldots, \Gamma^c(v_n)$.

    \textit{(4)} Sample a random, non-saturated vertex $v \in V$, and sample a random vertex $w$
    from $\Gamma^c(v)$.

    \textit{(5)} Add $\left\{ v, w \right\} $ to $E$. Remove $v$ from the list of non-neighbours of 
    $w$, and $w$ from the list of non-neighbours of $v$. 

    \textit{(6)} If $|E| = m$, finish. Otherwise go to \textit{4}.
\end{quote}

It is clear that the procedure always finishes, and since the tree is connected
the generated graph is connected. A pseudo-code algorithm may look as follows.

\begin{align*}
    &\textbf{Input: } n, m\\
    &(V, E) = \textbf{gen\_random\_tree}(n)\\
    &S = \big[~\Gamma^c(v_1), \ldots, \Gamma^c(v_n)~\big] &\left\{ \text{Non-neighbours} \right\} \\
    &C = \big[~ |S[1]|, \ldots, |S|[n]|~\big] & \left\{ \text{Cardinalities} \right\} \\
    &V = [1, \ldots, n] &\left\{ \text{Non-saturated vertices } \right\} \\
    &n' = n\\
    &\textbf{while } |E(T)| < m \textbf{ do } \\ 
    &\qquad i := \textbf{random}(1, n') \\ 
    &\qquad v := V[i]\\
    &\qquad \textbf{if } \left( d(v) == n - 1 \right) \textbf{ do } \\ 
    &\qquad \qquad \textbf{delete\_at}(V, i)\\ 
    &\qquad\qquad n' := n' - 1 \\ 
    &\qquad\textbf{else } \\ 
    &\qquad\qquad j = \textbf{random}(1, C[v])\\
    &\qquad\qquad w := S[v][j] \\ 
    &\qquad\qquad E(T) := E(T) \cup  \left\{ v, w \right\} \\
    &\qquad\qquad C[v] := C[v] - 1 \\ 
    &\qquad\qquad \textbf{delete\_at}(S[v], j)  \\ 
    &\qquad\qquad\textbf{delete\_element}(S[w], v)\\ 
    &\qquad\textbf{fi}\\
    &\textbf{od}\\
    &\textbf{return } T
\end{align*}

Generating a tree from a random Prüfer sequence is $O(n^2)$. Listing all the
non-neighbours is also $O(n^2)$. Within the while loop there are simply index
manipulations, so the complexity of the loop is $\varphi \times O(1) =
\varphi$, with $\varphi$ the complexity of the number of iterations. 

All iterations add an edge except those where a saturated vertex is chosen. A
saturated vertex may be chosen at most once. $\therefore $ There are $O(n)$
iterations where a saturated vertex is chosen. Since the rest of the 
iterations add an edge, their number is fixed: $m - (n-1)$, where $(n-1)$ is the number of 
edges in the spanned tree. $\therefore $ There are exactly $m -n + 1$ edge-adding
iterations.

$\therefore  \varphi = O(n) + O(m - n + 1) = O(n) + O(m) = O(n^2)$

$\therefore $ The complexity of the algorithm is $O(n^2) + O(n^2) = O(n^2)$.









\end{document}



