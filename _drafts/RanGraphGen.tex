\documentclass[a4paper, 12pt]{article}

\usepackage[utf8]{inputenc}
\usepackage[T1]{fontenc}
\usepackage{textcomp}
\usepackage{amssymb}
\usepackage{newtxtext} \usepackage{newtxmath}
\usepackage{amsmath, amssymb}
\newtheorem{problem}{Problem}
\newtheorem{example}{Example}
\newtheorem{lemma}{Lemma}
\newtheorem{theorem}{Theorem}
\newtheorem{problem}{Problem}
\newtheorem{example}{Example} \newtheorem{definition}{Definition}
\newtheorem{lemma}{Lemma}
\newtheorem{theorem}{Theorem}
\usepackage{parskip}

\begin{document}

Some observations:

\begin{itemize}
    \item $K_n$ can span every $G \in \mathcal{G}_n$.
    \item For any $G \in \mathcal{G}_{n, m}$, there are $M = \binom{n}{2} - m$ edges that must be 
        removed to span it from a $K_n$.
    \item The order in which the edges are removed does not matter.
    \item The space of prunable edges $\mathcal{E}$ is not constant, since an edge may become 
        a bridge and disappear from $\mathcal{E}$.
    \item Following the previous statement: $\mathcal{E}$ initializes as $\Lambda(n)$ but 
        loses an element per generated bridge.
\end{itemize}

We know $|\Lambda(n)| = \binom{n}{2}$. There are 

\begin{align*}
    \binom{\frac{n(n-1)}{2}}{\frac{n(n-1)}{2} - m} = \binom{\frac{n(n-1)}{2}}{m} =: C_{n, m}
\end{align*}

graphs in $\mathcal{G}_{n, m}$, but some of them are disconnected. The question
is: how many of them are disconnected. 

Let $\mathcal{A}$ be the class of all graphs. We wish to produce a generating
function for $\mathcal{A}$; this is, a series s.t. its $k$th coefficient is the
number of graphs with $n$ vertices, $m$ edges. We know this quantity
due to the derivation above, and all that is left is to expand it into a series
for each $n, m$.

The mixed exponential generating function for $\mathcal{A}$ is then

\begin{align*}
    A(x) &= \sum_{n=0}^{\infty}\left(\sum_{m = 0}^{\infty} \binom{\frac{n(n-1)}{2}}{m}  y^m\right) \frac{x^n}{n!}\\
                                                               &=\sum_{n=0}^{\infty}(1 + y)^{\frac{n(n-1)}{2}} \frac{x^n}{n!} \\ 
                                                               &= 1 + \sum_{n=1}^{\infty} (1+y)^{n(n-1)/2} \frac{x^n}{n!}
\end{align*}

Now, every graph in $\mathcal{A}$ is a set of connected graphs. In other words,
if we define $\mathcal{C}$ the class of connected graphs, the relationship
between these two clases is the set-of relation. This means 

\begin{equation*}
    A(x) = \exp C(x)
\end{equation*}

But we know $A(x)$, so we can find $C(x)$ by taking $\ln A(x)$:

\begin{align*}
    C(x) &= \ln \left[1 + \sum_{n=1}^{\infty} (1+y)^{n(n-1)/2} \frac{x^n}{n!}\right]
\end{align*}

Here, we recall that 

\begin{align*}
    \log (1 + u) &= \sum_{k=1}^{\infty} (-1)^{k+1} \frac{u^k}{k}
\end{align*}

which entails 

\begin{align*}
    C(x) &= \sum_{k=1}^{\infty} \frac{ (-1)^{k+1} }{k} \left[ \sum_{n=1}^{\infty}\left( 1+y \right)^{n (n-1) / 2} \frac{x^n}{n!}  \right]^{k} 
\end{align*}

Thus, $C(x)$ produces an enumeration of all connected graphs of $n$ vertices, and we can arrive at the expression for all connected graphs of $N$ vertices and $M$ edges:

\begin{align*}
    N! y^M x^N \sum_{k=1}^{N} \frac{ (-1)^{k+1} }{k} \left[ \sum_{n=1}^{N}\left( 1+y \right)^{n (n-1) / 2} \frac{x^n}{n!}  \right]^{k} 
\end{align*}

For example, for $M = N - 1$ across $N = 2, 3, \ldots$, this effectively produces the sequence 

\begin{align*}
    1, 1, 3, 16, 125, 1296, \ldots
\end{align*}

which matches the number of trees indicated by the Prufer sequence.
















\end{document}



