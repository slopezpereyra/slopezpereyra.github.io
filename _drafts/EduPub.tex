\documentclass[a4paper, 12pt]{article}

\usepackage[utf8]{inputenc}
\usepackage[T1]{fontenc}
\usepackage{textcomp}
\usepackage{amssymb}
\usepackage{newtxtext} \usepackage{newtxmath}
\usepackage{amsmath, amssymb}
\newtheorem{problem}{Problem}
\newtheorem{example}{Example}
\newtheorem{lemma}{Lemma}
\newtheorem{theorem}{Theorem}
\newtheorem{problem}{Problem}
\newtheorem{example}{Example} \newtheorem{definition}{Definition}
\newtheorem{lemma}{Lemma}
\newtheorem{theorem}{Theorem}


\begin{document}

El título de esta entrada es \textit{La educación pública}. Lo cierto es que me
limito a considerar la educación pública superior; en particular, las
universidades nacionales argentinas. Todo lo \textit{concreto} que diré remite a
estas instituciones. 

La educación pública argentina está siendo avasallada; al igual que la amenaza
del hambre fuerza a los individuos a enajenarse por dinero, la amenaza de la
quiebra, que ha tomado feroces proporciones frente a la no-renovación del
presupuesto universitario, forzará a las universidades o bien a enajenarse
también ellas mismas por vías del arancel, sacrificando así su propósito
original, o a cerrar definitivamente. Y el telón de fondo sobre el cual se teje
la discusión del presupuesto universitario concierne no a cuestiones concretas,
sino a principios filosóficos y éticas. 

Las primeras preguntas que uno debe hacerse son positivas. En primer lugar,
debemos preguntarnos si la educación universitaria superior debiera ser un
derecho. En segundo lugar, si beneficia a la sociedad, tanto desde una
perspectiva económica como desde una perspectiva ética---es decir, si aumenta la
felicidad de las personas---. La pregunta de si la educación que brinda es buena
o mala no es una pregunta de principio; lo cierto es que la educación superior
argentina posee una calidad sobradamente probada, cosa que no pretenderé mostrar
para no hacer esta entrada demasiado larga.

Luego debemos hacer preguntas negativas; es decir, preguntas rel








\end{document}



