\documentclass[a4paper, 12pt]{article}

\usepackage[utf8]{inputenc}
\usepackage[T1]{fontenc}
\usepackage{textcomp}
\usepackage{amssymb}
\usepackage{ebgaramond}
\usepackage{amsmath, amssymb}
\newtheorem{problem}{Problem}
\newtheorem{example}{Example}
\newtheorem{lemma}{Lemma}
\newtheorem{theorem}{Theorem}
\newtheorem{problem}{Problem}
\newtheorem{example}{Example} \newtheorem{definition}{Definition}
\newtheorem{lemma}{Lemma}
\newtheorem{theorem}{Theorem}


\begin{document}

When I was seventeen, I visited the Ranakpur Jain Temple. I learned there of a
curious practice not uncommon to this day. Jain monks live a life of silent
asceticism. Some of them, upon reaching supreme emancipation, abandon
themselves to meditation and into desintegration by means of inanition.

During the same trip, I visited the Jaisalmer Fort, built in the dawn of
the second millennia after Christ. During the thirteenth century,
succumbing to the siege of the Mongol invader Alauddin, while all the
women and children committed *Jahuar*—a ritual practice of mass
self-immolation—the few surviving men left the fort to die in battle.

Gödel, who was paranoid, died of inanition out of fear of someone poisoning his
food. Yukio Mishima, that exquisite writer, like many others before him,
practiced *seppuku* when his attempted *coup d'état* failed. Cato de Younger
sought death in similar fashion: defeated by Caesar, he plunged his sword into
his chest and tore his bowels with his hands. Seneca died a slow and
bureaucratic death by command of Nero. The Bolivian agent who executed Che
Guevara reported his last words to be: *Serénese y apunte bien: va a matar a un
hombre*. Socrates, out of respect to a ruling which he knew unfair, declined to
be released by his friends; when the time came, he drank the hemlock calmly,
and his farewell thought was of a debt he didn't want to leave unpaid.

Though voluntary death seems unnatural, it cannot be. It is a logical
impossibility that anything existing is unnatural; and it is in fact surprising
how decisively a person can rid itself of the will to live or embrace the will
to die. Thus, we find that nature is complex enough as to produce agents which
contradict it. A blind designer can bestow its creatures with the faculty of
rebellion. This apparent contradiction is the root of the ancient biblical
tale, as Milton put it:

> Because we freely love, as in our will<br>
> To love or not; in this we stand or<br>
> fall: And som are fall'n, to disobedience fall'n (...).

In desperate cases, as in the case of the women and children who
committed *Jahuar* in Jaisalmer Fort, voluntary death is the lesser
evil. In what comes to the ascetic surrender of Jain monks, I see that
it prevents no evil nor it produces any happiness. If anything, it seems
to be a rather a pointless endeavor to submit oneself to such
unnecessary suffering. Asceticism in general escapes my understanding.
As Hesse's Siddhartha put it:

> And Siddhartha said quietly, as if he was talking to himself: \"What
> is meditation? What is leaving one's body? What is fasting? What is
> holding one's breath? It is fleeing from the self, it is a short
> escape of the agony of being a self, it is a short numbing of the
> senses against the pain and the pointlessness of life. The same
> escape, the same short numbing is what the driver of an ox-cart finds
> in the inn, drinking a few bowls of rice-wine or fermented coconut
> milk.\"


The ethical value of any deed depends on whether it contributes to the
proliferation of happiness or the prevention of sorrow. Since many a faith is
worse than death, and death is not the worse of evils, voluntary death cannot
be universally wrong and must be judged according to the circumstances of the
case. It's true that most of us feel committed to life, and do not wish to "go
gentle into that good night". But, as the examples above show, there is place
for resignation, for acceptance, and even for gentleness in the embrace of
death.


The core of the Christian *ethos* is the willingness to die. God itself became
incarnate so as to suffer earthly death. Origen of Alexandria endured two years
of torture under Decius: he deemed death a lesser evil than abandoning his
faith. The Bible agreed with him: in \textit{Mark 14:31} Peter tells Jesus:
\textit{ If I should die with thee, I will not deny thee in any wise}. In
\textit{Chronicles 1:10} we are told that, defeated by the Philistines, 

\begin{quote}
    Saul said to his armourbearer: Draw thy sword, and thrust me through
    therewith; lest these uncircumcised come and abuse me. But his armourbearer
    would not; for he was sore afraid. So Saul took a sword, and fell upon it.
\end{quote}

I have never quite understood the attitude of modern Christians towards death
and even suicide. Their religion abounds in all its various kinds, from patient
resignation to gruesome martyrism. 

As far as I know we have no scientific understanding of suicide and voluntary
death. It seems in fact impossible to study this phenomenon. An account of the
examples which history provides us suffices to conclude that to perceive it as
unnatural is to confuse the facts. Resignation, willingness, embrace and even
satisfaction: All these may coexist with death. The chain which links human
existence to life, though strong, is not unbreakable---and every person seems
to possess, if not the key which gently frees it, the brutal tools which can
destroy it.


























\end{document}



