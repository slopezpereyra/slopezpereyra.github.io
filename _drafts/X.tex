\documentclass[a4paper, 12pt]{article}

\usepackage[utf8]{inputenc}
\usepackage[T1]{fontenc}
\usepackage{textcomp}
\usepackage{amssymb}
\usepackage{ebgaramond}
\usepackage{amsmath, amssymb}
\newtheorem{problem}{Problem}
\newtheorem{example}{Example}
\newtheorem{lemma}{Lemma}
\newtheorem{theorem}{Theorem}
\newtheorem{problem}{Problem}
\newtheorem{example}{Example} \newtheorem{definition}{Definition}
\newtheorem{lemma}{Lemma}
\newtheorem{theorem}{Theorem}


\begin{document}

When I was seventeen, I visited the Ranakpur Jain Temple. I learned there of a
curious practiced not uncommon to this day. Jain monks live a life of silent
asceticism. Some of them, upon reaching supreme emancipation, abandon
themselves to meditation and into desintegration by means of inanition. 

During the same trip, I visited the Jaisalmer Fort, built in the dawn of the
second millennia after Christ. During the thirteenth century, succumbing to the
siege of the Mongol invader Alauddin, while all the women and children
committed \textit{Jahuar}---a ritual practice of mass self-immolation--- the
few surviving men left the fort to die in battle.


Gödel, who was paranoid, died of
inanition out of fear of someone poisoning his food. Yukio Mishima, that
exquisite writer, like many others before him, practiced \textit{seppuku} when
his attempted \textit{coup d'état} failed. Seneca died a slow and bureaucratic
death by command of Nero. The Bolivian agent who executed Che Guevara reported
his last words to be: \textit{Serénese y apunte bien: va a matar un hombre}.
Socrates, out of respect to a ruling which he knew unfair, declined to be
released by his friends; when the time came, he drank the hemlock calmly, and
his farewell thought was of a debt he didn't want to leave unpaid. 

Though voluntary death appears to  be unnatural, it cannot be. Firstly, it
exists, and whatever exists exists in nature. It is a logical impossibility
that anything existing is unnatural. It is in fact surprising how decisively a
person can rid itself of the will to live or embrace the will to die. Thus, we
find that nature is complex enough as to produce agents which contradict it. A
blind designer can bestow its creatures with the faculty of rebellion. This
apparent contradiction is the root of the ancient biblical tale, as Milton put
it:

\begin{quote}
    Because we freely love, as in our will 

    To love or not; in this we stand or
    
    fall: And som are fall’n, to disobedience fall’n (\ldots).
\end{quote}

In desperate cases, as in the case of the women and children who committed
\textit{Jahuar} in Jaisalmer Fort, voluntary death is the lesser evil. In what
comes to the ascetic surrender of Jain monks, I see that it prevents no evil
nor it produces any happiness. If anything, it seems to be a rather a pointless 
endeavor to submit oneself to such unnecessary suffering. Asceticism in general 
escapes my understanding. As Hesse's Siddhartha put it:

\begin{quote} And Siddhartha said quietly, as if he was talking to himself:
    "What is meditation? What is leaving one's body? What is fasting? What is
    holding one's breath? It is fleeing from the self, it is a short escape of
    the agony of being a self, it is a short numbing of the senses against the
    pain and the pointlessness of life. The same escape, the same short numbing
    is what the driver of an ox-cart finds in the inn, drinking a few bowls of
    rice-wine or fermented coconut milk." 
\end{quote}















\end{document}



