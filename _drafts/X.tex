\documentclass[a4paper, 12pt]{article}

\usepackage[utf8]{inputenc}
\usepackage[T1]{fontenc}
\usepackage{textcomp}
\usepackage{amssymb}
\usepackage{amsmath, amssymb}
\newtheorem{problem}{Problem}
\newtheorem{example}{Example}
\newtheorem{lemma}{Lemma}
\newtheorem{theorem}{Theorem}
\newtheorem{problem}{Problem}
\newtheorem{example}{Example} \newtheorem{definition}{Definition}
\newtheorem{lemma}{Lemma}
\newtheorem{theorem}{Theorem}


\begin{document}

I take as self-evident that a life in accord to nature is superior to a life in
conflict with nature. A not unreasonably inflated sense of self is natural to
every person---a selfless life, \textit{in the radical sense}, is not entirely
ethical. As an example of what I mean by radical selflessness, one may take
a practice still common among Jainist monks: namely, to abandon oneself
to meditation and into desintegration by means of inanition. It is hard to see
how such a practice makes existence any better for any sentient being. In
general, any form of asceticism is negative to me. This was realized, for
example, by Hesse's Siddhartha:

\begin{quote} And Siddhartha said quietly, as if he was talking to himself:
    "What is meditation? What is leaving one's body? What is fasting? What is
    holding one's breath? It is fleeing from the self, it is a short escape of
    the agony of being a self, it is a short numbing of the senses against the
    pain and the pointlessness of life. The same escape, the same short numbing
is what the driver of an ox-cart finds in the inn, drinking a few bowls of
rice-wine or fermented coconut milk." \end{quote}

A reasonable sense of selflessness, the devotion of a considerable part of our
lives to the betterment of others, is not radical in my sense of the word. A
selflessness not understood as a sacrifice of all attachment to life, but as a
way of generously bonding with the world, with a concrete world of which we
are but a minuscule element, and in whose development we play a minuscule role---may we
be kings or beggars as it may---is certainly natural, and in fact the expected
result of abandoning the narcissistic tendencies of childhood. This kind of
generosity, though surely coexisting with envy and violence, and as much a part
of ourselves as hate and vengeance, comes with our natural endowment---every human 
being possesses the faculty of kindness. 

Of course, it is a logical impossibility that anything existing is unnatural.
Whatever exists, exists in nature. But nature is complex enough as to produce
agents which contradict it, like a designer which bestows his creatures with
the faculty of rebellion. This apparent contradiction is the root of the
ancient biblical tale, as Milton put it:

\begin{quote}
    Because wee freely love, as in our will 

    To love or not; in this we stand or
    
    fall: And som are fall’n, to disobedience fall’n (\ldots).
\end{quote}

A life in accord to nature: Such is a life of Unity. It is my belief that
people struggle to accept nature not because they fail to come to terms with their own
natural tendencies---for our natural inclinations cannot be felt as nothing but
natural and hence, at least in a certain sense, perfect---but because cowardice
or fear prevent them from respecting the natural tendencies of others. 
The clearest example of this is the complex conformed by sex, marriage, and
monogamy. Any man is capable of accepting that he feels attraction to people
other than his partner. And it is not an incapacity to reconcile this fact with
the love he feels for his partner what may prevent him to act on it---He feels
such attraction as natural, because it is. It is most likely the unpleasant
imagination of his partner acting upon their desire for other people what
drives his condemnation of sexual commerce outside the couple. And though he
feels---and thus he knows---that such attraction is natural, he cannot
extrapolate this feeling to his partner---at least not without considerable
discomfort. And so he is conflicted.

The essence of love, it seems to me, is this: That love can only change by
means of changes in its subjects. It shows a curious independence with regards
to the objective world. This is why time is said to be the major enemy of
love—Time effects change on love only because it effects change on people. But
provided that life is enjoyed in tolerable conditions—a necessary premise—I
think true love is independent of a circumstance, a fact, or particular
actions. The foundation of one's love must be very much in doubt if it could
not resist sexual commerce outside the couple. Sex is not a transformative
act—the fact that I am me remains intact despite it. If this is true for me,
then I cannot presume it isn’t true for other people. Hence, as long as love is
felt for an individual, it cannot be affected by such a relative triviality.

But love, contrarily to sex, is transformative. One cannot love and remain. An
inattentive person might presume that fear of our partner falling in love with
someone else, and not a superstitious relation with sex, is what sustains the
widespread fear of non-monogamy. But I very much doubt this is true, at least
in men. From what I gather, many men have a supernatural fear of female
sexuality. They make it seem as if the medieval witch, the one bestowed with
frenzied and orgiastic spells, was still a living archetype. Of course, this
fear is directed towards their long-term partners; their girlfriends, their
wives, and of course their mothers. Outside of this domain, women are either
incomprehensible creatures that dwell somewhere outside the familiar world, or
at worst an object from which to derive a satisfaction that grows more morbid
and obscene the more impure and vile it is conceived to be. Such is the triad
of all women before many men’s eyes: Sainct, mystery, filth. Nothing exists
in-between. There is no individual. And it is this, I think, above anything
else, what impedes most men to enjoy love somewhat more freely. 



Birth rates have decreased globally in the last fifty years, with the exception
of a few countries which were consistently relatively low---including Argentina
and Australia. This is not characteristic to Western nations, as it is often
said. Some of the factors which contribute to this development are obvious: the
development of prophylactics, the relatively recent possibility of developing a
career or independence for women, and in some nations the cultural praise of
self-realization and individual success. The only fact which matters when
considering sex, monogamy and marriage, is that sex and pregnancy have been
resolutely separated---and this development is likely irreversible. An argument
could be made that dislike for non-monogamy, at least in men, comes from the fear 
of caring for another man's child. I do not think this is true. In practice, at
least from what I gather, this is not the thought which terrifies men, and in
fact many men are more than capable of loving and caring for a non-biological
son. (It is true that domestic violence is more frequent in these cases,
and this seems to be a general fact of nature---but I only wish to point out
that love is not impossible in these kind of relations.)

Quite curiously, there is one question which is never raised when people think
about these matters---a question which, from my point of view, should be the
first and foremost one. Namely, is a world with more sex a desirable world?
This of course has equivalent formulations: Is sex a good? Does sex produce
happiness? Etc. The undeniable truth is that, given reasonable conditions
---e.g. protection is used to prevent the spread of diseases, there's no
coercion involved, etc.---sex is certainly a source of joy. It is true, at
least in my experience, that this joy grows proportionally with the amount of
love and trust which one feels for the other person, but the $y$-intercept---or
the baseline level of enjoyment, if one wishes---is certainly high itself.
Only once \textit{this} question is answered, can one proceed to evaluate 
ways to deal with the jealousy and possessiveness which we all possess.

At least in Argentina, it is easier to find women that accept the idea of their
partners having sexual commerce without the couple than finding men that do so.
This has to be a direct consequence of feminism, to which naturally women
relate more than men, and which, due to the very nature of the problems it
deals with, facilitates critical and liberal thought on these matters. However,
it still is rather uncommon. The few women that have sincerely shared with me
the reasons why they oppose non-mongamy were, I must say, slightly more twisted
in their conceptions. They seemed to be lest concerned with the actual practice
of sex---the imagination of which spreads fear in men---and more afraid of
losing a privileged position in their partner's life. One particular women I
met, who was absolutely liberal from a philosophical perspective---this is, who
expressed no condemnation of non-monogamy whatsoever---confessed to me that she
would intentionally ill-advise her boyfriends in what came to fashion---their
clothing, haircuts, their overall style---in an attempt to make them
unattractive to other women.


























\end{document}



