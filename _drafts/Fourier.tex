\documentclass[a4paper, 12pt]{article}

\usepackage[utf8]{inputenc}
\usepackage[T1]{fontenc}
\usepackage{textcomp}
\usepackage{amssymb}
\usepackage{amsmath, amssymb}
\newtheorem{problem}{Problem}
\newtheorem{example}{Example}
\newtheorem{lemma}{Lemma}
\newtheorem{theorem}{Theorem}
\newtheorem{problem}{Problem}
\newtheorem{example}{Example} \newtheorem{definition}{Definition}
\newtheorem{lemma}{Lemma}
\newtheorem{theorem}{Theorem}


\begin{document}

I presume the reader is familiar at least with the concept of the Fourier transform. Given a function of time $h(t)$, the Fourier transform of $h$ is 

\begin{align*}
    H(f) = \int_{-\infty}^{\infty} h(t) e^{2\pi i f t} ~ dt
\end{align*}

This is a representation of $h$ in the frequency domain. If $t$ is measured in
seconds, then $f$ is measured in cycles per second, or Hertz. There are
interesting relations between $H(f)$ and $H(-f)$, but I skip them because they
are not essential to the purpose of this entry.

The most important theorem regarding the total power of a signal is Parseval's theorem: 

\begin{align*}
    \text{total power} \equiv \int_{-\infty}^{\infty} |h(t)|^2 ~ dt = \int_{-\infty}^{\infty} |H(f)|^2 ~ df
\end{align*}



The power spectral density (PSD) of $h$ is defined as 

\begin{align*}
    P_h(f) := |H(f)|^2 + |H(-f)|^2  
\end{align*}

for $0\leq f  < \infty$. When $h(t)$ is real, the two terms are equal and thus we have 

\begin{align*}
    P_h(f) = 2|H(f)|^2
\end{align*}

One sometimes sees $P_h(f)$ without this factor of two, but this is only valid
when dealing with two-sided spectral densities. We will always deal with the
one-sided spectral density.

Sampled data is discrete, but so far we have discussed continuous functions. We
need to provide a few concepts before we can extend the Fourier transform to
discrete quantities.

For any sampling interval $\Delta$, there is a special frequency called the
Nyquist frequency, defined 

\begin{align*}
    f_c := \frac{1}{2\Delta}
\end{align*}

If a sine wave with frequency $f_c$ is sampled with sampling rate $\Delta$,
then the distance between a peak and a negative trough value is $f_c$. 

It is quite easy to observe that if we define the sampling rate as $f_s :=
\frac{1}{\Delta}$, we have $f_c = \frac{1}{2}f_s$. Sometimes this alternative
formulation is given.

The sampling theorem states that if a function $h(t)$ sampled with a rate
$\Delta$ happens to be bandwidth limited to frequencies smaller in magnitude
than $f_c$, then $h(t)$ is completely determined by its samples $h_n$, and 

\begin{align*}
    h(t) = \Delta \sum_{n=-\infty}^{\infty} h_n \frac{\sin \left( 2\pi f_c(t - n\Delta) \right) }{\pi(t - n\Delta)}
\end{align*}

This is nice because it shows that the "information content" of a function that
is bandwidth-limited can be totally captured through discrete sampling.

The downside of the theorem is that, when the signal being sampled is not
bandwidth limited, the spectral density outside the range $-f_c < f < f_c$ is
spuriously movied into that range. This is called \textit{aliasing}. Aliasing
can be avoided if one imposes a bandwidth limit artificially---e.g. via the
sampling method---or one samples at a sufficiently small $\Delta$.

The question we know face is that of estiamting $H(f)$ via a discrete sample
$h_1, \ldots, h_N$. 

With $N$ input values we may only output $N$ values; thus, instead of
estimating $H(f)$ for $f \in [-f_c, f_c]$, we may seek estimates only at 

\begin{align*}
    f_n := \frac{n}{N\Delta}  
\end{align*}

with $n = -\frac{N}{2}, \ldots, \frac{N}{2}$. The equation above gives us the
\textit{frequency domain} of our estimation.

Of course, $H(f)$ is approximated via a discrete sum: 

\begin{align*}
    H_n := \sum_{k=0}^{N - 1} h_k e^{2\pi i k n / N}
\end{align*}

Observe that $H_n$, the \textit{Discrete Fourier Transform}, is independent of any parameter other than the sequence $\left\{ h_n \right\} $, including $\Delta$. When interpreting the result in terms of the sampling frequency, the relationship between the continuous and the discrete Fourier transform is 

\begin{align*}
    H(f_n) \approx \Delta H_n
\end{align*}

The discrete form of Parseval's theorem is 

\begin{align*}
    \sum_{k=0}^{N-1} |h_k|^2 = \frac{1}{N} \sum_{n=0}^{N-1} |H_n|^2
\end{align*}

There is a famous algorithm, the \textit{Fast Fourier Transform}, for computing
the discrete Fourier transform of a vector. It is a very interesting algorithm
that is covered in detail in the book I have referenced. I will not explain it
here.


\section{Computing the power spectrum}

The spectral analysis of EEG data typically refers to two things: \textit{(1)}
the power spectrum and \textit{(2)} its spatio-temporal distribution. I will
not concern myself here with spatio-temporal analysis and will only provide
simple R functions that estimate the power spectrum. I will provide this
functions first with a simulated signal and then use them in an EEG fragment.

The main difficulty in power spectrum density (PSD) estimation over EEG data 
is the lack of a standard. Different researchers use different methods, and, 
as I will attempt to show, different methods produce different outputs. 

Let's start from the basis and let's simulate the sampling of a simple signal
wave in R. The signal we are to simulate is 

\begin{align*}
    h(t) = \sin(2\pi \times 100) + 4 \sin(2 \pi \times 110)
\end{align*}

This is, a signal that is a mixture of two sine waves with amplitudes $1$ and
$4$, respectively, and frequencies $100$ and $110$, respectively.

\begin{verbatim}
    
set.seed(123)

# Parameters
sampling_rate <- 500 # Sampling rate (samples per second)
duration <- 5-1/sampling_rate           # Duration of the signal in seconds
frequencies <- c(100, 110)  # Frequencies of the sine waves (in Hz)
amplitudes <- c(1, 4)   # Amplitudes of the sine waves

# Generate time vector
t <- seq(0, duration, by = 1/sampling_rate)
n <- length(t)

# Initialize signal as zeros
signal <- rep(0, n)

# Create a signal as a mixture of sine waves
for (i in seq_along(frequencies)) {
  frequency <- frequencies[i]
  amplitude <- amplitudes[i]
  
  # Generate the sine wave component
  sine_wave <- amplitude * sin(2 * pi * frequency * t)
  
  # Add the sine wave component to the signal
  signal <- signal + sine_wave
}

\end{verbatim}

The amplitude spectrum of the signal is given by 

\begin{align*}
    P(f) = \frac{2|H(f)|}{N}
\end{align*}

which is easily computed in R.

\begin{verbatim}
my_amplitude <- function(x, sampling_rate){
    N <- length(x)
    ft <- abs(fft(x)) # Absolute value of the FFT.
    ft <- ft[1:(N/2 + 1)] # Make one sided.
    freq <- c(0:(length(ft)-1))*sampling_rate/length(x) # Compute the frequency domain.
    amplitude <- 2/N * ft
    return(list(freq = freq, amplitude = amplitude))
}
\end{verbatim}

The PSD is defined as

\begin{align*}
    P(f) = \frac{2|H(f)|^2}{W}
\end{align*}

where $W = \sum_{i=0}^{N-1} w_i^2$ and $w_0, \ldots, w_{n-1}$ is the window
function used. If no window is used, this is equivalent to a rectangular window
and $W = N$. In R, this is again straightforward to compute:

\begin{verbatim}
my_psd <- function(x, sampling_rate, han = TRUE){
    N <- length(x)
    if (han){
        hann <- gsignal::hann(N) # Hanning window
        x <- x * hann
        normalization <- 2/(sum(hann^2)) 
    }else{
        normalization <- 2/(N) # Rectangular window
    }
    ft <- abs(fft(x)) 
    ft <- ft[1:(N/2 + 1)] # Make one sided
    freq <- c(0:(length(ft)-1))*sampling_rate/length(x) # One sided fft
    
    power <- ft^2 * normalization
    return(list(spec = power, freq = freq))
}
\end{verbatim}

Sometimes, one sees 

\begin{align*}
    P(f) = \frac{2|H(f)|^2}{W \times f_s}
\end{align*}

It is important to note that $f_s$ here is a scaling factor that translates
from units of power to units of power/Hz. This is useful when comparing 
powers from signals with different sampling rates. The power spectrum (in
decibels) of our signal, with and without a Hanning window, is the following.



Lastly, I provide a quick implementation of Welch's method. Welch's method consist 
of splitting the signal $h(t)$ into $K$ overlapping segments of length $L$, where 
the overlap is defined as a percentage of the segment length (and hence falls 
within $0$ and $1$). A slightly modified estimation of the power spectral density 
is computed for each segment, and then the estimations of all segments are 
averaged. The estimation in each segment, according to the original paper
(https://ieeexplore.ieee.org/document/1161901), is done as follows:

\begin{itemize}
    \item For each segment $s_i(t)$ of length $L$, compute $F_i := \frac{1}{L}S_i(f)$, where $S(f)$ is the 
        Fourier transform of $s_i$, and $f = \frac{n}{L}$ for $n = 0, 1, \ldots, \frac{L}{2}$.
    \item Compute $U := \frac{1}{L} \sum_{i=0}^{L-1} w_i^2$ with $w_0, \ldots, w_{L-1}$ the window function.
    \item Compute $\frac{L}{U} |F_i|^2$
\end{itemize}

This results in the power spectrum of the segment in question. If everything is put together, and 
we add a factor of two to account for the one-sided spectrum (this is not the case for the 
original paper), the
power spectrum of each segment is 

\begin{align*}
    A_i = \frac{2L}{U} \left[ |\frac{2}{L} S_i(f)| \right]^2 = \frac{2}{LU}|S_i(f)|^2 = \frac{2|S_i(f)|^2}{\sum w_i^2}
\end{align*}

This expression is prettier to program. In R,

\begin{verbatim}
overlaps <- function(signal, L, D){
    # Calculate the number of segments
    D <- L * D
    M <- ceiling((length(signal) - L) / (L - D)) + 1
    
    # Initialize a list to store segments
    segments <- list()
    
    # Create overlapping segments
    for (i in 1:M) {
      start <- (i - 1) * (L - D) + 1
      end <- start + L - 1
      if (end > length(signal)) {
        break
      }
      segments[[i]] <- signal[start:end]
    }
    return(segments)
}

my_pwelch <- function(signal, sampling_rate, L, D){
    

    modified_periodogram <- function(segment){
        L <- length(segment) # L to distinguish that this is the length of a segment, not the whole (original) signal!
        hann <- gsignal::hann(L) # Hanning window
        x <- segment * hann
        ft <- fft(x) # Aₖ(n)
        ft <- ft[1:(L/2 + 1)] # Lake one sided
        w <- 2*abs(ft)^2/sum(hann^2)
        freq <- c(0:(length(ft)-1))*sampling_rate/L 
        return( list(freq = freq, spec = w) )
    }

    segments <- overlaps(signal, L, D) 
    m_periodograms <- lapply(segments, modified_periodogram)
    spectrums <- lapply(m_periodograms, function(x) x$spec)
    n <- length(m_periodograms) # Number of segments
    avg_spectrum <- 1/n * rowSums(do.call(cbind, spectrums))
    return ( list(freq = m_periodograms[[1]]$freq, spec = avg_spectrum))
}
\end{verbatim}



\end{document}



