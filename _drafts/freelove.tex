\documentclass[a4paper, 12pt]{article}

\usepackage[utf8]{inputenc}
\usepackage[T1]{fontenc}
\usepackage{textcomp}
\usepackage{amssymb}
\usepackage{newtxtext} \usepackage{newtxmath}
\usepackage{amsmath, amssymb}
\newtheorem{problem}{Problem}
\newtheorem{example}{Example}
\newtheorem{lemma}{Lemma}
\newtheorem{theorem}{Theorem}
\newtheorem{problem}{Problem}
\newtheorem{example}{Example} \newtheorem{definition}{Definition}
\newtheorem{lemma}{Lemma}
\newtheorem{theorem}{Theorem}


\begin{document}

\section{Introduction}

The term \textit{free love} has two meanings, one philosophical and one
practical. Its philosophical meaning consists of a theory of
love---this is, a set of ideas with respect to what love is, how it ought to be
exercised, and what is its purpose and place in human life. 
The practical meaning is the consensual practice and exercise of free love with
others.

~ 

Before delving into these meanings and the way in which both relate to one
another, I think it is important to contextualize our discussion and understand
its place and relative importance in human affairs. From the perspectives of
social organization as well as the individual pursuit of satisfaction and
happiness, the matter is of definite importance. The organization of family,
with its wide range of variation across the temporal and cultural domains, has
implications on social order which cannot be disregarded, and which
anthropology has pointed out from Morgan and Engels onwards. Furthermore, love
is, beyond any doubt, one of the deepest wells of meaning, and one of the most
important drivers of human action alongside the search for knowledge, the
desire of money or power, and the pursuit of creative and artistic realization. 

~ 

Of course, as love is a source of joy, it is perhaps among the cruelest sources
of anguish, conflict, and even hatred. I grew up involved in the divorce of my
parents, which saw a genuinely loving couple disintegrate under the most
tempestuous storm of hatred, misunderstanding, wickedness and confusion; one
which drove otherwise decent people to commit cruel, manipulative
and unworthy acts. It is my firm belief that, as in most things, love is lived
with conflict and unhappiness due to a lack of a proper education and an
incorrect understanding of its nature and the way it bounds us with one
another. If we could reach a better understanding of love, one guided by
freedom, compassion and tolerance, we would reduce much of the pains it
typically brings about in individual life.

~

Notwithstanding its personal and social importance, it is crucial to understand
as well that the question of whether love ought to be exercised freely or
not---in the sense in which we will discuss---pales in comparison with other
matters. The concentration of power, the fragility of the democratic order, the
fight against state and private tyranny, and the threat of nuclear or climate
disaster, are all examples of questions which surpass the one we will discuss
in importance. 

~

It may seem superfluous to emphasize the relative inferiority of
our question with regards to, for instance, nuclear disaster or state tyranny.
But I believe my generation has mistakenly devoted too much energy into
questions surrounding love, sexuality and identity, while disregarding others
which are more important to human dignity and which affect a greater number of
people. It is impossible not to observe that, in general, only the privileged
have the time as well as the cultural resources to inquire on the philosophy of
free love, not to mention that so aristocratic form of irreverence that allows
certain people to follow practices which stand completely opposite to accepted
customs. As a rule, an irreverence which would have a significant social cost
to a marginalzed person is, with some luck, condescendingly tolerated as a curious
eccentricity in more educated or privileged circles.

~ 

\section{Philosophy of free love}

\subsection{What is free love}

Like anarchism, free love is not a definite ideology, but a set of flexible and
relatively unstructured beliefs inspired by a few common and fundamental principles. 
These principles are, in general, adaptations or extensions of more general
ones which are widely accepted. For instance, it is relatively easy to have
someone accept the following statement:

\begin{quote}
    As long as no-one's freedom is being limited by their actions, and no harm is being done unto others, I have no right to limit the freedom of another person.
\end{quote}

However widely accepted this principle is, people struggle to extrapolate it to
the freedom to love. But one cannot accept the general principle and abstain
from such extrapolation, specially when love is---as most people would
accept---a virtuous thing. 

~ 

The essential principle which guides free love was formulated in general terms by 
Bakunin in \textit{God and State}:

\begin{quote}
    The liberty of man consists solely in this: that he obeys natural laws
    because he has himself recognized them as such, and not because they have
    been externally imposed upon him by any extrinsic will whatever, divine or
    human, collective or individual. 
\end{quote}

Thus, believers in free love claim that everyone should be able to exercise their 
love of others without the exterior imposition of any limiting conditions. We accept 
axiomatically that love is a virtuous thing, and thus that limiting its
exercise and expression is morally equivalent to limiting the exercise and expression 
of generosity or kindness. 

~ 

Exterior impositions to love come in many ways. For centuries, the state has
taken upon itself to handle the private affairs of men and women alike. The
legal institution of marriage, the informal and consuetudinary precepts of this
or that culture, the various religious principles which establish what forms
of love are chaste and which are vile, are all examples of exterior limitations
to the free exercise of love. All of these, if one adheres at least rudimentary
to liberal principles or their anarchist continuation, ought to be utterly
rejected: their social existence should be dismantled, and their laws, when
organizing our individual life, should be exchanged for others which we have
set for ourselves.

~

Equivalently, if one believes in free love, one gives up completely any
pretension to impose limitations on the way others exercise, express, and
practice their love affairs. This includes our partners, our sons and
daughters, our friends, and our neighbours. One must recognize that an
individual's pursuit of affective or sexual satisfaction lies absolutely
without the scope of one's domain, surrender altogether all desire to influence
it, and achieve a radical acceptance of it. One must accept as axiomatic that 
romantic or sexual interactions between consenting adults lack any inherent evil
and, on the contrary, comprise at least potentially a virtuous and positive act.

\subsection{Free love and non-monogamy}

It should be self-evident that advocating for free love is not logically
equivalent to advocating the idea that one should love multiple people, or that
monogamy should not be practised. The only advocacy in place is this: that no
exterior element---be it a person or an institution---can impose any
restriction upon you with regards to who you love, how you express your love,
nor how many people you love. Such matters fall entirely within the boundaries
of your individual freedom.

~

Thus, free love is not equivalent to non-monogamy. A monogamous couple may be
in pure accordance with the principles of free love, and a non-monogamous
couple may fall in entire violation of them. The kind of contract which suits a
couple is a highly idiosyncratic matter, and one cannot seriously propose that
a \textit{generally} superior one exists. As stated earlier, to believe in free
love is to believe that \textit{exterior} impositions are utterly contrary to
the spirit of all virtue, including love and affection---interior conditions
are not only appropriate, but unavoidable.

~ 

Notwithstanding, two things ought to be said in this regard. Firstly, it is
easier to conceive non-monogamous arrangements if one believes in free love.
The philosophy of free love simply makes one more open to the idea, and free
from the misconception that love or sex among consenting adults can, under any
circumstance, be a sin. Secondly, an absolutely rigid form of monogamy---i.e. a
form of monogamy which allows absolutely no exceptions---is inconsistent with
the philosophy of free love. Everybody will, sooner or later, feel sexual or
romantic attraction to people beyond their partner---this is natural and lies
well within normal human nature, and only a dishonest denial of
facts could pretend otherwise. When this happens, insofar as the desire already
exists, though the freedom of the individual is left untouched---we are
assuming they have agreed to the monogamous contract freely and
voluntarily---the free expression of their romantic or sexual desire is being
limited. Their past self has agreed to this limitation, but their present self,
ignited by desire, infatuation or even love, will feel it is alien and
exterior. To avoid such contradiction, any agreement respecting of the philosophy 
of free love must lie, so to speak, in a semi-open interval between absolute anarchy
(included in the spectrum) and absolute monogamy (excluded from the spectrum).

~

Mary Wollstonecraft, in her famous \textit{Vindication}, said with respect to
women:

\begin{quote}
    Make them free, and they will quickly become wise and virtuous
\end{quote}

I venture, with regards to free love and non-monogamy, that an equivalent
statement holds: \textit{Make them free, and they will quickly become non-monogamous}.
This, in my opinion, is a positive thing: a world where love and affection are
abundant is superior to its alternative. Non-monogamy, just like monogamy, is
filled with bittersweetness, gains, and losses. However, human affection is the
purest form of ecstasy, and the benefits non-monogamy brings not only to the
individuals, but (surprisingly to many) to the couple involved, far surpass the
costs.



\subsection{Objections to free love}


I should wish to examine some common objections to free love and
non-monogamy. It is my belief that these objections are invalid and should not
be accepted. Some of these are the fact that people suffer when their partners
have romantic or sexual relations with others, the importance of true
commitment, the risk of spreading disease and unwanted pregnancies, and the
need for relational stability and a nurturing environment in the raising of children.

\subsubsection{Free love produces suffering}

Any self-respecting philosophy which deals with human affairs must be inspired
by compassion and kindness. It should strive to produce beliefs whose
corresponding acts induce the proliferation of joy and happiness or, at least,
the avoidance of suffering. Though a certain degree of pain is unavoidable 
in all human relations, I am not inclined to the romantic idea that inextricably 
ties love to suffering, turmoil, and sorrow. If conditions are met for love to
be exercised, it should be exercised with innocence, delight and candour.

~ 

However, as pointed out before, love can be a terrible source of anguish. Among the
causes which make love an unhappy matter, there are exogenous and
endogenous ones. Among the first, we have the existence of objective
obstacles which split lovers apart--e.g. distance, jail, war, or death---and
the submission to institutions which bound the expression and exercise of love
within socially acceptable bounds--e.g. marriage, religious prohibitions, state
intervention, etc---. The latter may be reduced to the lack of a
proper sentimental education which impedes people from enjoying love, even when its
object lies within their reach and its potential for free and spontaneous
expression is not limited from without. 


~ 

Discussing the exogenous factors which limit the free exercise of love lies not
within the scope of this writing. Those which are forceful factors, like war
and death, are not the subject of philosophy; those which are social or
political factors, like certain social institutions, have been widely discussed
in the anarchist and feminist traditions. So, I should wish to discuss the
manners in which we impose endogenous limitations both on the expression of our
love and on that of others. The suffering which is caused by the idea or the
fact of a loved one having a sexual or romantic interaction with another, which is 
the core of the objection we are dealing with, lies
within the category of endogenous evils against the enjoyment of love.

~

Any worthy sentimental education must begin by recognizing the facts. Thus, we
must accept that, in most people, true suffering is felt when their partners
have sexual or romantic affairs with others. Though there are exceptions, the
majority dislikes the idea of "sharing" their partners with others, due to
insecurity, possessiveness, jealousy, or fear of losing them. It is important
to recognize that this is a visceral reaction, which typically occurs without
much cognitive elaboration: it simply feels bad, and we may not know exactly
why. There is nothing inherently vile about these feelings, and they lie well
within the bounds of normal human experience.

~ 

It is also a fact, however, that these feelings can be nullified, or at least
alleviated, if a proper outlook is taken. Since these feelings are not equally 
caused in people by the idea of their partners loving someone else, on one hand,
and their partners having sex with someone else, on the other, we should discuss 
the sexual and (so to speak) romantic freedom of our partners, with the
feelings they cause in us, separately for the time being.

~ 

With regards to sex, it is obvious that it is not a transformative act. After a
sexual encounter, the same individuals persist. Thus, it is not rational to
conceive that, if we love another person, this love could suffer any change
after that person had sex with another. More radically, we must accept that 
sex, assuming normal conditions---no coercion involved, no diseases are spread,
unwanted pregnancies are avoided---is a positive thing. If a person feels sexual 
desire towards another, it is certainly a source of joy and happiness to
realize that desire, and a world where such realization is not bounded by
exterior limitations is morally superior to its alternative. 

~ 

With regards to love, the matter is different, insofar as love can be
transformative. The influence of love, when the passion is too strong, can
begin wars, induce the abandonment of children, and revolutionize a person's
life. Thus, it is not irrational to conceive that if a partner falls in-love
with someone else, they may forget or leave us. However, three things ought to
be said with regards to what we do with this legitimate fear.

~ 

Firstly, it is a fact that a person can love many others, and hence that
multiple loves do not cancel each other out. This is obviously true in 
friendship and it is a moral mistake not to acknowledge that it is true in 
all other kinds of love. Thus, if our partner falls in-love with someone else,
it really does not mean that it will love us any less. We may feel certain
sadness if our partner has less time for us now, but this feeling is not
peculiar to love: many feel the same when a friend befriends a new person. And
even if, for some strange reason, we cannot overcome these distasteful
feelings, to argue that because of this our partner should limit their romantic 
lives to us only is as absurd as arguing that our friends should limit their 
friendship to us. 

~ 

Secondly, we may assume our fear is true and that our partner finds somebody
who satisfies more than us, or brings them more joy. Should this be the case,
there is no non-selfish reason not to allow them to explore the opportunity. We
have absolutely no right to limit the scope of another person's reach,
specially in matters which relate directly to their personal development and
the exercise of their passions and virtues. And it is simply a display of
brutality to respond to this scenario with a limitation to another person's
freedom, instead of with a willingness to let people discover who they belong
with.

~

Thirdly, not exercising free love does not protect us from the suffering induced 
by the idea or fact of a partner loving, or having sex with, someone else.
Monogamous and non-free couples constantly suffer from this, either in imagination
(jealousy) or in fact (infidelity). Monogamy provides only an illusory
protection against the possibility of our partner (or ourselves, for that
matter) falling in-love with someone else. One may argue that allowing the
exploration of an infatuation increases the risk of such infatuation growing
into love, while the monogamic limit extinguishes the fire before it begins to
grow. However, I would argue the exact opposite: infatuation grows more
passionate and idealistic the less its exploration is allowed, and generally
the person which infatuates us quickly becomes humanized, or even
uninteresting, when we are allowed to pursue her. Thus, an illusion which may
cause serious distress in a monogamous arrangement can be rapidly dissolved in
non-monogamy, if only we allow ourselves and our partners to pursue a romantic
affair with the illusion master.

~

In summary, not only a reasonable understanding of sex and love educates our
feelings into accepting, alleviating, and perhaps extinguishing the pain of our
partner loving someone else, but it is also false that this pain is avoided in
non-monogamous arrangements. Thus, even if we do not accept the points in which
I conceive non-monogamy to be superior, we must recognize that it is \textit{at
least} as bad as monogamy in this regard.

~

A final point of principle is to be made. It is a general rule that we do not
have the right to limit the freedom of others simply because their acts dislike
us. No material nor moral harm is done unto us when our partner shares a sexual
or romantic bond with another person. Furthermore, if we accept love is a
virtue and sex is a positive thing, not only our partners aren't doing us any
harm: they are acting virtuously. Hence, the fact that we dislike our partners
seeing other people may be unpleasant, but it constitutes no argument in favour
of restricting our partner's liberty to do it. 

\subsubsection{The importance of true commitment}

When asked to justify their rejection of non-monogamy, it is quite common for
people to claim that they value true commitment. The hidden assumption is that
true commitment is impossible without monogamy. This assumption is of course
factually wrong, insofar as many cultures, across the temporal and cultural
domains, practise or have practised non-monogamy without a lack of allegiance,
responsibility, and dedication between the parts. However, putting aside its
disregard of facts, the assumption is also revealing of a perverse concept of
commitment. 

~ 

Commitment should consist of the voluntary dedication to another person, the
satisfaction of their affective needs, and the advancement of whatever life
project one has with them, if any. It is true that it is easier to become 
inattentive of a person's needs, or careless with a relationship, if one has 
to attend romantically to many people. However, once more, lack of commitment 
and inattentiveness are widespread problems in monogamous arrangements too. It seems
to me that, in this regard, \textit{les extrêmes se touchent}: both not seeing other 
people as well as seeing other people carry the risk of producing emotional 
negligence. 

~

People underestimate how much sharing a romantic affair with a person can boost
our romantic connection with another: my virtues and singularities become
apparent when you get the chance to explore those of others, while at the same
time the acknowledgement of flaws in others will allow you to become more
compassionate with my own. Monogamy produces negligence because it produces
boredom, it makes couples take each other for granted, it erodes the positive
impression which virtues produce, and it exacerbates our partner's flaws.
Seeing other people, for the reasons aforementioned, makes it easier for true
commitment and genuine connection to last in time.

~ 

Furthermore, depending on one's life conditions, it is conceivable that having
not too great a number of other romantic bonds still leaves time for managing a 
true commitment with another person. Here, however, I believe a certain hierarchy 
is unavoidable: a working adult has little time for leisure, and dividing such time 
equally among their partners will necessarily leave too little time for each, making 
commitment impossible. Some people dislike the idea of placing their romantic affairs 
in a hierarchy, but in my view this is a natural phenomenon which occurs spontaneously.
For instance, this occurs spontaneously in friendship: not all friends are
equally close to us, nor do we distribute our time equally among them. This is
reasonable, because we cannot expect to click with everybody in equally
profound fashion. And what applies to friendship, in most cases, applies to
love as well.

~

Thus, not only is free love compatible with true commitment, both from a
sentimental as well as a practical perspective, but it is also superior to
monogamy in allowing true commitment to survive in time. Furthermore, while it
is true that certain characteristics of free love, if handled irresponsibly,
can facilitate detachment, the same is true of monogamy. 

\subsubsection{The risk of disease and unwanted pregnancies}

The more sexual encounters one has, the greater the risk of having an unwanted
pregnancy or getting a disease. These two factors, I believe, were absolutely
paramount in making sense of monogamy. Maternal death was, and still is, a
serious issue, specially among the great majority of the
population living with limited resources.
(https://www.who.int/news-room/fact-sheets/detail/maternal-mortality) Assuming
the mother survived an unwanted pregnancy, the caring of an unplanned child
still implied severe changes in life conditions for the whole family. Sexually
transmitted diseases were more common and harder to avoid in the past, due to a 
lack of scientific sophistication in medicine as well as poorer understanding of 
sexual affairs in the general public.

~ 

If material conditions were the same as before, true sexual freedom would be
impossible. However, the existence of contraceptive pills and prophylactics,
the (slowly) increasing degree of sexual education among the general public,
and in some places the legalization of abortion, have radically changed the
risks associated with sexual relations. Such risk is not null, insofar as even
the best prophylactic measures have a margin of error, but it is close to
negligible, and even extreme promiscuity falls well within reasonable behavior,
when analyzed solely from this perspective, if safety measures are taken to
avoid pregnancy and disease transmission.

\subsubsection{The need for a nurturing environment in the raising of children}

It is unclear whether the exercise of free love is compatible with the raising
of children. The free love spectrum is too wide and comprises too many forms of
organizing an eventual family for a general position to be taken. For
instance, we cannot take as the same a polyamorous family where children live
with many adults in the same household, all or many of whom are romantically
related, with a quasi-monogamous family where the children live with two
biological parents who occasionally see other people outside the familiar home.
Furthermore, there are too many factors to control for: the same arrangement
may produce a stimulating and rich environment provided that children are
nurtured with quality education as well as material stability, while producing
a negative environment if some of these factors are lacking.

~

I would venture that those forms of free love which are closer to
monogamy are at least as bad for the raising of children, and arguably better,
than monogamy itself. But arguing in favour of this conjecture would take too
long, and it is best not to spend too much energy on a conjecture which I
myself deem so weak and insecure. Thus, in what comes to the raising of children,
I suspend my judgement for the time being, and concede that it is the one objection 
against the exercise of free love which has reasonable grounds.








\end{document}



