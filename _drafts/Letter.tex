\documentclass[a4paper, 12pt]{article}

\usepackage[utf8]{inputenc}
\usepackage[T1]{fontenc}
\usepackage{textcomp}
\usepackage{amssymb}
\usepackage{newtxtext} \usepackage{newtxmath}
\usepackage{amsmath, amssymb}
\newtheorem{problem}{Problem}
\newtheorem{example}{Example}
\newtheorem{lemma}{Lemma}
\newtheorem{theorem}{Theorem}
\newtheorem{problem}{Problem}
\newtheorem{example}{Example} \newtheorem{definition}{Definition}
\newtheorem{lemma}{Lemma}
\newtheorem{theorem}{Theorem}


\begin{document}

I have struggled to accept this, but I must speak the truth. I feel an abyss
growing wide and dark between us, blackening what once was the territory of
your light. It is not only that I am finding it more difficult to enjoy your
company, to be patient and compassionate with your flaws, to feel the bliss of
your laugh and the touch of your skin---It is that I cannot convince myself
that you do not feel the same. Yes, there are days when we both forget the
abyss, when we dance around each other with sincere delight, when we remember
what it was to be affectionate and young. But these are moments of
remembrance, of reminiscence, of recovering now lost chastities and wisdoms.
Love remains in both of us, but it is a love that grew in bygone times---a love
that persists beyond the people who incarnated it. You see it too, don't you?
You once adored me deeply---now you battle to accept me and, in your own words,
wish you didn't love me as much as you do. This marks a difference between 
you and I, a stark one: for I would never wish not to have felt, not to feel,
this love which has revolutionized the foundations of my soul. You wish 
to be free from it, to be unbound---I only wish to persist,
even if that requires adaptation, change, adjustment, transformation. Nothing 
in this world persists: even our love, as it was, has been eroded by the wind.
You may desire to abandon it: I contemplate the curious shapes it takes as the 
breeze transfigures it, and think: \textit{Yes, that too is love}\ldots 








\end{document}



