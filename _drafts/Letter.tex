\documentclass[a4paper, 12pt]{article}

\usepackage[utf8]{inputenc}
\usepackage[T1]{fontenc}
\usepackage{textcomp}
\usepackage{amssymb}
\usepackage{amsmath, amssymb}
\newtheorem{problem}{Problem}
\newtheorem{example}{Example}
\newtheorem{lemma}{Lemma}
\newtheorem{theorem}{Theorem}
\newtheorem{problem}{Problem}
\newtheorem{example}{Example} \newtheorem{definition}{Definition}
\newtheorem{lemma}{Lemma}
\newtheorem{theorem}{Theorem}


\begin{document}

    
Dear N. Robinson,

I write to you from Argentina to express my appreciation for your contributions
to critical journalism and your constant advocacy for the most noble of causes,
from Palestinian liberation to wealth distribution. I write to you, as well,
from an impoverished nation, one that suffers from the sudden reappearence of
a bitter far-right---one whose history of violence, oppression, and
blood is rich and perhaps cyclic.

I read \textit{Current affairs} daily, and if the material conditions around me
allowed for it, I'd become a paying subscriber a long time ago. I
endorse the magazine (as well as \textit{Why you should be a socialist}) to
everyone here, with relative successs. As we from the South observe the United
States, that distant hegemon, grow weaker and more somber everyday, we find in
your magazine a beam of light that blazes from within that terrible
obscurity.

I work at a neuroscience laboratory in the United States, and will be living
there in a couple of years, so the internal affairs are of relevance to me.
However, if you would allow me the impertinence, I would certainly enjoy more
articles on Latin American affairs. I know many of the people to which I
recommended the magazine here have felt this way. Perhaps in the future \textit{Current affairs}
will consider expanding the contents of its excellent articles further South. 

Again, I express my sincere appreciation for your work. It is the work of people
like yourself what keeps hope alive in this unjust world. 

Best,
Santiago











\end{document}



