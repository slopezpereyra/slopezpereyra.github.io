\documentclass[a4paper, 12pt]{article}

\usepackage[utf8]{inputenc}
\usepackage[T1]{fontenc}
\usepackage{textcomp}
\usepackage{amssymb}
\usepackage{newtxtext} \usepackage{newtxmath}
\usepackage{amsmath, amssymb}
\newtheorem{problem}{Problem}
\newtheorem{example}{Example}
\newtheorem{lemma}{Lemma}
\newtheorem{theorem}{Theorem}
\newtheorem{problem}{Problem}
\newtheorem{example}{Example} \newtheorem{definition}{Definition}
\newtheorem{lemma}{Lemma}
\newtheorem{theorem}{Theorem}


\begin{document}


Existe una forma de la tristeza que es, por así decirlo, positiva. Me refiero a
aquella que inclina nuestra pensamiento a la auto-crítica.
El desagrado que sentimos, o que, al menos en justa medida, deberíamos
sentir por nosotros mismos, tiene algo de constructivo.

Pienso que nuestros intentos de tratar constructivamente con nuestros malos
ánimos es similar, en su ironía, a nuestros infructuosos intentos de estudiar
el sueño. Me refiero a que en aquél, como en éste, todo espíritu analítico o
científico parece destinado al fracaso, y la interpretación poética supera, por
lo general, con creces al estudio meticuloso. Del sueño, la ciencia nos ha
provisto de poco más que ciertas fases distintivas—cuya utilidad científica
es, a lo sumo, dudosa, cuya caracterización es pobre, y cuyos criterios de
identificación son poco claros—. A esto se añade una poco interesante
acumulación de observaciones experimentales sobre los efectos de su privación,
apropiadamente sistematizada y estandarizada—es cierto—pero en sus rasgos
generales del todo obvia y sabida por el más ignorante de los hombres.
Toda esta falsa sabiduría es, a mi juicio, ampliamente superada por la 
observación de Shakespeare, *we are woven from the fabric of which our dreams
are made*, o la precisa intuición de Epicuro, que ya cité en otro artículo:


> τά τε τῶν μαινομένων φαντάσματα καὶ <τὰ> κατ' ὄναρ ἀληθῆ, κινεῖ γάρ· τὸ δὲ μὴ
> ὂν οὐ κινεῖ 
> 
> *The imagery of delusion or of dreams is real inasmuch as it is stimulating:
> what does not exist, does not affect.*

¡Tantos miles de papers, superados por versos e intuiciones!

En mi caso, me resulta difícil aprovechar el ímpetu que me otorga mi poca
conformidad conmigo mismo. Soy, o al menos así me parezco a veces, un alma
pésima y sin luz. Esto es particularmente cierto en el amor, que busco
incansablemente y nunca aprendo a disfrutar serenamente. Creo identificar al
menos cosas cosas de mí que, a este respecto, causan dolor en mí o en los
demás.

La primera es una inestabilidad terrible en mi representación de los demás. Si
bien no lo externalizo, mi fuero interno se mueve de la absoluta satisfacción
al desagrado o al repudio a veces en cuestión de minutos. Los demás (o así lo creo) 
no sufren de esta desgracia, porque, consciente de su carácter neurótico y endógeno,
la oculto en mi interior. Pero sí castiga mi corazón, causando malestar
innecesario en un cariño que podría fluir estable y continuamente.

La segunda es una dificultad atroz para simplemente aceptar aquello que se me
ofrece. Incluso con personas afectivamente cálidas o generosas, siento que
nunca es suficiente. Nunca termina de ser claro, ni siquiera para mí,
exactamente qué es aquello que podría satisfacerme plenamente. Soy consciente
de que abandonar toda demanda y simplemente aceptar aquello que el otro puede y
quiere darme es el camino a una vida más feliz y satisfecha. Pero en mi
corazón, jamás parezco estar saciado.

A todo esto, uno debe añadir que he sido terriblemente frío






























\end{document}



