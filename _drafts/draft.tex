\documentclass[a4paper]{article}

\usepackage[utf8]{inputenc}
\usepackage[T1]{fontenc}
\usepackage{textcomp}
\usepackage{amsmath, amssymb}
\edef\restoreparindent{\parindent=\the\parindent\relax}
\DeclareSymbolFont{letters}{OML}{ztmcm}{m}{it}
\usepackage{parskip}
\restoreparindent
\newtheorem{problem}{Problem}
\newtheorem{example}{Example}
\newtheorem{lemma}{Lemma}
\newtheorem{theorem}{Theorem}
\newtheorem{problem}{Problem}
\newtheorem{example}{Example}
\newtheorem{definition}{Definition}
\newtheorem{lemma}{Lemma}
\newtheorem{theorem}{Theorem}

\begin{document}

This is not a philosophical writing \textit{per se}, but a collection of
observations with regards to my interest in philosophy. I wish to obtain
clarity with respect to what matters of philosophy concern me, on a personal
level, and which I more or less disregard. It would perhaps be too much to
say that these notes conform a \textit{project}, an attempt at organizing
future studies. For the moment I should only claim that these are
\textit{notes} ---in an intentionally general and undetermined sense.

\textbf{A note on epistemology : } I hold a view that is common among analytic
philosophers; namely, that philosophy can not aspire to produce certain
knowledge and is a speculative matter only. In this regard, the observable trend
by virtue of which science more and more occupies the once central role played
by philosophy is in some sense justified. 

This claim requires some satisfactory justification of scientific knowledge in
epistemological terms. Science itself can not produce such justification,
insofar as it is concerned with empirical matters. In a sense, science is like
an illiterate but exceptional scribe. It is the best producer of that which it
will never get to comprehend. Of course, this follows directly from the fact
that science is nothing but a particular \textit{method}. But if the set of
scientific propositions is different from others by virtue of that method only,
the question is how can a method bestow a propositional formula with truth?

One needs not dig too much to see why the previous question is ill-posed.
Assume, for example, two different but both valid derivations of natural
deduction come to the same conclusion $\varphi$. Assume the hypotheses
sustaining one and the other differ in the following ---still undetermined---
sense: that those of the first are \textit{scientific} and that of the second
are not. What possible meaning could such distinction have? The only one is that
the hypotheses of the first derivation are of empirical nature: this is, that
they are verifiable through observation. If this were not true, then there would
be no difference between deriving a true proposition in a scientific manner
and, say, deriving it via the assumption of some arbitrary contradiction.

In short, science is not a system of logic, and its essence lies not in some
special set of inferential rules. It is a commitment to the idea that any
derivation is truth-preserving only if at least the set of maximal hypotheses
consists of purely empirical atomic propositions. In short, it is a radical
commitment to empiricism. And since empiricism is not an unproblematic stance,
it is important to develop an epistemological framework that supports it. Should
philosophy ---and science--- fail to do so, the advancement of science as the
superior method for the production of knowledge is unjustified.

From this follows that one of the central problems of philosophy is the
justification of knowledge; particularly, of how sense data is transformed into
knowledge. The most important matter on this regard are $a.$ some justification
for the claim that atomic propositions are empirically verifiable ---which
implies the postulate of an objective reality--- and $b.$ the justification of
inference ---which is tremendously difficult. I am of the opinion that a true
commitment to empiricism necessarily leads to basing our general conception of
knowledge upon probability. On this matter it is quite conceivable that a
Bayesian take on probability is more satisfactory than the traditional one, but
I do not want to go in detail into this. Suffices to say that a solid
justification for probabilistic claims is perhaps an incidental necessity
for the development of $a$ and $b$.

\textbf{A note on an important question : } It is a fact that we are only
seemingly ephimerous. Although it is probably true that our existence ends with
death, it certainly did not begin with birth. Every individual carries the
undeniable weight of its phylogenetic history. We are, so to speak, of thousands
of years old. It is crucially important to develop a form of empiricism that
takes into account our phylogenetic endowment without being self-contradictory.
An escape from contradiction is not altogether clear: the very statement that
our intellectual predispositions can not solely be based on experience is an
empirical observation. If such observation is true ---I find it undeniable---,
then it bears significant implications with regards to the limits and the nature
of our knowledge. (One is inevitably reminded of Nietzsche's extraordinary piece
\textit{On truth and lies in a nonmoral sense}.) 

I should quickly add that this fact does not contradict empiricism \textit{by
necessity}. In other words, it is not \textit{in principle} impossible to
develop a theory of empiricism that incorporates this fact. It is just that this
development is non-trivial.

\textbf{A note on ethics : } In my mind, it is impossible to produce an ethical
system whose propositions are worthy of the epithet of \textit{knowledge}. Here,
I follow the more or less general idea that the propositions "$X$ did $Y$, which
is wrong" and "$X$ did $Y$" are entirely equivalent from a propositional
perspective, where the aggregate "which is wrong" corresponds to an emotional
predisposition rather than a property of reality. In other words, ethical
predicates don't really say anything about their subject.

Furthermore, ethics is a matter of ends, and rational thinking is concerned with
means only. There really is no way to show an ethical end is superior to
another, since they necessarily appeal to the same tautological axiom. Indeed,
the necessary tautology upon which the whole of ethics is based is the claim
that "good is preferable to evil", which is as superfluous as saying "hotness is
warmer than coldness" or "joy is better than sadness". The axiom poses a
relationship between the atoms "goodness" and "evil", and appears in this sense
similar to Euclid's "all right angles are equal to one another". But while
"right angle" is a valid name, in the sense that it unambiguously refers to an
object, "goodness" and "evil" are not any kind of fact. The statement is true
insofar as "goodness" is \textit{by definition} that which is preferable to
evil, and "evil" is \textit{by definition} that which is not preferable to
goodness. They are the ethical equivalents to the concepts of \textit{function}
and \textit{expression} in mathematics, in the sense that a function is that
which is not an expression, and an expression that which is not a function
(Frege). But can be objectively said past this point?

In short, ethics is irreducible past the point of the tautological axiom that
claims the superiority of goodness over evil ---and yet provides no grasp of the
facts to which these terms correspond. Furthermore, such correspondence appears
to be nonexistent. It follows that ethics will always be in need of philosophy
---and a good deal of affection. 

\textbf{A note on metaphysics : } In general, the pursuit of knowledge unfolds
under highly limited conditions. The best questions are precise and touch upon
either particular objects or relationships among particular objects. From where
then comes the arrogant tendency of pretending to reduce reality, the whole of
reality, to a system? Any set of propositions that deals with the entirety of
reality is suspicious to me. The only true artisans in the art of metaphysics, I
come to think, were Schopenhauer and Kant. But this only either on the very
edges or on the very core of their philosophy: whatever fills the space between
them is more or less forgettable to me. In what comes to some other big names,
like Heidegger and Hegel, I am afraid I can only hold the claim ---perhaps
pretentious--- that they were utter charlatans. The \textit{Phenomenology of the
Spirit}, which I read when I was seventeen, impressed me at the moment, but as
the years went by I come to think it was nothing but a waste of time.
\textit{Being and Time}, on the other hand, is an unbearably stupid work that
consists only in the insistence to make the most empty statements seem like
philosophy by means of flamboyant terminology. (I should add that I hold not
only an intellectual dislike of these works, but a personal one too, insofar as
it seems to me that both to write them and to adhere to them, their
meaninglessness being so evident, is nothing but an act of vanity to my eyes.)

I should also say that the same reasons that make me weary of metaphysics also
prevented me from enjoying the work of authors that were not metaphysical at
all. A polemic example might be Foucault, in whose works I found very little of
value and the same sin of \textit{mucho ruido pocas nueces}. In general ---and
in this I agree with Chomsky--- the French school of philosophy disregards every
concern for falsifiability and truth and bargains it for (self-proclaimed)
originality. I personally prefer the humble pursuit of truth than the
fabrication of clever statements.

Lastly, I should wish to comment briefly on theology. Curiously enough, as I
have always been an atheist, some theological works are of great literary value
to me. When they are not too technical they can be very beautiful. Particularly,
Origen of Alexandria and Saint Agustine are very close to my heart. But in
philosophical terms it is needless to say they miss the mark from the very
beginning and can offer nothing to anyone that searches for some minimal amount
of truth.

Summarizing, I adhere to the position that philosophy should strive for sharp
edges and clarity, and that simple and humble systems are of greater value than
complicated and transcendent ones. This makes the whole of metaphysics, but not
only metaphysics, a quite unpleasant matter to me. In this, I believe the
greatest teaching of Nietzsche was to show that one could look at those hefty
volumes and claim that they are rubbish ---although regrettably enough the whole
of his philosophy is \textit{also} nothing more than Nietzsche's
\textit{opinion}, however penetrating.

\textbf{ A note on the philosophy of mind : } Philosophy of mind is very dear to
me, because it is biographically linked, together with logics, to my interest in
computer science and neuroscience. I believe neuroscience is the only hope at
attaining knowledge on the phenomena that falls under the broad category of
\textit{mental experiences}. (I for the moment disregard the question about
whether it is valid to speak of "mental" experiences ---suffices to say I adhere
to the most radical materialism on the matter.)

To my mind, philosophy plays a crucial role in the study of these phenomena.
That role consists in aiding scientists in the making of the \textit{right
questions}. It is a historical fact that philosophy has endured a better
education on the analysis of \textit{questions} ---although it has fallen behind
in the production of answers. Now, in general terms I do not accept the
postulate that the only role of philosophy is to perform this kind of analysis
---philosophy should strive to produce answers as well. But in the particular
case of the philosophy of mind, about whose object we have better means of
attaining answers than philosophical inquiry, I believe this is its noble place.


\end{document}
