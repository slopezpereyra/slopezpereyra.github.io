\documentclass[a4paper]{article}

\usepackage[utf8]{inputenc}
\usepackage[T1]{fontenc}
\usepackage{textcomp}
\usepackage{amsmath, amssymb}
\edef\restoreparindent{\parindent=\the\parindent\relax}
\DeclareSymbolFont{letters}{OML}{ztmcm}{m}{it}
\usepackage{parskip}
\restoreparindent
\newtheorem{problem}{Problem}
\newtheorem{example}{Example}
\newtheorem{lemma}{Lemma}
\newtheorem{theorem}{Theorem}
\newtheorem{problem}{Problem}
\newtheorem{example}{Example}
\newtheorem{definition}{Definition}
\newtheorem{lemma}{Lemma}
\newtheorem{theorem}{Theorem}

\begin{document}
    
This is not a philosophical writing *per se*, but a collection of
observations with regards to my interest in philosophy. I wish to obtain
clarity with respect to what matters of philosophy concern me, on a personal
level, and which I more or less disregard. It would perhaps be too much to
say that these notes conform a *project*, an attempt at organizing
future studies. For the moment I should only claim that these are
*notes* —in an intentionally general and undetermined sense.

**A note on epistemology :** I hold the common view that philosophy can not
aspire to produce certain knowledge. Its realm is that of speculation. This in
some sense justifies the prominence of science over philosophy. And yet it is
precisely on the matter of scientific knowledge that philosophy is, to my eyes,
most needed.

Scientific knowledge requires some satisfactory justification in epistemological
terms. Science itself, insofar as it is concerned with the production of
empirical assertions, can not produce such justification. The elementary
question is what do we mean when we say something is \textit{scientific
knowledge}. What is the contribution of the predicate \textit{scientific} to the
proposition? To say that the predicate refers to a special kind of method, to
which we ascribe a high degree of confidence, is not so trivial. If the set of
scientific propositions is different from others by virtue of a method only, the
question becomes: how can a method bestow a propositional formula with truth? A
method is ---speaking broadly--- nothing but a set of rules, and what makes a
set of rules superior or inferior to others must rely either in its assumptions
or its effects. So, when speaking of knowledge, to ascribe the importance of the
epithet \textit{scientific}, to a particular method simply begs the question.

We can restrict the context in which we are posing the question so as to gain a
bit of clarity. Assume two different ---but both valid--- derivations of natural
deduction with an identical conclusion $\varphi$. Assume the hypotheses
sustaining one and the other differ in the following —still undetermined— sense:
that those of the first are *scientific* and that of the second are not. What
possible meaning could such distinction have? Posing the question in this way is
better because it relieves us from the burden of dealing with the set of
scientific propositions and reduces the context to the particular hypotheses of
a specific derivation. And to my mind, such question can have one and only one
answer: that \textit{at least} every maximal ---and therefore atomic---
hypothesis is empirically observable and reproducible. I conceive the
scientific method as a logical \textit{consequence} of this fundamental
premise, which is of epistemological nature. If this were not true, then what
would be the difference between deriving a true proposition in a scientific
manner and, say, deriving it via the assumption of some arbitrary contradiction
or some suitable but arbitrary set of axioms? There are infinite paths to any
truth and almost all of them are untrue.

In short, science is not a system of logic, and its essence lies not in some
special set of inferential or methodological rules. It is a commitment to the
idea that empirically falsifiable propositional atoms are the only valid maximal
premises for any derivation. In short, it is a radical commitment to empiricism.
But empiricism is not an unproblematic, and it is important to develop an
epistemological framework that supports it. Should philosophy —and science— fail
to do so, the advancement of science as the superior method for the production
of knowledge is unjustified.

From this follows that one of the central problems of philosophy is the
justification of knowledge; particularly, of how sense data is transformed into
knowledge. The most important matter on this regard are $a.$ some justification
for the claim that atomic propositions are empirically verifiable —which implies
the postulate of an objective reality— and $b.$ the justification of inference
—which is tremendously difficult. I am of the opinion that a true commitment to
empiricism necessarily leads to basing our general conception of knowledge upon
probability. On this matter it is quite conceivable that a Bayesian take on
probability is more satisfactory than the traditional one, but I do not want to
go in detail into this. Suffices to say that a solid justification for
probabilistic claims is perhaps an incidental necessity for the development of
$a$ and $b$.

**A note on an important question :** It is a fact that we are only seemingly
ephimerous. Although it is probably true that our existence ends with death, it
certainly did not begin with birth. Every individual carries the undeniable
weight of its phylogenetic history. We are, so to speak, thousands of years old.
It is crucially important to develop a form of empiricism that takes into
account our phylogenetic endowment without being self-contradictory. An escape
from contradiction is not altogether clear: the very statement that our
intellectual predispositions can not solely be based on experience is an
empirical observation. If such observation is true —I find it undeniable—, then
it bears significant implications with regards to the limits and the nature of
our knowledge. (One is inevitably reminded of Nietzsche's extraordinary piece
*On truth and lies in a nonmoral sense*.) It is not *in principle* impossible to
develop a theory of empiricism that incorporates this fact, but its 
development is non-trivial.

**A note on ethics :** In my mind, it is impossible to produce an ethical system
whose propositions are worthy of the epithet of *knowledge*. Here, I follow the
more or less general idea that the propositions "$X$ did $Y$, which is wrong"
and "$X$ did $Y$" are entirely equivalent from a propositional perspective,
where the aggregate "which is wrong" corresponds to an emotional predisposition
rather than a property of reality. In other words, ethical predicates don't
really say anything about their subject.

Furthermore, ethics is a matter of ends, and rational thinking is concerned with
means only. There really is no way to show an ethical end is superior to
another, since they necessarily appeal to the same tautological axiom. Indeed,
the necessary tautology upon which the whole of ethics is based is the claim
that "good is preferable to evil", which is as superfluous as saying "hotness is
warmer than coldness" or "joy is better than sadness". The axiom poses a
relationship between the atoms "goodness" and "evil", and appears in this sense
similar to Euclid's "all right angles are equal to one another". But while
"right angle" is a valid name, in the sense that it unambiguously refers to an
object, "goodness" and "evil" are not any kind of fact. The statement is true
insofar as "goodness" is *by definition* that which is preferable to evil, and
"evil" is *by definition* that which is not preferable to goodness. They are the
ethical equivalents to the concepts of *function* and *expression* in
mathematics, in the sense that a function is that which is not an expression,
and an expression that which is not a function (Frege). But what can be objectively
said past this point?

In short, ethics is irreducible past the point of the tautological axiom that
claims the superiority of goodness over evil —and yet provides no grasp of the
facts to which these terms correspond. Furthermore, such correspondence appears
to be nonexistent. It follows that ethics will always be in need of philosophy
—and a good deal of affection. 

**A note on metaphysics :** In general, the pursuit of knowledge unfolds under
highly limited conditions. The best questions are precise and touch upon either
particular objects or relationships among particular objects. From where then
comes the arrogant tendency of pretending to reduce reality, the whole of
reality, to a system? Any set of propositions that deals with the entirety of
reality is suspicious to me. The only true artisans in the art of metaphysics, I
come to think, were Schopenhauer and Kant. But this only either on the very
edges or on the very core of their philosophies: whatever fills the space
in between is more or less forgettable to me. In what comes to some other big names,
like Heidegger and Hegel, I am afraid I can only hold the claim —perhaps
pretentious— that they were utter charlatans. The *Phenomenology of the Spirit*,
which I read when I was seventeen, impressed me at the moment, but as the years
went by I come to think it was nothing but a waste of time. *Being and Time*, on
the other hand, is an unbearably stupid work, consisting only of an obnoxious
insistence to make empty statements seem deep by means of
flamboyant terminology. (I should add that I hold not only an intellectual
dislike of these works, but a personal one too, insofar as it seems to me that
both to write them and to adhere to them, their meaninglessness being so
evident, can be nothing but an act of vanity.)

I should also say that the same reasons that make me weary of metaphysics also
prevented me from enjoying the work of authors that were not metaphysical at
all. A polemic example might be Foucault, in whose works I found very little of
value and the same sin of *mucho ruido pocas nueces*. In general —and in this I
agree with Chomsky— the French school of philosophy disregards every concern for
falsifiability and truth and bargains it for (self-proclaimed) originality. I
personally prefer the humble pursuit of truth than the fabrication of clever
statements.

In short, I adhere to the position that philosophy should strive for sharp
edges and clarity, and that simple and humble systems are of greater value than
complicated and transcendental ones. This makes the whole of metaphysics, but not
only metaphysics, a quite unpleasant matter to me. In this, I believe the
greatest teaching of Nietzsche was to show that one could look at those hefty
volumes and claim that they are rubbish —although regrettably enough the whole
of his philosophy is *also* nothing more than Nietzsche's *opinion*, however
penetrating.

\end{document}
