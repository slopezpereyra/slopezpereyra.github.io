\documentclass[a4paper, 12pt]{article}

\usepackage[utf8]{inputenc}
\usepackage[T1]{fontenc}
\usepackage{textcomp}
\usepackage{amssymb}
\usepackage{newtxtext} \usepackage{newtxmath}
\usepackage{amsmath, amssymb}
\newtheorem{problem}{Problem}
\newtheorem{example}{Example}
\newtheorem{lemma}{Lemma}
\newtheorem{theorem}{Theorem}
\newtheorem{problem}{Problem}
\newtheorem{example}{Example} \newtheorem{definition}{Definition}
\newtheorem{lemma}{Lemma}
\newtheorem{theorem}{Theorem}


\begin{document}

\section{V}

\subsection{}

Nada en el rostro de $V$ sugiere tristeza. La primera vez que la vi, fue
precisamente la calidez de su sonrisa, y la feliz inocencia que sus ojos
expresaban, lo que me atrajo más. Es curiosamente transparente: mientras
cortaba mi cabello, lo acariciaba con deliberada ternura. Tiene una belleza
extraña y singular. Cuando dice algo se pregunta demasiado si dijo la cosa
correcta---es notorio que lidiar con su timidez le exige una cuota genuina de
esfuerzo---pero en verdad, se expresa con una candidez tal que no admite la
posibilidad de error.

~ 

Hace algunos años, como dicen en Corrientes, \textit{se juntó}. Convivían y
estaban construyendo una casa juntos. Desconozco el desarrollo, pero el
desenlace es que una tarde se enteró que el muchacho estaba preso: había
manoseado a una joven en la calle y lo lincharon. En el trajín descubre que no
era la primera vez. Estimo que habrá cundido la confusión. Unos días después,
la madre del sujeto, que no habrá querido creer que su hijo fuera un desviado,
preguntó a $V$ si notaba algo extraño en él. Tal vez, en el fondo, intuía la
verdad. A los pocos días, la pobre mujer murió de un ataque cardiovascular,
probablemente desencadenado por la situación. El hombre acabó por confesar a
$V$ su compulsión, describiéndola como una pulsión incontrolable, una
enfermedad que él mismo condenaba pero no podía combatir. Al poco tiempo, tal
vez agotado y vencido en la batalla contra sus propios demonios, se suicidó.

~

Así, $V$ se vio, en pocas semanas, sin su pareja, sin su casa, sin su proyecto
de vida. Y, como tantas veces nos toca en esta vida, tuvo que empezar de nuevo.
Trabaja de peluquera y estudia danza árabe, y espera poder enseñar en el
futuro. Su salón de peluquería es el zaguán de la casita de estudiante que
alquila un amigo, venida a menos, llena de cacharros viejos y humedad. La
cadena del inodoro no funciona. Sin embargo, la pequeña parte del lugar que
corresponde a su salón está ordenada y limpia, y preserva un aire de absoluta
dignidad. Mientras me cortaba el pelo me expresó el deseo de tener su propio
salón en el futuro. También me contó de un videojuego de texto en que se simula
una vida alternativa: en él, ella es una exitosa estrella de música pop. Fuma
marihuana ocasionalmente, pero no parece ser un problema. Es trabajadora y no
ha podido disfrutar de su juventud con la misma facilidad que sus amigos.
Carece de resentimiento. Es afectuosa y fácil de querer.

\subsection{}

Hoy (02/05/2025) fuimos a tomar unas cervezas a Patagonia. Al llegar ya era de
noche, y el río parecía un inclemente horizonte negro. Llevaba un vestido con
patrones que recordaban a Kandinsky, blanco, rojo y negro. Cuando bajó del 
colectivo, nos saludamos con un beso y bromeó con eso, pues era nuestra
segunda cita. Después de los tragos, bajamos a zambullirnos al río. Sólo 
entonces vi la luna cruelmente roja que nos había estado llenando el corazón de
misterio y tiempo.

~

La luna se fue volviendo más roja a medida que retrocedía hacia el horizonte.
Cuando su cola rozaba la punta de los árboles chaqueños, ya parecía una gota de
sangre. Mientras tanto, llevábamos en los labios una conversación de agua y de
arena, y los suyos dejaron una nocturna huella sobre los míos. Fue entonces,
al pie del agua, cuando entendí que la quería.

~ 

Hace mucho, mucho tiempo no siento una conexión así con nadie. Por primera vez
desde que practico el amor libre, temo que un vínculo llegue a dejarme o perder
interés en mí, porque me hace mucha ilusión pensar que podamos construir algo
duradero en el tiempo. Le tomé cariño más rápido y más intensamente de lo que
estoy acostumbrado. Ahora temo como un tonto que me olvide, mi \textit{ñasaindí pytã},
cuando me vaya de Corrientes.




\end{document}



