\documentclass[a4paper, 12pt]{article}

\usepackage[utf8]{inputenc}
\usepackage[T1]{fontenc}
\usepackage{textcomp}
\usepackage{amssymb}
\usepackage{newtxtext} \usepackage{newtxmath}
\usepackage{amsmath, amssymb}
\newtheorem{problem}{Problem}
\newtheorem{example}{Example}
\newtheorem{lemma}{Lemma}
\newtheorem{theorem}{Theorem}
\newtheorem{problem}{Problem}
\newtheorem{example}{Example} \newtheorem{definition}{Definition}
\newtheorem{lemma}{Lemma}
\newtheorem{theorem}{Theorem}


\begin{document}

\section{V}

\subsection{}

Nada en el rostro de $V$ sugiere tristeza. La primera vez que la vi, fue
precisamente la calidez de su sonrisa, y la feliz inocencia que sus ojos
expresaban, lo que me atrajo más. Es curiosamente transparente: mientras
cortaba mi cabello, lo acariciaba con deliberada ternura. Tiene una belleza
extraña y singular. Cuando dice algo se pregunta demasiado si dijo la cosa
correcta---es notorio que lidiar con su timidez le exige una cuota genuina de
esfuerzo---pero en verdad, se expresa con una candidez tal que no admite la
posibilidad de error.

~ 

Hace algunos años, como dicen en Corrientes, \textit{se juntó}. Convivían y
estaban construyendo una casa juntos. Desconozco el desarrollo, pero el
desenlace es que una tarde se enteró que el muchacho estaba preso: había
manoseado a una joven en la calle y lo lincharon. En el trajín descubre que no
era la primera vez. Estimo que habrá cundido la confusión. Unos días después,
la madre del sujeto, que no habrá querido creer que su hijo fuera un desviado,
preguntó a $V$ si notaba algo extraño en él. Tal vez, en el fondo, intuía la
verdad. A los pocos días, la pobre mujer murió de un ataque cardiovascular,
probablemente desencadenado por la situación. El hombre acabó por confesar a
$V$ su compulsión, describiéndola como una pulsión incontrolable, una
enfermedad que él mismo condenaba pero no podía combatir. Al poco tiempo, tal
vez agotado y vencido en la batalla contra sus propios demonios, se suicidó.

~

Así, $V$ se vio, en pocas semanas, sin su pareja, sin su casa, sin su proyecto
de vida. Y, como tantas veces nos toca en esta vida, tuvo que empezar de nuevo.
Trabaja de peluquera y estudia danza árabe, y espera poder enseñar en el
futuro. Su salón de peluquería es el zaguán de la casita de estudiante que
alquila un amigo, venida a menos, llena de cacharros viejos y humedad. La
cadena del inodoro no funciona. Sin embargo, la pequeña parte del lugar que
corresponde a su salón está ordenada y limpia, y preserva un aire de absoluta
dignidad. Mientras me cortaba el pelo me expresó el deseo de tener su propio
salón en el futuro. También me contó de un videojuego de texto en que se simula
una vida alternativa: en él, ella es una exitosa estrella de música pop. Fuma
marihuana ocasionalmente, pero no parece ser un problema. Es trabajadora y no
ha podido disfrutar de su juventud con la misma facilidad que sus amigos.
Carece de resentimiento. Es afectuosa y fácil de querer.

\subsection{}

Hoy (02/05/2025) fuimos a tomar unas cervezas a Patagonia. Al llegar ya era de
noche, y el río parecía un inclemente horizonte negro. $V$ llevaba un vestido con
patrones que recordaban a Kandinsky, blanco, rojo y negro. Cuando bajó del 
colectivo, nos saludamos con un beso y bromeó con eso, pues era nuestra
segunda cita. Después de los tragos, bajamos a zambullirnos al río. Sólo 
entonces vi la luna cruelmente roja que nos había estado llenando el corazón de
misterio y tiempo.

~

La luna se fue volviendo más roja a medida que retrocedía hacia el horizonte.
Cuando su cola rozaba la punta de los árboles chaqueños, ya parecía una gota de
sangre. Mientras tanto, llevábamos en los labios una conversación de agua y de
arena, y los suyos dejaron una nocturna huella sobre los míos. Fue entonces,
al pie del agua, cuando entendí que la quería.

~ 

Hace mucho, mucho tiempo no siento una conexión así con nadie. Por primera vez
desde que practico el amor libre, temo que un vínculo llegue a dejarme o perder
interés en mí, porque me hace mucha ilusión pensar que podamos construir algo
duradero en el tiempo. Le tomé cariño más rápido y más intensamente de lo que
estoy acostumbrado. Ahora temo como un tonto que me olvide, mi \textit{ñasaindí pytã},
cuando me vaya de Corrientes.

\subsection{}

Hoy (02/08/2025), en un día lleno de idas y venidas, coincidimos otra vez.
Íbamos a vernos en el departamento de mi padre---el departamento no tenía luz y
el calor era intolerable, así que decidimos ir a la playa. Conversamos por un
rato, remediando el calor terrible con el agua apenas fresca del Paraná. Luego,
en cuestión de minutos, el cielo se volvió gris y un hermoso y violento viento
empezó a soplar. Una lluvia tropical feroz y apocalíptica nos empapó
completamente mientras huíamos a refugiarnos al departamento sin luz. En
verdad, fue un momento perfecto, y $V$ se veía hermosa y perfecta bajo la
lluvia cruel. 

~ 

Al llegar, nos bañamos (con ropa, tímidamente) para quitarnos la arena y el
peso del agua de lluvia. Ignoro si yo la besé primero o ella a mí, pero muy
pronto estábamos desnudos. Por reglar general, las primeras veces son
pésimas---esta fue la excepción que confirma la regla. $V$ me sigue pareciendo
un alma sensible y genuina y me gusta cada día más. Sufro al pensar en la
distancia que va a separarnos y temo lo que tal vez sea inevitable: que, una
vez me vaya de Corrientes, el doloroso movimiento de las lunas y los días
termine por borrar la humilde huella que pude haber dejado en ella.

~ 

Respecto a la huella que ella deja en mí, es más fuerte de lo que ella puede
sospechar. Por un lado, desde que practico el amor libre, es la primera
relación en que acabo sintiendo una conexión y un cariño genuinos. (Los
anteriores intentos dejaron mucho que desear: la gente es, por lo general,
decepcionante en lo concerniente a las citas y el amor romántico.) Además, es
la persona más linda y querible de todas las que conocí: la respeto, la
consisdero poseedora de virtudes hermosas, y adoro su cálida sencillez y sus
ganas de crecer y progresar en la vida. Es, lisa y llanamente, una persona 
hermosa. Y este es un recurso escaso y, por lo tanto, difícil de olvidar.

~ 

Me da un poco de ganas de cachetearla el hecho de que no sepa cuán hermosa es.
Ella me confiesa que siente inseguridad; yo lo veo poco pero, de vez en cuando,
un pequeño gesto o un comentario lo revelan. Es una chica laburante, digna, 
inteligente, honesta y linda. \textit{What the fuck?} ¿Inseguridad de qué \textit{katú}?
Bertrand Russell dijo, famosamente,

\begin{quote}
    The whole problem with the world is that fools and fanatics are always so certain of themselves, and wiser people so full of doubts.
\end{quote}

Análogamente, el mundo está lleno de gente mediocre o vil que es muy segura de
sí misma, y de gente virtuosa y digna que está llena de inseguridad. En
cualquier caso, confío en que algún día se dará cuenta de todo lo que tiene
para sentirse orgullosa. Entonces, probablemente, me deje de dar bola, porque
se va a dar cuenta de que le da para más (qué le vamos a hacer!)---pero
seguramente será más feliz. Se merece toda la felicidad del mundo.














\end{document}



